\documentclass[a4paper]{article}

\setlength{\parindent}{0pt}
\setlength{\parskip}{1em}

\pagestyle{headings}

\usepackage{amssymb}
\usepackage{amsmath}
\usepackage{amsthm}
\usepackage{mathtools}
\usepackage{graphicx}
\usepackage{hyperref}
\usepackage{color}
\usepackage{microtype}
\usepackage{tikz}
\usepackage{pgfplots}
\usepackage{pgfplotstable}

\newcommand{\N}{\mathbb{N}}
\newcommand{\Q}{\mathbb{Q}}
\newcommand{\Z}{\mathbb{Z}}
\newcommand{\R}{\mathbb{R}}
\newcommand{\C}{\mathbb{C}}
\newcommand{\D}{\mathcal{D}}
\renewcommand{\S}{\mathcal{S}}
\renewcommand{\P}{\mathbb{P}}
\newcommand{\F}{\mathbb{F}}
\newcommand{\E}{\mathbb{E}}

\graphicspath{{Image/}}

\hypersetup{
    colorlinks=true,
    linktoc=all,
    linkcolor=blue
}

\theoremstyle{definition}
\newtheorem*{axiom}{Axiom}
\newtheorem*{claim}{Claim}
\newtheorem*{conv}{Convention}
\newtheorem*{coro}{Corollary}
\newtheorem*{defi}{Definition}
\newtheorem*{eg}{Example}
\newtheorem*{lemma}{Lemma}
\newtheorem*{notation}{Notation}
\newtheorem*{prob}{Problem}
\newtheorem*{post}{Postulate}
\newtheorem*{prop}{Proposition}
\newtheorem*{rem}{Remark}
\newtheorem*{thm}{Theorem}

\DeclareMathOperator{\vdiv}{div}
\DeclareMathOperator{\grad}{grad}
\DeclareMathOperator{\curl}{curl}
\DeclareMathOperator{\Ann}{Ann}
\DeclareMathOperator{\Fit}{Fit}
\DeclareMathOperator{\Diag}{Diag}
\DeclareMathOperator{\tr}{tr}
\DeclareMathOperator{\im}{im}
\DeclareMathOperator{\Mat}{Mat}
\DeclareMathOperator{\Log}{Log}
\DeclareMathOperator{\Isom}{Isom}
\DeclareMathOperator{\Mesh}{Mesh}
\DeclareMathOperator{\Sym}{Sym}
\DeclareMathOperator{\Aut}{Aut}
\DeclareMathOperator{\cosech}{cosech}
\DeclareMathOperator{\Card}{Card}
\DeclareMathOperator{\Gal}{Gal}


\setcounter{section}{-1}

\begin{document}

\title{Quantum Computation}

\maketitle

\newpage

\tableofcontents

\newpage

\section{Introduction}
asdasd

\newpage

---Lecture 2---

\section{1}

Recall that we have an oracle $U_f$ for $f:\Z_M \to \Z_\N$ periodic, with period $r$, $A=M/r$. We want to find $r$ in $O(poly(m))$ time where $m=\log M$.

\subsection{The quantum algorithm}
Work on state space $\mathcal{H}_M \otimes \mathcal{N}$ with basis $\{|i\ket |k\ket\}_{i \in \Z_M, k \in \Z_N}$.\\
$\bullet$ Step 1. Make staet $\frac{1}{\sqrt{M}}\sum_{i=0}^{M-1} |i\ket|0\ket$.\\
$\bullet$ Step 2. Apply $U_f$ to get $\frac{1}{\sqrt{M}} \sum_{i=0}^{M-1} |i\ket |f(i)\ket$.\\
$\bullet$ Step 3. Measure the 2nd register to get a result $y$. By Born rule, the first register collapses to all those $i$'s (and only those) with $f(i)$ equal to the seen $y$, i.e. $i=x_0,x_0+r,...,x_0+(A-1)r$, where $0 \leq x_0 < r$ in 1st period has $f(m)=y$.\\
Discard 2nd register to get $|per\ket = \frac{1}{\sqrt{A}} \sum_{j=0}^{A-1}|x_0+jr\ket$.\\
Note: each of the $r$ possible function values $y$ occurs with same probability $1/r$, so $0 \leq x_0 < r$ has been chosen uniformly at random.\\
If we now measure $|per\ket$, we'd get a value $x_0+jr$ for uniformly random $j$, i.e. random element $(x_0^{th})$ of a random period $(j^{th})$, i.e. random element of $\Z_m$, so we could get no information about $r$.\\
$\bullet$ Step 4. Apply quantum Fourier transform mod $M$ (QFT) to  $|per\ket$. Recall the definition of QFT: $QFT: |x\ket \to \sum_{y=0}^{M-1} \omega^{xy} |y\ket$ for all $x \in \Z_M$ where $\omega = e^{2\pi i/M}$ is the $M$th root of unity. The existing result is that QFT mod $M$ can be implemented in $O(M^2)$ time.\\
Then we get 
\begin{equation*}
    \begin{aligned}
    QFT |per\ket &= \frac{1}{\sqrt{MA}} \sum_{j=0}^{A-1} \left(\sum_{y=0}^{M-1} \omega^{(x_0+jr)y}|y\ket\right)\\
    &= \frac{1}{\sqrt{MA}} \sum_{y=0}^{M-1} \omega^{x_0y} \left[\sum_{j=0}^{A-1} \omega^{jry}\right] |y\ket \ (*)
    \end{aligned}
\end{equation*}
where we group all the terms with the same $|y\ket$ together. One good thing is that the sum inside the square bracket is a geometric series, with ratio $\alpha = \omega^{ry} = e^{2\pi iry/M} = (e^{2\pi i/A})^y$.\\
Hence term inside bracket $=A$ if $\alpha=1$, i.e. $y=kA = k \frac{M}{r}$, $k=0,1,...,(r-1)$, and equals $0$ otherwise when $\alpha \neq 1$. Now\\
\begin{equation*}
    \begin{aligned}
        QFT |per\ket = \sqrt{\frac{A}{M}} \sum_{k=0}^{r-1} \omega^{x_0 k \frac{M}{r}} |k \frac{M}{r}\ket
    \end{aligned}
\end{equation*}
The random shift $x_0$ now appears only in phase, so measurement probabilities are now independent of $x_0$!

Measuring $QFT |per\ket$ gives a value $c$, where $c = k_0 \frac{M}{r}$ with $0 \leq k_0 \leq r-1$ chosen uniformly at random. Thus $\frac{k_0}{r} = \frac{c}{M}$, note that $c,M$ are known, $r$ is unknown (what we want), and $k_0$ is unknown but uniformly random.

So note that if we are lucky and get a $k_0$ that is coprime to $r$ then we could just simplify $\frac{c}{M}$ to get $r$. Obviously we cannot be always lucky every time, but by theorem in number theory, the number of integers $<r$ coprime to $r$ grows as $O(r/\log\log r)$ for large $r$, so we know probability of $k_0$ coprime to $r$ is $O(\frac{1}{\log\log r})$.

Then by some probability calculation we know that $O(1/p)$ trials are enough to achieve $1-\varepsilon$ probability of success.

So afer Step 4, cancel $c/M$ to the lowest terms $a/b$, giving $r$ as denominator $b$ (if $k_0$ is coprime to $r$). Check $b$ value by computing $f(0)$ and $f(b)$, since $b=r$ iff $f(0) = f(b)$.

Repeating $K=O(\log\log r)$ times gives $r$ with any desired probability.

Further insights into utility of QFT here:\\
Write $R = \{0,r,2r,...,(A-1)r\} \subseteq \Z_M$. $|R\ket = \frac{1}{\sqrt{A}} \sum_{k=0}^{A-1} |kr\ket$, and $|per\ket = |x_0 + R\ket = \frac{1}{\sqrt{A}} \sum_{k=0}^{A-1} |x_0+br \ket$ where $x_0$ is the random shift that caused problem previously.\\
For each $x_0 \in \Z_M$, consider mapping $k \to k+x_0$ (shift by $x_0$) on $\Z_M$, which is a 1-1 invertible map.

So linear map $U(x_0)$ on $\mathcal{H}_M$ defined by $U(x_0): |k\ket \to |k+x_0\ket$ is unitary, and $|x_0+R \ket = U(x_0) |R\ket$.

Since $(\Z_M,+)$ is abelian, $U(x_0)U(x_1) = U(x_0+x_1) =U(x_1)U(x_0)$ i.e. all $U(x_0)$'s commute as operators on $\mathcal{H}_M$.\\
So we have orthonormal basis of common eigenvectors $|\chi_k\ket\}_{k \in \Z_M}$, called \emph{shift invariant states}.

$U(x_0)|\chi_k\ket = \omega(x_0,k)|\chi_k\ket$ for all $x_0,k \in \Z_M$ with $|\omega(x_0,k)| = 1$. Now consider $|R\ket$ written in $|\chi\ket$ basis,\\
$|R\ket = \sum_{k=0}^{M-1} a_k | \chi_k\ket$ where $a_k$'s depending on $r$ (not $x_0$).\\
Then $|per\ket = U(x_0) |R\ket = \sum_{k=0}^{M-1}a_k \omega(x_0,k) |\chi_k\ket$, and measurement in the $\chi$-basis has $prob(k) = |a_k \omega(x_0,k)|^2 = |a_k|^2$ which is independent of $x_0$, i.e. giving information about $r$!

---Lecture 3---

Exercise classes: Sat 3 Nov 11am MR4, Sat 24 Nov 11am MR4, early next term (tba).\\
Thursday 8 November lecture is moved to Saturday 10 November 11am (still MR4).

Recall last time we had $\mathcal{H}_M$: shift operations $U(x_0) |y \ket = |y+x_0\ket$ for $x_0,y \in \Z_M$, which all permute, so have a common eigenbasis (shift invariant states) $\{|\chi_k\ket \}_{k \in \Z_M}$, $U(x_0) | x_k\ket = \omega(x_0,k) |\chi_k\ket$.\\
Measurement of $|x_0+R \ket = \frac{1}{\sqrt{A}} \sum_{l=0}^{A-1} |x_0 + l_r \ket = U(x_0) |R\ket$ in $|\chi\ket$ basis has output distribution independent of $x_0$, therefore gives information about $r$.

Introduce QFT as the unitary mapping that rotates $\chi$-basis to standard basis, i.e. define $QFT|\chi_k \ket = |k\ket$. So QFT followed by measurement implements $\chi$-basis measurement.

Explicit form of $|\chi_k\ket$ eigenspaces (!): consider
\begin{equation*}
    \begin{aligned}
        |\chi_k\ket = \frac{1}{\sqrt{M}}\sum_{l=0}^{M-1} e^{-2\pi i kl/M}|l\ket
    \end{aligned}
\end{equation*}
Then
\begin{equation*}
    \begin{aligned}
        U(x_0) |\chi_k \ket &= \frac{1}{\sqrt{M}} \sum_{l=0}^{M-1} e^{-2\pi i kl/M}|l+x_0\ket \\
        &= \frac{1}{\sqrt{M}} \sum_{\tilde{l}=0}^{M-1} e^{-2\pi i k (\tilde{l}-x_0)/M} | \tilde{l} \ket \text { where } \tilde{l} = l+x_0\\
        &= e^{2\pi i k x_0 / M} \cdot |\chi_k\ket
    \end{aligned}
\end{equation*}
i.e. these are the shift invariant staets, eigenvalues $\omega(x_0,k) = e^{2\pi i k x_0/M}$.

Matrix of QFT: So
\begin{equation*}
    \begin{aligned}
        [QFT^{-1}]_{lk} = \frac{1}{\sqrt{M}} e^{-2\pi i lk/M}
    \end{aligned}
\end{equation*}
(componets of $|\chi_k \ket = QFT^{-1} |k\ket$ as $k^{th}$ column). So
\begin{equation*}
    \begin{aligned}
        [QFT]_{kl} = \frac{1}{\sqrt{M}} e^{2\pi i lk / M}
    \end{aligned}
\end{equation*}
as expected.

\newpage

\section{The hidden subgroup problem (HSP)}

Let $G$ be a finite group of size $|G|$. Given (oracle for) function $f:G \to X$ ($X$ is some set), and promise that there is a subgroup $K < G$ such that $f$ is constant on (left) cosets of $K$ in $G$, and $f$ is distinct on distinct cosets.\\
The problem: determine the \emph{hidden subgroup} $K$ (e.g. output a set of generators, or sample uniformly from $K$).\\
We want to solve in time $O(poly(\log |G|))$ (an efficient algorithm) with any constant probability $1-\varepsilon$.

Examples of problems that can be cast(?) as HSPs:\\
(i) periodicity: $f:\Z_M \to X$, periodic with period $r$. Let $G=(\Z_m,+)$, the hidden subgroup is $K=\{0,r,2r,...\} < G$, cosets $x_0+K = \{x_0,x_0+r,x_0+2r,...\}$. The period $r$ is generator of $K$.\\
(ii) discrete logarithm: for prime $p$, $\Z_p^* = \{1,2,...,p-1\}$ with multiplication mod $p$. $g \in \Z_p^*$ is a generator (or primitive root mod $p$). If powers generate all of $\Z_p^*$, $\Z_p^* = \{g^0 = 1,g^1,...,g^{p-2}\}$, then also $g^{p-1} \equiv 1 \pmod p$ (easy number theory).\\
Fact: the generator always exists if $p$ is prime. So any $x \in \Z_p^*$ can be written $x = g^y$ for some $y \in \Z_{p-1}$, write $y=\log_g x$ called the discrete log of $x$ to base $g$.

Discrete log problem: given a generator $g$ and $x \in \Z^*_p$, compute $y=\log_g x$ (classically hard).\\
To express as HSP, consider $f:\Z_{p-1} \times \Z_{p-1} \to \Z_p^*$: $f(a,b) = g^a x^{-b} \ mod \ p = g^{a-yb} \ mod \ p$.\\
Then check: $f(a_1,b_1) = f(a_2,b_2)$ iff $(a_2,b_2) = (a_1,b_1)+\lambda(y,1)$ where $\lambda \in \Z_{p-1}$.

So if $G = \Z_{p-1} \times \Z_{p-1}$, $K=\{\lambda(y,1):\lambda \in \Z_{p-1}\} < G$. Then $f$ is constant and distinct on the cosets of $K$ in $G$, and generator $(y,1)$ gives $y=\log_g x$.\\

(iii) graph problems ($G$ non-abelian now): consider undirected graph $A = \{V,E\}$, $|V| =n$, with at most one edge between any two vertices. Label vertices by $[n] = \{1,2,...,n\}$.\\
Introduce the permutation group $\mathcal{P}_n$ of $[n]$. Define $Aut(A)$ to be the group of automorphisms of $A$, which is a subgroup of $\mathcal{P}_n$, containing exactly the permutations $\pi \in \mathcal{P}_n$ such that for all $i,j \in [n]$, $(i,j) \in E \iff (\pi(i),\pi(j)) \in E$, i.e. the labelled graph $\pi(A)$ obtained by permuting labels of $A$ by $\pi$ is the same \emph{labelled} graph as $A$.

Associated HSP: Take $G = \mathcal{P}_n$. Let $X$ be set of all labelled graphs on $n$ vertices. Given $A$, consider $f_A: \mathcal{P}_n \to X$ by $f_A(\pi) = \pi(A)$, $A$ with labels permuted by $\pi$. The associated hiiden subroup is $Aut(A) = K$.

Application: if we can sample uniformly from this $K$, then we can solve graph isomorphism problem (GI): two labelled graphs $A,B$ are isomorphic if there is 1-1 map $\pi:[n] \to [n]$ such that for all $i,j \in [n]$, $i,j$ is an edge in $A$ iff $\pi(i),\pi(j)$ is an edge in $B$, i.e. $A$ and $B$ are the same graph but just labelled differently.

\end{document}
