\documentclass[a4paper]{article}

\setlength{\parindent}{0pt}
\setlength{\parskip}{1em}

\pagestyle{headings}

\usepackage{amssymb}
\usepackage{amsmath}
\usepackage{amsthm}
\usepackage{mathtools}
\usepackage{graphicx}
\usepackage{hyperref}
\usepackage{color}
\usepackage{microtype}
\usepackage{tikz}
\usepackage{pgfplots}
\usepackage{pgfplotstable}

\newcommand{\N}{\mathbb{N}}
\newcommand{\Q}{\mathbb{Q}}
\newcommand{\Z}{\mathbb{Z}}
\newcommand{\R}{\mathbb{R}}
\newcommand{\C}{\mathbb{C}}
\newcommand{\D}{\mathcal{D}}
\renewcommand{\S}{\mathcal{S}}
\renewcommand{\P}{\mathbb{P}}
\newcommand{\F}{\mathbb{F}}
\newcommand{\E}{\mathbb{E}}

\graphicspath{{Image/}}

\hypersetup{
    colorlinks=true,
    linktoc=all,
    linkcolor=blue
}

\theoremstyle{definition}
\newtheorem*{axiom}{Axiom}
\newtheorem*{claim}{Claim}
\newtheorem*{conv}{Convention}
\newtheorem*{coro}{Corollary}
\newtheorem*{defi}{Definition}
\newtheorem*{eg}{Example}
\newtheorem*{lemma}{Lemma}
\newtheorem*{notation}{Notation}
\newtheorem*{prob}{Problem}
\newtheorem*{post}{Postulate}
\newtheorem*{prop}{Proposition}
\newtheorem*{rem}{Remark}
\newtheorem*{thm}{Theorem}

\DeclareMathOperator{\vdiv}{div}
\DeclareMathOperator{\grad}{grad}
\DeclareMathOperator{\curl}{curl}
\DeclareMathOperator{\Ann}{Ann}
\DeclareMathOperator{\Fit}{Fit}
\DeclareMathOperator{\Diag}{Diag}
\DeclareMathOperator{\tr}{tr}
\DeclareMathOperator{\im}{im}
\DeclareMathOperator{\Mat}{Mat}
\DeclareMathOperator{\Log}{Log}
\DeclareMathOperator{\Isom}{Isom}
\DeclareMathOperator{\Mesh}{Mesh}
\DeclareMathOperator{\Sym}{Sym}
\DeclareMathOperator{\Aut}{Aut}
\DeclareMathOperator{\cosech}{cosech}
\DeclareMathOperator{\Card}{Card}
\DeclareMathOperator{\Gal}{Gal}


\setcounter{section}{-1}

\begin{document}

\title{Combinatorics}

\maketitle

\newpage

\tableofcontents

\newpage

\section{Introduction}

In this course we'll discuss three main aspects:\\
$\bullet$ Set systems;\\
$\bullet$ Isoperimetric Inequalities;\\
$\bullet$ Projections (combinatorics in continuous settings).

References:\\
\emph{Combinatorics}, Bocabas, Cambridge University Press, 1986 (chapter 1,2);\\
\emph{Combinatorics and finite sets}, Anderson, Oxford University Press, 1987 (chapter 1).

\newpage

\section{Set Systems}

Let $X$ be a set. A \emph{set system} on $X$ (or family of subsets of $X$) is a family $\mathcal{A} \subset \P(X)$.\\
For example, we define $X^{(r)} = \{A \subset X: |A| = r\}$.

Unless otherwise stated, $X=[n] = \{1,2,...,n\}$. For example, $|X^{(r)}| = {n \choose r}$ (assume finiteness). So $[4]^{(2)} = \{12,13,14,23,24,34\}$.

We often make $\P(x)$ into a graph, called $Q_n$, by joining $A$ to $B$ if $|A \triangle B| = 1$ (symmetric difference).

(examples of $Q_3,Q_n$)

If we identify a set $A \subset X$ with a 0-1 sequence of length $n$ via $A \leftrightarrow 1_A$ (characteristic function), then $Q_3$ acn be thought of as a cube. In general, $Q_n$ is an $n$-dimensional cube (hypercube/discretecube/$n$-cube/...).

\subsection{Chains and antichains}
A family $\mathcal{A} \subset \P(X)$ is a \emph{chain} if $\forall A,B \in \mathcal{A}, A \subset B$ or $B \subset A$. It is an antichain if $\forall A \neq B \in \mathcal{A}$, $A \not\in B$.

Obviously the maximum size of a chain in $X$ is $n+1$.

For antichains, we can take $X^{\lfloor\frac{n}{2}\rfloor}$, which has size ${n \choose {\lfloor n/2 \rfloor}}$. The result is that wee can't beat this, but the proof is not trivial.

\end{document}
