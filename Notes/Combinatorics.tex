\documentclass[a4paper]{article}

\setlength{\parindent}{0pt}
\setlength{\parskip}{1em}

\pagestyle{headings}

\usepackage{amssymb}
\usepackage{amsmath}
\usepackage{amsthm}
\usepackage{mathtools}
\usepackage{graphicx}
\usepackage{hyperref}
\usepackage{color}
\usepackage{microtype}
\usepackage{tikz}
\usepackage{pgfplots}
\usepackage{pgfplotstable}

\newcommand{\N}{\mathbb{N}}
\newcommand{\Q}{\mathbb{Q}}
\newcommand{\Z}{\mathbb{Z}}
\newcommand{\R}{\mathbb{R}}
\newcommand{\C}{\mathbb{C}}
\newcommand{\D}{\mathcal{D}}
\renewcommand{\S}{\mathcal{S}}
\renewcommand{\P}{\mathbb{P}}
\newcommand{\F}{\mathbb{F}}
\newcommand{\E}{\mathbb{E}}
\newcommand{\bra}{\langle}
\newcommand{\ket}{\rangle}


\graphicspath{{Image/}}

\hypersetup{
    colorlinks=true,
    linktoc=all,
    linkcolor=blue
}

\theoremstyle{definition}
\newtheorem*{axiom}{Axiom}
\newtheorem*{claim}{Claim}
\newtheorem*{conv}{Convention}
\newtheorem*{coro}{Corollary}
\newtheorem*{defi}{Definition}
\newtheorem*{eg}{Example}
\newtheorem*{lemma}{Lemma}
\newtheorem*{notation}{Notation}
\newtheorem*{prob}{Problem}
\newtheorem*{post}{Postulate}
\newtheorem*{prop}{Proposition}
\newtheorem*{rem}{Remark}
\newtheorem*{thm}{Theorem}

\DeclareMathOperator{\vdiv}{div}
\DeclareMathOperator{\grad}{grad}
\DeclareMathOperator{\curl}{curl}
\DeclareMathOperator{\Ann}{Ann}
\DeclareMathOperator{\Fit}{Fit}
\DeclareMathOperator{\Diag}{Diag}
\DeclareMathOperator{\tr}{tr}
\DeclareMathOperator{\im}{im}
\DeclareMathOperator{\Mat}{Mat}
\DeclareMathOperator{\Log}{Log}
\DeclareMathOperator{\Isom}{Isom}
\DeclareMathOperator{\Mesh}{Mesh}
\DeclareMathOperator{\Sym}{Sym}
\DeclareMathOperator{\Aut}{Aut}
\DeclareMathOperator{\cosech}{cosech}
\DeclareMathOperator{\Card}{Card}
\DeclareMathOperator{\Gal}{Gal}


\setcounter{section}{-1}

\begin{document}

\title{Combinatorics}

\maketitle

\newpage

\tableofcontents

\newpage

\section{Introduction}

In this course we'll discuss three main aspects:\\
$\bullet$ Set systems;\\
$\bullet$ Isoperimetric Inequalities;\\
$\bullet$ Projections (combinatorics in continuous settings).

References:\\
\emph{Combinatorics}, Bocabas, Cambridge University Press, 1986 (chapter 1,2);\\
\emph{Combinatorics and finite sets}, Anderson, Oxford University Press, 1987 (chapter 1).

\newpage

\section{Set Systems}

Let $X$ be a set. A \emph{set system} on $X$ (or family of subsets of $X$) is a family $\mathcal{A} \subset \P(X)$.\\
For example, we define $X^{(r)} = \{A \subset X: |A| = r\}$.

Unless otherwise stated, $X=[n] = \{1,2,...,n\}$. For example, $|X^{(r)}| = {n \choose r}$ (assume finiteness). So $[4]^{(2)} = \{12,13,14,23,24,34\}$.

We often make $\P(x)$ into a graph, called $Q_n$, by joining $A$ to $B$ if $|A \triangle B| = 1$ (symmetric difference).

(examples of $Q_3,Q_n$)

If we identify a set $A \subset X$ with a 0-1 sequence of length $n$ via $A \leftrightarrow 1_A$ (characteristic function), then $Q_3$ acn be thought of as a cube. In general, $Q_n$ is an $n$-dimensional cube (hypercube/discretecube/$n$-cube/...).

\subsection{Chains and antichains}
A family $\mathcal{A} \subset \P(X)$ is a \emph{chain} if $\forall A,B \in \mathcal{A}, A \subset B$ or $B \subset A$. It is an antichain if $\forall A \neq B \in \mathcal{A}$, $A \not\in B$.

Obviously the maximum size of a chain in $X$ is $n+1$.

For antichains, we can take $X^{\lfloor\frac{n}{2}\rfloor}$, which has size ${n \choose {\lfloor n/2 \rfloor}}$. The result is that wee can't beat this, but the proof is not trivial.

---Lecture 2---

No lecture this thursday (11 Oct 2018)!

Idea: inspired by \emph{each chain meets each level $X^{(r)}$ in at most one place} -- try to decompose $Q_n$ into chains.

\begin{thm} (Sperner's Lemma)\\
    Let $A \subset \P(X)$ be an antichain. Then $|A| \leq {n \choose {\lfloor n/2 \rfloor}}$.
    \begin{proof}
        It's sufficient to partitoin $\P(X)$ into that many chains (since an anti-chain and a chain can have at most one common vertex).\\
        For this, it's sufficient to show:\\
        $\bullet$ $\forall r < n/2$, there exists a matching (set of disjoint edges) from $X^{(r)}$ to $X^{(r+1)}$;\\
        $\bullet$ $\forall r > n/2$, there exists a matching from $X^{(r)}$ to $X^{(r-1)}$.\\
        (Then put these matchings together to form chains, each passing through $X^{(\lfloor n/2 \rfloor)})$, so the result.\\
        By taking complements it's sufficient to prove (i).\\
        Consider subgraph of $Q_n$ spanned by $X^{(r)} \cup X^{(r+1)}$ which is bipartite. For any $B \subset X^{(r)}$, we have:\\
        $\bullet$ number of $B-\P(B)$ edges = $|B| (n-r)$; (each point in $X^{(r)}$ has degree $(n-r)$)\\
        $\bullet$ number of $B-\P(B)$ edges $\leq$ $|\P(B)|(r+1)$. (each point in $X^{(r+1)})$ has degree $r+1$)\\
        Thus $|\P(B)| \geq |B| \frac{n-r}{r+1} \geq |B|$, as $r < n/2$.\\
        Hence by Hall's theorem there exists a matching.
    \end{proof}
\end{thm}

\begin{rem}
    $\bullet$ 1. ${n \choose {\lfloor n/2 \rfloor}}$ is achievable by just taking $X^{(\lfloor n/2 \rfloor)}$.\\
    $\bullet$ 2. This proof says nothing about extremal cases: which antichains have size ${n \choose {\lfloor n/2 \rfloor}}$?
\end{rem}

Aim: For $\mathcal{A}$ an antichain, $\sum_{r=0}^n \frac{|\mathcal{A} \cap X^{(r)}}{{n \choose r}} \leq 1$. Note that this trivally implies Sperner's lemma.

Let $\mathcal{A} \subset X^{(r)}$ for some $1 \leq r \leq n$. The \emph{shadow} or \emph{lower shadow} of $\mathcal{A}$ is
\begin{equation*}
    \begin{aligned}
        \partial A = \partial^- A = \{A-\{i\}:A \in \mathcal{A},i \in A\}
    \end{aligned}
\end{equation*}
So $\partial A \subset X^{(r-1)}$.

For example, if $\mathcal{A} = \{123,124,134,135\} \subset X^{(3)}$, then $\partial A =\{12,13,23,14,24,34,15,35\} \subset X^{(2)}$.

\begin{lemma} (Local LYM)\\
    Let $\mathcal{A} \subset X^{(r)}$, $1 \leq r \leq n$. Then 
    \begin{equation*}
        \begin{aligned}
            \frac{|\partial \mathcal{A}|}{{n \choose {r-1}}} \geq \frac{|\mathcal{A}|}{{n \choose r}}
        \end{aligned}
    \end{equation*}
    (\emph{the fraction of the layer occupied increases when we take the shadow}.)
    \begin{proof}
        $\bullet$ Number of $\mathcal{A}-\partial\mathcal{A}$ edges (in $Q_n$) = $r|\mathcal{A}|$ (counting from above);\\
        $\bullet$ Number of $\mathcal{A}-\partial\mathcal{A}$ edges $\leq$ $(n-r+1)|\partial \mathcal{A}|$ (counting from below).\\
        So
        \begin{equation*}
        \begin{aligned}
            \frac{|\partial \mathcal{A}|}{|\mathcal{A}|} \geq \frac{r}{n-r+1}
        \end{aligned}
        \end{equation*}
        However RHS is the ratio of size between the two layers.
    \end{proof}
\end{lemma}
Let's consider when is equality achieved in local LYM. we need $A-\{i\} \cup \{j\} \in \mathcal{A}$ $\forall a \in \mathcal{A}, i \in A, j \not\in A$.\\
Hence $\mathcal{A} = X^{(r)}$ or $\phi$.

\begin{thm} (Lubell-Yamamoto-Meshalkin inequality)\\
    Let $\mathcal{A} \subset \P(X)$ be an antichain. Then $\sum_{r=0}^n \frac{|\mathcal{A} \cap X^{(r)}}{{n \choose r}} \leq 1$.
    \begin{proof} (1, \emph{Bubble down with local LYM})\\
        Let's start with $X^{(r)}$. Write $\mathcal{A}_r$ for $\mathcal{A} \cap X^{(r)}$.\\
        We have $\frac{|\mathcal{A}_n|}{{n\choose n}} \leq 1$ (trivially).\\
        Also, $\partial \mathcal{A}_n$ and $\mathcal{A}_{n-1}$ are disjoint (as $\mathcal{A}$ is an antichain). So 
        \begin{equation*}
        \begin{aligned}
            \frac{|\partial \mathcal{A}_n|}{{n \choose {n-1}}} + \frac{|\mathcal{A}_{n-1}|}{{n \choose {n-1}}} = \frac{|\partial \mathcal{A}_n \cup \mathcal{A}_{n-1}}{{n \choose {n-1}}} \leq 1
        \end{aligned}
        \end{equation*}
        So
        \begin{equation*}
        \begin{aligned}
            \frac{|\mathcal{A}_n|}{{n \choose n}} + \frac{|\mathcal{A}_{n-1}|}{{n \choose {n-1}}} \leq 1
        \end{aligned}
        \end{equation*}
        by local LYM. Note that we have successfully expanded LHS to two terms.\\
        Also, $\partial (\partial \mathcal{A}_n \cup \mathcal{A}_{n-1})$ is disjoint from $\mathcal{A}_{n-2}$ again since $\mathcal{A}$ is an antichain. So 
        \begin{equation*}
            \begin{aligned}
                \frac{|\partial(\partial \mathcal{A}_n \cup \mathcal{A}_{n-1})|}{{n \choose {n-2}}} + \frac{|\mathcal{A}_{n-2}|}{{n \choose {n-2}}} \leq 1
            \end{aligned}
        \end{equation*}
        So
        \begin{equation*}
            \begin{aligned}
                \frac{|\partial \mathcal{A}_n \cup \mathcal{A}_{n-1}|}{{n \choose {n-1}}} + \frac{|\mathcal{A}_{n-2}|}{{n \choose {n-2}}} \leq 1
            \end{aligned}
        \end{equation*}
        So
        \begin{equation*}
            \begin{aligned}
                \frac{|\mathcal{A}_n|}{{n \choose {n}}} + \frac{|\mathcal{A}_{n-1}|}{{n \choose {n-1}}} + \frac{|\mathcal{A}_{n-2}|}{{n \choose {n-2}}} \leq 1
            \end{aligned}
        \end{equation*}
        Keep going and we obtain the desired result.
    \end{proof}
\end{thm}

When is equality achieved in LYM? We must have equality in each use of local LYM, so the first $r$ with $\mathcal{A}_r \neq \phi$ must have $\mathcal{A}_r = X^{(r)}$, i.e. $\mathcal{A} = X^{(r)}$.

Hence equality in Sperner's lemma is only achieved when $\mathcal{A} = X^{\lfloor n/2 \rfloor}$ for $n$ even, or also $X^{\lceil n/2 \rceil}$ when $n$ is odd.
\end{document}
