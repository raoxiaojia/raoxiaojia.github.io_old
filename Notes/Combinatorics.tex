\documentclass[a4paper]{article}

\setlength{\parindent}{0pt}
\setlength{\parskip}{1em}

\pagestyle{headings}

\usepackage{amssymb}
\usepackage{amsmath}
\usepackage{amsthm}
\usepackage{mathtools}
\usepackage{graphicx}
\usepackage{hyperref}
\usepackage{color}
\usepackage{microtype}
\usepackage{tikz}
\usepackage{pgfplots}
\usepackage{pgfplotstable}

\newcommand{\N}{\mathbb{N}}
\newcommand{\Q}{\mathbb{Q}}
\newcommand{\Z}{\mathbb{Z}}
\newcommand{\R}{\mathbb{R}}
\newcommand{\C}{\mathbb{C}}
\newcommand{\D}{\mathcal{D}}
\renewcommand{\S}{\mathcal{S}}
\renewcommand{\P}{\mathbb{P}}
\newcommand{\F}{\mathbb{F}}
\newcommand{\E}{\mathbb{E}}
\newcommand{\bra}{\langle}
\newcommand{\ket}{\rangle}


\graphicspath{{Image/}}

\hypersetup{
    colorlinks=true,
    linktoc=all,
    linkcolor=blue
}

\theoremstyle{definition}
\newtheorem*{axiom}{Axiom}
\newtheorem*{claim}{Claim}
\newtheorem*{conv}{Convention}
\newtheorem*{coro}{Corollary}
\newtheorem*{defi}{Definition}
\newtheorem*{eg}{Example}
\newtheorem*{lemma}{Lemma}
\newtheorem*{notation}{Notation}
\newtheorem*{prob}{Problem}
\newtheorem*{post}{Postulate}
\newtheorem*{prop}{Proposition}
\newtheorem*{rem}{Remark}
\newtheorem*{thm}{Theorem}

\DeclareMathOperator{\vdiv}{div}
\DeclareMathOperator{\grad}{grad}
\DeclareMathOperator{\curl}{curl}
\DeclareMathOperator{\Ann}{Ann}
\DeclareMathOperator{\Fit}{Fit}
\DeclareMathOperator{\Diag}{Diag}
\DeclareMathOperator{\tr}{tr}
\DeclareMathOperator{\im}{im}
\DeclareMathOperator{\Mat}{Mat}
\DeclareMathOperator{\Log}{Log}
\DeclareMathOperator{\Isom}{Isom}
\DeclareMathOperator{\Mesh}{Mesh}
\DeclareMathOperator{\Sym}{Sym}
\DeclareMathOperator{\Aut}{Aut}
\DeclareMathOperator{\cosech}{cosech}
\DeclareMathOperator{\Card}{Card}
\DeclareMathOperator{\Gal}{Gal}


\setcounter{section}{-1}

\begin{document}

\title{Combinatorics}

\maketitle

\newpage

\tableofcontents

\newpage

\section{Introduction}

In this course we'll discuss three main aspects:\\
$\bullet$ Set systems;\\
$\bullet$ Isoperimetric Inequalities;\\
$\bullet$ Projections (combinatorics in continuous settings).

References:\\
\emph{Combinatorics}, Bocabas, Cambridge University Press, 1986 (chapter 1,2);\\
\emph{Combinatorics and finite sets}, Anderson, Oxford University Press, 1987 (chapter 1).

\newpage

\section{Set Systems}

Let $X$ be a set. A \emph{set system} on $X$ (or family of subsets of $X$) is a family $\mathcal{A} \subset \P(X)$.\\
For example, we define $X^{(r)} = \{A \subset X: |A| = r\}$.

Unless otherwise stated, $X=[n] = \{1,2,...,n\}$. For example, $|X^{(r)}| = {n \choose r}$ (assume finiteness). So $[4]^{(2)} = \{12,13,14,23,24,34\}$.

We often make $\P(x)$ into a graph, called $Q_n$, by joining $A$ to $B$ if $|A \triangle B| = 1$ (symmetric difference).

(examples of $Q_3,Q_n$)

If we identify a set $A \subset X$ with a 0-1 sequence of length $n$ via $A \leftrightarrow 1_A$ (characteristic function), then $Q_3$ acn be thought of as a cube. In general, $Q_n$ is an $n$-dimensional cube (hypercube/discretecube/$n$-cube/...).

\subsection{Chains and antichains}
A family $\mathcal{A} \subset \P(X)$ is a \emph{chain} if $\forall A,B \in \mathcal{A}, A \subset B$ or $B \subset A$. It is an antichain if $\forall A \neq B \in \mathcal{A}$, $A \not\in B$.

Obviously the maximum size of a chain in $X$ is $n+1$.

For antichains, we can take $X^{\lfloor\frac{n}{2}\rfloor}$, which has size ${n \choose {\lfloor n/2 \rfloor}}$. The result is that wee can't beat this, but the proof is not trivial.

---Lecture 2---

No lecture this thursday (11 Oct 2018)!

Idea: inspired by \emph{each chain meets each level $X^{(r)}$ in at most one place} -- try to decompose $Q_n$ into chains.

\begin{thm} (Sperner's Lemma)\\
    Let $A \subset \P(X)$ be an antichain. Then $|A| \leq {n \choose {\lfloor n/2 \rfloor}}$.
    \begin{proof}
        It's sufficient to partition $\P(X)$ into that many chains (since an anti-chain and a chain can have at most one common vertex).\\
        For this, it's sufficient to show:\\
        $\bullet$ $\forall r < n/2$, there exists a matching (set of disjoint edges) from $X^{(r)}$ to $X^{(r+1)}$;\\
        $\bullet$ $\forall r > n/2$, there exists a matching from $X^{(r)}$ to $X^{(r-1)}$.\\
        (Then put these matchings together to form chains, each passing through $X^{(\lfloor n/2 \rfloor)})$, so the result.\\
        By taking complements it's sufficient to prove (i).\\
        Consider subgraph of $Q_n$ spanned by $X^{(r)} \cup X^{(r+1)}$ which is bipartite. For any $B \subset X^{(r)}$, we have:\\
        $\bullet$ number of $B-\P(B)$ edges = $|B| (n-r)$; (each point in $X^{(r)}$ has degree $(n-r)$)\\
        $\bullet$ number of $B-\P(B)$ edges $\leq$ $|\P(B)|(r+1)$. (each point in $X^{(r+1)})$ has degree $r+1$)\\
        Thus $|\P(B)| \geq |B| \frac{n-r}{r+1} \geq |B|$, as $r < n/2$.\\
        Hence by Hall's theorem there exists a matching.
    \end{proof}
\end{thm}

\begin{rem}
    $\bullet$ 1. ${n \choose {\lfloor n/2 \rfloor}}$ is achievable by just taking $X^{(\lfloor n/2 \rfloor)}$.\\
    $\bullet$ 2. This proof says nothing about extremal cases: which antichains have size ${n \choose {\lfloor n/2 \rfloor}}$?
\end{rem}

Aim: For $\mathcal{A}$ an antichain, $\sum_{r=0}^n \frac{|\mathcal{A} \cap X^{(r)}|}{{n \choose r}} \leq 1$. Note that this trivally implies Sperner's lemma.

Let $\mathcal{A} \subset X^{(r)}$ for some $1 \leq r \leq n$. The \emph{shadow} or \emph{lower shadow} of $\mathcal{A}$ is
\begin{equation*}
    \begin{aligned}
        \partial A = \partial^- A = \{A-\{i\}:A \in \mathcal{A},i \in A\}
    \end{aligned}
\end{equation*}
So $\partial A \subset X^{(r-1)}$.

For example, if $\mathcal{A} = \{123,124,134,135\} \subset X^{(3)}$, then $\partial A =\{12,13,23,14,24,34,15,35\} \subset X^{(2)}$.

\begin{lemma} (Local LYM)\\
    Let $\mathcal{A} \subset X^{(r)}$, $1 \leq r \leq n$. Then 
    \begin{equation*}
        \begin{aligned}
            \frac{|\partial \mathcal{A}|}{{n \choose {r-1}}} \geq \frac{|\mathcal{A}|}{{n \choose r}}
        \end{aligned}
    \end{equation*}
    (\emph{the fraction of the layer occupied increases when we take the shadow}.)
    \begin{proof}
        $\bullet$ Number of $\mathcal{A}-\partial\mathcal{A}$ edges (in $Q_n$) = $r|\mathcal{A}|$ (counting from above);\\
        $\bullet$ Number of $\mathcal{A}-\partial\mathcal{A}$ edges $\leq$ $(n-r+1)|\partial \mathcal{A}|$ (counting from below).\\
        So
        \begin{equation*}
        \begin{aligned}
            \frac{|\partial \mathcal{A}|}{|\mathcal{A}|} \geq \frac{r}{n-r+1}
        \end{aligned}
        \end{equation*}
        However RHS is the ratio of size between the two layers.
    \end{proof}
\end{lemma}
Let's consider when is equality achieved in local LYM. we need $A-\{i\} \cup \{j\} \in \mathcal{A}$ $\forall A \in \mathcal{A}, i \in A, j \not\in A$.\\
Hence $\mathcal{A} = X^{(r)}$ or $\phi$.

\begin{thm} (Lubell-Yamamoto-Meshalkin inequality)\\
    Let $\mathcal{A} \subset \P(X)$ be an antichain. Then $\sum_{r=0}^n \frac{|\mathcal{A} \cap X^{(r)}|}{{n \choose r}} \leq 1$.
    \begin{proof} (1, \emph{Bubble down with local LYM})\\
        Let's start with $X^{(r)}$. Write $\mathcal{A}_r$ for $\mathcal{A} \cap X^{(r)}$.\\
        We have $\frac{|\mathcal{A}_n|}{{n\choose n}} \leq 1$ (trivially).\\
        Also, $\partial \mathcal{A}_n$ and $\mathcal{A}_{n-1}$ are disjoint (as $\mathcal{A}$ is an antichain). So 
        \begin{equation*}
        \begin{aligned}
            \frac{|\partial \mathcal{A}_n|}{{n \choose {n-1}}} + \frac{|\mathcal{A}_{n-1}|}{{n \choose {n-1}}} = \frac{|\partial \mathcal{A}_n \cup \mathcal{A}_{n-1}}{{n \choose {n-1}}} \leq 1
        \end{aligned}
        \end{equation*}
        So
        \begin{equation*}
        \begin{aligned}
            \frac{|\mathcal{A}_n|}{{n \choose n}} + \frac{|\mathcal{A}_{n-1}|}{{n \choose {n-1}}} \leq 1
        \end{aligned}
        \end{equation*}
        by local LYM. Note that we have successfully expanded LHS to two terms.\\
        Also, $\partial (\partial \mathcal{A}_n \cup \mathcal{A}_{n-1})$ is disjoint from $\mathcal{A}_{n-2}$ again since $\mathcal{A}$ is an antichain. So 
        \begin{equation*}
            \begin{aligned}
                \frac{|\partial(\partial \mathcal{A}_n \cup \mathcal{A}_{n-1})|}{{n \choose {n-2}}} + \frac{|\mathcal{A}_{n-2}|}{{n \choose {n-2}}} \leq 1
            \end{aligned}
        \end{equation*}
        So
        \begin{equation*}
            \begin{aligned}
                \frac{|\partial \mathcal{A}_n \cup \mathcal{A}_{n-1}|}{{n \choose {n-1}}} + \frac{|\mathcal{A}_{n-2}|}{{n \choose {n-2}}} \leq 1
            \end{aligned}
        \end{equation*}
        So
        \begin{equation*}
            \begin{aligned}
                \frac{|\mathcal{A}_n|}{{n \choose {n}}} + \frac{|\mathcal{A}_{n-1}|}{{n \choose {n-1}}} + \frac{|\mathcal{A}_{n-2}|}{{n \choose {n-2}}} \leq 1
            \end{aligned}
        \end{equation*}
        Keep going and we obtain the desired result.
    \end{proof}
\end{thm}

When is equality achieved in LYM? We must have equality in each use of local LYM, so the first $r$ with $\mathcal{A}_r \neq \phi$ must have $\mathcal{A}_r = X^{(r)}$, i.e. $\mathcal{A} = X^{(r)}$.

Hence equality in Sperner's lemma is only achieved when $\mathcal{A} = X^{\lfloor n/2 \rfloor}$ for $n$ even, or also $X^{\lceil n/2 \rceil}$ when $n$ is odd.

---Lecture 3---

Now let's look at another proof to LYM inequality:
\begin{proof} (2) \\
    Choose, uniformly at random, a maximal chain $\mathcal{C}$ (i.e. $C_0 \subset C_1 \subset ... \subset C_n$ with $|C_i| =i \forall i$). For a given $r$-set $A$ (which is just one vertex in our graph, if you rememeber what our vertices mean), $\P(A \in \mathcal{C}) = \frac{1}{{n \choose r}}$. So $\P(\mathcal{A}_r \text{ meets } \mathcal{C}) = \frac{|\mathcal{A}_r|}{{n \choose r}}$ (events are disjoint).\\
    So $\P(\mathcal{A} \text{ meets } \mathcal{C}) = \sum_{r=0}^n \frac{|\mathcal{A}_r}{{n \choose r}}$, but that can be no greater than 1.
\end{proof}

\begin{rem}
    Equivalently, we could also do counting: the number of maximal chains is $n!$, and the number containing a given $r$-sets is $r!(n-r)!$. So we get $\sum|\mathcal{A}_r|r!(n-r)! \leq n!$ -- we can rearrange to get LYM as well.
\end{rem}

\subsection{Shadows}
For $\mathcal{A} \subset X^{(r)}$, we know $|\partial A| \geq |\mathcal{A}| \frac{r}{n-r+1}$, but equality is rare (only for extreme cases $\phi$ or $X^{(r)}$).\\
It then comes to our interests how we should choose $\mathcal{A} \subset X^{(r)}$ to minimize $|\partial A|$ for any fixed given $|\mathcal{A}|$, which is in some sense, how \emph{tightly} can we \emph{pack} some $r$-sets.\\
One trivial observation: if $|\mathcal{A}| = {k \choose r}$, it's believable that we would take $\mathcal{A} = [k]^{(r)}$ which gives $\partial A =[k]^{(r-1)}$.\\
What if ${k \choose r} < |\mathcal{A}| < {{k+1} \choose r}$? Naturally we expect to take $[k]^{(r)}$ with some others. For example, if $\mathcal{A} \subset X^{(3)}$ with $|\mathcal{A}| = {7 \choose 3} + {4 \choose 2}$, we'd take $\mathcal{A} = [7]^{(3)} \cup \{A \cup \{8\}: A \in [4]^{(2)}\}$. If we play around with this method and look at the $\mathcal{A}$ we choose each time, we note that there seems to be some order in $X^{(r)}$ that, whenever we are given $|\mathcal{A}| = m$, we should just pick the first $m$ $r-$sets in that order.

\subsubsection{Total orderings on $X^{(r)}$}
\begin{defi}
Given $A,B \in X^{(r)}$, say $A = a_1...a_r,B=b_1...b_r$ where $a_1<...<a_r$ and same for $b_i$. We say $A<B$ in the \emph{lexicographic} (or \emph{lex}) order, if for some $i$ we have $a_i < b_i$ and $a_j = b_j$ $\forall j < i$. Equivalently, $a_i<b_i$, where $i=\min\{j: a_j \neq b_j\}$ (use small numbers).\\
Given $A<B$ in the \emph{colexicographic} or \emph{colex} order if, for some $i$ have $a_i<b_i$, and $a_j=b_j$ $\forall j>i$. Equivalently, $a_i < b_i$ where $i = \max\{j:a_j \neq b_j\}$ (avoid large numbers). State it in a cooler way, $A<B$ if $\sum_{i \in A} 2^i < \sum_{i \in B} 2^i$.\\
(some useless examples)
\end{defi}

Note: in colex, $[k]^{(r)}$ is an initial segment of $[k+1]^{(r)}$, so we could view colex as an enumeration of $\N^{(r)}$d (but not for lex -- we'll have to know the size of the ground set first before deciding what's coming next)!\\
Indeed, the colex order is what we need to use for the shadow problem, i.e. if $\mathcal{A} \subset X^{(r)}$, and $\mathcal{C} \subset X^{(r)}$ is the first $|\mathcal{A}|$ $r$-sets in colex, then $|\partial \mathcal{A}| \geq |\partial \mathcal{C}|$ (\href{https://en.wikipedia.org/wiki/Kruskal%E2%80%93Katona_theorem}{Kruskal-Katona theorem}). In particular, $|\mathcal{A}| = {k \choose r} \implies |\partial \mathcal{A}| \geq {k \choose {r-1}}$).

\subsubsection{Compressions}
Idea: we want to \emph{replace} $\mathcal{A} \subset X^{(r)}$ with some $A' \subset X^{(r)}$, such that\\
(i) $|\mathcal{A}'| = |\mathcal{A}|$;\\
(ii) $|\partial \mathcal{A}'| \leq |\partial \mathcal{A}|$;\\
(iii) $\mathcal{A}'$ \emph{looks more like $\mathcal{C}$} then $\mathcal{A}$ did.

Ideally, we'd compress $\mathcal{A} \to \mathcal{A}' \to \mathcal{A}'' \to ... \to \mathcal{B}$ where either $\mathcal{B} = \mathcal{C}$, or $\mathcal{B}$ is so similar to $\mathcal{C}$ that we can see directly that $|\partial \mathcal{B}| \geq |\partial \mathcal{C}|$.

---Lecture 4---

We'll follow two general ideas to obtain our desired result:

$\bullet$ \emph{Colex prefers 1 to 2} inspires:\\
For $1 \leq i < j \leq n$, the $ij-$compression $C_{ij}$ is defined by: for $A \subset X$,
\begin{equation*}
    \begin{aligned}
        C_{ij}(A) = \left\{
            \begin{array}{ll}
                A-j \cup i & \text{ if } j \in A, i \not\in A\\
                A & \text{ otherwise}
            \end{array}
        \right.
    \end{aligned}
\end{equation*}
and for $\mathcal{A} \subset \P(X)$, $C_{ij}(\mathcal{A}) = \{C_{ij}(A):A \in \mathcal{A}\} \cup \{A \in \mathcal{A}:C_{ij}(A) \in \mathcal{A}\}$.\\
Note that $|C_{ij}(\mathcal{A})| = |\mathcal{A}|$.

We say $\mathcal{A}$ is \emph{$ij$-compressed} if $C_{ij}(\mathcal{A}) = \mathcal{A}$.

\begin{prop} (4)\\
    Let $\mathcal{A} \subset X^{(r)}$, $1 \leq i < j \leq n$. Then $|\partial C_{ij}(\mathcal{A}) | \leq |\partial \mathcal{A}|$.\\
    (Shadow of a compressed set is no larger than that of the original.)
    \begin{proof}
        Write $\mathcal{A'}$ for $C_{ij}(\mathcal{A})$. We'll show that if $B \in \partial \mathcal{A}' - \partial A$, then $i \in B, j \not \in B$, and $B \cup j - i \in \partial \mathcal{A} - \partial{A'}$, then we are done since for each new elememt that we probably introduced, there's one element removed.\\
        We have $B \cup x \in \mathcal{A}'$ for some $x \not\in B$, and $B \cup x \not\in \mathcal{A}$ (as $B \not\in \partial \mathcal{A}$).\\
        Hence $i \in B \cup x$, $j \not\in B \cup x$, and $(B \cup x ) \cup j - i \in \mathcal{A}$. Note that $i \neq x$, since otherwise $B \cup j \in \mathcal{A}$.\\
        Certainly $B \cup j - i \in \partial \mathcal{A}$. Now we claim that $B \cup j - i \not\in \partial \mathcal{A}'$: suppose $(B \cup j - i) \cup y \in \mathcal{A}'$. We cannot have $y=i$ for else $B \cup j \in \mathcal{A}'$, then $B\cup j$ have to be in $\mathcal{A}$ by definition of compression; but that's not allowed.\\
        Thus $j \in (B\cup j - i) \cup y$, $i \not\in (B \cup j - i) \cup y$. So $(B \cup j - i) \cup y \in \mathcal{A}$, so $B \cup y \in \mathcal{A}$ by definition of compression; but that's similarly a contraditction.
    \end{proof}
\end{prop}

\begin{rem}
    We've actually shown that $\partial C_{ij}(\mathcal{A}) \subset C_{ij}(\partial \mathcal{A})$.
\end{rem}

We say $\mathcal{A} \subset X^{(r)}$ is \emph{left-compressed} if $C_{ij}(\mathcal{A}) = \mathcal{A}$ $\forall i < j$.

\begin{prop}
    Let $\mathcal{A} \subset X^{(r)}$. Then there is a left-compressed $\mathcal{B} \in X^{(r)}$ with $|\mathcal{B}| = |\mathcal{A}|$, and $|\partial \mathcal{B}| \leq |\partial \mathcal{A}|$.
    \begin{proof}
        Among all $\mathcal{B} \subset X^{(r)}$ with $|\mathcal{B}| = |\mathcal{A}|$ and $|\partial \mathcal{B}| \leq |\partial \mathcal{A}|$, choose one with $\sum_{A \in \mathcal{B}} \sum_{x \in A} x$ minimal.\\
        Then $\mathcal{B}$ is left-compressed, else we can compress it to reduce the above sum.
    \end{proof}
\end{prop}

Note: we can also apply $C_{ij}$ repeatedly -- this must terminate (by counting on the above sum). In fact, we can apply each $C_{ij}$ at most once if we choose a sensible order.

Certainly, initial segments of colex are left-compressed. However the converse can be very false (consider $\{123,124,125,126,127\}$).

$\bullet$ \emph{Colex prefers 23 to 14} inspires:\\
For $U,V \subset X$ with $|U| = |V|$, and $U \cap V = \phi$, the $UV$-compression $C_{UV}$ is defined by: for $A \subset X$,
\begin{equation*}
    \begin{aligned}
        C_{UV}(A) = \left\{
            \begin{array}{ll}
                A\cup U - V & \text{ if } V \subset A, U \cap A = \phi\\
                A & \text{ otherwise}
            \end{array}
        \right.
    \end{aligned}
\end{equation*}
and for $\mathcal{A} \subset X^{(r)}$, $C_{UV}(A) = \{C_{UV}(A):A \in \mathcal{A}\} \cup \{A \in \mathcal{A}: C_{UV}(A)\in \mathcal{A}\}$.

Note that also the two sets have equal size. We say $\mathcal{A}$ is \emph{$UV$-compressed} if $C_{UV}(\mathcal{A}) = \mathcal{A}$.

Sadly $C_{UV}$ doesn't necessarily decrease the shadow.

However, it turns out to be fine if we have done the smaller ones.

\begin{prop}
    Let $\mathcal{A} \subset X^{(r)}$ and $U,V \subset X$ with $|U|=|V|$ and $U \cap V = \phi$. Suppose that\\
    (*) $\forall x \in U, \exists y \in V$ s.t. $\mathcal{A}$ is $(U-x,V-y)$-compressed.\\
    Then $|\partial C_{UV}(\mathcal{A})|\leq |\partial \mathcal{A}|$.
\end{prop}

\end{document}
