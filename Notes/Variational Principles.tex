\documentclass[a4paper]{article}

\setlength{\parindent}{0pt}
\setlength{\parskip}{1em}

\pagestyle{headings}

\usepackage{amssymb}
\usepackage{amsmath}
\usepackage{amsthm}
\usepackage{mathtools}
\usepackage{graphicx}
\usepackage{hyperref}
\usepackage{color}
\usepackage{microtype}
\usepackage{tikz}
\usepackage{pgfplots}
\usepackage{pgfplotstable}

\newcommand{\N}{\mathbb{N}}
\newcommand{\Q}{\mathbb{Q}}
\newcommand{\Z}{\mathbb{Z}}
\newcommand{\R}{\mathbb{R}}
\newcommand{\C}{\mathbb{C}}
\newcommand{\D}{\mathcal{D}}
\renewcommand{\S}{\mathcal{S}}
\renewcommand{\P}{\mathbb{P}}
\newcommand{\F}{\mathbb{F}}
\newcommand{\E}{\mathbb{E}}
\newcommand{\bra}{\langle}
\newcommand{\ket}{\rangle}


\graphicspath{{Image/}}

\hypersetup{
    colorlinks=true,
    linktoc=all,
    linkcolor=blue
}

\theoremstyle{definition}
\newtheorem*{axiom}{Axiom}
\newtheorem*{claim}{Claim}
\newtheorem*{conv}{Convention}
\newtheorem*{coro}{Corollary}
\newtheorem*{defi}{Definition}
\newtheorem*{eg}{Example}
\newtheorem*{lemma}{Lemma}
\newtheorem*{notation}{Notation}
\newtheorem*{prob}{Problem}
\newtheorem*{post}{Postulate}
\newtheorem*{prop}{Proposition}
\newtheorem*{rem}{Remark}
\newtheorem*{thm}{Theorem}

\DeclareMathOperator{\vdiv}{div}
\DeclareMathOperator{\grad}{grad}
\DeclareMathOperator{\curl}{curl}
\DeclareMathOperator{\Ann}{Ann}
\DeclareMathOperator{\Fit}{Fit}
\DeclareMathOperator{\Diag}{Diag}
\DeclareMathOperator{\tr}{tr}
\DeclareMathOperator{\im}{im}
\DeclareMathOperator{\Mat}{Mat}
\DeclareMathOperator{\Log}{Log}
\DeclareMathOperator{\Isom}{Isom}
\DeclareMathOperator{\Mesh}{Mesh}
\DeclareMathOperator{\Sym}{Sym}
\DeclareMathOperator{\Aut}{Aut}
\DeclareMathOperator{\cosech}{cosech}
\DeclareMathOperator{\Card}{Card}
\DeclareMathOperator{\Gal}{Gal}


\begin{document}

\title{Variational Principles}
\date{Easter 2016}

\maketitle

\newpage

\tableofcontents

\newpage

\section{Introduction}
Variational principles discusses how we maximise/minimise something that depends on other things that we can vary.\\
For many problems we have a continuous infinity of independent variables.

\begin{eg} (Dido's problem)\\
What curve of fixed length maximises an enclosed area?
\end{eg}

\begin{eg} (Newton's problem, in Principia 1687)\\
What surface of revolution minimises resistance to motion in a fluid?
\end{eg}

\begin{eg} (The \emph{brachistochrone})\\
What curve of wire minimises the time for the bead to fall from rest?
\end{eg}

\newpage

\section{Variational principles}
$\bullet$ Hero's principle (-100 BC): 'Light travels by the shortest path'.\\
$\bullet$ Fermat's principle (1662): Light travels on a path of \emph{least time}. He assumed that light is slowed when it passes into a more dense material. Effectively he was assuming that the velocity is inversely proportional to the refraction index. He used this principle to derive Snell's law:
\begin{equation*}
\begin{aligned}
n_1 \sin \theta_1 = n_2 \sin \theta_2.
\end{aligned}
\end{equation*}
For a medium of variable refractive index $n\left(\mathbf{x}\right)$, Fermat's principle is equivalent to the minimum optical path length:
\begin{equation*}
\begin{aligned}
P=\int_C n\left(x\right) dx
\end{aligned}
\end{equation*}
vary $C$ to find the minimum of $P$. This led to analogies principles for mechanics.\\

Calculus for functions of many variables:
\begin{equation*}
\begin{aligned}
f: \R^n &\to \R\\
\mathbf{x} = \left(x_1,...,x_n\right) &\to f\left(\mathbf{x}\right) \in \R
\end{aligned}
\end{equation*}
Assume $f$ is smooth. Stationary points of $f$ are points in $\R^n$ for which $\nabla f=\mathbf{0}$, i.e.
\begin{equation*}
\begin{aligned}
\left(\frac{\partial f}{\partial x_1},\frac{\partial f}{\partial x_2},...,\frac{\partial f}{\partial x_n}\right) = \left(0,0,...,0\right).
\end{aligned}
\end{equation*}
Expanding in Taylor series about a stationary point $\mathbf{x}=\mathbf{a}$:
\begin{equation*}
\begin{aligned}
f\left(\mathbf{x}\right) = f\left(\mathbf{a}\right) + 0 + \frac{1}{2} \sum_{i,j} \left(x_i-a_i\right)\left(x_j-a_j\right) H_{ij} \left(\mathbf{a}\right) + O\left(\left(x-a\right)^3\right)
\end{aligned}
\end{equation*}
Where $H$ is the Hessian matrix which is symmetric,
\begin{equation*}
\begin{aligned}
H_{ij}\left(\mathbf{x}\right) = \frac{\partial f\left(\mathbf{x}\right)}{\partial x_i \partial x_j}
\end{aligned}
\end{equation*}

Focus on one stationary point: assume wlog that $\mathbf{a}=\mathbf{0}$. Then
\begin{equation*}
\begin{aligned}
f\left(\mathbf{x}\right) - f\left(\mathbf{0}\right) = \frac{1}{2}x_i H_{ij}\left(\mathbf{0}\right) x_j
\end{aligned}
\end{equation*}

Now let $x_i=R_{ij} x_j'$ for some rotation matrix $R$, such that the new Hessian $H'$ in this basis is diagonal, i.e. $H' = diag \left(\lambda_1,...,\lambda_n\right)$. Then
\begin{equation*}
\begin{aligned}
f\left(\mathbf{x}\right) - f\left(\mathbf{0}\right) = \frac{1}{2}\sum_i \lambda_i \left(x_i'\right)^2 + O\left(x'^3\right)
\end{aligned}
\end{equation*}
This is positive definite if $\lambda_i>0$ for all $i$, i.e. a local minimum. It's negative definite if $\lambda_i<0$ for all $i$, in which we have a local maximum. If some of the eigenvalues are positive while others are negative, we have a saddle point. If some eigenvalues are zero then we have a degenerate stationary point and need to investigate further (go to higher orders).

\begin{eg}
Consider n=2.
\begin{equation*}
\begin{aligned}
\det H = \lambda_1 \lambda_2\\
\tr H = \lambda_1 + \lambda_2
\end{aligned}
\end{equation*}
If both are positive then this is a local minimum; if $\det H>0$ and $\tr H<0$ then this is a local maximum; if $\det H<0$ this is a saddle point; if $\det H=0$ then this case is degenerate.
\end{eg}

\begin{eg}
Find stationary points of 
\begin{equation*}
\begin{aligned}
f\left(x,y\right) = x^3+y^3-3xy
\end{aligned}
\end{equation*}
\end{eg}

\end{document}