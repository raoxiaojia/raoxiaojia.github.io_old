\documentclass[a4paper]{article}

\setlength{\parindent}{0pt}
\setlength{\parskip}{1em}

\pagestyle{headings}

\usepackage{amssymb}
\usepackage{amsmath}
\usepackage{amsthm}
\usepackage{mathtools}
\usepackage{graphicx}
\usepackage{hyperref}
\usepackage{color}
\usepackage{microtype}
\usepackage{tikz}
\usepackage{pgfplots}
\usepackage{pgfplotstable}

\newcommand{\N}{\mathbb{N}}
\newcommand{\Q}{\mathbb{Q}}
\newcommand{\Z}{\mathbb{Z}}
\newcommand{\R}{\mathbb{R}}
\newcommand{\C}{\mathbb{C}}
\newcommand{\D}{\mathcal{D}}
\renewcommand{\S}{\mathcal{S}}
\renewcommand{\P}{\mathbb{P}}
\newcommand{\F}{\mathbb{F}}
\newcommand{\E}{\mathbb{E}}
\newcommand{\bra}{\langle}
\newcommand{\ket}{\rangle}


\graphicspath{{Image/}}

\hypersetup{
    colorlinks=true,
    linktoc=all,
    linkcolor=blue
}

\theoremstyle{definition}
\newtheorem*{axiom}{Axiom}
\newtheorem*{claim}{Claim}
\newtheorem*{conv}{Convention}
\newtheorem*{coro}{Corollary}
\newtheorem*{defi}{Definition}
\newtheorem*{eg}{Example}
\newtheorem*{lemma}{Lemma}
\newtheorem*{notation}{Notation}
\newtheorem*{prob}{Problem}
\newtheorem*{post}{Postulate}
\newtheorem*{prop}{Proposition}
\newtheorem*{rem}{Remark}
\newtheorem*{thm}{Theorem}

\DeclareMathOperator{\vdiv}{div}
\DeclareMathOperator{\grad}{grad}
\DeclareMathOperator{\curl}{curl}
\DeclareMathOperator{\Ann}{Ann}
\DeclareMathOperator{\Fit}{Fit}
\DeclareMathOperator{\Diag}{Diag}
\DeclareMathOperator{\tr}{tr}
\DeclareMathOperator{\im}{im}
\DeclareMathOperator{\Mat}{Mat}
\DeclareMathOperator{\Log}{Log}
\DeclareMathOperator{\Isom}{Isom}
\DeclareMathOperator{\Mesh}{Mesh}
\DeclareMathOperator{\Sym}{Sym}
\DeclareMathOperator{\Aut}{Aut}
\DeclareMathOperator{\cosech}{cosech}
\DeclareMathOperator{\Card}{Card}
\DeclareMathOperator{\Gal}{Gal}


\begin{document}

\title{Quantum Mechanics}

\maketitle

\newpage

\tableofcontents

\newpage

\section{Wave Functions and Operators}
We introduce some of the mathematical structure of quantum mechanics (QM) by considering a particle in one dimension.

\subsection{Wave Function and States}
A classical point particle in one dimension has a position $x$ at each time. In QM a particle has a \emph{state} at each time given by a complex \emph{wave function} $\psi\left(x\right)$.

\begin{post}
A measurement of position gives a result $x$ with probability density $|\psi\left(x\right)|^2$, i.e.
\begin{equation*}
\begin{aligned}
|\psi\left(x\right)|^2 \delta x
\end{aligned}
\end{equation*}
will be the probability that particle is found between $x$ and $x+\delta x$, or
\begin{equation*}
\begin{aligned}
\int_a^b |\psi\left(x\right)|^2 dx
\end{aligned}
\end{equation*}
is the probability that the particle is found in the interval $a\leq x \leq b$.
This obviously requires
\begin{equation*}
\begin{aligned}
\int_{-\infty}^\infty |\psi\left(x\right)|^2 dx = 1
\end{aligned}
\end{equation*}
We say $\psi$ is \emph{normalised} if it satisfies this condition.
\end{post}

\begin{eg} (Gaussian wave function) Let 
\begin{equation*}
\begin{aligned}
\psi\left(x\right) = C e^{-\frac{\left(x-x_0\right)^2}{2\alpha}}
\end{aligned}
\end{equation*}
For some real $\alpha > 0$.\\
If $\alpha$ is small, $|\psi|^2$ will be sharply peaked around $x=x_0$.\\
If $\alpha$ is large, $|\psi|^2$ is more spread out.

Since $\psi$ needs to be normalised,
\begin{equation*}
\begin{aligned}
\int_{-\infty}^\infty |\psi\left(x\right)|^2 dx &= |C|^2 \int_{-\infty}^\infty e^{-\frac{\left(x-x_0\right)^2}{\alpha}}dx\\
&= |C|^2 \left(2\pi\right)^{1/2}\\
&= 1
\end{aligned}
\end{equation*}
So $\psi$ is normalised if 
\begin{equation*}
\begin{aligned}
C=\left(\frac{1}{2\pi}\right)^{1/4}
\end{aligned}
\end{equation*}
\end{eg}

It is convenient to deal more generally with \emph{normalisable} wave functions that are \emph{not} identically zero, and satisfy
\begin{equation*}
\begin{aligned}
\int_{-\infty}^\infty |\psi\left(x\right)|^2 dx < \infty
\end{aligned}
\end{equation*}
In fact, $\psi\left(x\right)$ and $\phi\left(x\right) = \lambda \psi\left(x\right)$ contain the same physical information for any complex $\lambda \neq 0$.

If $\psi\left(x\right)$ is normalisable, we can choose $\lambda$ to ensure $\psi\left(x\right)$ is normalised. Note also $\psi\left(x\right)$ and $e^{i\alpha} \psi\left(x\right)$ are physically equivalent for any real $\alpha$, and when $\psi$ is normalised,
\begin{equation*}
\begin{aligned}
|\psi\left(x\right)|^2 = |e^{i\alpha} \psi\left(x\right)|^2
\end{aligned}
\end{equation*}
i.e. they have the same probability distribution.

In general we will consider normalisable wave functions $\psi,\phi,\chi,...$ that are \emph{smooth} (differentiable any number of times) except at isolated points (see examples below). Also, $\psi,\psi',... \to 0$ as $|x| \to \infty$.

Given states $\psi_1\left(x\right)$ and $\psi_2\left(x\right)$ can form new state
\begin{equation*}
\begin{aligned}
\psi\left(x\right) = \psi_1\left(x\right) + \psi_2\left(x\right)
\end{aligned}
\end{equation*}
which is called superposition.

\begin{eg}
Take
\begin{equation*}
\begin{aligned}
\psi\left(x\right) = B\left(e^{-\frac{x^2}{2\alpha}} + e^{-\frac{\left(x-x_0\right)^2}{2\alpha}}\right)
\end{aligned}
\end{equation*}
i.e. the superposition of Gaussian wave function and itself at different positions.

By drawing the image of $|\psi\left(x\right)|^2$ we can get the probability distribution for single particle, with appropriate choice of $B$.
\end{eg}

\subsection{Operators and Observables}
A quantum state contains information about other physical quantities \emph{observables}, for example, momentum and energy, in addition to position. In QM, each observable is represented by an \emph{operator} acting on wave functions:\\
Position:
\begin{equation*}
\begin{aligned}
\hat{x} = x\\
\left(\hat{x}\psi\right)\left(x\right) = x\psi\left(x\right)
\end{aligned}
\end{equation*}
Momentum:
\begin{equation*}
\begin{aligned}
\hat{p} = -i\hbar \frac{d}{dx}\\
\left(\hat{p}\psi\right)\left(x\right) = -i\hbar \frac{d\psi}{dx} = -i\hbar \psi'\left(x\right)
\end{aligned}
\end{equation*}
Energy, or \emph{Hamiltonian}, for a particle of mass $m$ moving at potential $V\left(x\right)$:
\begin{equation*}
\begin{aligned}
H &= \frac{\hat{p}^2}{2m} + V\left(\hat{x}\right)\\
&= -\frac{\hbar^2}{2m} \frac{d^2}{dx^2} + V\left(x\right)
\end{aligned}
\end{equation*}

If an observable $Q$ is measured when the system is in a state $\psi$, we would like to know:\\
(i) what results are possible;\\
(ii) what is the probability for each result.

\subsubsection{Expectation values}
For any normalisable $\psi\left(x\right)$ and $\phi\left(x\right)$, define
\begin{equation*}
\begin{aligned}
\left(\psi,\phi\right) = \int_{-\infty}^\infty \psi\left(x\right)^* \phi\left(x\right) dx
\end{aligned}
\end{equation*}

For normalised $\psi\left(x\right)$, define the \emph{expectation value} of $Q$ (some operator) in this state to be
\begin{equation*}
\begin{aligned}
\left<Q\right>_\psi = \left(\psi,Q\psi\right) = \int_{-\infty}^\infty \psi^* Q\psi dx
\end{aligned}
\end{equation*}

Note for $Q=\hat{x}$,
\begin{equation*}
\begin{aligned}
\left<\hat{x}\right>_\psi = \left(\psi,\hat{x}\psi\right)=\int_{-\infty}^\infty x|\psi\left(x\right)|^2 dx
\end{aligned}
\end{equation*}
standard expression for mean or expectation of $x$, and for $Q=\hat{p}$,
\begin{equation*}
\begin{aligned}
\left<\hat{p}\right>_\psi = \left(\psi,\hat{p}\psi\right) = \int_{-\infty}^\infty -i\hbar \psi^* \psi' dx
\end{aligned}
\end{equation*}

\begin{post}
For any observable, $\left<Q\right>_\psi$ is the mean result of measuring $Q$ when the system is in state $\psi$.
\end{post}

Now consider $\phi$ and $\psi$ (normalised) with
\begin{equation*}
\begin{aligned}
\phi\left(x\right) = \psi\left(x\right) e^{ikx}
\end{aligned}
\end{equation*}
for real $k$. Then $|\phi\left(x\right)|^2 = |\psi\left(x\right)|^2$. As a result, $\left<\hat{x}\right>_\phi = \left<\hat{x}\right>_\psi$, but
\begin{equation*}
\begin{aligned}
\left<\hat{p}\right>_\phi &= \int_{-\infty}^\infty -i\hbar \phi^*\phi' dx\\
&= \int_{-\infty}^\infty -i\hbar \psi^* \psi' + \int_{-\infty}^\infty \hbar
k \psi^* \psi\\
&= \left<\hat{p}\right>_\psi + \hbar k
\end{aligned}
\end{equation*}
So additional factor of $e^{ikx}$ changes momentum by $hk$.

\begin{eg}
Let
\begin{equation*}
\begin{aligned}
\psi = C e^{-\frac{x^2}{2\alpha}}
\end{aligned}
\end{equation*}
as in the last subsection but with $x_0=0$.\\
Then $\left<\hat{p}\right>_\psi = 0$,
\begin{equation*}
\begin{aligned}
\phi = C e^{-\frac{x^2}{2\alpha}} e^{ikx}
\end{aligned}
\end{equation*}
with $\left<\hat{p}\right>_\phi = \hbar k$.
\end{eg}

\subsubsection{Eigenvalues and Eigenstates}
Consider an operator corresponding to an observable, $Q$, with
\begin{equation*}
\begin{aligned}
Q \psi = q\psi
\end{aligned}
\end{equation*}
for some number $q$. Then $\psi\left(x\right)$ is called an \emph{eigenfunction}, or \emph{eigenstate} of $Q$ with eigenvalue $q$.

\begin{post}
If $Q$ is measured when the system is in an eigenstate $\psi$, then the result is the eigenvalue $q$ with probability 1.
\end{post}

\begin{eg}
$Q = \hat{x}$. This has no normalisable eigenfunctions, since if
\begin{equation*}
\begin{aligned}
\hat{x} \psi\left(x\right) = x\psi\left(x\right) = q\psi\left(x\right)
\end{aligned}
\end{equation*}
for some $q$, then $\psi\left(x\right) = 0$ for all $x\neq q$.
\end{eg}

\begin{eg}
Set $Q = \hat{p} = -ih\frac{d}{dx}$, and $q=p$. Then $\psi = C e^{ikx}$ is an eigenfunction with $p = \hbar k$. However, this is not normalisable on the real line (we'll return to this later).
\end{eg}

\begin{eg}
Set $Q=H$ with $V\left(x\right) = \frac{1}{2} kx^2$ with $k>0$, a harmonic oscillator, and $q = E$. Then the eigenvalue equation
\begin{equation*}
\begin{aligned}
H \psi &= -\frac{\hbar^2}{2m} \psi'' + \frac{1}{2} kx^2 \psi = E\psi
\end{aligned}
\end{equation*}
is satisfied by
\begin{equation*}
\begin{aligned}
\psi = Ce^{-x^2/2\alpha}
\end{aligned}
\end{equation*}
(where $C$ is chosen to normalise $\psi$) for $\alpha^2 = \hbar^2/km$ and $E = \frac{\hbar}{2} \sqrt{\frac{k}{m}}$.
\end{eg}

In general, the energy eigenvalue equation
\begin{equation*}
\begin{aligned}
H_\psi = E_\psi
\end{aligned}
\end{equation*}
or
\begin{equation*}
\begin{aligned}
-\frac{\hbar}{2m} \frac{d^2 \psi}{dx^2} +V\left(x\right) \psi = E\psi\left(x\right)
\end{aligned}
\end{equation*}
for particle in potential $V\left(x\right)$ is called the \emph{time-independent} \emph{Schr\"{o}dinger Equation}(SE). Solving this determines all states of definite energy.

\subsubsection{Additional comments}
\begin{rem}
If $\psi$ is any normalised state, then $\left<\hat{p}\right>_\psi$ and $\left<H\right>_\psi$ are real. This follows from definitions using integration by parts, e.g.
\begin{equation*}
\begin{aligned}
\left<\hat{p}\right>^* &= \left(\int_{-\infty}^\infty -i\hbar \psi^* \psi' dx\right)^* \\
&= \left(\int_{-\infty}^\infty i\hbar \psi\left(\psi^*\right)' dx\right)\\
&= i\hbar \left[\psi \psi^*\right]_{\infty}^\infty (=0) - i\hbar \int_{-\infty}^\infty \psi'\psi^* dx\\
&= \left<\hat{p}\right>_\psi
\end{aligned}
\end{equation*}
Similarly we can check
\begin{equation*}
\begin{aligned}
\left<H\right>_\psi = \int_{-\infty}^\infty \left(-\frac{\hbar^2}{2m} \psi^* \psi'' + \psi^* V \psi\right) dx
\end{aligned}
\end{equation*}
is real, integrate first term by parts twice after taking complex conjugate (assume V real).
\end{rem}

\begin{rem}
Postulate 3 is consistent with Postulate 2 since
\begin{equation*}
\begin{aligned}
&H\psi = E\psi\\
\implies & \left<H\right>_\psi = \int_{-\infty}^\infty \psi^* H \psi dx = \int_{-\infty}^\infty E \psi^* \psi dx = E
\end{aligned}
\end{equation*}
for $\psi$ normalised.
\end{rem}

\begin{rem}
From the previous two remarks, the energy eigenvalue for a normalised eigenstate $\psi$ is always real.
\end{rem}

\subsection{Infinite well or particle in a box}

Let
\begin{equation*}
\begin{aligned}
V\left(x\right) = \left\{
\begin{array}{ll}
0 & |x| \leq a\\
\infty & |x| > a
\end{array}
\right.
\end{aligned}
\end{equation*}
assume $\psi\left(\pm a\right) = 0$ and justify at end and $\psi\left(x\right) = 0$ for $|x|>a$.

Consider SE for $-a\leq x \leq a$
\begin{equation*}
\begin{aligned}
-\frac{\hbar^2}{2m} \psi'' = E\psi
\end{aligned}
\end{equation*}
for $E>0$, set $E = \frac{\hbar^2 k^2}{2m}$ where $k>0$ so that SE becomes
\begin{equation*}
\begin{aligned}
\psi'' + k^2 \psi = 0
\end{aligned}
\end{equation*}
So
\begin{equation*}
\begin{aligned}
\psi = A \cos kx + B \sin kx
\end{aligned}
\end{equation*}
But $A\cos ka \pm B \sin ka = 0$ from boundary condition. That implies either
\begin{equation*}
\begin{aligned}
B=0,\ ka = \frac{n\pi}{2} \ n=1,3,...
\end{aligned}
\end{equation*}
or
\begin{equation*}
\begin{aligned}
A=0,\ ka = \frac{n\pi}{2} \ n=2,4,...
\end{aligned}
\end{equation*}

Solutions:
\begin{equation*}
\begin{aligned}
\psi_n\left(x\right) = \left(\frac{1}{a}\right)^{1/2} \left\{
\begin{array}{ll}
\cos \\
\sin
\end{array}
\right\} \frac{n\pi}{2a}x \text{ for } n>0 \left\{
\begin{array}{ll}
\text{odd} \\
\text{even}
\end{array}
\right\}
\end{aligned}
\end{equation*}
energy eigenfunctions and discrete set of energy eigenvalues
\begin{equation*}
\begin{aligned}
E_n = \frac{\hbar^2 \pi^2 n^2}{8ma^2}
\end{aligned}
\end{equation*}

For $E<0$, set
\begin{equation*}
\begin{aligned}
E= -\frac{\hbar^2 k^2}{2m}
\end{aligned}
\end{equation*}
with $k>0$ so that SE becomes\begin{equation*}
\begin{aligned}
\psi'' - k^2 \psi = 0
\end{aligned}
\end{equation*}
Solutions are
\begin{equation*}
\begin{aligned}
\psi = A e^{kx} + B e^{-kx}
\end{aligned}
\end{equation*}
and cannot satisfy boundary conditions, except by $A=B=0$.

To justify boundary conditions, consider potential with $V\left(x\right) = U \gg E$ for $|x|>a$. Setting
\begin{equation*}
\begin{aligned}
U-E = \frac{\hbar^2 k^2}{2m}
\end{aligned}
\end{equation*}
and SE is
\begin{equation*}
\begin{aligned}
\psi'' - k^2 \psi = 0
\end{aligned}
\end{equation*}
for $|x|>a$. So for normalisable solutions, we need
\begin{equation*}
\begin{aligned}
\psi = \left\{
\begin{array}{ll}
Ae^{-kx} & x>a\\
Be^{+kx} & x<-a
\end{array}
\right.
\end{aligned}
\end{equation*}
Taking $U \to \infty$ with $E$ fixed, $k \to \infty$, and $\psi \to 0$ for $|x| > a$.

\newpage

\section{The Schr\"{o}dinger Equation}
To continue our development of QM (in one dimension), we need to consider how things evolve in time.

Classical dynamics of a particle can be specified by the potential $V\left(x\right)$ (Force $f\left(x\right) = -V'\left(x\right)$). Quantum dynamics is also specified by Hamiltonian
\begin{equation*}
\begin{aligned}
H = \frac{\hat{p}^2}{2m} + V\left(\hat{x}\right)
\end{aligned}
\end{equation*}
also determined by potential.

Evolution of a quantum state in time is described by a $t$-dependent wave function $\Psi\left(x,t\right)$ which satisfies
\begin{equation}\label{1}
\begin{aligned}
i\hbar \frac{\partial}{\partial t} \Psi = H \Psi
\end{aligned}
\end{equation}
the \emph{time-dependent Schr\"{o}dinger Equation}.

The operators $\hat{x}$ and $\hat{p}$ do not change in time, and \eqref{1} is
\begin{equation*}
\begin{aligned}
i\hbar \frac{\partial \Psi}{\partial t} = -\frac{\hbar^2}{2m} \frac{\partial^2 \Psi}{\partial x^2} + V\left(x\right) \psi
\end{aligned}
\end{equation*}
a PDE linear in $\Psi$ and first order in $t$, so specify $\Psi\left(x,0\right)$ and $\Psi\left(x,t\right)$ can be determined uniquely.

\subsection{Stationary states}
Consider a wave function of definite frequency:
\begin{equation*}
\begin{aligned}
\Psi\left(x,t\right) = \psi\left(x\right) e^{-i\omega t}
\end{aligned}
\end{equation*}
Substituting in \eqref{1} gives
\begin{equation*}
\begin{aligned}
\psi \hbar \omega e^{-i\omega t} = \left(H\psi\right) e^{-i\omega t}
\end{aligned}
\end{equation*}
This holds if and only if
\begin{equation*}
\begin{aligned}
H \psi = E\psi
\end{aligned}
\end{equation*}
with $E = \hbar \omega$.

Alternatively, look for a separable solution
\begin{equation*}
\begin{aligned}
\Psi\left(x,t\right) = f\left(t\right)\psi\left(x\right)
\end{aligned}
\end{equation*}
and find
\begin{equation*}
\begin{aligned}
\frac{1}{\psi} H \psi = \frac{i\hbar}{f}\dot{f} = E
\end{aligned}
\end{equation*}
which is a separation constant. This implies $H\psi = E\psi$ and $f\left(t\right) = f\left(0\right) e^{-iEt/\hbar}$.

A solution of this special form is called a \emph{stationary state}. Special properties of stationary states:\\
(i)
\begin{equation*}
\begin{aligned}
\left|\Psi \left(x,t\right)\right|^2 = \left|\psi\left(x\right)\right|^2
\end{aligned}
\end{equation*}
So probability density does not change with time;\\
(ii)
\begin{equation*}
\begin{aligned}
\Psi\left(x,t\right) = \psi\left(x\right) e^{-iEt/\hbar}
\end{aligned}
\end{equation*}
is the unique solution with $\Psi\left(x,0\right) = \psi\left(x\right)$ and $H\psi = E\psi$. Then $H\Psi = E\Psi$ implies that the measurement of energy gives result $E$ with certainty (probability 1) for all $t$.

\begin{eg}
Consider particle in a box in chapter 1.3: found energy eigenstates $\psi_n\left(s\right)$ $\left(\begin{array}{ll}\sin \\ \cos\end{array}\right)$ with
\begin{equation*}
\begin{aligned}
E_n = \frac{\hbar^2 \pi^2}{8ma^2} n^2
\end{aligned}
\end{equation*}
for $n=1,2,...$.\\
Stationary state solutions of time dependent SE:
\begin{equation*}
\begin{aligned}
\Psi_n\left(x,t\right) = \psi_n\left(x\right) e^{-iE_n t/\hbar}
\end{aligned}
\end{equation*}
Note however $\psi_1+\psi_2$ is \emph{not} an energy eigenstate:
\begin{equation*}
\begin{aligned}
H\left(\psi_1+\psi_2\right) = E_1\psi_1 E_2 \psi_2 \not\propto \psi_1+\psi_2
\end{aligned}
\end{equation*}
\end{eg}

\subsection{Conservation of Probability}
The probability density
\begin{equation*}
\begin{aligned}
P\left(x,t\right) = \left|\Psi\left(x,t\right)\right|^2
\end{aligned}
\end{equation*}
obeys a conservation equation
\begin{equation*}
\begin{aligned}
\frac{\partial P}{\partial t} = -\frac{\partial J}{\partial x}
\end{aligned}
\end{equation*}
where
\begin{equation*}
\begin{aligned}
J\left(x,t\right) = -\frac{i\hbar}{2m} \left(\Psi^* \Psi' - \left.\Psi'\right.^* \Psi\right)
\end{aligned}
\end{equation*}
which is real (here $' = \frac{\partial}{\partial x}$), and is called the \emph{probability current}.

This follows from time dependent SE
\begin{equation*}
\begin{aligned}
i\hbar \frac{\partial\Psi}{\partial t} = -\frac{\hbar^2}{2m} \Psi'' + V\Psi
\end{aligned}
\end{equation*}
and its conjugate
\begin{equation*}
\begin{aligned}
-i\hbar \frac{\partial\Psi^*}{\partial t} = -\frac{\hbar^2}{2m} \left.\Psi^*\right.'' + V\Psi^*
\end{aligned}
\end{equation*}
So
\begin{equation*}
\begin{aligned}
\frac{\partial P}{\partial t} &= \Psi^* \frac{\partial \Psi}{\partial t} + \frac{\partial \Psi^*}{\partial t} \Psi \\
&= \Psi^* \frac{i\hbar}{2m} \Psi'' - \frac{i\hbar}{2m} \left.\Psi^*\right.'' \Psi
\end{aligned}
\end{equation*}
Since the potential terms ($V$) cancel each other. That is equal to
\begin{equation*}
\begin{aligned}
-\frac{\partial J}{\partial x}
\end{aligned}
\end{equation*}
as claimed.

The conservation equation implies
\begin{equation*}
\begin{aligned}
\frac{d}{dt}\int_a^b P\left(x,t\right)dx &= \int_a^b + \frac{\partial P}{\partial t}\left(x,t\right)dx\\
&=\int_a^b -\frac{\partial J}{\partial x} \left(x,t\right) dx\\
&= -J\left(b,t\right) + J\left(a,t\right)
\end{aligned}
\end{equation*}

Then boundary conditions $\Psi, J \to 0$ as $x \to \pm \infty$ (for fixed $t$) gives
\begin{equation*}
\begin{aligned}
\int_{-\infty}^\infty \left|\Psi\left(x,t\right)\right|^2 dx
\end{aligned}
\end{equation*}
which is independent of time.

Hence $\Psi\left(x,0\right)$ normalised $\implies$ $\Psi\left(x,t\right)$ normalised for all $t\geq 0$.

\subsection{Wave packets and particles}
Any wave function that represents a particle localised in space (about some point, on some scale) is called a \emph{wave packet}.

For example, consider Gaussian
\begin{equation*}
\begin{aligned}
\psi\left(x\right) = A \frac{1}{\alpha^{1/2}} e^{-x^2/2\alpha}
\end{aligned}
\end{equation*}
where $A = \left(\alpha/\pi\right)^{1/4}$.\\
Wave packet localised around $x=0$ on length scale $\sqrt{\alpha}$.

So the solution of time dependent SE with $V=0$ (free particle) and $\Psi\left(x,0\right) = \psi\left(x\right)$. Then
\begin{equation*}
\begin{aligned}
\Psi\left(x,t\right) = A\frac{1}{\gamma\left(t\right)^{1/2}} e^{-x^2/2\gamma\left(t\right)}
\end{aligned}
\end{equation*}
with
\begin{equation*}
\begin{aligned}
\gamma\left(t\right) = \alpha + \frac{i\hbar t}{m}
\end{aligned}
\end{equation*}
(see example sheet 1).

Then the probability density is
\begin{equation*}
\begin{aligned}
P_\Psi\left(x,t\right) &= \left|\Psi\left(x,t\right)\right|^2 \\
&= \frac{|a|^2}{|\gamma\left(t\right)|} e^{-\alpha x^2/|\gamma\left(t\right)|^2}
\end{aligned}
\end{equation*}
localised around $x=0$ but the length scale $|\gamma\left(t\right)/\sqrt{\alpha}$ spreads with $t$.

However it's easy to check
\begin{equation*}
\begin{aligned}
\left<\hat{x}\right>_\Psi = \left<\hat{p}\right>_\Psi = 0
\end{aligned}
\end{equation*}
for all $t$.

Previously noted $\phi\left(x\right) = \psi\left(x\right) e^{ikx}$ has expectation value $\hbar k$ for momentum. Solution to SE with $\Phi\left(x,0\right) = \phi \left(x\right)$ is
\begin{equation*}
\begin{aligned}
\Phi\left(x,t\right) = \Psi\left(x-ut,t\right)e^{ikx}e^{-i\left(\hbar k^2/2m\right)t}
\end{aligned}
\end{equation*}
with $mu = \hbar k$ (can be checked directly). We can also check
\begin{equation*}
\begin{aligned}
\left<\hat{x}\right>_\Phi = ut
\end{aligned}
\end{equation*}
and
\begin{equation*}
\begin{aligned}
\left<\hat{p}\right>_\Phi = \hbar k
\end{aligned}
\end{equation*}

Moreover,
\begin{equation*}
\begin{aligned}
P_\Phi = \left|\Phi\left(x,t\right)\right|^2  = \left|\Psi\left(x-ut,t\right)\right|^2 = P_\Psi\left(x-ut,t\right)
\end{aligned}
\end{equation*}

So $\Phi$ corresponds to the moving particle with $mu = \hbar k$ momentum.

Gaussian wave function spreads out on time-scale $\tau \sim \frac{m\alpha}{\hbar}$.

For example, consider an electron $m=m_e$ and $\sqrt{\alpha} = 10^{-12}$ meter. Then $ \tau \sim 10^{-20}$ sec.

Now take $m=10^{-6}$ kg and $\alpha = 10^{-6}$ meter. Then $\tau \sim 10^{16}$ sec.

\newpage

\section{Bound States in One Dimension}

A bound state for a particle of mass $m$ in a potential $V\left(x\right)$ is a normalisable energy eigenstate (or stationary state)
\begin{equation*}
\begin{aligned}
H\psi = -\frac{\hbar^2}{2m} \psi'' + V\left(x\right) \psi = E\psi
\end{aligned}
\end{equation*}
(time-independent SE). This corresponds to a bounded classical orbit.

If $V\left(x\right) \to 0$ as $|x| \to \infty$, need $E < 0$ (see section 3.2 below).

\subsection{Potential Well}
Consider a potential well
\begin{equation*}
\begin{aligned}
V\left(x\right) = \left\{ \begin{array}{ll}
-U & |x|<a \\
0 & |x| \geq a
\end{array}
\right.
\end{aligned}
\end{equation*}
Seek solutions of time-independent SE with $-U<E<0$:
\begin{equation*}
\begin{aligned}
-\frac{\hbar^2}{2m}\psi'' = \left(E+U\right)\psi & &|x|<a\\
-\frac{\hbar^2}{2m}\psi'' = E\psi & &|x|>n
\end{aligned}
\end{equation*}

Set $U+E = \frac{\hbar^2 k^2}{2m}$ and $E=-\frac{\hbar^2\kappa^2}{2m}$ for $k,\kappa \in \R^+$. Then
\begin{equation*}
\begin{aligned}
k^2 + \kappa^2 = \frac{2mU}{\hbar^2}
\end{aligned}
\end{equation*}
Then SE becomes
\begin{equation*}
\begin{aligned}
\psi'' + k^2 \psi = 0 & & |x|<a\\
\psi'' - \kappa^2 \psi = 0 & & |x|>a
\end{aligned}
\end{equation*}
Need $\psi,\psi'$ continuous but $\psi''$ discontinuous at $x = \pm a$.

Consider \emph{even parity} solutions with $\psi\left(-x\right) = \psi\left(x\right)$:\begin{equation*}
\begin{aligned}
\psi = \left\{\begin{array}{ll}
A\cos kx & |x|<a\\
Be^{-\kappa x} & |x|>a
\end{array}
\right.
\end{aligned}
\end{equation*}
Matching $\psi$ and $\psi'$ at $x=a$ ($x=-a$ automatic for $\psi$ even):
\begin{equation*}
\begin{aligned}
A\cos ka = Be^{-\kappa a}\\
-Ak\sin ka = -B \kappa e^{-\kappa a}
\end{aligned}
\end{equation*}

These give the same result for $A/B$ if and only if
\begin{equation*}
\begin{aligned}
k\tan ka = \kappa
\end{aligned}
\end{equation*}
To see when solutions exists, convenient to set
\begin{equation*}
\begin{aligned}
\xi = ak, \eta = a\kappa
\end{aligned}
\end{equation*}
which are dimensionless (and positive), and consider
\begin{equation*}
\begin{aligned}
\eta = \xi \tan \xi,\\
\xi^2 + \eta^2 = \frac{2ma^2}{\hbar^2} U
\end{aligned}
\end{equation*}

For each point of intersection, we get one solution for $\xi,\eta$ or $k,\kappa$ and corresponding value of $E$.\\
Hence there is exactly one solution for $\frac{2ma^2}{\hbar^2}U < \pi^2$.

In general, there are $n$ solutions for $\left(n-1\right)^2 \pi^2 < \frac{2ma^2U}{h\hbar^2} < n^2\pi^2$.

There are finite number of allowed energy eigenstates.

\begin{tikzpicture}
\draw (0,0) -- (1,0);
\draw (1,0) -- (0,1);
\draw (0.5,1) -- (0.25,-0.25);
\draw (0,0) .. controls (1,0) and (0.5,1) .. (1,1);
\draw (2,2) .. controls (1.65,-0.95) and (-0.3,3.4) .. (1,0);
\end{tikzpicture}

Note that now we have non-zero probability density $|\psi\left(x\right)|^2$ of measuring particle \emph{outside} classically allowed region $|x|<a$ (for $E<0$).

We can consider \emph{odd parity} solutions $\psi\left(-x\right) = -\psi\left(x\right)$ similarly (see example sheet 1).

\subsection{General Properties}

\subsubsection{Bound state energies}
Consider time-independent SE with $V\left(x\right) \to 0$ as $x \to \pm \infty$. For 2nd order ODE, there are 2 complex constants in general solution.\\
But this is linear in $\psi$, so one complex constants corresponds to $\psi \to \lambda \psi$.

Now
\begin{equation*}
\begin{aligned}
-\frac{\hbar^2}{2m}\psi'' \sim E\psi
\end{aligned}
\end{equation*}
as $x \to \pm \infty$. So
\begin{equation*}
\begin{aligned}
\psi \sim A_\pm e^{ikx} + B_\pm e^{-ikx} & & E=\frac{\hbar^2 k^2}{2m} > 0\\
\psi \sim A_\pm e^{\kappa x} + B_\pm e^{-\kappa x} & & E=-\frac{\hbar^2\kappa^2}{2m} < 0
\end{aligned}
\end{equation*}
For $E>0$ there is no normalisable solution.

For $E<0$ we have normalisable solution if
\begin{equation*}
\begin{aligned}
\psi \sim \left\{ \begin{array}{ll}
B_+ e^{-\kappa x} & x\to +\infty \  \left(A_+ = 0\right)\\
A_- e^{\kappa x} & x \to -\infty \  \left(B_- = 0\right)
\end{array}
\right.
\end{aligned}
\end{equation*}

Only one complex constant left to choose, so specifying behaviour at both boundaries $\implies$ over-determined system, solutions exist for \emph{particular} values of $E$ $\implies$ bound state energies quantised.

We may have several bound states:

\begin{tikzpicture}
\draw (-2,0) -- (2,0);
\draw (0,-2) -- (0,2);
\draw (-2,-0.1) .. controls (-1.0,-0.15) and (-0.5,-1.0) .. (0,-1);
\draw (2,-0.1) .. controls (1.0,-0.15) and (0.5,-1.0) .. (0,-1);
\end{tikzpicture}

or none:

\begin{tikzpicture}
\draw (-2,0) -- (2,0);
\draw (0,-2) -- (0,2);
\draw (-2,0.1) .. controls (-1.0,0.15) and (-0.5,1.0) .. (0,1);
\draw (2,0.1) .. controls (1.0,0.15) and (0.5,1.0) .. (0,1);
\end{tikzpicture}

Furthermore, if $V\left(x\right) \geq V_0$ (constant), then for $\psi$ normalisable,
\begin{equation*}
\begin{aligned}
H\psi = E\psi \implies E &= \left<H\right>_\psi\\
&=\int_{-\infty}^\infty \left(-\frac{\hbar^2}{2m} \psi^* \psi'' + V\left(x\right) |\psi\left(x\right)|^2\right) dx\\
&= \int_{-\infty}^\infty \left(\frac{\hbar^2}{2m}|\psi'|^2 + V\left(x\right) |\psi|^2\right) dx\\
&\geq 0+V_0
\end{aligned}
\end{equation*}
(integration by parts). So $0>E>V_0$ for any bounded state.

\newpage

\section{Expectation and Uncertainty}
\subsection{Hermitian Operators}
Recall earlier definition
\begin{equation*}
\begin{aligned}
\left(\phi,\psi\right) = \int_{-\infty}^\infty \phi\left(x\right)^* \psi\left(x\right) dx
\end{aligned}
\end{equation*}
with properties
\begin{equation*}
\begin{aligned}
\left(\psi,\alpha\psi\right)=\alpha\left(\phi,\psi\right) = \left(\alpha^*\phi,\psi\right)
\end{aligned}
\end{equation*}
and similarly
\begin{equation*}
\begin{aligned}
\left(\phi,\psi\right)^* = \left(\psi,\phi\right)
\end{aligned}
\end{equation*}

Regarding this as an inner product on wave functions, define the \emph{norm} of $psi$, denoted $||\psi||$, by
\begin{equation*}
\begin{aligned}
||\psi||^2 = \left(\psi,\psi\right) = \int_{-\infty}^\infty |\psi\left(x\right)|^2 dx
\end{aligned}
\end{equation*}
which is real and positive, and $||\psi|| = 1$ is $\psi$ is normalised.

An operator $Q$ is \emph{hermitian} if
\begin{equation*}
\begin{aligned}
\left(\phi,Q\psi\right) = \left(Q\phi,\psi\right)
\end{aligned}
\end{equation*}
for all normalisable $\psi,\phi$. This implies
\begin{equation*}
\begin{aligned}
\left(\psi,Q\psi\right) = \left(Q\psi,\psi\right) = \left(\psi,Q\psi\right)^*\\
\implies \left<Q\right>_\psi = \left<Q\right>_\psi^*
\end{aligned}
\end{equation*}

The operators $\hat{x},\hat{p}$, and $H=\frac{\hat{p}^2}{2m} + V\left(\hat{x}\right)$ are hermitian (for $V$ real).

Check:
\begin{equation*}
\begin{aligned}
&\left(\phi,\hat{x}\psi\right) = \left(\hat{x}\phi,\psi\right)\\
\iff & \int_{-\infty}^\infty \phi\left(x\right)^* \left(x\psi\left(x\right)\right)dx = \int_{-\infty}^\infty \left(x\phi\left(x\right)\right)^* \psi\left(x\right) dx
\end{aligned}
\end{equation*}
which is true ($x$ is real).

\begin{equation*}
\begin{aligned}
&\left(\phi,\hat{p}\psi\right) = \left(\hat{p}\phi,\psi\right)\\
&\iff \int_{-\infty}^\infty \phi^* \left(-i\hbar \psi'\right) dx = \int_{-\infty}^\infty \left(-ih\phi'\right)^* \psi dx
\end{aligned}
\end{equation*}
by parts and $\left[\phi^*\psi\right]_{-\infty}^\infty = 0$.

To show $\left(\phi,H\psi\right) = \left(H\phi,\psi\right)$, check KE and PE terms separately:\\
KE: $\left(\phi,\psi''\right) = -\left(\phi',\psi'\right) = \left(\phi'',\psi\right)$;\\
PE: $\left(\phi,V\left(x\right)\psi\right) = \left(V\left(x\right)\phi,\psi\right)$ for $V$ real.

(Later in chapter 6 we'll prove other general properties of hermitian operators, e.g. eigenvalues are real, eigenstates with distinct eigenvalues are orthogonal with respect to inner product.)

\subsection{Ehrenfest's Theorem}

Consider normalised $\Psi\left(x,t\right)$ satisfying SE
\begin{equation*}
\begin{aligned}
i\hbar\dot{\Psi} = H\Psi = \left(\frac{\hat{p}^2}{2m} + V\left(\hat{x}\right)\right)\Psi = -\frac{\hbar^2}{2m}\Psi'' + V\left(x\right) \Psi
\end{aligned}
\end{equation*}
The expectation values
\begin{equation*}
\begin{aligned}
\left<\hat{x}\right>_\Psi = \left(\Psi,\hat{x}\Psi\right)
\end{aligned}
\end{equation*}
and
\begin{equation*}
\begin{aligned}
\left<\hat{p}\right>_\Psi = \left(\Psi,\hat{p}\Psi\right)
\end{aligned}
\end{equation*}
depend on $t$ through $\Psi$. Ehrenfest's Theorem states
\begin{equation*}
\begin{aligned}
\frac{d}{dt}\left<\hat{x}\right>_\Psi = \frac{1}{m} \left<\hat{p}\right>_\Psi
\end{aligned}
\end{equation*}
and
\begin{equation*}
\begin{aligned}
\frac{d}{dt}\left<\hat{p}\right>_\Psi = -\left<V'\left(\hat{x}\right)\right>_\Psi
\end{aligned}
\end{equation*}
which is the quantum counterparts to classical equations of motion (in first order form).
\begin{proof}
\begin{equation*}
\begin{aligned}
\frac{d}{dt}\left<\hat{x}\right>_\Psi &= \left(\dot{\Psi},\hat{x}\Psi\right) + \left(\Psi,\hat{x}\dot{\Psi}\right)\\
&= \left(\frac{1}{i\hbar} H \Psi, \hat{x}\Psi\right) + \left(\Psi,\hat{x}\frac{1}{i\hbar}H\Psi\right)
\end{aligned}
\end{equation*}

Since $H$ is hermitian,
\begin{equation*}
\begin{aligned}
&-\frac{1}{i\hbar}\left(H\Psi,\hat{x}\Psi\right) + \frac{1}{i\hbar}\left(\Psi,\hat{x}H\Psi\right)\\
&= -\frac{1}{i\hbar}\left(\Psi,H\hat{x}\Psi\right)+\frac{1}{i\hbar}\left(\Psi,\hat{x}H\Psi\right)\\
&= \frac{1}{i\hbar}\left(\Psi,\left(\hat{x}H-H\hat{x}\right)\Psi\right)
\end{aligned}
\end{equation*}
But
\begin{equation*}
\begin{aligned}
\left(\hat{x}H-H\hat{x}\right)\Psi &= \frac{-\hbar^2}{2m}\left(x\Psi'' - \left(x\Psi\right)''\right) + \left(xV-Vx\right)\Psi\\
&= +\frac{\hbar^2}{2m}2\Psi'\\
&= \frac{i\hbar}{m}\hat{p}\Psi
\end{aligned}
\end{equation*}
as required.

Similarly,
\begin{equation*}
\begin{aligned}
\frac{d}{dt}\left<\hat{p}\right>_\Psi &= \left(\dot{\Psi},\hat{p}\Psi\right)+\left(\Psi,\hat{p}\dot{\Psi}\right)\\
&= \left(\frac{1}{i\hbar}H\Psi,\hat{p}\Psi\right) + \left(\Psi,\hat{p}\frac{1}{i\hbar}H\Psi\right)\\
&= \frac{1}{i\hbar}\left(\Psi,\left(\hat{p}H-H\hat{p}\right)\Psi\right)
\end{aligned}
\end{equation*}
But
\begin{equation*}
\begin{aligned}
\left(\hat{p}H-H\hat{p}\right)\Psi &= -i\hbar \left(-\frac{\hbar^2}{2m}\right)\left(\left(\Psi''\right)'-\left(\Psi'\right)''\right) (=0) - i\hbar\left(\left(V\Psi\right)'-V\Psi'\right)\\
&= -i\hbar V'\left(x\right)\Psi
\end{aligned}
\end{equation*}
as required.

\subsection{The Uncertainty Principle}
If $\psi$ is any normalised state (at fixed time) define the \emph{uncertainty} in position $\left(\Delta x\right)_\psi$ and momentum $\left(\Delta p\right)_\psi$ by
\begin{equation*}
\begin{aligned}
\left(\Delta x\right)_\psi^2 = \left<\left(\hat{x}-\left<\hat{x}\right>-\psi\right)^2\right>_\psi \neq = \left<\hat{x}^2\right>_\psi - \left<\hat{x}\right>_\psi^2
\end{aligned}
\end{equation*}
and the same formula for $\left(\Delta p\right)_\psi$.

These quantify 'spread' of possible results of measurements.

Heisenberg's Uncertainty Principle states
\begin{equation*}
\begin{aligned}
\left(\Delta x\right)_\psi \left(\Delta p\right)_\psi \geq \frac{\hbar}{2}
\end{aligned}
\end{equation*}
Interpretation: we can never reduce combined uncertainty in measurements of position and momentum below this threshold.

Note: $X=\hat{x}-\alpha$ and $P=\hat{p}-\beta$ are both hermitian for any real $\alpha,\beta$.

\begin{equation*}
\begin{aligned}
\left(\psi,X^2\psi\right) = \left(X\psi,X\psi\right) = ||X\psi||^2 \geq 0,\
\left(\psi,P^2\psi\right) = \left(P\psi,P\psi\right) = ||P\psi||^2 \geq 0
\end{aligned}
\end{equation*}
Choosing $\alpha = \left<\hat{x}\right>_\psi$ and $\beta = \left<\hat{p}\right>_\psi$, we deduce $\left(\Delta x\right)_\psi^2$ and $\left(\Delta p\right)_\psi^2$ are indeed real and positive, as required in our definition.

\begin{eg}
For Gaussian
\begin{equation*}
\begin{aligned}
\psi\left(x\right) = \left(\frac{1}{\alpha\pi}\right)^\frac{1}{4} e^{-x^2/2\alpha}
\end{aligned}
\end{equation*}
find
\begin{equation*}
\begin{aligned}
\left<\hat{x}\right>_\psi = \left<\hat{p}\right>_\psi = 0
\end{aligned}
\end{equation*}
and
\begin{equation*}
\begin{aligned}
\left(\Delta x\right)_\psi^2 = \alpha/2,\left(\Delta p\right)_\psi^2 = \hbar^2 / 2\alpha
\end{aligned}
\end{equation*}
So
\begin{equation*}
\begin{aligned}
\left(\Delta x\right)_\psi \left(\Delta p\right)_\psi = \frac{\hbar}{2}
\end{aligned}
\end{equation*}
\end{eg}

\end{proof}

\end{document}