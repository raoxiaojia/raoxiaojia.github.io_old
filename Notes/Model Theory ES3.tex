\documentclass[a4paper]{article}

\setlength{\parindent}{0pt}
\setlength{\parskip}{1em}

\pagestyle{headings}

\usepackage{amssymb}
\usepackage{amsmath}
\usepackage{amsthm}
\usepackage{mathtools}
\usepackage{graphicx}
\usepackage{hyperref}
\usepackage{color}
\usepackage{microtype}
\usepackage{tikz}
\usepackage{pgfplots}
\usepackage{pgfplotstable}

\newcommand{\N}{\mathbb{N}}
\newcommand{\Q}{\mathbb{Q}}
\newcommand{\Z}{\mathbb{Z}}
\newcommand{\R}{\mathbb{R}}
\newcommand{\C}{\mathbb{C}}
\newcommand{\D}{\mathcal{D}}
\renewcommand{\S}{\mathcal{S}}
\renewcommand{\P}{\mathbb{P}}
\newcommand{\F}{\mathbb{F}}
\newcommand{\E}{\mathbb{E}}

\graphicspath{{Image/}}

\hypersetup{
    colorlinks=true,
    linktoc=all,
    linkcolor=blue
}

\theoremstyle{definition}
\newtheorem*{axiom}{Axiom}
\newtheorem*{claim}{Claim}
\newtheorem*{conv}{Convention}
\newtheorem*{coro}{Corollary}
\newtheorem*{defi}{Definition}
\newtheorem*{eg}{Example}
\newtheorem*{lemma}{Lemma}
\newtheorem*{notation}{Notation}
\newtheorem*{prob}{Problem}
\newtheorem*{post}{Postulate}
\newtheorem*{prop}{Proposition}
\newtheorem*{rem}{Remark}
\newtheorem*{thm}{Theorem}

\DeclareMathOperator{\vdiv}{div}
\DeclareMathOperator{\grad}{grad}
\DeclareMathOperator{\curl}{curl}
\DeclareMathOperator{\Ann}{Ann}
\DeclareMathOperator{\Fit}{Fit}
\DeclareMathOperator{\Diag}{Diag}
\DeclareMathOperator{\tr}{tr}
\DeclareMathOperator{\im}{im}
\DeclareMathOperator{\Mat}{Mat}
\DeclareMathOperator{\Log}{Log}
\DeclareMathOperator{\Isom}{Isom}
\DeclareMathOperator{\Mesh}{Mesh}
\DeclareMathOperator{\Sym}{Sym}
\DeclareMathOperator{\Aut}{Aut}
\DeclareMathOperator{\cosech}{cosech}
\DeclareMathOperator{\Card}{Card}
\DeclareMathOperator{\Gal}{Gal}


\begin{document}

\title{Model Theory Example Sheet 3}

\author{Xiaojia Rao}

\maketitle

\setcounter{secnumdepth}{0}

\section{Question 4}

Suppose $a \not\in acl(A)$. Then there exists a model $\mathcal{M} \preccurlyeq U$ which contains $A$ but not $a$. But $a \in acl(A,b)$ for any realization $b$ of $\phi(z)$, so $\mathcal{M}$ cannot contain any of those $b$, i.e. $\mathcal{M}$ does not contain any realization of $\phi(z)$. But it does contain $A$, so the sentence $\exists z \phi(z)$ should be true in $\mathcal{M}$ since it is true in $U$ and by elementarity. Contradiction.

For types it's similar: similarly we have a model $\mathcal{M}$ containing $A$ that does not contain $a$ and  has no realization of $p(z)$. Note that by elementarity $p(z)$ is still finitely consistent in $\mathcal{M}$, so $p(z)$ must be an infinite type.

\section{Question 7}

Suppose $C \cap acl(A) = \phi$. Then for each $c \in C$, consider its type over $A$, $p_c(x):=tp(c/A)$: for $\phi(x) \in p_c(x)$, we have $|\phi(U)| \geq \omega$. So by the same argument from lecture we know $|p_c(U)| = |U|$.\\
We continue mimicking the proof: let $\mathcal{M} \supset A$ be a model, then $p_c(U) \setminus \mathcal{M} \neq \phi$ for each $c \in C$. Pick $b_c \in p_c(U) \setminus \mathcal{M}$, then by homogeneity

\end{document}
