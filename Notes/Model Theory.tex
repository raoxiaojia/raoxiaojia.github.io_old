\documentclass[a4paper]{article}

\setlength{\parindent}{0pt}
\setlength{\parskip}{1em}

\pagestyle{headings}

\usepackage{amssymb}
\usepackage{amsmath}
\usepackage{amsthm}
\usepackage{mathtools}
\usepackage{graphicx}
\usepackage{hyperref}
\usepackage{color}
\usepackage{microtype}
\usepackage{tikz}
\usepackage{pgfplots}
\usepackage{pgfplotstable}

\newcommand{\N}{\mathbb{N}}
\newcommand{\Q}{\mathbb{Q}}
\newcommand{\Z}{\mathbb{Z}}
\newcommand{\R}{\mathbb{R}}
\newcommand{\C}{\mathbb{C}}
\newcommand{\D}{\mathcal{D}}
\renewcommand{\S}{\mathcal{S}}
\renewcommand{\P}{\mathbb{P}}
\newcommand{\F}{\mathbb{F}}
\newcommand{\E}{\mathbb{E}}
\newcommand{\bra}{\langle}
\newcommand{\ket}{\rangle}


\graphicspath{{Image/}}

\hypersetup{
    colorlinks=true,
    linktoc=all,
    linkcolor=blue
}

\theoremstyle{definition}
\newtheorem*{axiom}{Axiom}
\newtheorem*{claim}{Claim}
\newtheorem*{conv}{Convention}
\newtheorem*{coro}{Corollary}
\newtheorem*{defi}{Definition}
\newtheorem*{eg}{Example}
\newtheorem*{lemma}{Lemma}
\newtheorem*{notation}{Notation}
\newtheorem*{prob}{Problem}
\newtheorem*{post}{Postulate}
\newtheorem*{prop}{Proposition}
\newtheorem*{rem}{Remark}
\newtheorem*{thm}{Theorem}

\DeclareMathOperator{\vdiv}{div}
\DeclareMathOperator{\grad}{grad}
\DeclareMathOperator{\curl}{curl}
\DeclareMathOperator{\Ann}{Ann}
\DeclareMathOperator{\Fit}{Fit}
\DeclareMathOperator{\Diag}{Diag}
\DeclareMathOperator{\tr}{tr}
\DeclareMathOperator{\im}{im}
\DeclareMathOperator{\Mat}{Mat}
\DeclareMathOperator{\Log}{Log}
\DeclareMathOperator{\Isom}{Isom}
\DeclareMathOperator{\Mesh}{Mesh}
\DeclareMathOperator{\Sym}{Sym}
\DeclareMathOperator{\Aut}{Aut}
\DeclareMathOperator{\cosech}{cosech}
\DeclareMathOperator{\Card}{Card}
\DeclareMathOperator{\Gal}{Gal}


\setcounter{section}{-1}

\begin{document}

\title{Model Theory}

\maketitle

\newpage

\tableofcontents

\newpage

\section{Reviews}

\subsection{Langauges and structures}

\begin{defi} (1.1)
A language $L$ consists of:\\
$\bullet$(i) a set $\mathcal{F}$ of function symbols, and for each $f \in \mathcal{F}$, a positive integer $n_f$, the arity of $f$;\\
$\bullet$(ii) a set $\mathcal{R}$ of relation symbols, and for each $R \in \mathcal{R}$, a positive integer $n_R$, the arity of $R$;\\
$\bullet$(iii) a set $\mathcal{C}$ of constant symbols.\\
Note that each of the above three sets can be empty.
\end{defi}

\begin{eg}
$L=\{\{\cdot,-1\},\{1\}\}$ where $\cdot$ is a binary function, $-1$ is a unary function, and $1$ is a constant. We call this $L_{gp}$ (language of groups);\\
$L_{lo} = \{<\}$, where $<$ is a binary relation (linear order).
\end{eg}

\begin{defi} (1.2)
Given a language $L$, say, an $L$-structure consists of:\\
(i) a set $M$, the \emph{domain};\\
(ii) for each $f \in \mathcal{F}$, a function $f^M:M^{n_f} \to M$;\\
(iii) for each $R \in \mathcal{R}$, a relation $R^M \subseteq M^{n_R}$;\\
(iv) for each $c \in \mathcal{C}$, an element $c^M \in M$.

$f^M,R^M,c^M$ are called the \emph{interpretation} of $f,R,c$ respectively.
\end{defi}

\begin{notation} (1.3)\\
We often fail to distinguish between the symbols in the language $L$ and their interpretations in a $L$-structure, if the context allows.

We may write $\mathcal{M} = <M,\mathcal{F},\mathcal{R},\mathcal{C}>$.
\end{notation}

\begin{eg} (1.4)\\
(a) $\mathcal{R} = <\R^+,\{\cdot,-1\},1>$ is an $L_{gp}$-structure.\\
$\mathcal{Z} = <\Z,\{+,-\},0>$ is also an $L_{gp}$-structure (here $+$ is a binary and $-$ is the unary negation function).\\
$\mathcal{Q} = <\Q,<>$ is an $L_{lo}$ structure ($<$ is the interpretation of relation).
\end{eg}

\begin{defi} (1.5)\\
Let $L$ be a language, let $\mathcal{M}$ and $\mathcal{N}$ be $L$-structures.\\
An \emph{embedding} of $\mathcal{M}$ into $\mathcal{N}$ is an injection $\alpha:M \to N$ that preserves the structure:\\
(i) For all $f \in \mathcal{F}$, and $a_1,...,a_{n_f} \in M$,
\begin{equation*}
\begin{aligned}
\alpha(f^M(a_1,...,a_{n_f})) = f^N(\alpha(a_1),...,\alpha(a_{n_f}))
\end{aligned}
\end{equation*}
(ii) For all $R \in \mathcal{R}$, and $a_1,...,a_{n_R} \in M$,
\begin{equation*}
\begin{aligned}
(a_1,...,a_{n_R}) \in R^M \iff (\alpha(a_1),...,\alpha(a_{n_R})) \in R^N
\end{aligned}
\end{equation*}
Note that this is an if and only if.
(iii) For all $c \in \mathcal{C}$, we need 
\begin{equation*}
\begin{aligned}
\alpha(c^M) = c^N
\end{aligned}
\end{equation*}
As anyone could expect, a surjective embedding $\mathcal{M} \to \mathcal{N}$ is also called an \emph{isomorphism} of $\mathcal{M}$ onto $\mathcal{N}$.
\end{defi}

(1.6) Exercise. Let $G_1,G_2$ be groups, regarded as $L_{gp}$-structures.\\
Check that $G_1 \cong G_2$ in the usual algebra sense, if and only if there is an isomprhism $\alpha:G_1\ to G_2$ in the sense of above definition 1.5.

\subsection{Terms, formulae, and their interpretations}
In addition to the symbols of $L$, we also have:\\
(i) infinitely many variables, $\{x_i\}_{i \in I}$;\\
(ii) logical connectives, $\wedge,\neg$ (also express $\vee, \to, \leftrightarrow$);\\
(iii) quantifier $\exists$ (also express $\forall$);\\
(iv) punctuations $(,)$.

\begin{defi} (2.1) \\
\emph{$L$-terms} are defined recursively as follows:\\
$\bullet$ any variable $x_i$ is a term;\\
$\bullet$ any constant symbol is a term;\\
$\bullet$ for any $f \in \mathcal{F}$, 
\begin{equation*}
\begin{aligned}
f(t_1,...,t_{n_f})
\end{aligned}
\end{equation*}
for any terms $t_1,...,t_{n_f}$ is a term;\\
$\bullet$ nothing else is a term.
\end{defi}

Notation: we write $t(x_1,...,x_n)$ to mean that the variables appearing in $t$ are among $x_1,..,x_n$.

\begin{eg}
In $\mathcal{R} = <\R,\cdot,-1,1>$,\\
$\bullet$ $(\cdot (x_1,x_2),x_3)$ is a term ($x_1 \cdot x_2) \cdot x_3$);\\
$\bullet$ $(\cdot (1,x_1))^{-1}$ is a term ($1 \cdot x)^{-1}$.
\end{eg}

\begin{defi} (2.2)\\
If $\mathcal{M}$ is an $L$-structure, to each $L$-term $t(x_1,...,x_k)$ we assign a function
\begin{equation*}
\begin{aligned}
t^M: M^k \to M
\end{aligned}
\end{equation*}
defined as follows:\\
(i) If $t = x_i, t^M [a_1,...,a_k] = a_i$;\\
(ii) If $t=c$ is a constant, $t^M [a_1,...,a_k] = c^m$;\\
(iii) If $t=f(t_1(x_1,...,x_k),...,t_{n_f}(x_1,...,x_k))$, 
\begin{equation*}
\begin{aligned}
t^M (a_1,...,a_k) = f^M(t_1^M(a_1,...,a_k),...,t_{n_f}^M(a_1,...,a_k))
\end{aligned}
\end{equation*}
\end{defi}

\end{document}
