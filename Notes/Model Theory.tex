\documentclass[a4paper]{article}

\setlength{\parindent}{0pt}
\setlength{\parskip}{1em}

\pagestyle{headings}

\usepackage{amssymb}
\usepackage{amsmath}
\usepackage{amsthm}
\usepackage{mathtools}
\usepackage{graphicx}
\usepackage{hyperref}
\usepackage{color}
\usepackage{microtype}
\usepackage{tikz}
\usepackage{pgfplots}
\usepackage{pgfplotstable}

\newcommand{\N}{\mathbb{N}}
\newcommand{\Q}{\mathbb{Q}}
\newcommand{\Z}{\mathbb{Z}}
\newcommand{\R}{\mathbb{R}}
\newcommand{\C}{\mathbb{C}}
\newcommand{\D}{\mathcal{D}}
\renewcommand{\S}{\mathcal{S}}
\renewcommand{\P}{\mathbb{P}}
\newcommand{\F}{\mathbb{F}}
\newcommand{\E}{\mathbb{E}}
\newcommand{\bra}{\langle}
\newcommand{\ket}{\rangle}


\graphicspath{{Image/}}

\hypersetup{
    colorlinks=true,
    linktoc=all,
    linkcolor=blue
}

\theoremstyle{definition}
\newtheorem*{axiom}{Axiom}
\newtheorem*{claim}{Claim}
\newtheorem*{conv}{Convention}
\newtheorem*{coro}{Corollary}
\newtheorem*{defi}{Definition}
\newtheorem*{eg}{Example}
\newtheorem*{lemma}{Lemma}
\newtheorem*{notation}{Notation}
\newtheorem*{prob}{Problem}
\newtheorem*{post}{Postulate}
\newtheorem*{prop}{Proposition}
\newtheorem*{rem}{Remark}
\newtheorem*{thm}{Theorem}

\DeclareMathOperator{\vdiv}{div}
\DeclareMathOperator{\grad}{grad}
\DeclareMathOperator{\curl}{curl}
\DeclareMathOperator{\Ann}{Ann}
\DeclareMathOperator{\Fit}{Fit}
\DeclareMathOperator{\Diag}{Diag}
\DeclareMathOperator{\tr}{tr}
\DeclareMathOperator{\im}{im}
\DeclareMathOperator{\Mat}{Mat}
\DeclareMathOperator{\Log}{Log}
\DeclareMathOperator{\Isom}{Isom}
\DeclareMathOperator{\Mesh}{Mesh}
\DeclareMathOperator{\Sym}{Sym}
\DeclareMathOperator{\Aut}{Aut}
\DeclareMathOperator{\cosech}{cosech}
\DeclareMathOperator{\Card}{Card}
\DeclareMathOperator{\Gal}{Gal}


\setcounter{section}{-1}

\begin{document}

\title{Model Theory}

\maketitle

\newpage

\tableofcontents

\newpage

\section{Reviews}

\subsection{Langauges and structures}

\begin{defi} (1.1)
A language $L$ consists of:\\
$\bullet$(i) a set $\mathcal{F}$ of function symbols, and for each $f \in \mathcal{F}$, a positive integer $n_f$, the arity of $f$;\\
$\bullet$(ii) a set $\mathcal{R}$ of relation symbols, and for each $R \in \mathcal{R}$, a positive integer $n_R$, the arity of $R$;\\
$\bullet$(iii) a set $\mathcal{C}$ of constant symbols.\\
Note that each of the above three sets can be empty.
\end{defi}

\begin{eg}
$L=\{\{\cdot,-1\},\{1\}\}$ where $\cdot$ is a binary function, $-1$ is a unary function, and $1$ is a constant. We call this $L_{gp}$ (language of groups);\\
$L_{lo} = \{<\}$, where $<$ is a binary relation (linear order).
\end{eg}

\begin{defi} (1.2)
Given a language $L$, say, an $L$-structure consists of:\\
(i) a set $M$, the \emph{domain};\\
(ii) for each $f \in \mathcal{F}$, a function $f^M:M^{n_f} \to M$;\\
(iii) for each $R \in \mathcal{R}$, a relation $R^M \subseteq M^{n_R}$;\\
(iv) for each $c \in \mathcal{C}$, an element $c^M \in M$.

$f^M,R^M,c^M$ are called the \emph{interpretation} of $f,R,c$ respectively.
\end{defi}

\begin{notation} (1.3)\\
We often fail to distinguish between the symbols in the language $L$ and their interpretations in a $L$-structure, if the context allows.

We may write $\mathcal{M} = \bra M,\mathcal{F},\mathcal{R},\mathcal{C}\ket$.
\end{notation}

\begin{eg} (1.4)\\
(a) $\mathcal{R} = \bra\R^+,\{\cdot,-1\},1\ket$ is an $L_{gp}$-structure.\\
$\mathcal{Z} = \bra\Z,\{+,-\},0\ket$ is also an $L_{gp}$-structure (here $+$ is a binary and $-$ is the unary negation function).\\
$\mathcal{Q} = \bra\Q,<\ket$ is an $L_{lo}$ structure ($<$ is the interpretation of relation).
\end{eg}

\begin{defi} (1.5)\\
Let $L$ be a language, let $\mathcal{M}$ and $\mathcal{N}$ be $L$-structures.\\
An \emph{embedding} of $\mathcal{M}$ into $\mathcal{N}$ is an injection $\alpha:M \to N$ that preserves the structure:\\
(i) For all $f \in \mathcal{F}$, and $a_1,...,a_{n_f} \in M$,
\begin{equation*}
\begin{aligned}
\alpha(f^M(a_1,...,a_{n_f})) = f^N(\alpha(a_1),...,\alpha(a_{n_f}))
\end{aligned}
\end{equation*}
(ii) For all $R \in \mathcal{R}$, and $a_1,...,a_{n_R} \in M$,
\begin{equation*}
\begin{aligned}
(a_1,...,a_{n_R}) \in R^M \iff (\alpha(a_1),...,\alpha(a_{n_R})) \in R^N
\end{aligned}
\end{equation*}
Note that this is an if and only if.
(iii) For all $c \in \mathcal{C}$, we need 
\begin{equation*}
\begin{aligned}
\alpha(c^M) = c^N
\end{aligned}
\end{equation*}
As anyone could expect, a surjective embedding $\mathcal{M} \to \mathcal{N}$ is also called an \emph{isomorphism} of $\mathcal{M}$ onto $\mathcal{N}$.
\end{defi}

(1.6) Exercise. Let $G_1,G_2$ be groups, regarded as $L_{gp}$-structures.\\
Check that $G_1 \cong G_2$ in the usual algebra sense, if and only if there is an isomprhism $\alpha:G_1\ \to G_2$ in the sense of above definition 1.5.

\subsection{Terms, formulae, and their interpretations}
In addition to the symbols of $L$, we also have:\\
(i) infinitely many variables, $\{x_i\}_{i \in I}$;\\
(ii) logical connectives, $\wedge,\neg$ (also express $\vee, \to, \leftrightarrow$);\\
(iii) quantifier $\exists$ (also express $\forall$);\\
(iv) punctuations $(,)$.

\begin{defi} (2.1) \\
\emph{$L$-terms} are defined recursively as follows:\\
$\bullet$ any variable $x_i$ is a term;\\
$\bullet$ any constant symbol is a term;\\
$\bullet$ for any $f \in \mathcal{F}$, 
\begin{equation*}
\begin{aligned}
f(t_1,...,t_{n_f})
\end{aligned}
\end{equation*}
for any terms $t_1,...,t_{n_f}$ is a term;\\
$\bullet$ nothing else is a term.
\end{defi}

Notation: we write $t(x_1,...,x_n)$ to mean that the variables appearing in $t$ are among $x_1,..,x_n$.

\begin{eg}
In $\mathcal{R} = <\R,\cdot,-1,1>$,\\
$\bullet$ $(\cdot (x_1,x_2),x_3)$ is a term ($x_1 \cdot x_2) \cdot x_3$);\\
$\bullet$ $(\cdot (1,x_1))^{-1}$ is a term ($1 \cdot x)^{-1}$.
\end{eg}

\begin{defi} (2.2)\\
If $\mathcal{M}$ is an $L$-structure, to each $L$-term $t(x_1,...,x_k)$ we assign a function
\begin{equation*}
\begin{aligned}
t^M: M^k \to M
\end{aligned}
\end{equation*}
defined as follows:\\
(i) If $t = x_i, t^M [a_1,...,a_k] = a_i$;\\
(ii) If $t=c$ is a constant, $t^M [a_1,...,a_k] = c^m$;\\
(iii) If $t=f(t_1(x_1,...,x_k),...,t_{n_f}(x_1,...,x_k))$, 
\begin{equation*}
\begin{aligned}
t^M (a_1,...,a_k) = f^M(t_1^M(a_1,...,a_k),...,t_{n_f}^M(a_1,...,a_k))
\end{aligned}
\end{equation*}
\end{defi} 

---Lecture 2---

No lecture this friday (12th Oct)! Will have an extra one on Monday 22 Oct at 12 (MR12).

First example class: Monday 29th Oct at 12.

Info on course and notes on $http:\\users.mct.open.ac.uk/sb27627/MT.html$ (it seems that it only comes after lecture, and is hand-written, so this notes still continues), or google \emph{Silvia Barbina MCT} and follow link \emph{Part III Model Theory} on lecturer's homepage.

\begin{rem}
    (The lecture forgot about this last time) Any language $L$ includes an equality symbol $=$.
\end{rem}

Last time we assigned a function $t^m$. In $L_{gp}$, the term $x_2 \cdot x_3$ can be described as, say $t_1(x_1,x_2,x_3),t_2(x_1,x_2,x_3,x_4),...$.\\
Then the term $x_2 \cdot x_3$ can be assigned to functions $t_1^M:M^3\to M:(a_1,a_2,a_3) \to (a_2 \cdot a_3)$, or $t_2^M: M^4 \to M: (a_1,a_2,a_3,a_4) \to (a_2 \cdot a_3)$. These syntactic things are not really important -- we just have to know that there is a corresponding action for each term.

We now define the \emph{complexity} of a term $t$ to be the number of symbols of $L$ occuring in $t$.

Fact (2.3): Let $\mathcal{M}$ and $\mathcal{N}$ be $L$-structures, and let $\alpha:\mathcal{M} \to \mathcal{N}$ be an embedding. For any $L$-term $t(x_1,...,x_k)$ and $a_1,...,a_k \in M$, we have
\begin{equation*}
    \begin{aligned}
        \alpha(t^M(a_1,...,a_k)) = t^N(\alpha(a_1),...,\alpha(a_k))
    \end{aligned}
\end{equation*}
\begin{proof}
Prove by induction on complexity of $t$.\\
Let $\bar{a} = (a_1,...,a_k)$ and $\bar{x} = (x_1,...,x_l)$. Then:\\
(i) if $t=x_i$a a variable, then $t^M(\bar{a}) = a_i$, and $t^N(\alpha(a_1),...,\alpha(a_k)) = \alpha(a_i)$, so the conclusion holds;\\
(ii) if $t=c$ is a constant, then $t^M(\bar{a}) = c^M$, and $t^N(\alpha(\bar{a})) = c^N$ by definition of a term. The key here is that, since $\alpha$ is an embedding we have $\alpha(c^M) = c^N$;\\
(iii) if $t = f(t_1(\bar{x},...,t_{n_f}(\bar{x})))$, then
\begin{equation*}
    \begin{aligned}
    \alpha(f^M(t_1^M(\bar{a}),...,t_{n_f}(\bar{a}))) &= f^N(\alpha (t_1^M(\bar{a})),...,\alpha(t_{n_f}^M(\bar{a})))
    \end{aligned}
\end{equation*}
as $\alpha$ is an embedding. But $t_1(\bar{x}),...,t_{n_f}(\bar{x})$ have lower complexity than $t$, so the inductive hypothesis applies.
\end{proof}

Exercise (2.4): conclude the proof of the above fact.\\
(Actually is it not done?)

\begin{defi} (2.5)\\
    The set of \emph{atmoic formulas} of $L$ is defined as follows:\\
    (i) if $t_1,t_2$ are $L$-terms, then $t_1 = t_2$ is an atomic formula;\\
    (ii) if $R$ is a relation symbol, and $t_1,...,t_{n_R}$ are $L$-terms, then $R(t_1,...,t_{n_R})$ is an atomic formula;\\
    (iii) nothing else is an atomic formula.
\end{defi}

\begin{defi} (2.6)\\
    The set of $L$-formulas is defined as follows:\\
    (i) any atomic formula is an $L$-formula;\\
    (ii) if $\phi$ is an $L$-formula, then so is $\neg \phi$;\\
    (iii) if $\phi$ and $\psi$ are $L$-formulas, then so is $\phi \wedge \psi$;\\
    (iv) if $\phi$ is an $L$-formula, for any $i \geq 1$, $\exists x_i \phi$ is a formula;\\
    (v) nothing else is a formula (note that $\forall$ can be constructed by $\neg$ and $\exists$).
\end{defi}

\begin{eg}
    In $L_{gp}$, $x_1\cdot x_1 = x_2$, or $x_1\cdot x_2=1$ are both atomic formulas;\\
    $\exists x_1(x_1 \cdot x_2) = 1$ is an $L$-formula, but (obviously) not atomic.
\end{eg}

A variable occurs \emph{freely} in a formula if it does not occur within the scope of a quantifier $\exists$. We sometimes also say that the variable is \emph{free} (from Part II Logic and Sets). Otherwise we say the variable is \emph{bound}.

We'll use the convention that no variable occurs both freely and as a bound variable in the same formula.

A \emph{sentence} is a formula with no free variables. For example, $\exists x_1\exists x_2 (x_1\cdot x_2=1)$ is an $L_{gp}$-sentence.

Notation: $\phi(x_1,...,x_k)$ means that the free variables in $\phi$ are among $x_1,...,x_k$.

Now we introduce a long and inductive (and also in logic and sets) definition for which sentences are \emph{true}:
\begin{defi} (2.7)\\
    Let $\phi(x_1,...,x_k)$ be an $L$-formula, let $\mathcal{M}$ be an $L$-structure, and let $\bar{a} = a_1,...,a_k$ be elements of $\mathcal{M}$.\\
    We define $\mathcal{M} \vDash \phi(\bar{a})$ (syntactic implication, read as \emph{M models $\phi(\bar{a})$}) as follows:\\
    (i) if $\phi$ is $t_1=t_2$, then $\mathcal{M} \vDash \phi(\bar{a}) \iff t_1^M(\bar{a}) = t_2^M(\bar{a})$;\\
    (ii) if $\phi$ is $R(t_1,...,t_{n_R})$, then $\mathcal{M} \vDash \phi(\bar{a})$ iff
    \begin{equation*}
        \begin{aligned}
            \left(t_1^M(\bar{a}),...,t_{n_R}^M(\bar{a})\right) \in R^M
        \end{aligned}
    \end{equation*}
    (iii) if $\phi$ is a conjunction, say $\psi \wedge \chi$, then $\mathcal{M} \vDash \phi(\bar{a})$ iff $\mathcal{M} \vDash \psi(\bar{a})$ and $\mathcal{M} \vDash \chi(\bar{a})$;\\
    (iv) if $\phi$ is $\exists x_j \chi(x_1,...,x_k,x_j)$ (where we'll assume that $x_j$ is not one of the free variables $x_1,...,x_k$), then $\mathcal{M} \vDash \phi(\bar{a})$ iff there exists $b \in \mathcal{M}$ s.t. $\mathcal{M} \vDash \chi(a_1,...,a_k,b)$;\\
    (v) (lecture forgets this, this should probably be more in front rather than in the end) if $\phi$ is $\neg \psi$, then $\mathcal{M} \vDash \phi(\bar{a})$ iff $\mathcal{M} \not\vDash \psi(\bar{a})$.
\end{defi}

\begin{eg}
    Consider $\mathcal{R} = \bra \R^*,\cdot ,-1,1\ket$, the multiplicative group of non-negative reals, and suppose we have $\phi(x_1) = \exists x_2 (x_2 \cdot x_2 = x_1)$, then $\mathcal{R} \vDash \phi(1)$, but $\mathcal{R} \not\vDash \phi(-1)$.
\end{eg}

Notation (2.8) (useful abbreviations, closer to real life. The precise formulas are not that important -- the abbreviations mean what we expect in real life):\\
$\bullet$ $\phi \vee \psi$ for $\neg(\neg\phi \wedge \neg \psi)$;\\
$\bullet$ $\phi \to \psi$ for $\neg \phi \vee \psi$;\\
$\bullet$ $\phi \leftrightarrow \psi$ for $(\phi \to \psi) \wedge (\psi \to \phi)$;\\
$\bullet$ $\forall x_i \phi$ for $\neg \exists x_i (\neg \phi)$.

\begin{prop} (2.9)\\
    Let $\mathcal{M}$ and $\mathcal{N}$ be $L$-structures, and let $\alpha: \mathcal{M} \to \mathcal{N}$ be an embedding.\\
    Let $\phi(\bar{x})$ be an atomic(!) formula, and $\bar{a} \in M^k$ (from now on, when we write a tuple like $\bar{a}$, we will assume that it has the correct length without explicitly stating that), then
    \begin{equation*}
        \begin{aligned}
            \mathcal{M} \vDash \phi(\bar{a}) \iff \mathcal{N} \vDash \phi(\alpha(\bar{a}))
        \end{aligned}
    \end{equation*}
\end{prop}

Question: if $\phi$ is an $L$-formula, not necessarily atomic, does (2.9) still hold? (the answer is no!)

\end{document}
