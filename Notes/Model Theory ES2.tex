\documentclass[a4paper]{article}

\setlength{\parindent}{0pt}
\setlength{\parskip}{1em}

\pagestyle{headings}

\usepackage{amssymb}
\usepackage{amsmath}
\usepackage{amsthm}
\usepackage{mathtools}
\usepackage{graphicx}
\usepackage{hyperref}
\usepackage{color}
\usepackage{microtype}
\usepackage{tikz}
\usepackage{pgfplots}
\usepackage{pgfplotstable}

\newcommand{\N}{\mathbb{N}}
\newcommand{\Q}{\mathbb{Q}}
\newcommand{\Z}{\mathbb{Z}}
\newcommand{\R}{\mathbb{R}}
\newcommand{\C}{\mathbb{C}}
\newcommand{\D}{\mathcal{D}}
\renewcommand{\S}{\mathcal{S}}
\renewcommand{\P}{\mathbb{P}}
\newcommand{\F}{\mathbb{F}}
\newcommand{\E}{\mathbb{E}}

\graphicspath{{Image/}}

\hypersetup{
    colorlinks=true,
    linktoc=all,
    linkcolor=blue
}

\theoremstyle{definition}
\newtheorem*{axiom}{Axiom}
\newtheorem*{claim}{Claim}
\newtheorem*{conv}{Convention}
\newtheorem*{coro}{Corollary}
\newtheorem*{defi}{Definition}
\newtheorem*{eg}{Example}
\newtheorem*{lemma}{Lemma}
\newtheorem*{notation}{Notation}
\newtheorem*{prob}{Problem}
\newtheorem*{post}{Postulate}
\newtheorem*{prop}{Proposition}
\newtheorem*{rem}{Remark}
\newtheorem*{thm}{Theorem}

\DeclareMathOperator{\vdiv}{div}
\DeclareMathOperator{\grad}{grad}
\DeclareMathOperator{\curl}{curl}
\DeclareMathOperator{\Ann}{Ann}
\DeclareMathOperator{\Fit}{Fit}
\DeclareMathOperator{\Diag}{Diag}
\DeclareMathOperator{\tr}{tr}
\DeclareMathOperator{\im}{im}
\DeclareMathOperator{\Mat}{Mat}
\DeclareMathOperator{\Log}{Log}
\DeclareMathOperator{\Isom}{Isom}
\DeclareMathOperator{\Mesh}{Mesh}
\DeclareMathOperator{\Sym}{Sym}
\DeclareMathOperator{\Aut}{Aut}
\DeclareMathOperator{\cosech}{cosech}
\DeclareMathOperator{\Card}{Card}
\DeclareMathOperator{\Gal}{Gal}


\begin{document}

\title{Model Theory Example Sheet 2}

\author{Xiaojia Rao}

\maketitle

\setcounter{secnumdepth}{0}

\section{Question 3}

For the first part, $\mathcal{N}$ is not connected, so it's not a random graph.

For the second part simply use $\psi(x,y) = \exists z (R(z,x) \wedge R(z,y))$ -- it will be true if $x$ and $y$ are from the same random subgraph by the property of random graphs, and will be false otherwise since the two random subgraphs are disconnected.

\section{Question 7}

(ii) $\implies$ (i) is trivial as $|\mathcal{N}|$ is infinite by definition of saturation.\\
For (i) $\implies$ (ii), suppose otherwise, that $|\phi(\mathcal{N})| < |\mathcal{N}|$. Then let $\bra a_i:i<|\phi(\mathcal{N})|\ket$ enumerate $\phi(\mathcal{N})$, and consider the formula
$$p(x) = \{\phi(x) \wedge (x \neq a_i): i < |\phi(\mathcal{N})|\}$$
this is finitely satisfiable (as $\phi(\mathcal{N})$ is infinite), and has parameters $\phi(\mathcal{N})$ which, by assumption, has cardinality less than $\mathcal{N}$. Since $\mathcal{N}$ is saturated, $\mathcal{N}$ realizes $p(x)$. But this is impossible.

\end{document}
