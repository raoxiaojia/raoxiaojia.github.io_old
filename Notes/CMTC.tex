\documentclass[a4paper]{article}

\setlength{\parindent}{0pt}
\setlength{\parskip}{1em}

\pagestyle{headings}

\usepackage{amssymb}
\usepackage{amsmath}
\usepackage{amsthm}
\usepackage{mathtools}
\usepackage{graphicx}
\usepackage{hyperref}
\usepackage{color}
\usepackage{microtype}
\usepackage{tikz}
\usepackage{pgfplots}
\usepackage{pgfplotstable}

\newcommand{\N}{\mathbb{N}}
\newcommand{\Q}{\mathbb{Q}}
\newcommand{\Z}{\mathbb{Z}}
\newcommand{\R}{\mathbb{R}}
\newcommand{\C}{\mathbb{C}}
\newcommand{\D}{\mathcal{D}}
\renewcommand{\S}{\mathcal{S}}
\renewcommand{\P}{\mathbb{P}}
\newcommand{\F}{\mathbb{F}}
\newcommand{\E}{\mathbb{E}}
\newcommand{\bra}{\langle}
\newcommand{\ket}{\rangle}


\graphicspath{{Image/}}

\hypersetup{
    colorlinks=true,
    linktoc=all,
    linkcolor=blue
}

\theoremstyle{definition}
\newtheorem*{axiom}{Axiom}
\newtheorem*{claim}{Claim}
\newtheorem*{conv}{Convention}
\newtheorem*{coro}{Corollary}
\newtheorem*{defi}{Definition}
\newtheorem*{eg}{Example}
\newtheorem*{lemma}{Lemma}
\newtheorem*{notation}{Notation}
\newtheorem*{prob}{Problem}
\newtheorem*{post}{Postulate}
\newtheorem*{prop}{Proposition}
\newtheorem*{rem}{Remark}
\newtheorem*{thm}{Theorem}

\DeclareMathOperator{\vdiv}{div}
\DeclareMathOperator{\grad}{grad}
\DeclareMathOperator{\curl}{curl}
\DeclareMathOperator{\Ann}{Ann}
\DeclareMathOperator{\Fit}{Fit}
\DeclareMathOperator{\Diag}{Diag}
\DeclareMathOperator{\tr}{tr}
\DeclareMathOperator{\im}{im}
\DeclareMathOperator{\Mat}{Mat}
\DeclareMathOperator{\Log}{Log}
\DeclareMathOperator{\Isom}{Isom}
\DeclareMathOperator{\Mesh}{Mesh}
\DeclareMathOperator{\Sym}{Sym}
\DeclareMathOperator{\Aut}{Aut}
\DeclareMathOperator{\cosech}{cosech}
\DeclareMathOperator{\Card}{Card}
\DeclareMathOperator{\Gal}{Gal}


\setcounter{section}{-1}

\begin{document}

\title{Connection between Model theory and Combinatorics}

\maketitle

\newpage

\tableofcontents

\newpage

\section{Introduction}
---Lecture 1---

Several things we'll look into in this course:\\
$\bullet$ Intro to stability;\\
$\bullet$ stable Ramsey/Erd$\ddot{o}$s-Hajnal;\\
$\bullet$ stable regularity lemma;\\
$\bullet$ independence property;\\
$\bullet$ dividing lines in unstable theories.

\newpage

\section{Introduction to stability}

\subsection{History}

$\bullet$ (definition of first-order language, some examples including $L_{gp},L_{agp},L_{lo}$ and $L_{gr}$)\\
$\bullet$ (definition of $L$-structures, also some examples)\\
$\bullet$ (definition of $L$-formulas, with some examples)\\
$\bullet$ (definition of $L$-sentences)\\
$\bullet$ (definition of $L$-theory)\\
$\bullet$ (definition of models of an $L$-theory)

We use the notation $I_T(\kappa)$ to mean the number of models of $T$ of size $\kappa$ up to isomorphism.

Result by Morley: let $T$ be a conutable theory. If $I_T(\kappa) = 1$ for some uncountable $\kappa$, then $I_T(\kappa) = 1 \forall \kappa$. Examples include the theory of vector spaces over a fixed field, and the theory of algebraically closed fields (ACF).

\subsection{The order property}

\begin{defi}
Let $T$ be a theory, $\mathcal{M} \vDash T$, $k \geq 1$ an integer. A formula $\phi(x,y)$ (for the time being, let it have two free variables) is said to have the $k$-order property ($k$-OP) if there are sequences $a_i,b_j,i,j=1,...,k$ s.t. $\mathcal{M} \vDash \phi(a_i,b_j)$ iff $i \leq j$.\\
A formula $\phi(x,y)$ is said to be $k$-stable if it does not have $k$-OP.
\end{defi}

\begin{eg}
$\bullet$ Consider $Th_{gr}$ and a model $G=\bra V,E\ket$. Then the formula $E(x,y)$ is $k$-stable if $G$ does not contain a half-graph of height $k$ (from the definition it's obvious what it means) as an induced bipartite subgraph. We'll sometimes say the graph $G$ is $k$-stable.\\
$\bullet$ Consider $Th_{agp}$ and a model $\bra G,+,-,A\ket$, where $A$ is a unary relation (so basically specifies a subset of $G$). The formula $\phi(x,y)=A(x+y)$ (i.e. $x+y \in A$) is $k$-stable if $G$ does not contain sequences $a_i,b_j$ of length $k$ s.t. $a_i+b_j \in A$ iff $i \leq j$. We'll sometimes say the set $A$ is $k$-stable.\\
\end{eg}

\begin{lemma}
Let $G$ be an abelian group. If $H \leq G$, then $H$ is $2$-stable.
\begin{proof}
We want to show that $H$ can't have $2$-OP. Suppose there are $a_1,a_2,b_1,b_2 \in G$ s.t. $a_i+b_j \in H$ for $1 \leq i \leq j \leq 2$, i.e. $a_1+b_1,a_1+b_2,a_2+b_2 \in H$, but $a_2+b_1$ not in $H$. But that is not possible because $a_2+b_1 = (a_1+b_1)-(a_1+b_2)+(a_2+b_2)$.
\end{proof}
\end{lemma}

\begin{lemma}
Let $G$ be an abelian group, $H \leq G$, and $U$ a union of $k$ cosets of $H$. Then $U$ is $(k+1)$-stable.
\begin{proof}
Suppose we had $a_1,...,a_{k+1},b_1,...,b_{k+1} \in G$ witnessing $(k+1)$-OP. Then by pigeonhole principle, there exists $1 \leq i < j \leq k+1$ s.t. $a_1+b_i$ and $a_1+b_j$ lie in the same coset of $H$. Then $b_i+H = b_j+H$, and then $a_j+b_i =\underbrace{(a_j+b_j)}_{\in U} + \underbrace{(b_i-b_j)}_{\in H} \in U$.
\end{proof}
\end{lemma}

Exercise. Let $A \subseteq G$ be a Sidon set, i.e. it contains no non-trivial solutions to $x+y=z+w$. Show that $A$ is 3-stable. Are all 3-stable sets Sidon sets?

\end{document}
