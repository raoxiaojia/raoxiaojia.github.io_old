\documentclass[a4paper]{article}

\setlength{\parindent}{0pt}
\setlength{\parskip}{1em}

\pagestyle{headings}

\usepackage{amssymb}
\usepackage{amsmath}
\usepackage{amsthm}
\usepackage{mathtools}
\usepackage{graphicx}
\usepackage{hyperref}
\usepackage{color}
\usepackage{microtype}
\usepackage{tikz}
\usepackage{pgfplots}
\usepackage{pgfplotstable}

\newcommand{\N}{\mathbb{N}}
\newcommand{\Q}{\mathbb{Q}}
\newcommand{\Z}{\mathbb{Z}}
\newcommand{\R}{\mathbb{R}}
\newcommand{\C}{\mathbb{C}}
\newcommand{\D}{\mathcal{D}}
\renewcommand{\S}{\mathcal{S}}
\renewcommand{\P}{\mathbb{P}}
\newcommand{\F}{\mathbb{F}}
\newcommand{\E}{\mathbb{E}}

\graphicspath{{Image/}}

\hypersetup{
    colorlinks=true,
    linktoc=all,
    linkcolor=blue
}

\theoremstyle{definition}
\newtheorem*{axiom}{Axiom}
\newtheorem*{claim}{Claim}
\newtheorem*{conv}{Convention}
\newtheorem*{coro}{Corollary}
\newtheorem*{defi}{Definition}
\newtheorem*{eg}{Example}
\newtheorem*{lemma}{Lemma}
\newtheorem*{notation}{Notation}
\newtheorem*{prob}{Problem}
\newtheorem*{post}{Postulate}
\newtheorem*{prop}{Proposition}
\newtheorem*{rem}{Remark}
\newtheorem*{thm}{Theorem}

\DeclareMathOperator{\vdiv}{div}
\DeclareMathOperator{\grad}{grad}
\DeclareMathOperator{\curl}{curl}
\DeclareMathOperator{\Ann}{Ann}
\DeclareMathOperator{\Fit}{Fit}
\DeclareMathOperator{\Diag}{Diag}
\DeclareMathOperator{\tr}{tr}
\DeclareMathOperator{\im}{im}
\DeclareMathOperator{\Mat}{Mat}
\DeclareMathOperator{\Log}{Log}
\DeclareMathOperator{\Isom}{Isom}
\DeclareMathOperator{\Mesh}{Mesh}
\DeclareMathOperator{\Sym}{Sym}
\DeclareMathOperator{\Aut}{Aut}
\DeclareMathOperator{\cosech}{cosech}
\DeclareMathOperator{\Card}{Card}
\DeclareMathOperator{\Gal}{Gal}


\setcounter{section}{-1}

\begin{document}

\title{Topics in Set Theory Sheet 4}

\author{Xiaojia Rao}

\maketitle

\newpage

\section*{34.}
Fix $A \in M[G]$, a formula $\varphi(x,y,x_1,...,x_n)$ which specifies the function that we want to use for replacement, and fix parameters $a_1,...,a_n \in M[G]$. We want a name for
\[
B:=\{y: M[G] \vDash \exists x \in A \varphi(x,y,a_1,...,a_n)\}
\]
Take a name $\sigma$ for $A$ and some names $\tau_1,...,\tau_n$ for $a_1,...,a_n$. Define a formula
\[
\psi(x,y,x_1,...,x_n,A) := x \in A \wedge \varphi(x,y,x_1,...,x_n)
\]
and now consider the name
\[
\rho := \{(\pi,p): p \Vdash^* \exists x \psi(x,\pi,\tau_1,...,\tau_n,\sigma)\}
\]
We claim that $val(\rho,G) = B$.\\
$\subseteq$: suppose $y \in val(\rho,G)$. So there is $(\pi,p) \in \rho$ with $p \in G$ and $val(\pi,G) = y$. So $p \Vdash^* \pi \in \exists x \psi(x,\pi,\tau_1,...,\tau_n,\sigma)$. By definition, this means that the set
\[
D:=\{r:\exists \mu \in M^\P (r \Vdash^* \psi(\mu,\pi,\tau_1,...,\tau_n,\sigma))\}
\]
is dense below $p$. In particular $G \cap D \neq \phi$, so take some $q \in G \cap D$. $q \in D$ means there exists a name $\mu$ s.t.
\[
q\Vdash^* \psi(\mu,\pi,\tau_1,...,\tau_n,\sigma)
\]
But $q \in G$ as well, so by FT,
\[
M[G] \vDash \psi(\mu,\pi,\tau_1,...,\tau_n,\sigma)
\]
which translates to: there exists $x$ ($:=val(\mu,G)$) that
\[
&M[G] \vDash \psi(x,y,a_1,...,a_n,A)\\
\iff &M[G] \vDash x \in A \wedge \varphi(x,y,x_1,...,x_n)
\]
we can rewrite this as
\[
M[G] \vDash \exists x (x \in A \wedge \varphi(x,y,x_1,...,x_n))
\]
i.e. $y \in B$.\\
$\supseteq$: suppose $y \in B$. So $M[G] \vDash \exists x \psi(x,y,a_1,....a_n,A)$. Now take a name $\pi$ for $y$; so by FT, there exists $p \in G$ s.t. $p \Vdash^* \exists x \psi(x,\pi,\tau_1,...,\tau_n,\sigma)$. But this is exactly the requirement for $(\pi,p) \in \rho$. So $y \in val(\rho,G)$.

\section*{35.}
Fix $A \in M[G]$, and $\phi \not\in A$. We want to prove that there exists $f \in M[G]$ s.t. $f$ is a function from $A \to \cup A$, and $x \in A \implies f(x) \in x$.

Take a name $\sigma$ of $A$. Now $dom(\sigma) \in M$, and $M \vDash AC$, so there exists a choice function $g$ for $dom(\sigma)$. In other words, for each $(\pi,p) \in \sigma$, we have $g(\pi) \in \pi$. 


\end{document}
