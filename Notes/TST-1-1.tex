\documentclass[a4paper]{article}

\setlength{\parindent}{0pt}
\setlength{\parskip}{1em}

\pagestyle{headings}

\usepackage{amssymb}
\usepackage{amsmath}
\usepackage{amsthm}
\usepackage{mathtools}
\usepackage{graphicx}
\usepackage{hyperref}
\usepackage{color}
\usepackage{microtype}
\usepackage{tikz}
\usepackage{pgfplots}
\usepackage{pgfplotstable}

\newcommand{\N}{\mathbb{N}}
\newcommand{\Q}{\mathbb{Q}}
\newcommand{\Z}{\mathbb{Z}}
\newcommand{\R}{\mathbb{R}}
\newcommand{\C}{\mathbb{C}}
\newcommand{\D}{\mathcal{D}}
\renewcommand{\S}{\mathcal{S}}
\renewcommand{\P}{\mathbb{P}}
\newcommand{\F}{\mathbb{F}}
\newcommand{\E}{\mathbb{E}}
\newcommand{\bra}{\langle}
\newcommand{\ket}{\rangle}


\graphicspath{{Image/}}

\hypersetup{
    colorlinks=true,
    linktoc=all,
    linkcolor=blue
}

\theoremstyle{definition}
\newtheorem*{axiom}{Axiom}
\newtheorem*{claim}{Claim}
\newtheorem*{conv}{Convention}
\newtheorem*{coro}{Corollary}
\newtheorem*{defi}{Definition}
\newtheorem*{eg}{Example}
\newtheorem*{lemma}{Lemma}
\newtheorem*{notation}{Notation}
\newtheorem*{prob}{Problem}
\newtheorem*{post}{Postulate}
\newtheorem*{prop}{Proposition}
\newtheorem*{rem}{Remark}
\newtheorem*{thm}{Theorem}

\DeclareMathOperator{\vdiv}{div}
\DeclareMathOperator{\grad}{grad}
\DeclareMathOperator{\curl}{curl}
\DeclareMathOperator{\Ann}{Ann}
\DeclareMathOperator{\Fit}{Fit}
\DeclareMathOperator{\Diag}{Diag}
\DeclareMathOperator{\tr}{tr}
\DeclareMathOperator{\im}{im}
\DeclareMathOperator{\Mat}{Mat}
\DeclareMathOperator{\Log}{Log}
\DeclareMathOperator{\Isom}{Isom}
\DeclareMathOperator{\Mesh}{Mesh}
\DeclareMathOperator{\Sym}{Sym}
\DeclareMathOperator{\Aut}{Aut}
\DeclareMathOperator{\cosech}{cosech}
\DeclareMathOperator{\Card}{Card}
\DeclareMathOperator{\Gal}{Gal}


\usetikzlibrary{positioning}

\setcounter{section}{-1}

\begin{document}

\title{Topics in Set Theory Sheet 4}

\author{Xiaojia Rao}

\maketitle

\newpage

\section*{34.}
Fix $A \in M[G]$, a formula $\varphi(x,y,x_1,...,x_n)$ which specifies the function that we want to use for replacement, and fix parameters $a_1,...,a_n \in M[G]$. We want a name for
\[
B:=\{y: M[G] \vDash \exists x \in A \varphi(x,y,a_1,...,a_n)\}
\]
Take a name $\sigma$ for $A$ and some names $\tau_1,...,\tau_n$ for $a_1,...,a_n$. Define a formula
\[
\psi(x,y,x_1,...,x_n,A) := x \in A \wedge \varphi(x,y,x_1,...,x_n)
\]
and now consider the name
\[
\rho := \{(\pi,p): p \Vdash^* \exists x \psi(x,\pi,\tau_1,...,\tau_n,\sigma)\}
\]
We claim that $val(\rho,G) = B$.\\
$\subseteq$: suppose $y \in val(\rho,G)$. So there is $(\pi,p) \in \rho$ with $p \in G$ and $val(\pi,G) = y$. So $p \Vdash^* \pi \in \exists x \psi(x,\pi,\tau_1,...,\tau_n,\sigma)$. By definition, this means that the set
\[
D:=\{r:\exists \mu \in M^\P (r \Vdash^* \psi(\mu,\pi,\tau_1,...,\tau_n,\sigma))\}
\]
is dense below $p$. \underline{Also $p \in G$, so $G \cap D \neq \phi$},\footnote{On lecture 19 this was just stated as a fact without proof, but I can't see why it's trivial. My proof of this claim: suppose otherwise, that $G \cap D = \phi$. Add all elements $q$ s.t. there's no $r \in D$ s.t. $r \leq q$ to $D$ to form a new set $D'$. Then $D'$ is dense, so $G \cap D' \neq \phi$, so pick $r \in G \cap D'$. Now if $r \not\in D$, then since $G$ is a filter, pick $q \in G$ s.t. $q \leq p,r$. Since $D$ is dense below $p$ and $q \leq p$, we can pick $s \in D$ s.t. $s \in q$. But then $s \leq q \leq r$, and $s \in D$, $r \in D'\setminus D$, contradicting our definition of $D'$.} so take some $q \in G \cap D$. $q \in D$ means there exists a name $\mu$ s.t.
\[
q\Vdash^* \psi(\mu,\pi,\tau_1,...,\tau_n,\sigma)
\]
But $q \in G$ as well, so by FT,
\[
M[G] \vDash \psi(\mu,\pi,\tau_1,...,\tau_n,\sigma)
\]
which translates to: there exists $x$ ($:=val(\mu,G)$) that
\[
&M[G] \vDash \psi(x,y,a_1,...,a_n,A)\\
\iff &M[G] \vDash x \in A \wedge \varphi(x,y,x_1,...,x_n)
\]
we can rewrite this as
\[
M[G] \vDash \exists x (x \in A \wedge \varphi(x,y,x_1,...,x_n))
\]
i.e. $y \in B$.\\
$\supseteq$: suppose $y \in B$. So $M[G] \vDash \exists x \psi(x,y,a_1,....a_n,A)$. Now take a name $\pi$ for $y$; so by FT, there exists $p \in G$ s.t. $p \Vdash^* \exists x \psi(x,\pi,\tau_1,...,\tau_n,\sigma)$. But this is exactly the requirement for $(\pi,p) \in \rho$. So $y \in val(\rho,G)$.

\section*{35.}
Fix $A \in M[G]$, and $\phi \not\in A$. We want to prove that there exists $f \in M[G]$ s.t. $f$ is a function from $A \to \cup A$, and $x \in A \implies f(x) \in x$.

Take a name $\sigma$ of $A$. We'd like a name $\pi$ that satisfies the following:\\
(1) $val(\pi,G)$ is a function $A \to \cup A$;\\
(2) if $(\tau_1,p_1),(\tau_2,p_2) \in \pi$, $p_1,p_2 \in G$, and

\section*{36.}
I'm assuming every instance of '$\P$-generic over $M$' actually means '$\P$-generic filter over $M$' (and similarly for later questions).

$G$ is a $\P$-generic filter:\\
$\bullet$ Let $p \in G$, and $p \leq p'$. Then $i(p) \leq i(p')$. But $H$ is a filter, and $i(p) \in H$, so $i(p') \in H$. So $p' \in G$.\\
$\bullet$ Let $p,p' \in G$; we want to prove that $p,p'$ have a witness of compatibility in $G$. Suppose otherwise, so if $p'' \leq p,p'$ then $i(p'') \not\in H$. But $i(p)$ and $i(p')$ still need 
to have a witness of compatibility in $H$, say $q \in H$; so this $q$ cannot be in the image $i(\P)$.\\
$\bullet$ We've proved that $G$ is a filter; now we prove that $G$ is $\P$-generic over $M$. In fact we prove that it's $\P$-antichain generic over $M$. Let $A \subset \P$ be a maximal antichain, we want to prove that $\exists a \in A (i(a) \in H)$.\\
Suppose otherwise. Write $i(A)$ for the set containing images of element of $A$ under $i$. Then $i(A)$ cannot be an maximal antichain (as $H$ is $\Q$-antichain generic, so every maximal antichain has non-empty intersection with $H$). Obviously by (b), $i(A)$ is still an antichain. So (as $M[H] \vDash AC$) we extend it to a maximal antichain $A'$. Now $A' \cap H \neq \phi$, so take $q \in A' \cap H$. By assumption, $q \not\in i(A)$. Now apply (c), we get a $p \in \P$ s.t. $p' \leq p \implies i(p')$ and $q$ are compatible. But $A$ is a maximal antichain in $\P$, so $p$ has to be compatible with some element of $A$, say $p_a$. So let $p_c \in \P$ witness that $p$ and $p_a$ are compatible; in particular, $p_c \leq p$. By (c), $i(p_c)$ and $q$ are compatible. But $p_c \leq p_a$ and hence $i(p_c) \leq i(p_a)$, so $i(p_a)$ and $q$ are compatible; but both $i(p_a)$ and $q$ are in the antichain $A'$ of $\Q$. Contradiction (see diagram below).

\begin{tikzpicture}[node distance = 2cm]
\node(pa)			{$p_a$};
\node(p)	  [above right=0.5cm and 1.5cm of pa] {$p$};
\node(pc)	  [below right of= pa] {$p_c$};

\node(ipa)	[right=4cm of pa] {$i(p_a)$};
\node(q) [right of =ipa] {$q$};
\node(ipc) [below right of =ipa] {$i(p_c)$};

\draw(pa) -- (pc);
\draw(p)--(pc);
\draw(ipa) -- (ipc) node [midway, fill=white] {$?$};
\draw(q)--(ipc) node [midway, fill=white] {$?$};
\end{tikzpicture}


\end{document}
