\documentclass[a4paper]{article}

\setlength{\parindent}{0pt}
\setlength{\parskip}{1em}

\pagestyle{headings}

\usepackage{amssymb}
\usepackage{amsmath}
\usepackage{amsthm}
\usepackage{mathtools}
\usepackage{graphicx}
\usepackage{hyperref}
\usepackage{color}
\usepackage{microtype}
\usepackage{tikz}
\usepackage{pgfplots}
\usepackage{pgfplotstable}

\newcommand{\N}{\mathbb{N}}
\newcommand{\Q}{\mathbb{Q}}
\newcommand{\Z}{\mathbb{Z}}
\newcommand{\R}{\mathbb{R}}
\newcommand{\C}{\mathbb{C}}
\newcommand{\D}{\mathcal{D}}
\renewcommand{\S}{\mathcal{S}}
\renewcommand{\P}{\mathbb{P}}
\newcommand{\F}{\mathbb{F}}
\newcommand{\E}{\mathbb{E}}

\graphicspath{{Image/}}

\hypersetup{
    colorlinks=true,
    linktoc=all,
    linkcolor=blue
}

\theoremstyle{definition}
\newtheorem*{axiom}{Axiom}
\newtheorem*{claim}{Claim}
\newtheorem*{conv}{Convention}
\newtheorem*{coro}{Corollary}
\newtheorem*{defi}{Definition}
\newtheorem*{eg}{Example}
\newtheorem*{lemma}{Lemma}
\newtheorem*{notation}{Notation}
\newtheorem*{prob}{Problem}
\newtheorem*{post}{Postulate}
\newtheorem*{prop}{Proposition}
\newtheorem*{rem}{Remark}
\newtheorem*{thm}{Theorem}

\DeclareMathOperator{\vdiv}{div}
\DeclareMathOperator{\grad}{grad}
\DeclareMathOperator{\curl}{curl}
\DeclareMathOperator{\Ann}{Ann}
\DeclareMathOperator{\Fit}{Fit}
\DeclareMathOperator{\Diag}{Diag}
\DeclareMathOperator{\tr}{tr}
\DeclareMathOperator{\im}{im}
\DeclareMathOperator{\Mat}{Mat}
\DeclareMathOperator{\Log}{Log}
\DeclareMathOperator{\Isom}{Isom}
\DeclareMathOperator{\Mesh}{Mesh}
\DeclareMathOperator{\Sym}{Sym}
\DeclareMathOperator{\Aut}{Aut}
\DeclareMathOperator{\cosech}{cosech}
\DeclareMathOperator{\Card}{Card}
\DeclareMathOperator{\Gal}{Gal}


\usetikzlibrary{positioning}

\setcounter{section}{-1}

\begin{document}

\title{Topics in Set Theory Sheet 4}

\author{Xiaojia Rao}

\maketitle

\newpage

\section*{34.}
Fix $A \in M[G]$, a formula $\varphi(x,y,x_1,...,x_n)$ which specifies the function that we want to use for replacement, and fix parameters $a_1,...,a_n \in M[G]$. We want a name for
\[
B:=\{y: M[G] \vDash \exists x \in A \varphi(x,y,a_1,...,a_n)\}
\]
Take a name $\sigma$ for $A$ and some names $\tau_1,...,\tau_n$ for $a_1,...,a_n$. Define a formula
\[
\psi(x,y,x_1,...,x_n,A) := x \in A \wedge \varphi(x,y,x_1,...,x_n)
\]
and now consider the name
\[
\rho := \{(\pi,p): p \Vdash^* \exists x \psi(x,\pi,\tau_1,...,\tau_n,\sigma)\}
\]
We claim that $val(\rho,G) = B$.\\
$\subseteq$: suppose $y \in val(\rho,G)$. So there is $(\pi,p) \in \rho$ with $p \in G$ and $val(\pi,G) = y$. So $p \Vdash^* \pi \in \exists x \psi(x,\pi,\tau_1,...,\tau_n,\sigma)$. By definition, this means that the set
\[
D:=\{r:\exists \mu \in M^\P (r \Vdash^* \psi(\mu,\pi,\tau_1,...,\tau_n,\sigma))\}
\]
is dense below $p$. \underline{Also $p \in G$, so $G \cap D \neq \phi$},\footnote{On lecture 19 this was just stated as a fact without proof, but I can't see why it's trivial. My proof of this claim: suppose otherwise, that $G \cap D = \phi$. Add all elements $q$ s.t. there's no $r \in D$ s.t. $r \leq q$ to $D$ to form a new set $D'$. Then $D'$ is dense, so $G \cap D' \neq \phi$, so pick $r \in G \cap D'$. Now if $r \not\in D$, then since $G$ is a filter, pick $q \in G$ s.t. $q \leq p,r$. Since $D$ is dense below $p$ and $q \leq p$, we can pick $s \in D$ s.t. $s \leq q$. But then $s \leq q \leq r$, and $s \in D$, $r \in D'\setminus D$, contradicting our definition of $D'$.} so take some $q \in G \cap D$. $q \in D$ means there exists a name $\mu$ s.t.
\[
q\Vdash^* \psi(\mu,\pi,\tau_1,...,\tau_n,\sigma)
\]
But $q \in G$ as well, so by FT,
\[
M[G] \vDash \psi(\mu,\pi,\tau_1,...,\tau_n,\sigma)
\]
which translates to: there exists $x$ ($:=val(\mu,G)$) that
\[
&M[G] \vDash \psi(x,y,a_1,...,a_n,A)\\
\iff &M[G] \vDash x \in A \wedge \varphi(x,y,x_1,...,x_n)
\]
we can rewrite this as
\[
M[G] \vDash \exists x (x \in A \wedge \varphi(x,y,x_1,...,x_n))
\]
i.e. $y \in B$.\\
$\supseteq$: suppose $y \in B$. So $M[G] \vDash \exists x \psi(x,y,a_1,....a_n,A)$. Now take a name $\pi$ for $y$; so by FT, there exists $p \in G$ s.t. $p \Vdash^* \exists x \psi(x,\pi,\tau_1,...,\tau_n,\sigma)$. But this is exactly the requirement for $(\pi,p) \in \rho$. So $y \in val(\rho,G)$.

\section*{36.}
I'm assuming every instance of '$\P$-generic over $M$' actually means '$\P$-generic filter over $M$' (and similarly for later questions).

Also, in this proof we omit all suffixes on $\leq_\P,\leq_\Q$ because in all places it's quite obvious which $\leq$ we're referring to; we also just say \emph{$p$ and $p'$ are compatible} instead of saying \emph{$p$ and $p'$ are compatible w.r.t. $\P$} for the same reason.

$G$ is a $\P$-generic filter:\\
$\bullet$ Let $p \in G$, and $p \leq p'$. Then $i(p) \leq i(p')$. But $H$ is a filter, and $i(p) \in H$, so $i(p') \in H$. So $p' \in G$.

$\bullet$ Let $p,p' \in G$; we want to prove that $p,p'$ have a witness of compatibility in $G$. Now $i(p),i(p') \in H$, so they have some witnesses of compatibility in $H$. By (b), $p$ and $p'$ are also compatible, but we don't know yet that there's a witness in $G$.\\
Let $q \in H$ s.t. $q \leq i(p),i(p')$. We first claim that the set
\[
B:=\{i(p_c): p_c \leq p,p', p_c \in \P\}
\]
is $\Q$-bar below $q$ (obvious meaning): suppose $q' \leq q$. By (c), take some $p_{q'} \in \P$ s.t. it satisfies the condition in (c). Then $i(p_{q'
})$ and $q'$ are compatible. But $q' \leq q$, so $q' \leq i(p),i(p')$. So $i(p_{q'})$ and $i(p),i(p')$ are both compatible. By (b), $p_{q'}$ and $p,p'$ are both compatible. So take some $p_1 \leq p_{q'},p$, $p_1 \in \P$; but now $p_1 \leq p_{q'}$, so by (c), $i(p_1)$ and $q'$ are still compatible, i.e. $i(p_1)$ and $i(p')$ are still compatible. So $p_1$ and $p'$ are compatible. So take come $p_c \leq p_1,p'$ in $\P$. Now $p_c \leq p,p',p_{q'}$, so $i(p_c) \in B$; and it is compatible with $q'$ by (c).

Now we prove that $B\cap H \neq \phi$ (\emph{again, this might already be trivial, but I can't see why -- see footnote on the previous page}): we know $H$ is a $\Q$-generic filter, so it's a $\Q$-bar generic filter. Suppose $B \cap H = \phi$, so $B$ is not a $\Q$-bar. Add all elements $s \in \Q$ that witnesses this fact to form a new set $B'$, then $B'$ is a $\Q$-bar, so pick $r \in H \cap B'$. Now if $r \not\in B'$, then since $H$ is a filter, pick $s \in H$ s.t. $s \leq r,q$. Since $B$ is $\Q$-bar below $q$ and $s \leq q$, we can pick $t \in B$ s.t. $t$ and $s$ are compatible. But $s \leq r$, so $t$ and $r$ are compatible, contradicting our definition of $B'$.

So $G$ is a filter.

$\bullet$ Now we prove that $G$ is $\P$-generic over $M$. In fact we prove that it's $\P$-antichain generic over $M$. Let $A \subset \P$ be a maximal antichain, we want to prove that $\exists a \in A (i(a) \in H)$.\\
Suppose otherwise. Write $i(A)$ for the set containing images of element of $A$ under $i$. Then $i(A)$ cannot be an maximal antichain (as $H$ is $\Q$-antichain generic, so every maximal antichain has non-empty intersection with $H$). Obviously by (b), $i(A)$ is still an antichain. So (as $M[H] \vDash AC$) we extend it to a maximal antichain $A'$. Now $A' \cap H \neq \phi$, so take $q \in A' \cap H$. By assumption, $q \not\in i(A)$. Now apply (c), we get a $p \in \P$ s.t. $p' \leq p \implies i(p')$ and $q$ are compatible. But $A$ is a maximal antichain in $\P$, so $p$ has to be compatible with some element of $A$, say $p_a$. So let $p_c \in \P$ witness that $p$ and $p_a$ are compatible; in particular, $p_c \leq p$. By (c), $i(p_c)$ and $q$ are compatible. But $p_c \leq p_a$ and hence $i(p_c) \leq i(p_a)$, so $i(p_a)$ and $q$ are compatible; but both $i(p_a)$ and $q$ are in the antichain $A'$ of $\Q$. Contradiction (see diagram below).

\begin{tikzpicture}[node distance = 2cm]
\node(pa)			{$p_a$};
\node(p)	  [above right=0.5cm and 1.5cm of pa] {$p$};
\node(pc)	  [below right of= pa] {$p_c$};

\node(ipa)	[right=4cm of pa] {$i(p_a)$};
\node(q) [right of =ipa] {$q$};
\node(ipc) [below right of =ipa] {$i(p_c)$};

\draw(pa) -- (pc);
\draw(p)--(pc);
\draw(ipa) -- (ipc) node [midway, fill=white] {$?$};
\draw(q)--(ipc) node [midway, fill=white] {$?$};
\end{tikzpicture}

So $G$ is a $\P$-generic filter over $M$.

Now suppose $x \in M[G]$. So pick a name $\sigma$ for it. For each $(\pi,p) \in \sigma$, we \emph{transform} $p$ to $i(p)$, and we do so recursively, i.e. not only for the elements $(\pi,p)$ of $\sigma$, but also for all $(\tau,p') \in \pi$ where $(\pi,p) \in \sigma$, and so on. After this recursive transformation we get a name in $M^\Q$, let's call it $\sigma_q$.

We claim that $val(\sigma_q,H)=x(=val(\sigma,G))$. Prove this by induction on complexity of name $\sigma$: the case $\sigma=\phi$ is trivial. Now suppose $\sigma \neq \phi$. Note that because of our definition of $\sigma_q$, for each $(\pi_q,q) \in \sigma_q$ we can find a corresponding 'preimage' $(\pi,p) \in \sigma$ where $(\pi_q,q)$ is the result of our transformation from $(\pi,p)$. We know two things about them:\\
$\bullet$ $i(p) = q$;\\
$\bullet$ $val(\pi,G) = val(\pi_q,H)$ by induction hypothesis.

So we only need to show that $p \in G$ iff $i(p) \in H$, but fortunately that's exactly how we defined $G$. So done.

\section*{37.}
36(a) is trivial. For 36(c), suppose $q \in \Q$. By 37(c) we have some $p \in \P$ s.t. $i(p) \leq q$. Now if $p' \leq p$ then $i(p') \leq i(p) \leq q$, so in particular $i(p')$ and $q$ are compatible.

Now 37(b) gives that if $i(p)$ and $i(p')$ are compatible then $p$ and $p'$ are compatible (needs care here because \emph{compatible} and $\perp$ are opposite things). So the direction we need to prove in 36(b) is that if $p$ and $p'$ are compatible then so are $i(p)$ and $i(p')$. But this is trivial because $i$ preserves order, so if $p_c$ witnesses the compatibility of $p$ and $p'$ then $i(p_c)$ witnesses the compatibility of $i(p)$ and $i(p')$.

\section*{38.}
Note that in $\T$, $t$ and $t'$ being compatible necessarily means that $t\leq t'$ or $t' \leq t$, as otherwise $t$ and $t'$ must have already differed at some place in the sequence and there's no way to repair it (unlike the case of subsets of $\N$). So in graph theorical sense, if we only count edges between sequences of length that differ by 1, then $\T$ looks like a tree where the root is the empty sequence, and each node has $\N$ many offsprings.

Enumerate all elements in $\Q$ as $(q_i)$ (we can write this since $\Q$ is countable). Now define the required map $i:\T \to \Q$ recursively:\\
$\bullet$ the empty sequence is mapped to $1_\Q$;\\
$\bullet$ suppose a sequence $(s_i)$ of length $n \geq 0$ is mapped to some $q \in \Q$. Now, temporarily, we restrict our enumeration of $(q_i)$ to only those elements that are $\leq q$, but preserving the original order of enumeration; i.e. we get a subsequence of the original enumeration. We call that subsequence $q_i$ temporarily for convenience.\\
Now define recursively the image of $(s_1,...,s_n,j)$: for each $j$, let $k$ be the smallest number s.t. $q_k$ is incompatible with any of the images of $(s_1,...,s_n,i)$ where $i<j$ (we'll prove that this exists later). Now if $q_k$ is incompatible with some other elements that are not yet compatible with any of the images of $(s_1,...,s_n,i)$ where $i<j$ (let's call this condition(*)), then we let the image of $(s_1,...,s_n,j)$ be $q_k$; otherwise, since $\Q$ is splitting, take some $q_{k_1},q_{k_2} \leq q_k$ that are incompatible, then $q_{k_1}$ will have the above property that we desire to make it the image of $(s_1,...,s_n,j)$, and we do so. We see from the construction that our claim of existence above is justified, since when we choose the image of $(s_1,...,s_n,j)$, we always leave some elements that are not yet compatible with any of the previous images out.

It's obvious that this construction will satisfy 37(a) and (b), so we'll just need to prove 37(c). Take any $q \in \Q$. It must be $q_i$ in the enumeration for some $i$. There are three cases:\\
$\bullet$ we picked $q_i$ as the image of some sequence $(s_i)$ in the above construction. In this we already get an image of $\T$ below $q_i$ (which is itself);

$\bullet$ $q_i$ is not chosen in the above construction of picking images for sequences, because doing so would violate our condition (*) in the construction above. But in that case, we have taken some $q_{i_1},q_{i_2} \leq q_i$ that are incompatible, and chosen one of them as an image of some sequence. Therefore we also have an image of some sequence below $q_i$;

$\bullet$ $q_i$ is skipped in the process of picking images for sequence, because it is already compatible to some previous $q_j$ ($j<i$) that has been picked as an image of some sequence. Suppose $(s_n)$ is mapped to that $q_j$. Then when we recursively construct image of extensions of $(s_n)$ and restrict our vision to only those elements that are $\leq q_j$, $q_i$ will now have a lower index. For each time this case happens, the index (after restricted vision) of $q_i$ decreases by at least 1, so this case cannot take place infinitely many times, and we must get one of the above two cases after finitely many steps. So done.

\section*{39.}
We first prove that the composition of two dense embeddings is still a dense embedding. 37(a) and (b) are quite trivial. For 37(c), suppose $i:\P \to \Q$, $j:\Q \to \R$ are dense embeddings. Let $r \in \R$. Then there's some $q \in \Q$ s.t. $j(q) \leq r$. But then there's some $p \in \P$ s.t. $i(p) \leq j(q)$. So $j\circ i(p) \leq r$. So the image of $\P$ under $j \circ i$ is dense in $\R$.

Now consider $\T_3$, the partial order of finite 0-1-2 sequences. We claim that there's no dense embedding from $\T_{bin} \to \T_3$: suppose $i$ is such a dense embedding, and WLOG let $i(\phi_{bin}) = \phi_3$ (meaning of notations should be clear). Now consider what the images of $(0)_{bin}$ and $(1)_{bin}$ under $i$ are. In particular, the first term of those two images can only take two values among $\{0,1,2\}$, so at least one of $(0)_3,(1)_3,(2)_3$ is not compatible with either of the two images; suppose $(2)_3$ is so. But we've argued in q38 that being dense in $\T$ (and similarly, $\T_{bin}$ and $\T_3$) is equivalent to one is $\leq$ than the other. That means there's no element in $\T_{bin}$ that has image below $(2)_3$, so the image of $\T_{bin}$ under $i$ is not dense in $\T_3$. Contradiction.

However, $\T_3$ is still a countable splitting partial order, so by q38, there is a dense embedding $\T \to \T_3$. Combining our previous two results we see that there cannot be a dense embedding from $\T_{bin} \to \T$.

\section*{41.}
Following the same argument in lecture, we could get a similar version of lemma that gets us an approximation $F \in M$ for any function $f$ in $M[G]$ s.t. $|F(x)| < \kappa$ for any $x$. Now suppose $\P$ does not preserve some $\lambda \geq \kappa$. So $M[G] \vDash \exists \gamma < \lambda$ s.t. there's a bijection $f:\gamma \to \lambda$. Apply the lemma to get an $F$, and by surjectivity of $f$ we know $\lambda = \bigcup_{\alpha < \gamma} F(\alpha)$; then we get a similar contradiction that $M \vDash |\lambda| < |\gamma \times \kappa| \leq \lambda$ if $\lambda \geq \kappa$.

\section*{42.}
Consider instead $\P := Fn(\omega,2)$, and consider $f_G:=\bigcup G: \N \to 2$. We interpret this as characteristic function of a subset. Now let
\[
S_G :&= \{n \in \N: f_G(n) = 1\}\\
&=\{n \in \N, \exists p \in G(p(n)=1)\}
\]
Is $S_G$ splitting over $M$?\\
Consider
\[
D_{A,m}:=\{p: \text{ there are } a_1,...,a_m \in A s.t. p(a_1) = p(a_2)=...=p(a_m) = 1\}
\]
and let $\bar{D}_{A,m}$ be the same thing but with 1 replaced by 0. These are both dense (in $M$!).

For $Fn(\omega,\omega)$, one way is to consider the mod 2 function, and use $\bar{f}(n) := f(n) \pmod 2$.

\section*{43.}
Note that $A$ has to play a role here, otherwise this forcing is just the one in the previous question; if $\Q = Fn(\omega,\omega)$ and $f_G$ $\Q$-generic, we say $f$ dominates $M$ if $\forall g \in M$ $\exists N \forall n \geq N g(n) < f(n)$. Obviously if $g$ dominates $M$ then $g$ bounds $M$; we claim that $f_G$ does not dominate $M$.\\
Why is that? Fix $g \in M$. Then consider the set 
\[
D_{g,n}:= \{p: \text{ there are } m_1 < ... < m_n (p(m_i) \leq g(m_i))\}
\]

Let's talk about some nomenclature. The partial order $Fn(\omega,\omega)$ is usually called Cohen forcing, which doesn't actually do anything. The one we actually used was $Fn(\kappa \times \omega,\omega)$ was the forcing to add $\kappa$ many Cohen reals (generic object for Cohen forcing). We've also looked that $Fn(\omega,\kappa)$ which is called the collapse forcing.

Most of the forcings (except one) are named by the person who named them. The forcing in this question is called the Mathias forcing.

In this case, in a pair $(s,A)$, $s$ is the part that matters (defines what function we have), and $A$ is sometimes called the \emph{reservoir set}; it is the rule of how to extend the function when you want to get stronger (specifying the range).

If $G$ is $\P$-generic over $M$, $f_G:\N \to \N$: $f_G(n) = k$ if there is $(s,A) \in G$ s.t. $s(n) = k$.

So following the same convention, we call this Mathias reals (doesn't really matter).

Claim that $f_G$ bounds $M$: consider the dense set
\[
D_{g,n}:=\{(s,A) : \text{ there are } \geq n \text{ points } s.t. s > g\}
\]
To prove this is dense, suppose given $s$. We copy $s_0,...,s_k$, then extend $a_1>g(k+1),...,a_n>g(k+n)$ and call this $s^*$. Then $(s^*,A) \leq (s,A)$ (note $A$ is infinite, so it contains arbitrarily large numbers).

45 is actually hard. Let's get back to this at the end.

\end{document}
