\documentclass[a4paper]{article}

\setlength{\parindent}{0pt}
\setlength{\parskip}{1em}

\pagestyle{headings}

\usepackage{amssymb}
\usepackage{amsmath}
\usepackage{amsthm}
\usepackage{mathtools}
\usepackage{graphicx}
\usepackage{hyperref}
\usepackage{color}
\usepackage{microtype}
\usepackage{tikz}
\usepackage{pgfplots}
\usepackage{pgfplotstable}

\newcommand{\N}{\mathbb{N}}
\newcommand{\Q}{\mathbb{Q}}
\newcommand{\Z}{\mathbb{Z}}
\newcommand{\R}{\mathbb{R}}
\newcommand{\C}{\mathbb{C}}
\newcommand{\D}{\mathcal{D}}
\renewcommand{\S}{\mathcal{S}}
\renewcommand{\P}{\mathbb{P}}
\newcommand{\F}{\mathbb{F}}
\newcommand{\E}{\mathbb{E}}
\newcommand{\bra}{\langle}
\newcommand{\ket}{\rangle}


\graphicspath{{Image/}}

\hypersetup{
    colorlinks=true,
    linktoc=all,
    linkcolor=blue
}

\theoremstyle{definition}
\newtheorem*{axiom}{Axiom}
\newtheorem*{claim}{Claim}
\newtheorem*{conv}{Convention}
\newtheorem*{coro}{Corollary}
\newtheorem*{defi}{Definition}
\newtheorem*{eg}{Example}
\newtheorem*{lemma}{Lemma}
\newtheorem*{notation}{Notation}
\newtheorem*{prob}{Problem}
\newtheorem*{post}{Postulate}
\newtheorem*{prop}{Proposition}
\newtheorem*{rem}{Remark}
\newtheorem*{thm}{Theorem}

\DeclareMathOperator{\vdiv}{div}
\DeclareMathOperator{\grad}{grad}
\DeclareMathOperator{\curl}{curl}
\DeclareMathOperator{\Ann}{Ann}
\DeclareMathOperator{\Fit}{Fit}
\DeclareMathOperator{\Diag}{Diag}
\DeclareMathOperator{\tr}{tr}
\DeclareMathOperator{\im}{im}
\DeclareMathOperator{\Mat}{Mat}
\DeclareMathOperator{\Log}{Log}
\DeclareMathOperator{\Isom}{Isom}
\DeclareMathOperator{\Mesh}{Mesh}
\DeclareMathOperator{\Sym}{Sym}
\DeclareMathOperator{\Aut}{Aut}
\DeclareMathOperator{\cosech}{cosech}
\DeclareMathOperator{\Card}{Card}
\DeclareMathOperator{\Gal}{Gal}


\setcounter{section}{-1}

\begin{document}

\title{Stochastic Financial Models}

\maketitle

\newpage

\tableofcontents

\newpage

\section{Motivation}
An investor needs a certain quantity of a share (or currency, good, etc), however, not right now ($t=0$) but at a later time ($t=1$). The price of the share $S(w)$ at time $t=1$ is random, but already today one has to make calculation with it so there is risk. For example, 500USD $\approx$ 370 GBP today. What about in one year?

Possible solution: purchase a financial derivative such as:\\
$\bullet$ forward contract: right and obligation to buy a share at time $t=1$ for a strike price $K$ specified at time $t=0$. Its value at time $t=1$ should be $H(w) = S(w) - K$ is positive if $S(w) > K$, and negative i	f $S(w) < K$;\\
$\bullet$ call-option: the right, but no obligation, to do the same thing as above. Its value at $t=1$ should be $H(w) = (S(w)-K)^+$, i.e. $S(w)-K$ if that is positive, and 0 otherwise (no obligation to exercise the option).

One question: what is the fair price for such a derivative?

1) Classical approach: Regard payoff $H(w)$ as lottery, modelled by a random variable on $(\Omega,F,\P)$, where $\P$ is the 'objective probability measure'.\\
a) very classical: fair price = expected discounted payoff = $\E[\frac{H}{1+r}]$, where $r$ is the interest rate for funds/loans from $t=0$ to $t=1$.\\
Assumption: both interest rates are the same for large investors.\\
b) classical: subjective assessment of the risk (by the seller of $H$) by utility functions.

2) More modern approach: suppose the primary risk (share) can only be traded in $t=0$ and $t=1$.\\
Hedging strategy: $\theta^1$ = number of shares held between $t=0$ and $t=1$;\\
$\theta^0$ = balance on bank account with interest rate $r$.\\
Here we allow $\theta^1$ to be either positive or negative (i.e. allow short-selling).

Price at $t=0$: $\theta^0+\theta^1 \pi^1 = V_0$, where $\pi^1$ is the price of one share at $t=0$.\\
The value of this portfolio at $t=1$: $\theta^0(1+r) + \theta^1 S(w) = V(w)$.

Requirement: value of derivative = value of strategy, $H(w) = V(w)$ for all $w \in \Omega$.

For example, for forward contract: $S(w)-K = V(w) = \theta^0(1+r)+\theta^1 S(w)$, we should choose $\theta^1 = 1,\theta^0=\frac{-K}{1+r}$, so $V_0 = \pi^1 - \frac{K}{1+r}$. The seller of $H$ has no risk if he uses this strategy(?).

Even more, $\pi(H)=V_0$ is the unique fair price for the forward contract. Any other price $\tilde{\pi} \neq V_0$ would lead to \emph{arbitrage}: a riskless opportunity to make profit, which should be excluded). For example, if $\tilde{\pi} > V_0$, at $t=0$ sell forward for $\tilde{\pi}$ and buy the strategy for $V_0$. Then in $t=1$ deliver share and repay the loan. We gain a pure profit at $t=1$: $(\tilde{\pi}-V_0) (1+r)>0$, i.e. arbitrage.

Questions: how to characterize arbitrage-free market? How to determine fair prices of options and derivatives?

\newpage

\section{Utility and mean variance}

The market is interaction of agents trading goods. Individual agents have preferences over different contingent(?) claims (=specified random payment). Agents' preferences are expressed by an expected utility representation. $Y$ is preferred to $X$ means $\E[U(X)] \leq \E[U(Y)]$ with utility function $U: \R \to [-\infty,\infty)$ which is non-decreasing. We assume $U$ to be concave, in the sense that we expect agents to dislike risks.

\begin{defi} 
(1.1) A function $U:\R \to [-\infty,\infty)$ is \emph{concave} if $\forall p \in [0,1]$, $pU(x) + (1-p) U(y) \leq U(px+(1-p)y)$. Let $P(U) = \{x: U(x)>-\infty\}$.
\end{defi}

\end{document}
