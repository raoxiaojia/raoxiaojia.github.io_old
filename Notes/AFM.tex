\documentclass[a4paper]{article}

\setlength{\parindent}{0pt}
\setlength{\parskip}{1em}

\pagestyle{headings}

\usepackage{amssymb}
\usepackage{amsmath}
\usepackage{amsthm}
\usepackage{mathtools}
\usepackage{graphicx}
\usepackage{hyperref}
\usepackage{color}
\usepackage{microtype}
\usepackage{tikz}
\usepackage{pgfplots}
\usepackage{pgfplotstable}

\newcommand{\N}{\mathbb{N}}
\newcommand{\Q}{\mathbb{Q}}
\newcommand{\Z}{\mathbb{Z}}
\newcommand{\R}{\mathbb{R}}
\newcommand{\C}{\mathbb{C}}
\newcommand{\D}{\mathcal{D}}
\renewcommand{\S}{\mathcal{S}}
\renewcommand{\P}{\mathbb{P}}
\newcommand{\F}{\mathbb{F}}
\newcommand{\E}{\mathbb{E}}
\newcommand{\bra}{\langle}
\newcommand{\ket}{\rangle}


\graphicspath{{Image/}}

\hypersetup{
    colorlinks=true,
    linktoc=all,
    linkcolor=blue
}

\theoremstyle{definition}
\newtheorem*{axiom}{Axiom}
\newtheorem*{claim}{Claim}
\newtheorem*{conv}{Convention}
\newtheorem*{coro}{Corollary}
\newtheorem*{defi}{Definition}
\newtheorem*{eg}{Example}
\newtheorem*{lemma}{Lemma}
\newtheorem*{notation}{Notation}
\newtheorem*{prob}{Problem}
\newtheorem*{post}{Postulate}
\newtheorem*{prop}{Proposition}
\newtheorem*{rem}{Remark}
\newtheorem*{thm}{Theorem}

\DeclareMathOperator{\vdiv}{div}
\DeclareMathOperator{\grad}{grad}
\DeclareMathOperator{\curl}{curl}
\DeclareMathOperator{\Ann}{Ann}
\DeclareMathOperator{\Fit}{Fit}
\DeclareMathOperator{\Diag}{Diag}
\DeclareMathOperator{\tr}{tr}
\DeclareMathOperator{\im}{im}
\DeclareMathOperator{\Mat}{Mat}
\DeclareMathOperator{\Log}{Log}
\DeclareMathOperator{\Isom}{Isom}
\DeclareMathOperator{\Mesh}{Mesh}
\DeclareMathOperator{\Sym}{Sym}
\DeclareMathOperator{\Aut}{Aut}
\DeclareMathOperator{\cosech}{cosech}
\DeclareMathOperator{\Card}{Card}
\DeclareMathOperator{\Gal}{Gal}


\setcounter{section}{-1}

\begin{document}

\title{Advanced Financial Models}

\maketitle

\newpage

\tableofcontents

\newpage

\section{Introduction}

$www.staslab.cam.ac.uk/~mike/AFM/$ for course material. However lecture notes only come after lectures, so taking notes is still necessary..

$m.tehranchi@statslab.cam.ac.uk$

Assumptions for this course:\\
No dividends, zero tick size (continuous), no transaction costs, no short-selling constraints, infinitely divisible assets, no bid-ask spread, infinite market depth, agents have preferences for expected utility.

\newpage

\section{Discrete time models}

We'll assume there are $n$ assets with price $P_t^i$ at time $t$ for asset $i$. Apparently $P_t^i$ is a random variable on some probability space $(\Omega,\mathcal{F},\P)$.

We'll use the notation $P=(P_t^1,...,P_t^n)_{t \geq 0}$ which is a $n$-dimensional stochastic process.

Information available at time $t$ is modelled by a $\sigma$-algebra $\mathcal{F}_t \subseteq \mathcal{F}$.

The assumption will be $\mathcal{F}_s \subseteq \mathcal{F}_t$ for $s \leq t$ (in other words, $\mathcal{F}$ is a filtration):

\begin{defi} (Filtration)\\
A \emph{Filtration} is a collection of $\sigma$-algebra $(\mathcal{F}_t)_{t \geq 0}$ such that $\mathcal{F}_s \subseteq \mathcal{F}_t$ for $s \leq t$.
\end{defi}

We'll assume that $\mathcal{F}_0$ is trivial, i.e. if $A$ is $\mathcal{F}_0$ measure then $\mathbb{P} = 0$ or $1$. As a result $P_0^1,...,P_0^n$ are constants.

We assume that $(P_t)_{t \geq 0}$ is adapted to the filtration, i.e. $P_t$ is $\mathcal{F}_t$-measurable for all $t \geq 0$. Usually we assume the filtration is generated by $P$ itself (so all the information is in the price).

Let $c_t$ be the amount consumed at time $t$ which will be a $\mathcal{F}_t$-measurable scalar, $H_t = (H_t^1,...,H_t^n)$ be the vector of portfolio weights, so $H_t^i$ is the number of shares held at $(t-1,t]$ (remember we are in discrete time), which will be $\mathcal{F}_{t-1}$-measurable. $(c_t)_{t \geq 0}$ is adapted ($\mathcal{F}_t$-measurable), $(H_t)_{t \geq 1}$ is predictable/previsible ($\mathcal{F}_{t-1}$-measurable, $n$-dimensional. 

\begin{defi}
The process $(c,H)$ is self-financing if $H_t\cdot P_t = c_t+H_{t+1} \cdot P_t$ for all $t \geq 1$.
\end{defi}

For example, $X_0$ be the initial wealth, $X_0-c_0$ is the post-consumption wealth ($=H_1 \cdot P_0$), $X_1=H_1 \cdot P_1$ is the pre-consumption wealth at time 1, $X_1-c_1$ is the post consumption wealth (which is $H_2\cdot P_1$. In other words, at any specific time, consumption takes place and then price updates.

Background assumption on behaviour of an agent:\\
Say $c^2$ is preffered to $c^1$ iff $\E U(c_0^1,c_1^1,...) < \E U(c_0^2,c_1^2,...)$ where $U$ is some investor utility function, which is increasing in all $c_i$, concave (so risk-avert)

\begin{defi}
An arbitrage is a self-financing investment-consumption strategy $(c,H)$ such that there exists a non-random time $T>0$ s.t. $c_0 = -H_1 \cdot P_0$, $c_t = (H_t - H_{t+1}) \cdot P_t$ for $1 \leq t \leq T-1$, and $c_T =H_T \cdot P_T$ (in words, with initial wealth $X_0 = 0$ and post-consumption wealth at $T$ $X_T-C_T=0$), that $\P(c_t \geq 0)$ for all $0 \leq t \leq T) = 1$, and $\P(c_t > 0$ for some $0 \leq t \leq T) > 0$.
\end{defi}

Suppose $(c^1,H^1)$ is self-financing with iniital wealth $X_0$, $(c^2,H^2) = (c^1,H^1) + (c,H)$, where $(c,H)$ is an arbitrage. Then $c^2$ is preferred to $c^1$.

The inverstor who believes there is an arbitrage would have no optimal investment-consumption policy.

Even further background assumption: the market is in equillibrium (supply = demand).

\begin{defi}
Given the market model, a martingale deflator is positive adapted process $(Y_t)_{t \geq 0}$ such that $(Y_t \cdot P_t)_{t \geq 0}$ is a martingale.
\end{defi}

\begin{thm} (First fundamental theorem of asset pricing)\\
The market has no arbitrage if and only if there exists a martingale deflator.
\end{thm}

\begin{defi}
Given an $(\Omega,\mathcal{F},\P)$-integrable $X$ ($\E|X| < \infty$), and $\mathcal{G} \subseteq \mathcal{F}$) a sub-$\sigma$-algebra of $\mathcal{F}$, a conditional expectation of $X$ given $\mathcal{G}$ is an integrable $Y$ that is $\mathcal{G}$-measurable and such that $\E(X 1_G) = \E(Y 1_G)$ for all $G \in \mathcal{G}$.
\end{defi}

\begin{thm}
The conditional expectations exist and are unique in the sense that, if $Y^1,Y^2$ are both conditional expecations, then $Y^1=Y^2$ almost surely.
\end{thm}

We'll use the notation $Y = \E(X|\mathcal{G})$.

\begin{eg}
Let $(G_n)_n$ be a partition of $\Omega$, $\mathcal{G} = \sigma({G_n})$. Then $\E(X|\mathcal{G})(\omega) = \frac{\E(X1_{G_n})}{\P(G_n)} = \E(X|G_n)$ if $\omega \in G_n$ and $\P(G_n>0$, or anything else if $\P(G_n) = 0$ is a valid conditional expectation.
\end{eg}

\end{document}
