\documentclass[a4paper]{article}

\setlength{\parindent}{0pt}
\setlength{\parskip}{1em}

\pagestyle{headings}

\usepackage{amssymb}
\usepackage{amsmath}
\usepackage{amsthm}
\usepackage{mathtools}
\usepackage{graphicx}
\usepackage{hyperref}
\usepackage{color}
\usepackage{microtype}
\usepackage{tikz}
\usepackage{pgfplots}
\usepackage{pgfplotstable}

\newcommand{\N}{\mathbb{N}}
\newcommand{\Q}{\mathbb{Q}}
\newcommand{\Z}{\mathbb{Z}}
\newcommand{\R}{\mathbb{R}}
\newcommand{\C}{\mathbb{C}}
\newcommand{\D}{\mathcal{D}}
\renewcommand{\S}{\mathcal{S}}
\renewcommand{\P}{\mathbb{P}}
\newcommand{\F}{\mathbb{F}}
\newcommand{\E}{\mathbb{E}}

\graphicspath{{Image/}}

\hypersetup{
    colorlinks=true,
    linktoc=all,
    linkcolor=blue
}

\theoremstyle{definition}
\newtheorem*{axiom}{Axiom}
\newtheorem*{claim}{Claim}
\newtheorem*{conv}{Convention}
\newtheorem*{coro}{Corollary}
\newtheorem*{defi}{Definition}
\newtheorem*{eg}{Example}
\newtheorem*{lemma}{Lemma}
\newtheorem*{notation}{Notation}
\newtheorem*{prob}{Problem}
\newtheorem*{post}{Postulate}
\newtheorem*{prop}{Proposition}
\newtheorem*{rem}{Remark}
\newtheorem*{thm}{Theorem}

\DeclareMathOperator{\vdiv}{div}
\DeclareMathOperator{\grad}{grad}
\DeclareMathOperator{\curl}{curl}
\DeclareMathOperator{\Ann}{Ann}
\DeclareMathOperator{\Fit}{Fit}
\DeclareMathOperator{\Diag}{Diag}
\DeclareMathOperator{\tr}{tr}
\DeclareMathOperator{\im}{im}
\DeclareMathOperator{\Mat}{Mat}
\DeclareMathOperator{\Log}{Log}
\DeclareMathOperator{\Isom}{Isom}
\DeclareMathOperator{\Mesh}{Mesh}
\DeclareMathOperator{\Sym}{Sym}
\DeclareMathOperator{\Aut}{Aut}
\DeclareMathOperator{\cosech}{cosech}
\DeclareMathOperator{\Card}{Card}
\DeclareMathOperator{\Gal}{Gal}


\setcounter{section}{-1}

\begin{document}

\title{Advanced Financial Models}

\maketitle

\newpage

\tableofcontents

\newpage

\section{Introduction}

$www.staslab.cam.ac.uk/~mike/AFM/$ for course material. However lecture notes only come after lectures, so taking notes is still necessary..

$m.tehranchi@statslab.cam.ac.uk$

Assumptions for this course:\\
No dividends, zero tick size (continuous), no transaction costs, no short-selling constraints, infinitely divisible assets, no bid-ask spread, infinite market depth, agents have preferences for expected utility.

\newpage

\section{Discrete time models}

We'll assume there are $n$ assets with price $P_t^i$ at time $t$ for asset $i$. Apparently $P_t^i$ is a random variable on some probability space $(\Omega,\mathcal{F},\P)$.

We'll use the notation $P=(P_t^1,...,P_t^n)_{t \geq 0}$ which is a $n$-dimensional stochastic process.

Information available at time $t$ is modelled by a $\sigma$-algebra $\mathcal{F}_t \subseteq \mathcal{F}$.

The assumption will be $\mathcal{F}_s \subseteq \mathcal{F}_t$ for $s \leq t$ (in other words, $\mathcal{F}$ is a filtration):

\begin{defi} (Filtration)\\
A \emph{Filtration} is a collection of $\sigma$-algebra $(\mathcal{F}_t)_{t \geq 0}$ such that $\mathcal{F}_s \subseteq \mathcal{F}_t$ for $s \leq t$.
\end{defi}

We'll assume that $\mathcal{F}_0$ is trivial, i.e. if $A$ is $\mathcal{F}_0$ measure then $\mathbb{P} = 0$ or $1$.

\end{document}
