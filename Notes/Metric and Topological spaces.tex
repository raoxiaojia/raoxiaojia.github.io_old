\documentclass[a4paper]{article}

\setlength{\parindent}{0pt}
\setlength{\parskip}{1em}

\pagestyle{headings}

\usepackage{amssymb}
\usepackage{amsmath}
\usepackage{amsthm}
\usepackage{mathtools}
\usepackage{graphicx}
\usepackage{hyperref}
\usepackage{color}
\usepackage{microtype}
\usepackage{tikz}
\usepackage{pgfplots}
\usepackage{pgfplotstable}

\newcommand{\N}{\mathbb{N}}
\newcommand{\Q}{\mathbb{Q}}
\newcommand{\Z}{\mathbb{Z}}
\newcommand{\R}{\mathbb{R}}
\newcommand{\C}{\mathbb{C}}
\newcommand{\D}{\mathcal{D}}
\renewcommand{\S}{\mathcal{S}}
\renewcommand{\P}{\mathbb{P}}
\newcommand{\F}{\mathbb{F}}
\newcommand{\E}{\mathbb{E}}

\graphicspath{{Image/}}

\hypersetup{
    colorlinks=true,
    linktoc=all,
    linkcolor=blue
}

\theoremstyle{definition}
\newtheorem*{axiom}{Axiom}
\newtheorem*{claim}{Claim}
\newtheorem*{conv}{Convention}
\newtheorem*{coro}{Corollary}
\newtheorem*{defi}{Definition}
\newtheorem*{eg}{Example}
\newtheorem*{lemma}{Lemma}
\newtheorem*{notation}{Notation}
\newtheorem*{prob}{Problem}
\newtheorem*{post}{Postulate}
\newtheorem*{prop}{Proposition}
\newtheorem*{rem}{Remark}
\newtheorem*{thm}{Theorem}

\DeclareMathOperator{\vdiv}{div}
\DeclareMathOperator{\grad}{grad}
\DeclareMathOperator{\curl}{curl}
\DeclareMathOperator{\Ann}{Ann}
\DeclareMathOperator{\Fit}{Fit}
\DeclareMathOperator{\Diag}{Diag}
\DeclareMathOperator{\tr}{tr}
\DeclareMathOperator{\im}{im}
\DeclareMathOperator{\Mat}{Mat}
\DeclareMathOperator{\Log}{Log}
\DeclareMathOperator{\Isom}{Isom}
\DeclareMathOperator{\Mesh}{Mesh}
\DeclareMathOperator{\Sym}{Sym}
\DeclareMathOperator{\Aut}{Aut}
\DeclareMathOperator{\cosech}{cosech}
\DeclareMathOperator{\Card}{Card}
\DeclareMathOperator{\Gal}{Gal}


\begin{document}

\title{Metric and Topological spaces}
\date{Easter 2016}

\maketitle

\newpage

\tableofcontents

\newpage

\section{Introduction}
Suppose $\left(x_n\right)$ is a sequence of real numbers. $\left(x_n\right)$ converges to $x\in \R$ if for every $\varepsilon>0$, there exists some $N$ such that $|x_n-x|<\varepsilon$ whenever $n>N$. We know that a function $f:\R \to \R$ is \emph{continuous} if $\left(f\left(x_n\right)\right)\to f\left(x\right)$ whenever $\left(x_n\right) \to x$. But what does it mean for a sequence $\left(x_n\right)$ to converge if $x_n\in X \neq \R$?

\begin{eg}
Let
\begin{equation*}
\begin{aligned}
X=\left\{A\in M_{n\times n} \left(\R\right) | A^T =A\right\}
\end{aligned}
\end{equation*}
and let $f\left(A\right)$ be the largest eigenvalue of $A$. Is $f$ continuous?
\end{eg}

\begin{eg}
Let
\begin{equation*}
\begin{aligned}
&X=\left\{f:\left[0,1\right] \to \R | f  \text{ is continuous} \right\} \\
&F\left(f\right) = f\left(\frac{1}{2}\right)
\end{aligned}
\end{equation*}
\end{eg}
Is $F$ continuous?

\begin{thm} (Intermediate value theorem)\\
If $f:\left[0,1\right]\to \R$ is continuous and $f\left(0\right)\left<0, f\left(1\right)\right>0$, then $\exists x \in \left[0,1\right]$ with $f\left(x\right)=0$.
\end{thm}

\begin{thm} (Maximum value theorem)\\
If $f:\left[0,1\right] \to \R$ is continuous, then $\exists x \in \left[0,1\right]$ such that $f\left(x\right) \geq f\left(y\right)$ for all $y\in\left[0,1\right]$.
\end{thm}

These are both theorems about $\left[0,1\right]$. However, they might not be true if we change to another domain. For example, neither of these theorems hold for $\left[0,1\right] \cap \Q$.

\begin{eg}
Let
\begin{equation*}
\begin{aligned}
f\left(x\right) = x^2-\frac{1}{2}
\end{aligned}
\end{equation*}
Then $f\left(0\right)\left<0, f\left(1\right)\right>0$, but there is no $x\in\left[0,1\right] \cap \Q$ such that $f\left(x\right) = 0$.
\end{eg}

\newpage
\section{Metric spaces}
Suppose $\left(\mathbf{v}_n\right) = \left(\left(x_n,y_n\right)\right)$ is a sequence in $\R^2$. What should it mean for $\left(\mathbf{v}_n\right)$ to converge to $\mathbf{v} = \left(x,y\right)$?\\
A natural thought would be that the sequence ocnverges to $\left(x,y\right)$ if $\left(x_n\right) \to x$ and $\left(y_n\right) \to y$. That is then equivalent to
\begin{equation*}
\begin{aligned}
&\left(x_n-x\right)^2 + \left(y_n-y\right)^2 \to 0 \iff
&|\mathbf{v}_n-\mathbf{v}| \to 0.
\end{aligned}
\end{equation*}

\begin{defi} (Metric space)\\
A \emph{metric space} is a pair $\left(X,d\right)$ where $X$ is a set and $d:X\times X \to \R$, called the \emph{metric}, or \emph{distance function}, that satisfies the following properties:
\begin{equation*}
\begin{aligned}
&\bullet d\left(x,y\right) \geq 0 \forall x,y\in X;\\
&\bullet d\left(x,y\right) = 0 \iff x=y;\\
&\bullet d\left(x,y\right) = d\left(y,x\right);\\
&\bullet d\left(x,z\right) \leq d\left(x,y\right) + d\left(y,z\right).
\end{aligned}
\end{equation*}
\end{defi}

\begin{eg}(Euclidean metric)\\
Let $X=\R^n$, $d\left(\mathbf{x},\mathbf{y}\right) = |\mathbf{x}-\mathbf{y}|$, where
\begin{equation*}
\begin{aligned}
|\mathbf{v}| = \left(\sum_{i=1}^n v_i^2\right)^\frac{1}{2}
\end{aligned}
\end{equation*}
is the \emph{Euclidean metric} on $\R^n$. Note that this satisfies the triangle inequality.
\end{eg}

\begin{eg} (Metric subspace)\\
Suppose $\left(X,d_X\right)$ is a metric space and $Y\subset X$. Then $\left(Y,d_Y\right)$ is a metric space, where $d_Y\left(y_1,y_2\right) = d_X\left(y_1,y_2\right)$, i.e. $d_Y=d_X |_{Y\times Y}$.\\
We say $\left(Y,d_Y\right)$ is a \emph{subspace} of $\left(X,d_X\right)$, or just say $Y$ is a subspace of $X$.
\end{eg}

\begin{eg}
Consider
\begin{equation*}
\begin{aligned}
\left\{A \in M_{n\times n} \left(\R\right) | A^T = A \right\} \subset \R^{n^2}.
\end{aligned}
\end{equation*} 
\end{eg}

\begin{defi} (Convergence)\\
If $\left(X,d\right)$ is a metric space and $\left(x_n\right)$ is a sequence in $X$, we say $\left(x_n\right)$ converges to $x$ if for every $\varepsilon >0$ there exists $N$ such that $d\left(x_n,x\right) < \varepsilon$ whenever $n>N$.
\end{defi}

\begin{prop}
Suppose $\left(x_n\right)$ is a sequence in a metric space $\left(X,d\right)$ that $\left(x_n\right) \to x$ and $\left(x_n\right) \to y$. Then $x=y$.
\begin{proof}
Given $\varepsilon>0$, pick $N_1,N_2$ with
\begin{equation*}
\begin{aligned}
&n>N_1 \implies d\left(x_n,x\right) < \frac{\varepsilon}{2},\\
&n>N_2 \implies d\left(x_n,y\right) < \frac{\varepsilon}{2}
\end{aligned}
\end{equation*}
Then if $n>\max\left\{N_1,N_2\right\}$,
\begin{equation*}
\begin{aligned}
0\leq d\left(x,y\right)\leq d\left(x,x_n\right) + d\left(x_n,y\right) = d\left(x,x_n\right) + d\left(y,x_n\right) < \frac{\varepsilon}{2} + \frac{\varepsilon}{2} = \varepsilon.
\end{aligned}
\end{equation*}
Since $\varepsilon$ is arbitrary, $d\left(x,y\right)=0$. So $x=y$.
\end{proof}
\end{prop}

\begin{defi} (Continuity)\\
If $\left(X,d_X\right)$ and $\left(Y,d_Y\right)$ are metric spaces, then $f:X\to Y$ is \emph{continuous} if $\left(f\left(x_n\right)\right) \to f\left(x\right)$ whenever $\left(x_n\right) \to x$.
\end{defi}

\begin{prop}
If $\left(X,d_X\right), \left(Y,d_Y\right)$ and $\left(Z,d_Z\right)$ are metric spaces, and $f:X\to Y$ and $g:Y\to Z$ are both continuous, then $g \circ f:X\to Z$ is also continuous.
\begin{proof}
Suppose $\left(x_n\right) \to x$ in $X$. Since $f$ is continuous, 
\begin{equation*}
\begin{aligned}
\left(f\left(x_n\right)\right) \to f\left(x\right)
\end{aligned}
\end{equation*}
in $Y$. Since $g$ is also continuous,
\begin{equation*}
\begin{aligned}
\left(g\left(f\left(x_n\right)\right)\right) \to g\left(f\left(x\right)\right)
\end{aligned}
\end{equation*}
in $Z$, i.e.
\begin{equation*}
\begin{aligned}
\left(g\circ f\left(x_n\right)\right) \to g\circ f\left(x\right).
\end{aligned}
\end{equation*}
So $g\circ f$ is continuous.
\end{proof}
\end{prop}

\begin{eg} (Discrete metric)\\
Let $X$ be any set, and let
\begin{equation*}
\begin{aligned}
d\left(x,y\right) = \left\{
\begin{array}{ll}
1 & x\neq y\\
0 & x=y
\end{array}
\right.
\end{aligned}
\end{equation*}
This is a metric space.
\end{eg}

\begin{eg}
Let $X=\R^2$, $\mathbf{x}=\left(x_1,x_2\right)$, $\mathbf{x}'=\left(x_1',x_2'\right)$, and
\begin{equation*}
\begin{aligned}
d\left(\mathbf{x},\mathbf{x}'\right) = |x_1-x_1'| + |x_2-x_2'|.
\end{aligned}
\end{equation*}
Triangle inequality follows by:
\begin{equation*}
\begin{aligned}
d\left(x,z\right) &= |x_1-z_1| + |x_2-x_2| \\
&\leq |x_1-y_1| + |y_1-z_1| + |x_2 - y_2| + |y_2 - z_2| \\
&= d\left(x,y\right) + d\left(y,z\right).
\end{aligned}
\end{equation*}
\end{eg}

\begin{eg} (British Railway metric)\\
Let $X=\R^2$, and
\begin{equation*}
\begin{aligned}
d\left(\mathbf{x},\mathbf{y}\right) = \left\{
\begin{array}{ll}
|\mathbf{x}-\mathbf{y}| &  \mathbf{x}=k\mathbf{y}\\
|\mathbf{x}| + |\mathbf{y}| & \text{ otherwise}
\end{array}
\right.
\end{aligned}
\end{equation*}
(Unless you, your destination and London are collinear, you always have to take a train to London first and transfer there.)
\end{eg}

\begin{defi}
If $V$ is a real vector space, a \emph{norm} on $V$ is a map $||\cdot|| : V \to \R$ satisfying
\begin{equation*}
\begin{aligned}
&\bullet ||\mathbf{v}|| \geq 0 \forall \mathbf{v}\in V;\\
&\bullet ||\mathbf{v}|| = 0 \iff \mathbf{v}=\mathbf{0} \in V;\\
&\bullet ||\lambda \mathbf{v}|| = |\lambda| ||\mathbf{v}||, \lambda \in \R, \mathbf{v} \in V;\\
&\bullet ||\mathbf{v}+\mathbf{w}|| \leq ||\mathbf{v}|| + ||\mathbf{w}||.
\end{aligned}
\end{equation*}
\end{defi}

\begin{eg}
Let $V = \R^n$,
\begin{equation*}
\begin{aligned}
||\mathbf{v}|| = \sum_{i=1}^n |v_i|
\end{aligned}
\end{equation*}
(this is called the Manhattan norm).
\end{eg}

\begin{lemma}
If $||\cdot||$ is a norm on $V$, then 
\begin{equation*}
\begin{aligned}
d\left(\mathbf{x},\mathbf{y}\right) = ||\mathbf{x}-\mathbf{y}||
\end{aligned}
\end{equation*}
defines a metric on $V$.
\begin{proof}
Firstly
\begin{equation*}
\begin{aligned}
d\left(\mathbf{x},\mathbf{y}\right) = ||\mathbf{x}-\mathbf{y}|| \geq 0
\end{aligned}
\end{equation*}
by our definition of norm. The second holds by
\begin{equation*}
\begin{aligned}
d\left(\mathbf{x},\mathbf{y}\right) = 0 &\iff ||\mathbf{x}-\mathbf{y}|| = 0\\
&\iff \mathbf{x}-\mathbf{y}=0\\
&\iff \mathbf{x} = \mathbf{y}.
\end{aligned}
\end{equation*}
For the third condition,
\begin{equation*}
\begin{aligned}
d\left(\mathbf{y}-\mathbf{x}\right) &= ||\mathbf{y}-\mathbf{x}|| \\
&= ||\left(-1\right)\left(\mathbf{x}-\mathbf{y}\right)||\\
&= |-1| ||\mathbf{x}-\mathbf{y}||\\
&= d\left(\mathbf{x},\mathbf{y}\right)
\end{aligned}
\end{equation*}
Lastly
\begin{equation*}
\begin{aligned}
d\left(\mathbf{x},\mathbf{y}\right) + d\left(\mathbf{y},\mathbf{z}\right) &= ||\mathbf{x}-\mathbf{y}|| + ||\mathbf{y}-\mathbf{z}||\\
&\geq \mathbf{x}-\mathbf{z}\\
&= d\left(\mathbf{x},\mathbf{z}\right)
\end{aligned}
\end{equation*}
\end{proof}
\end{lemma}

The Manhattan metric on $\R^2$ is induced by the Manhattan norm (exercise).

\begin{defi}
Let $V$ be a real vector space. An \emph{inner product} on $V$ is a map $\left<\cdot, \cdot\right>: V\times V \to \R$, satisfying
\begin{equation*}
\begin{aligned}
&\bullet \left<\mathbf{v},\mathbf{v}\right> \geq 0 \forall \mathbf{v}\in V;\\
&\bullet \left<\mathbf{v},\mathbf{v}\right> = 0 \iff \mathbf{v}=\mathbf{0};\\
&\bullet \left<\mathbf{v},\mathbf{w}\right> = \left<\mathbf{w},\mathbf{v}\right>;\\
&\bullet \left<\mathbf{v},\mathbf{w_1}+\lambda \mathbf{w_2}\right> = \left<\mathbf{v},\mathbf{w_1}\right> + \lambda\left<\mathbf{v},\mathbf{w_2}\right>.
\end{aligned}
\end{equation*}
\end{defi}

The standard inner product on $\R^n$ is given by
\begin{equation*}
\begin{aligned}
\left<\mathbf{v},\mathbf{w}\right> = \mathbf{v}\cdot\mathbf{w}=\sum_{i=1}^n v_i w_i
\end{aligned}
\end{equation*}

\begin{thm} (Cauchy-Schwartz inequality)\\
If $\left<\cdot,\cdot\right>$ is an inner product on $V$, then
\begin{equation*}
\begin{aligned}
\left<\mathbf{v},\mathbf{w}\right>^2 \leq \left<\mathbf{v},\mathbf{v}\right>\left<\mathbf{w},\mathbf{w}\right>.
\end{aligned}
\end{equation*}
\begin{proof}
Consider $\left<\mathbf{v}+\lambda\mathbf{w},\mathbf{v}+\lambda\mathbf{w}\right>\geq 0$ by definition. So
\begin{equation*}
\begin{aligned}
\left<\mathbf{v},\mathbf{v}\right> + 2\lambda\left<\mathbf{v},\mathbf{w}\right>+\lambda^2\left<\mathbf{w},\mathbf{w}\right> \geq 0
\end{aligned}
\end{equation*}
Now if $ax^2 + bx + c\geq 0$ for all $x\in \R$, then $ax^2+bx+c=0$ has at most one real root. So $b^2-4ac \leq 0$.\\
Now we've just seen $a\lambda^2+b\lambda+c\geq 0$ for all $\lambda\in\R$, where $a=\left<\mathbf{w},\mathbf{w}\right>$, $b=2\left<\mathbf{v},\mathbf{w}\right>$, $c=\left<\mathbf{v},\mathbf{v}\right>$. So
\begin{equation*}
\begin{aligned}
4\left<\mathbf{v},\mathbf{w}\right>^2 \leq 4\left<\mathbf{w},\mathbf{w}\right>\left<\mathbf{v},\mathbf{v}\right>
\end{aligned}
\end{equation*}
which is the desired result.
\end{proof}
\end{thm}

\begin{coro}
If $\left<,\right>$ is an inner product on $V$, then $||\mathbf{v}|| = \left(\left<\mathbf{v},\mathbf{v}\right>\right)^{\frac{1}{2}}$ is a norm on $V$.
For the proof we just need to verify all the conditions again. Cauchy-Schwartz inequality is required for verifying the fourth condition.
\end{coro}

Another exercise: The norm on $\R^n$ induced by the standard inner product is Euclidean length:
\begin{equation*}
\begin{aligned}
||\mathbf{v}|| = |\mathbf{v}| = \left(\sum_{i=1}^n v_i^2\right)^\frac{1}{2}
\end{aligned}
\end{equation*}
The metric induced by the norm is the Euclidean norm.\\
And another exercise: is the Manhattan norm induced by an inner product on $\R^n$?

\begin{eg}
Let $V=\R^n$.\\
1) 
\begin{equation*}
\begin{aligned}
||\mathbf{v}||_1=\sum_{i=1}^n |v_i|
\end{aligned}
\end{equation*}
The Manhattan norm;\\
2)
\begin{equation*}
\begin{aligned}
||\mathbf{v}||_2 = \left(\sum_{i=1}^n v_i^2\right)^\frac{1}{2}
\end{aligned}
\end{equation*}
The Euclidean norm;\\
3)
\begin{equation*}
\begin{aligned}
||\mathbf{v}||_\infty = \max_{1\leq i \leq n |v_i|}
\end{aligned}
\end{equation*}
(In example sheet:
\begin{equation*}
\begin{aligned}
||\mathbf{v}||_p = \left(\sum_{i=1}^n |v_i|^p \right)^\frac{1}{p}
\end{aligned}
\end{equation*}
is a norm for all $1\leq p < \infty$.)
\end{eg}

\begin{eg}
Let
\begin{equation*}
\begin{aligned}
V=C\left[0,1\right] := \left\{ f:\left[0,1\right] \to \R | f \text{ is continuous} \right\}
\end{aligned}
\end{equation*}
where
\begin{equation*}
\begin{aligned}
&\left(\lambda f\right)\left(x\right) = \lambda f\left(x\right),\\
&\left(f+g\right) \left(x\right) = f\left(x\right) + g\left(x\right)
\end{aligned}
\end{equation*}
\end{eg}
1)
\begin{equation*}
\begin{aligned}
||f||_1 = \int_0^1 |f\left(x\right)| dx
\end{aligned}
\end{equation*}
2)
\begin{equation*}
\begin{aligned}
||f||_2 = \left(\int_0^1 f\left(x\right)^2\right)^\frac{1}{2} dx
\end{aligned}
\end{equation*}
3)
\begin{equation*}
\begin{aligned}
||f||_\infty = \max_{x \in \left[0,1\right]} |f\left(x\right)|
\end{aligned}
\end{equation*}
This is usually called the 'sup norm'.\\
How do we know that the first two norms satisfy the second condition, i.e. the norm is zero if and only if the function is zero? This is actually a standard result in analysis:

\begin{lemma}
Suppose $f\left[0,1\right]\to\R$ is continuous. If $f \not\cong 0$, then
\begin{equation*}
\begin{aligned}
\int_0^1 |f\left(x\right)|>0
\end{aligned}
\end{equation*}
\begin{proof}
Pick $x_0\in\left[0,1\right]$ with $f\left(x_0\right) \neq 0.$ Since $f$ is continuous, there exists $\delta > 0$ such that
\begin{equation*}
\begin{aligned}
|f\left(x\right) - f\left(x_0\right)| < \frac{|f\left(x_0\right)|}{2}
\end{aligned}
\end{equation*}
whenever $|x-x_0| < \delta$. Then for $|x-x_0|<\delta$, we have
\begin{equation*}
\begin{aligned}
|f\left(x\right)| > |f\left(x_0\right)| - \frac{|f\left(x_0\right)|}{2} \geq \frac{|f\left(x_0\right)|}{2}
\end{aligned}
\end{equation*}
So if we define
\begin{equation*}
\begin{aligned}
g\left(x\right) = \left\{
\begin{array}{ll}
\frac{|f\left(x_0\right)|}{2} & |x-x_0| < \delta\\
0 & \text{  otherwise}
\end{array}
\right.
\end{aligned}
\end{equation*}
Then $|f\left(x\right)| \geq g\left(x\right)$ for all $x\in\left[0,1\right]$.\\
So
\begin{equation*}
\begin{aligned}
\int_0^1 |f\left(x\right)| dx \geq \int_0^1 |g\left(x\right)| dx = 2\delta \frac{|f\left(x_0\right)|}{2} = \delta |f\left(x_0\right)|>0.
\end{aligned}
\end{equation*}
\end{proof}
\end{lemma}

\begin{eg}
\begin{equation*}
\begin{aligned}
f_n\left(x\right) = \left\{ 
\begin{array}{ll}
1-nx & x\in\left[0,\frac{1}{n}\right]\\
0 & x\in\left[\frac{1}{n},1\right]
\end{array}
\right.
\end{aligned}
\end{equation*}
Then
\begin{equation*}
\begin{aligned}
||f_n||_1 = \int_0^1 |f\left(x\right)| dx = \frac{1}{2}\left(\frac{1}{n}\right)\cdot 1 = \frac{1}{2n} \to 0
\end{aligned}
\end{equation*}
as $n\to \infty$.\\
So $\left(f_n\right) \to 0$ with respect to the metric induced by $|| \cdot ||_1$.\\
On the other hand,
\begin{equation*}
\begin{aligned}
||f_n||_\infty = 1
\end{aligned}
\end{equation*}
regardless of $n$, since $f\left(0\right)=1$.\\
This means that $\left(f_n\right)$ does not converge to 0 as $n\to \infty$ with respect to the metric induced by $||\cdot||_\infty$, i.e. the sup norm.
\end{eg}

Exercise: is it possible to find a sequence that converges with respect to $||\cdot||_1$ on $\R^2$, but not with respect to $||\cdot||_\infty$?

\begin{eg} (The \emph{p-adic} norm)\\
Let $p \in \N$ be a prime number. If $a \in \Z$, let $|a|_p = p^{-k}$, where $p^k | a$, but $p^{k+1} \nmid a$.\\
So
\begin{equation*}
\begin{aligned}
&|0|_p = 0,\\
&|ab|_p = |a|_p |b|_p,\\
&|a+b|_p \leq \max\left\{|a|_p, |b|_p\right\} \leq |a|_p + |b|_p
\end{aligned}
\end{equation*}

If $r=\frac{a}{b} \in \Q$, define
\begin{equation*}
\begin{aligned}
|r|_p  = \frac{|a|_p}{|b|_p}
\end{aligned}
\end{equation*}
Then the \emph{p-adic metric} on $\Q$ is defined by
\begin{equation*}
\begin{aligned}
d_p \left(r,s\right) = |r-s|_p
\end{aligned}
\end{equation*}
\end{eg}

Exercise: 1,2,4,8,16, ... $\to 0$ with respect to $d_2$.

\newpage
\section{Open and Closed Subsets}
Let $\left(X,d\right)$ be a metric space.

\begin{defi}
If $x\in X$ and $r\in \R$ with $r>0$, let
\begin{equation*}
\begin{aligned}
B_r\left(x\right) = \left\{ y\in X | d\left(x,y\right) < r \right\}
\end{aligned}
\end{equation*}
is the \emph{open ball} of radius $r$ with centre $x$,
\begin{equation*}
\begin{aligned}
\bar{B}_r \left(x\right) = \left\{y \in X | d\left(x,y\right) \leq r \right\}
\end{aligned}
\end{equation*}
is the \emph{closed ball} of radius $r$ with centre $x$.
\end{defi}

\begin{eg}
Let $X=\R$ with the usual metric. Then
\begin{equation*}
\begin{aligned}
&B_r\left(a\right) = \left(a-r,a+r\right),\\
&\bar{B}_r\left(a\right) = \left[a-r,a+r\right].
\end{aligned}
\end{equation*}
These are just the open and closed interval respectively.
\end{eg}

\begin{eg}
Let $X=\R^3$, with the Euclidean metric. Then
\begin{equation*}
\begin{aligned}
\bar{B}_1\left(0\right) = \left\{ \mathbf{v}\in \R^3 | |\mathbf{v}| \leq 1\right\}
\end{aligned}
\end{equation*}
which is a sphere with centre 0 and radius 1.
\end{eg}

\begin{eg}
Let $X=\R^2$, with the metric induced by\\ 1) the Manhattan norm, $||\cdot ||_1$;\\
2) the Euclidean norm, $||\cdot ||_2$;\\
3) the sup norm, $||\cdot ||_\infty$.\\
Now consider $\bar{B}_1\left(0\right)$. For the Manhattan norm, this is a square with with centre 0 and edges parallel to the axes(so with edge length 2); for the Euclidean norm this is a disc with centre 0 and radius 1; for the sup norm this is a square with centre 0 but edges making an angle of $\frac{\pi}{4}$ to the axes (so with edge length $\sqrt{2}$).
\end{eg}

\begin{eg}
Let $X=\R^2$, $d$ is induced by an inner product $\left<,\right>$: if $\mathbf{v}=x\mathbf{e}_1+y\mathbf{e}_2$, then
\begin{equation*}
\begin{aligned}
\left<\mathbf{v},\mathbf{v}\right> = ax^2 + 2bxy + cy^2
\end{aligned}
\end{equation*}
Then
\begin{equation*}
\begin{aligned}
B_1\left(0\right) = \left\{\left(x,y\right) | ax^2+2bxy+cy^2 \leq 1\right\}
\end{aligned}
\end{equation*}
where
\begin{equation*}
\begin{aligned}
a=\left<\mathbf{e}_1,\mathbf{e}_1\right>,\\
b=\left<\mathbf{e}_1,\mathbf{e}_2\right>,\\
c=\left<\mathbf{e}_2,\mathbf{e}_2\right>.
\end{aligned}
\end{equation*}
This is an ellipse.\\
So we know that $||\cdot ||_\infty$, $||\cdot ||_1$ are not induced from an inner product on $\R^2$.
\end{eg}

\begin{notation}
If $A\subset X$ is a subset, then $X\backslash A$, or $X-A$, is defined as
\begin{equation*}
\begin{aligned}
X-A = \left\{x\in X | x\not\in A\right\}
\end{aligned}
\end{equation*}
which is called the complement of $A$ (in $X$).
\end{notation}

\begin{defi} (Open and closed subset)\\
If $A\subset X$, we say $A$ is an \emph{open subset} if for every $x\in A$, $\exists r>0$ such that $B_r\left(x\right) \subset A$.\\
If $C\subset X$, we say $C$ is a \emph{closed subset} if $X-C$ is an open subset of $X$.
\end{defi}

\begin{prop}
If $X$ is a metric space, then $B_r\left(x\right)$ is an open subset of $X$, and $\bar{B}_r\left(x\right)$ is a closed subset of $X$.
\begin{proof}
Suppose $b\in B_r\left(a\right)$. Then $d\left(a,b\right) < r$. So $r_0 = r-d\left(a,b\right)>0$. Now if $c\in B_{r_0}\left(b\right)$, then
\begin{equation*}
\begin{aligned}
d\left(a,c\right) \leq d\left(a,b\right)+d\left(b,c\right)<d\left(a,b\right)+r_0=r
\end{aligned}
\end{equation*}
So $c\in B_r\left(a\right)$. Thus $B_{r_0}\left(b\right) \subset B_r\left(a\right)$. So $B_r\left(a\right)$ is open.\\
For the second part, we need to show that
\begin{equation*}
\begin{aligned}
X-\bar{B}_r\left(a\right) = \left\{ x\in X | d\left(x,a\right)>r\right\}
\end{aligned}
\end{equation*}
is open. The proof is analogous to the previous proof.
\end{proof}
\end{prop}

\begin{rem}
Note that $A\subset X$ may be \emph{neither} open nor closed. In fact if we take a random subset from a random metric space $X$, this is most likely the case. For example, consider $X=\R$ with the usual metric. then an obvious example is $\left[0,1\right)$,which is neither open nor closed; $\Q \subset \R$ is a more subtle example.
\end{rem}

\begin{rem}
$A\subset X$ can also be \emph{both} open and closed. Consider
\begin{equation*}
\begin{aligned}
X=\left[-1,1\right]-\left\{0\right\}
\end{aligned}
\end{equation*}
as a subspace of $\R$. Then
\begin{equation*}
\begin{aligned}
A=\left[-1,0\right)=B_1\left(-1\right) \subset X
\end{aligned}
\end{equation*}
is open. On the other hand, this is also a closed ball, since it is also
\begin{equation*}
\begin{aligned}
\bar{B}_{\frac{1}{2}} \left(-\frac{1}{2}\right) \subset X
\end{aligned}
\end{equation*}
So $A$ is both closed and open in $X$.
\end{rem}

\begin{rem}
Note that being open or closed is \emph{not} a property of a space, but a property of a subset of a space.\\
We've seen that $\left[-1,0\right)$ is not a closed subset of $\R$, but it \emph{is} a closed subset of $\left[-1,1\right] - \left\{0\right\}$.
\end{rem}

\begin{defi}
If $\left(X,d\right)$ is a metric space and $x\in X$, then $U$ is an \emph{open neighbourhood} of $x$ if $U\subset X$ is open and $x\in U$.
\end{defi}

\begin{prop}
Suppose sequence $\left(x_n\right) \to x$ and $U$ is an open neighbourhood of $x$. Then there exists $N$ such that $x_n\in U$ for all $n>N$.
\begin{proof}
Since $U\subset X$ is open, $x\in U$, $\exists \varepsilon > 0$ with
\begin{equation*}
\begin{aligned}
B_\varepsilon \left(x\right) \subset U.
\end{aligned}
\end{equation*}
Since $\left(x_n\right) \to x$, $\exists N$ such that $d\left(x_n,x\right) < \varepsilon$ whenever $n>N$. This implies that
\begin{equation*}
\begin{aligned}
x_n \in B_\varepsilon \left(x\right)
\end{aligned}
\end{equation*}
whenever $n>N$. So $x_n\in U$ whenever $n>N$.
\end{proof}
\end{prop}

\begin{defi} (Limit point)\\
If $\left(X,d\right)$ is a metric space and $A \subset X$, we say $x\in X$ is a \emph{limit point} of $A$ if there is a sequence $\left(x_n\right) \to x$ with $x_n \in A$ for all $n$.
\end{defi}

\begin{eg}
If $a\in A$, then $a$ is a limit point of $A$. We can just take a constant sequence with value $a$, which clearly converges to $a$.
\end{eg}

\begin{eg}
Let $X = \R$ with the usual metric, $A=\left(0,1\right)$. Then
\begin{equation*}
\begin{aligned}
\left(\frac{1}{n}\right) \to 0
\end{aligned}
\end{equation*}
and $\frac{1}{n} \in A$ for all $n\geq 1$. So $0$ is a limit point of $A$.
\end{eg}

\begin{prop}
Suppose $C\subset X$ is a closed subset and $x$ is a limit point of $C$. Then $x \in C$.
\begin{proof}
Suppose $\left(x_n\right) \to x$, where $x_n \in C$, but $x\not\in C$.\\
Since $C$ is closed, $U=X-C$ is open. So
\begin{equation*}
\begin{aligned}
x \not\in C \implies X \in U
\end{aligned}
\end{equation*}
So $U$ is an open neighbourhood of $x$.\\
Now $\left(x_n\right)$ converges to $x$ and $U$ is an open neighbourhood. So by the previous proposition, there exists an $N$ such that $x_n \in U$ for all $n>N$. But $x_n \in C$ for all $n$. Contradiction.
\end{proof}
\end{prop}

\begin{thm}
Suppose $f:\left(X,d_X\right) \to \left(Y,d_Y\right)$. Then the following conditions are equivalent:\\
1) $f$ is continuous;\\
2) $f^{-1}\left(U\right)$ is open in $X$ whenever $U$ is open in $Y$;\\
3) $f^{-1}\left(C\right)$ is closed in $X$ whenever $U$ is closed in $Y$.
\end{thm}


\end{document}