\documentclass[a4paper]{article}

\setlength{\parindent}{0pt}
\setlength{\parskip}{1em}

\pagestyle{headings}

\usepackage{amssymb}
\usepackage{amsmath}
\usepackage{amsthm}
\usepackage{mathtools}
\usepackage{graphicx}
\usepackage{hyperref}
\usepackage{color}
\usepackage{microtype}
\usepackage{tikz}
\usepackage{pgfplots}
\usepackage{pgfplotstable}

\newcommand{\N}{\mathbb{N}}
\newcommand{\Q}{\mathbb{Q}}
\newcommand{\Z}{\mathbb{Z}}
\newcommand{\R}{\mathbb{R}}
\newcommand{\C}{\mathbb{C}}
\newcommand{\D}{\mathcal{D}}
\renewcommand{\S}{\mathcal{S}}
\renewcommand{\P}{\mathbb{P}}
\newcommand{\F}{\mathbb{F}}
\newcommand{\E}{\mathbb{E}}
\newcommand{\bra}{\langle}
\newcommand{\ket}{\rangle}


\graphicspath{{Image/}}

\hypersetup{
    colorlinks=true,
    linktoc=all,
    linkcolor=blue
}

\theoremstyle{definition}
\newtheorem*{axiom}{Axiom}
\newtheorem*{claim}{Claim}
\newtheorem*{conv}{Convention}
\newtheorem*{coro}{Corollary}
\newtheorem*{defi}{Definition}
\newtheorem*{eg}{Example}
\newtheorem*{lemma}{Lemma}
\newtheorem*{notation}{Notation}
\newtheorem*{prob}{Problem}
\newtheorem*{post}{Postulate}
\newtheorem*{prop}{Proposition}
\newtheorem*{rem}{Remark}
\newtheorem*{thm}{Theorem}

\DeclareMathOperator{\vdiv}{div}
\DeclareMathOperator{\grad}{grad}
\DeclareMathOperator{\curl}{curl}
\DeclareMathOperator{\Ann}{Ann}
\DeclareMathOperator{\Fit}{Fit}
\DeclareMathOperator{\Diag}{Diag}
\DeclareMathOperator{\tr}{tr}
\DeclareMathOperator{\im}{im}
\DeclareMathOperator{\Mat}{Mat}
\DeclareMathOperator{\Log}{Log}
\DeclareMathOperator{\Isom}{Isom}
\DeclareMathOperator{\Mesh}{Mesh}
\DeclareMathOperator{\Sym}{Sym}
\DeclareMathOperator{\Aut}{Aut}
\DeclareMathOperator{\cosech}{cosech}
\DeclareMathOperator{\Card}{Card}
\DeclareMathOperator{\Gal}{Gal}


\begin{document}

\title{Analysis}
\date{Lent 2016}

\maketitle

\newpage

\tableofcontents

\newpage

\section{Differentiation}
\begin{thm}(IFT)
Assume $I$ is an open interval, $f:I\to \R$ is differentiable on $I$, $f'\left(x\right)\neq 0 \forall x\in I$. Then $J=f\left(I\right)$ is an open interval, $f$ is strictly monotonic, and hence bijection $I\to J$. Moreover, $f^{-1}:J\to I$ is differentiable, and
\begin{equation*}
\left(f^{-1}\right)'\left(y\right)=\frac{1}{f'\left(f^{-1}\left(y\right)\right)}.
\end{equation*}
\begin{proof}
\underline{$f$ is injective}: if $x<y$ and $f\left(x\right)=f\left(y\right)$. Then by Rolle's theorem ($f$ differentiable and hence continuous on $I$),$\exists c\in \left(x,y\right)$s.t. $f'\left(c\right)=0$. Contradiction.\\
\underline{$f$ is strictly monotonic}: $\forall x<y$ in $I$, either $f\left(x\right)<f\left(y\right)$ or $f\left(x\right)>f\left(y\right)$. We next show that $\forall a<b<c$ in $I$ either $f\left(a\right)<f\left(b\right)<f\left(c\right)$ or $f\left(a\right)>f\left(b\right)>f\left(c\right)$. If not then $\exists a<b<c$ in $I$ s.t.\\
either $f\left(a\right)<f\left(b\right), f\left(b\right)>f\left(c\right)$\\
or $f\left(a\right)>f\left(b\right), f\left(b\right)<f\left(c\right)$.\\
In the first case, fix $w$ s.t. $\max\left(f\left(a\right),f\left(c\right)\right)<w<f\left(b\right)$. Then $f\left(a\right)<w<f\left(b\right)$ so by IVT, $\exists x\in \left(a,b\right)$ s.t. $f\left(x\right)=w$, and $f\left(b\right)>w>f\left(c\right)$ so by IVT $\exists y\in \left(b,c\right)$ s.t. $f\left(y\right)=w$. Contradicts with injectivity.\\
The other case is similar (apply the first case to $\left(-f\right)$).\\
Fix $a<b$ in $I$. We show that if $f\left(a\right)<f\left(b\right)$ then $f$ is strictly increasing on $I$. The case $f\left(a\right)>f\left(b\right)$ will be similar and then $f$ is strictly decreasing on $I$.\\
Let $x\in I$. If $x<a$ then considering the triple $x<a<b$ we obtain $f\left(x\right)<f\left(a\right)$. If $a<x$ then\\
either $x<b$ and considering $a<x<b$, get $f\left(a\right)<f\left(x\right)$\\
or $x>b$ and considering $a<b<x$, get $f\left(a\right)<f\left(x\right)$\\
or $x=b$ and then $f\left(a\right)<f\left(b\right)=f\left(x\right)$.\\
So far we have $\forall x<y$ in $I$ if $x=a$ or $y=a$ then $f\left(x\right)<f\left(y\right)$.\\
For arbitrary $x<y$ in $I$ with $a\neq x$ and $a \neq y$ we have 3 cases: $a<x<y$,$x<a<y$,$x<y<a$, applying the previous claim we get $f\left(x\right)<f\left(y\right)$.\\
\underline{$J$ is an interval}: Let $x<y<z$ in $\R$ s.t. $x,z\in J$. We have $a,b\in I$ s.t. $x=f\left(a\right)$,$z=f\left(b\right)$. So by IVT, $\exists c$ between $a,b$ s.t. $f\left(c\right)=y$, so $y\in J$.\\
\underline{$J$ is an open interval}: Given $y\in J$,$\exists b\in I$ s.t. $f\left(b\right)=y$.\\
$I$ is an open interval, so $\exists a,c\in I$,$a<b<c$.\\
Then either $f\left(a\right)<f\left(b\right)<f\left(c\right)$ or $f\left(a\right)>f\left(b\right)>f\left(c\right)$.\\
So $y$ is not an endpoint of $J$.\\

Now $f:I\to J$ is a strictly monotonic bijection, so $f^{-1}:J\to I$ is continuous by Theorem 3.6.\\
\underline{$f^{-1}$ differentiable}: Let $y\in J$. We consider
\begin{equation*}
\frac{f^{-1}\left(y+k\right)-f^{-1}\left(y\right)}{k} \text{ as } k\to 0.
\end{equation*}
For given $k$, let $h=f^{-1}\left(y+k\right)-f^{-1}\left(y\right)$ and let $x=f^{-1}\left(y\right)$.\\
Then $f^{-1}\left(y+k\right)=h+f^{-1}\left(y\right)=x+h$,\\
$k=f\left(x+h\right)-f\left(x\right)$,
\begin{equation*}
\begin{aligned}
\frac{f^{-1}\left(y+k\right)-f^{-1}\left(y\right)}{k}&=\frac{h}{f\left(x+h\right)-f\left(x\right)}\\
&=\frac{1}{\frac{f\left(x+h\right)-f\left(x\right)}{h}}
\end{aligned}
\end{equation*}
Here $h=h\left(k\right)$ depends on $k$, $h\left(k\right) \neq 0$ if $k\neq 0$ and $h\left(k\right) \to 0$ as $k \to 0$ since $f^{-1}$ is continuous.\\
So
\begin{equation*}
\frac{f^{-1}\left(y+k\right)-f^{-1}\left(y\right)}{k} \to \frac{1}{f'\left(x\right)} = \frac{1}{f'\left(f^{-1}\left(y\right)\right)}.
\end{equation*}
\end{proof}
\end{thm}

\begin{eg}
Fix $n\in \N$, consider $f:\left(0,\infty\right)\to\R$,$f\left(x\right)=x^n$.\\
$f$ is strictly increasing, onto $\left(0,\infty\right)$, differentiable, $f'\left(x\right)=nx^{n-1}$.\\
We have $f^{-1}:\left(0,\infty\right)\to\left(0,\infty\right)$,$f^{-1}\left(x\right)=x^{\frac{1}{n}}$(definition).\\
The extra information from IFT is that $f^{-1}$ is differentiable, and
\begin{equation*}
\begin{aligned}
\left(f^{-1}\right)'\left(y\right)&=\frac{1}{f'\left(f^{-1}\left(y\right)\right)}\\
&=\frac{1}{n\left(y^{\frac{1}{n}}\right)^{n-1}}\\
&=\frac{1}{n} y^{\frac{1}{n}-1}.
\end{aligned}
\end{equation*}
\end{eg}
For $\alpha=\frac{p}{q}$, $p,q\in\N$, $x^\alpha = \left(x^{\frac{1}{q}}\right)^p$ is differentiable by Chain Rule:\\
$\frac{d}{dx}\left(x^\alpha\right)=p\left(x^\frac{1}{q}\right)^{p-1}\cdot\frac{1}{q}x^{\frac{1}{q}-1}=\alpha x^{\alpha-1}$.\\
For $\alpha\in\Q$,$\alpha<0$,$x^\alpha = \frac{1}{x^{-\alpha}}$ is differentiable, and\\
$\frac{d}{dx}\left(x^\alpha\right)=-\frac{1}{\left(x^{-\alpha}\right)^2}\cdot\left(-\alpha\right)x^{-\alpha-1}=\alpha x^{\alpha-1}$.\\
Need $exp$, $log$ for $\alpha \in \R$.\\

\subsection{Complex differentiation}
Given $f:\C\to\C$, $a\in\C$, we say $f$ is complex differentiable at $a$ if 
\begin{equation*}
\begin{aligned}
\lim_{h\to 0} \frac{f(\left(a+h\right)-f\left(a\right)}{h}
\end{aligned}
\end{equation*}
exists and we denote the limit by $f'\left(a\right)$ and call it the \emph{derivative of $f$ at $a$}.\\
Say $f$ is \emph{complex differentiable} on $\C$ (or \emph{holomorphic}) if it's complex differentiable at every $a\in\C$.\\
$f$ is complex differentiable at $a$ $\iff$ $\exists \lambda\in\C$ $f\left(a+\right)=f\left(a\right)+\lambda h+\epsilon\left(h\right)\cdot h$, where $\epsilon\left(h\right)\to 0$ as $h\to 0$. Then $\lambda = f'\left(a\right)$.\\

\begin{prop}
$f$ is complex differentiable at $a$ $\implies$ $f$ is continuous at $a$.
\end{prop}
Property 2, Theorem 3 also hold.\\

For $z\in\C$, $\sum_{n=0}^\infty \frac{z^n}{n!}$ converges absolutely and hence converges.\\
For $z=0$ ok, for $z\neq 0$ we use the ratio test:
\begin{equation*}
\frac{|\frac{z^{n+1}}{\left(n+1\right)!}|}{|\frac{z^n}{n!}|}=\frac{|z|}{n+1}\to 0 \text{ as } n\to \infty.
\end{equation*}

We define the \emph{exponential function} $\exp :\C\to\C$ by $\exp\left(z\right)=\sum_{n=0}^\infty \frac{z^n}{n!}$.\\

\begin{thm} (Properties of $\exp$)\\
1) $\exp\left(z+n\right)=\exp\left(z\right)\exp\left(w\right)$ $\forall z,w\in\C$;\\
2) $\exp\left(0\right)=1$,$\exp\left(z\right)\neq 0 \forall z\in\C$;\\
3) $\overline{\exp\left(z\right)}=\exp\left(\overline{z}\right)$;\\
4) $\exp\left(x\right)\in\R \forall x\in\R$;\\
5) $\exp\left(ix\right)\in T=\left\{z\in\C | |z|=1\right\} \forall x\in\R$;\\
6) $\exp$ is complex differentiable at $0$, $\exp'\left(0\right)=1$;\\
7) $\exp$ is holomorphic, and $\exp'\left(z\right) = \exp\left(z\right)$.\\
\begin{proof}
1) Let $a_{k}=\frac{z^n}{n'}$,$b_{n}=\frac{w^n}{n!}$,$c_{n}=\frac{\left(z+w\right)^n}{n!}$ for $n\geq 0$.\\
$c_{n} = \frac{1}{n!}\sum_{j=0}^n {n \choose j} z^j w^{n-j} = \sum_{j=0}^n \frac{1}{j!\left(n-j\right)!}z^j w^{n-j} = \sum_{j+k=n} a_{j} b_{j}$.\\
\begin{equation*}
\begin{aligned}
|\sum_{n=0}^N c_{n} - \left(\sum_{n=0}^N a_{n}\right)\left(\sum_{n=0}^N b_{n}\right)|\\
&= |\sum_{n=0}^N \sum_{j+k=n} a_{j} b_{k} - \sum_{j,k=0}^N a_{j} b_{k}|\\
&= |\sum_{j,k=0,j+k>N}^N a_{j} b_{k}|\\
&\leq \sum_{j,k=0,j+k>N}^N |a_{j}||b_{k}|\\
&\leq \sum_{j,k=0,j>\frac{N}{2} \text{ or } k>\frac{N}{2}}^N |a_{j}| |b_{k}|\\
&\leq \sum_{\frac{N}{2}<j\leq N} |a_{j}| \cdot \sum_{k=0}^N |b_{k}| + \sum_{\frac{N}{2}<k\leq N} |b_{k}| \cdot \sum_{j=0}^N |a_{j}|\\
&\to 0 \text{ as } N\to\infty.
\end{aligned}
\end{equation*}
So $\sum_{n=0}^\infty c_{n} = \left(\sum_{n=0}^\infty a_{n}\right)\left(\sum_{n=0}^\infty b_{n}\right)$.\\

2) $\exp\left(0\right)$ by definition.\\
$1=\exp\left(0\right)=\exp\left(z+\left(-z\right)\right)=\exp\left(z\right)\exp\left(-z\right)$.\\
So $\exp\left(z\right)\neq 0 \forall z\in \C$.\\

3) $\overline{\exp\left(z\right)}=\exp\left(\overline{z}\right) \forall z\in \C$.\\
\begin{equation*}
\begin{aligned}
\overline{\exp\left(z\right)}&=\left(\lim_{N\to\infty} \sum_{n=0}^N \frac{z^n}{n!}\right)\\
&=\lim_{N\to\infty} \overline{\left(\sum_{n=0}^N \frac{z^n}{n!}\right)}\\
&=\lim_{N\to\infty} \sum_{n=0}^N \frac{\left(\overline{z}\right)^n}{n!}\\
&=\exp\left(\overline{z}\right).
\end{aligned}
\end{equation*}

4) $\exp\left(x\right)\in\R \forall x\in\R$ and $\exp\left(ix\right)\in T \forall x\in\R$.\\
$\overline{\exp\left(x\right)} = \exp\left(\overline{x}\right)=\exp\left(x\right)$, so $\exp\left(x\right)\in\R$.\\
\begin{equation*}
\begin{aligned}
|\exp\left(ix\right)|^2 &= \exp\left(ix\right)\\
\overline{\exp\left(ix\right)}\\
&=\exp\left(ix\right)\exp\left(-ix\right)\\
&=\exp\left(0\right)\\
&=1.
\end{aligned}
\end{equation*}

5) $\exp'\left(0\right)=1$.\\
\begin{equation*}
\begin{aligned}
\exp\left(h\right)&=\sum_{n=0}^\infty \frac{h^n}{n!}\\
&= 1+h+\sum_{n=2}^\infty \frac{h^n}{n!}\\
&= \exp\left(0\right)+h+h\sum_{n=2}^\infty \frac{h^{n-1}}{n!}.
\end{aligned}
\end{equation*}
Define $\epsilon\left(h\right) = \sum_{n=2}^\infty \frac{h^{n-1}}{n!}$. Need $\epsilon\left(h\right)\to 0$ as $h\to 0$.\\
We have
\begin{equation*}
\begin{aligned}
|\epsilon\left(h\right)|&\leq \sum_{n=2}^\infty |\frac{h^{n-1}}{n!}\\
&= \sum_{n=2}^\infty \frac{|h|^{n-1}}{n!}\\
&\leq \sum_{n=2}^\infty |h|^{n-1} \text{  assume }|h| \leq 1\\
&=\frac{|h|}{1-|h|}
\end{aligned}
\end{equation*}
So $\epsilon\left(h\right)\to 0$ as $h\to 0$. Done.\\

6) $\exp: \C\to\C$ is holomorphic.\\
\begin{equation*}
\begin{aligned}
\frac{\exp\left(z+h\right)-\exp\left(z\right)}{h}\\
&= \frac{\exp\left(z\right)\cdot\exp\left(h\right)-\exp\left(z\right)}{h}\\
&=\exp\left(z\right)\frac{\exp\left(h\right)-\exp\left(0\right)}{h}\\
&\to \exp\left(z\right)
\end{aligned}
\end{equation*}
as $h\to 0$.\\
So $\exp'\left(z\right) = \exp\left(z\right)$.
\end{proof}
\end{thm}

By 4) we have a real function $\exp:\R\to\R$.\\

\begin{thm}
$\exp:\R\to\R$ is a strictly increasing, differentiable bijection of $\R$ onto $\R^+$;\\
For $x\geq 0$, $\exp\left(x\right)\geq 1+x$ so $\exp\left(x\right)\to\infty$ as $x\to\infty$;\\
For $x\leq 0$, $\exp\left(x\right)=\frac{1}{\exp\left(-x\right)}\to 0$ as $x\to -\infty$.\\
\begin{proof}
For $x\geq 0$, $\exp\left(x\right)=\sum_{n=0}^\infty \frac{x^n}{n!} \geq 1+x > 0$.
So for $x \leq 0$,
\begin{equation*}
\begin{aligned}
1&=\exp\left(x+\left(-x\right)\right)\\
&= \exp\left(x\right)\exp\left(-x\right)
\end{aligned}
\end{equation*}
So $\exp\left(x\right) = \frac{1}{\exp\left(-x\right)} > 0$\\
since $\exp'\left(x\right)=\exp\left(x\right)>0 \forall x\in\R$.
\end{proof}
\end{thm}

By Corollary 6, $\exp:\R\to\R^+$ is strictly increasing.\\
Given $y\in\R^+$, choose $n\in\N$ s.t. $n>y>\frac{1}{n}$. So $\exp\left(n\right)\geq 1+n > y$, and $\exp\left(-n\right)=\frac{1}{\exp\left(n\right)}\leq \frac{1}{1+n}<\frac{1}{n}<y$.\\
By IVT, $\exists x\in\left(-n,n\right)$, $\exp\left(x\right) = y$.\\
Finally, $\exp\left(x\right)\geq 1+x\to\infty$ as $x\to\infty$ for $x\geq 0$.\\
For $x\leq 0$, $\exp\left(x\right) = \frac{1}{\exp\left(-x\right)}\to 0$ as $x\to -\infty$.\\

We define the \emph{logarithm} to be the function $\log:\R^+\to\R$ that is the inverse of $\exp:\R\to\R^+$.

\begin{thm}
$\log:\R^+\to\R$ is a strictly increasing, differentiable bijection. For $y>0$, $\log'\left(y\right)=\frac{1}{y}$, $\log1=0$,$\log\left(xy\right)=\log x+\log y \forall x,y>0$, $\log x\to\infty$ as $x\to\infty$, $\log x\to\infty$ as $x\to 0$.\\
\begin{proof}
If $0<x<y$ and $\log x\geq \log y$ then\\
$x=\exp\left(\log x\right)\geq \exp\left(\log y\right) = y$, contradiction.\\
Since $\exp'\left(x\right)=\exp\left(x\right)\neq 0 \forall x\in\R$, by IFT, $\log$ is differentiable, and\\
\begin{equation*}
\begin{aligned}
\log'\left(y\right)=\frac{1}{\exp\left(\log y\right)}=\frac{1}{y}.
\end{aligned}
\end{equation*}
$\log 1=0$ since $1=\exp\left(0\right)$.\\
$\exp\left(\log x+\log y\right) = \exp\left(\log x\right)\exp\left(\log y\right)=xy$;\\
Apply log:\\
$\log x+\log y = \log \left(xy\right)$.\\
Since $\log$, $\exp$, are strictly increasing, $\log x>c \iff x>\exp c$, $\log x<c \iff x<\exp c$.\\
So it follows immediately that\\
$\log x\to\infty$ as $x\to\infty$,\\
$\log x\to -\infty$ as $x\to 0^+$.
\end{proof}
\end{thm}

\begin{defi}
Define for $x>0$, $\alpha\in\R$
\begin{equation*}
\begin{aligned}
x^\alpha = \exp\left(\alpha \log x\right).
\end{aligned}
\end{equation*}
\end{defi}

\begin{thm}
$\bullet$ 1) for $\alpha \in \Q$, $x^\alpha$ agrees with the previous definition.\\
$\bullet$ 2) for $\alpha > 0$, $x\to x^\alpha$ is a strictly increasing differentiable bijection: $\R^+\to\R^+$;\\
for $\alpha < 0$, $x\to x^\alpha$ is a strictly decreasing differentiable bijection: $\R^+\to\R^+$;
$\forall \alpha$, $f\left(x\right)=x^\alpha$, $f'\left(x\right)=\alpha x^{\alpha-1}$;\\

$\forall x,y>0, \forall \alpha,\beta\in\R$:\\
$\bullet$ 3) $\left(xy\right)^\alpha = x^\alpha y^\alpha$,\\
$x^{\alpha+\beta} = x^\alpha x^\beta$,
$\left(x^\alpha\right)^\beta = x^{\alpha\beta}$;\\
$\bullet$ 4) $\frac{x^\alpha}{\exp\left(x\right)} \to 0$ as $x\to \infty \forall x\in\R$,\\
$\frac{\log x}{x^\alpha} \to 0$ as $x\to\infty \forall x>0$.
\begin{proof}
1)
\begin{equation*}
\begin{aligned}
x^n&=\exp\left(n\log x\right)\\
&= \exp\left(\log x+\log x+...+\log x\right) \text{ n times}\\
&= \exp\left(\log x\right)\exp\left(\log x\right)...\exp\left(\log x\right) \text{ n times}\\
&= x\cdot x\cdot...\cdot x \text{ n times}
\end{aligned}
\end{equation*}
which is the old definition of $x^n$.\\
\begin{equation*}
\begin{aligned}
\left(x^\frac{1}{n}\right)^n &= \left(\exp\left(\frac{1}{n}\log x\right)\right)^n\\
&= \exp\left(\frac{1}{n}\log x\right)\cdot\exp\left(\frac{1}{n}\log x\right)\cdot...\cdot\exp\left(\frac{1}{n}\log x\right) \text{ n times}\\
&= \exp\left(\log x\right)\\
&= x.
\end{aligned}
\end{equation*}
So the new $x^\frac{1}{n}$ is the unique $y>0$ such that $y^n=x$, also same as the old definition.\\
So now for $\alpha\in\Q^+$, it follows that the two definitions coincide.\\
For $\alpha\in\Q^-$,
\begin{equation*}
\begin{aligned}
x^\alpha &= \exp\left(\alpha \log x\right)\\
&= \exp\left(-\left(-\alpha\right)\log x\right)\\
&= \frac{1}{\exp\left(-\alpha \log x\right)}\\
&= \frac{1}{x^{-\alpha}}
\end{aligned}
\end{equation*}
also the old definition.\\

2) Immediate from properties of $\log$ and $\exp$.\\
e.g. for $f\left(x\right)=x^\alpha = \exp\left(\alpha \log x\right)$, by chain rule, \\
\begin{equation*}
\begin{aligned}
f'\left(x\right)&=\exp\left(\alpha \log x\right)\alpha \frac{1}{x}\\
&=\alpha \frac{\exp\left(\alpha \log x\right)}{\exp\left(\log x\right)}\\
&= \alpha \exp\left(\alpha \log x-\log x\right)\\
&=\alpha \exp\left(\left(\alpha-1\right)\log x\right)\\
=\alpha x^{\alpha-1}.
\end{aligned}
\end{equation*}

3) also immediate from properties of $\log$ and $\exp$. (exercise)\\

4) For $x>0$, $\exp\left(x\right)>\frac{x^n}{n!}$ for any $n\in\N$.\\
Given $\alpha\in\R$, choose $n\in\N, n>\alpha$, then\\
\begin{equation*}
\begin{aligned}
\frac{x^\alpha}{\exp\left(x\right)}&<\frac{x^\alpha}{x^{-n} n!}\\
&=\left(n!\right) x^{\alpha - n}\\
&=\left(n!\right)\exp\left(\left(\alpha - n\right)\log x\right) \to 0
\end{aligned}
\end{equation*}
as $x\to\infty$.

Now let $y=\log\left(x\alpha\right) = \alpha \log x \to \infty$ as $x \to \infty$, so $\frac{\log x}{x^\alpha}=\frac{1}{\alpha}\frac{y}{\exp\left(y\right)}\to 0$ as $x\to\infty$.
\end{proof}
\end{thm}

We can define $x^\alpha = \exp\left(\alpha\log x\right)$ for $x\in\R, x>0$ and $\alpha\in\C$.\\
Exercise: define $e=\lim_{n\to\infty} \left(1+\frac{1}{n}\right)^n$.\\
Show that $e=\exp\left(1\right)$, $e^z=\exp\left(z\right)$.\\
Need $e^z=e^w \iff z-w\in 2\pi\Z$ (what is $\pi$?)

\subsection{Trigonometric and Hyperbolic functions}
\begin{defi}
Define functions $\sin,\cos,\sinh,\cosh:\C\to\C$:
\begin{equation*}
\begin{aligned}
\sin z &= \frac{e^iz-e^-iz}{2i} = \sum_{n=0}^\infty \frac{\left(-1\right)^n z^{2n+1}}{\left(2n+1\right)!} = z-\frac{z^3}{6}+\frac{z^5}{120}-...\\
\cos z &= \frac{e^iz+e^-iz}{2} = \sum_{n=0}^\infty \frac{\left(-1\right)^n z^{2n}}{\left(2n\right)!} = 1-\frac{z^2}{2}+\frac{z^4}{24}-...\\
\sinh z &= \frac{e^z-e^-z}{2} = \sum_{n=0}^\infty \frac{z^{2n+1}}{\left(2n+1\right)!}\\
\cosh z &= \frac{e^z+e^-z}{2} = \sum_{n=0}^\infty \frac{z^{2n}}{\left(2n\right)!}
\end{aligned}
\end{equation*}
\end{defi}

\begin{prop} (Properties of trigonometric functions)\\
1) If $f$ is any of these trigonometric functions, then $\overline{f\left(z\right)}>f\left(\overline{z}\right)$.\\
Also $f\left(-z\right)=-f\left(z\right)$ for $f=\sin, \sinh$ (odd),\\
$f\left(-z\right)=f\left(z\right)$ for $f=\cos,\cosh$ (even),\\
$\sin\left(0\right) =\sinh\left(0\right) = 0$, $\cos\left(0\right) = \cosh \left(0\right) = 1$.\\

2)
\begin{equation*}
\begin{aligned}
\sin\left(z+w\right)&=\sin z\cos w + \cos z \sin w, \sin\left(2z\right)=2\sin z\cos z\\
\cos\left(z+w\right)&=\cos z\cos w - \sin z \sin w, \cos\left(2z\right)=\cos^2 z-\sin^2 z\\
\sinh \left(z+w\right)&=\sinh z\cosh w + \cosh z\sinh w, \sinh \left(2z\right)=2\sinh z\cosh z\\
\cosh\left(z+w\right)&=\cosh z\cosh w + \sinh z\sinh w, \cosh\left(2z\right)=\cosh^2 z+\sinh^2 z.
\end{aligned}
\end{equation*}
3)
\begin{equation*}
\begin{aligned}
1&=\cos\left(0\right)=\cos^2 z+\sin^2 z\\
1&=\cosh\left(0\right) = \cosh^2 z - \sinh^2 z
\end{aligned}
\end{equation*}
4)
\begin{equation*}
\begin{aligned}
e^{iz} &= \cos z + i\sin z\\
e^z &= \cosh z + \sinh z
\end{aligned}
\end{equation*}
5) All the four functions are complex differentiable, with
\begin{equation*}
\begin{aligned}
\sin'z = \cos z, \cos'z = -\sin z, \sinh'z = \cosh z, \cosh' z= \sinh z.
\end{aligned}
\end{equation*}
\begin{proof} Immediate from the previous theorem.
\end{proof}
\end{prop}

This implies that $\sin x, \cos x \in \R$ for $x\in\R$.\\
Since $\cos^2 x + \sin^2 x = 1$, we have $\cos x, \sin x \in [-1,1]$.\\
Have functions $\sin,\cos: \R \to [-1,1]$\\
\begin{equation*}
\begin{aligned}
\cos x &= \sum_{n=0}^\infty \left(-1\right)^n \frac{x^{2n}}{\left(2n\right)!} = 1-\frac{x^2}{2} + \frac{x^4}{24}+\left(-\frac{x^6}{6!}+\frac{x^8}{8!}\right) + \left(-\frac{-x^10}{10!} + \frac{x^12}{12!}\right)+...
\end{aligned}
\end{equation*}
At $x=2$, each term in the brackets is negative.\\
So 
\begin{equation*}
\begin{aligned}
\cos\left(z\right) < 1-\frac{2^2}{2}+\frac{2^4}{4!} = 1-2+\frac{16}{84}<0
\end{aligned}
\end{equation*}
Since $\cos \left(0\right)=1>0$ and $\cos$ is continuous, by IVT
\begin{equation*}
\begin{aligned}
\exists z\in\left(0,2\right) s.t. \cos\left(z\right)=0.
\end{aligned}
\end{equation*}
So $A=\left\{z\geq 0|\cos\left(z\right)=0\right\} \neq \phi$ is bounded below by 0, so $\inf A$ exists.\\

\begin{defi}
$\pi = 2\times \inf A$, i.e. $\frac{\pi}{2} = \inf A$.
\end{defi}
Claim:
\begin{equation*}
\begin{aligned}
\cos\frac{\pi}{2} = 0
\end{aligned}
\end{equation*}
and so $\frac{\pi}{2}\geq 0$, and it is the least positive zero of $\cos$.\\
\begin{proof}
$\forall n\in\N$, $\frac{\pi}{2}+\frac{1}{n} > \inf A$, so $\exists x_n \in A$ s.t. $\frac{\pi}{2} + \frac{1}{n} > x_n \geq \frac{\pi}{2}$.\\
So $x_n \to \frac{\pi}{2}$ and have $\cos \left(x_n\right) \to \cos\frac{\pi}{2}$.($\cos$ is continuous). So $\cos\frac{\pi}{2} = 0$.\\
$\cos\left(x\right)>0$ for $x\in\left(0,\frac{\pi}{2}\right)$. So $\sin'\left(x\right)=\cos x<0$ and hence $\sin$ is strictly increasing on $[0,\frac{\pi}{2}]$.
\end{proof}
So $\sin^2\frac{\pi}{2} = 1-\cos^2\frac{\pi}{2}=1$, so $\sin\frac{\pi}{2} = 1$.\\
And $\sin x>0$ on $\left(0,\frac{\pi}{2}\right)$ and hence $\cos$ is strictly decreasing on $[0,\frac{\pi}{2}]$.\\
$\sin\pi = 2\sin\frac{\pi}{2}\cos\frac{\pi}{2}=0$, $\cos\pi=\cos^2\frac{\pi}{2}-\sin^2\frac{\pi}{2}=-1$.\\
$\sin\left(\pi-x\right)=\sin\pi\cos\left(-x\right)+\cos\pi\sin\left(-x\right)=\sin x$.\\
$x\to\frac{\pi}{2}+x$, $\sin\left(\frac{\pi}{2}-x\right) =\sin\left(\frac{\pi}{2}+x\right)$;\\
$x\to -x$, $\sin\left(\pi+x\right)=\sin\left(-x\right)=-\sin\left(x\right)=-\sin\left(\pi+x\right)$;\\
$\sin\left(2\pi+x\right)=-\sin\left(-x\right)=\sin x$.\\
So $\sin\left(2\pi n+x\right)=\sin x \forall x\in \R \forall n\in\Z$.\\
$\sin\left(x+\frac{\pi}{2}\right)=\sin x\cos\frac{\pi}{2} + \cos x \sin \frac{\pi}{2} = \cos x \implies$ usual properties of $\cos$ (symmetry, periodicity).

\begin{prop}
For $z\in\C,e^z=1 \iff z\in2\pi i \Z$.\\
So $e^z = e^w \iff z-w\in2\pi i\Z$.
\begin{proof}
If $z=2\pi in,n\in \Z$ then
\begin{equation*}
\begin{aligned}
e^z &= e^{\left(2\pi n\right)}i\\
&= \cos\left(2\pi n\right)+2\sin\left(2\pi n\right)\\
&= \cos\left(0\right)+i\sin\left(0\right)\\
&=1.
\end{aligned}
\end{equation*}
Conversely, assume $e^z=1$ and write $z=x+iy$, $x,y\in\R$,
\begin{equation*}
\begin{aligned}
1=e^z=e^x e^{iy}
\end{aligned}
\end{equation*}
taking modulus,
\begin{equation*}
\begin{aligned}
1=e^x
\end{aligned}
\end{equation*}
So$x=0$.\\
So $1=e^{iy}=\cos\left(y\right)+i\sin\left(y\right)$ and hence $\cos\left(y\right) = 1$.\\
Since $\cos$ is strictly decreasing on $[0,\pi]$, we have
\begin{equation*}
\begin{aligned}
\cos t = 1, t\in[0,\pi] \iff t=0
\end{aligned}
\end{equation*}
Since $\cos$ is symmetric in $x=\pi$, for $t\in[0,2\pi)$,
\begin{equation*}
\begin{aligned}
\cos t=1 \iff t=0
\end{aligned}
\end{equation*}
Now choose $n\in\Z$ s.t. $y-2\pi n\in[0,2\pi)$.\\
Then $\cos\left(y-2\pi n\right)=\cos\left(y\right)=1$ and so $y-2\pi n = 0$. Hence $z\in 2\pi i\Z$.
\end{proof}
\end{prop}

For $x\in\R$,
\begin{equation*}
\begin{aligned}
\sinh x&=\frac{e^x-e^{-x}}{2} \in \R, \sinh:\R\to\R\\
\cosh x&=\frac{e^x+e^{-x}}{2} \in \R, \cosh:\R\to\R
\end{aligned}
\end{equation*}
$\cosh x>0$, $\sinh'\left(x\right)=\cosh\left(x\right)$. So $\sinh$ is strictly increasing.\\
$\cosh'\left(x\right)=\sinh\left(x\right) > 0$ for $x>0$.\\
So $\cosh$ is strictly increasing on $[0,\infty)$.\\
$\cosh x > \sinh x$, and\\
\begin{equation*}
\begin{aligned}
\frac{\cosh x}{\sinh x} = \frac{1+e^{-2x}}{1-e^{-2x}} \to 1
\end{aligned}
\end{equation*}
as $x\to\infty$.\\
Define $\tan z = \frac{\sin z}{\cos z}$, $\tanh z = \frac{\sinh z}{\cosh z}$.

\subsection{Derivative of higher orders}
\begin{defi}
Let $A\subset \R, f:A\to\R$ be a function.\\
For $a\in A$, say $f$ is twice differentiable at $a$ if $f$ is defined and differentiable on some open interval $I$ containing $a$ ($I\subset A$), and $f':I\to \R$, $x\to f'\left(x\right)$ is differentiable at $a$.
The second derivative of $f$ at $a$ is $\left(f'\right)'\left(a\right)$.\\
We denote this by $f''\left(a\right)$ or $f^{\left(2\right)} \left(a\right)$ (also sometimes write $f^{\left(1\right)}$ for $f'$,$f^{\left(0\right)}$ for $f$).\\
$f$ is twice differentiable on $A$ if $f$ is twice differentiable at every $a\in A$, then the second derivative of $f$ is the function $f'':A\to \R, x\to f''\left(x\right)$.\\
In this case $\forall a\in A \exists r>0, \left(a-r,a+r\right)\subset A$.\\
Typically $A=\R$ or some open interval or $\R \backslash \left\{0\right\}$.\\

In general for $n\geq 2$, $f$ is $n$ times differentiable at $a$ if $f$ is $n-1$ times differentiable on some open interval $I$ containing $a$, and $f^{\left(n-1\right)}:I\to \R, x \to f^{\left(n-1\right)} \left(x\right)$ is differentiable at $a$. We write $f^{\left(n\right)}\left(a\right)$ for $\left(f^{\left(n-1\right)}\right)'\left(a\right)$ called the $n^{th}$ derivative of $f$ at $a$.\\
$f$ is $n$ times differentiable on $A$ if $f$ is $n$ times differentiable at every $a\in A$. Then the $n^{th}$ derivative of $f$ on $A$ is the function: $f^{\left(n\right)}:A \to \R, x \to f^{\left(n\right)}\left(x\right)$.\\
$f$ is infinitely differentiable on $A$ (or $C^\infty$) if $f$ is $n$ times differentiable on $A$ $\forall n\in \N$.\\
$f$ is $n$ times continuously differentiable on $A$ (or $C^n$) if $f$ is $n$ times differentiable on $A$, and $f^{\left(n\right)}: A\to\R$ is continuous.
\end{defi}

\begin{eg}
Let
\begin{equation*}
\begin{aligned}
p\left(x\right) = a_0 + a_1 x + a_2 x^2 + ... + a_n x^n 
\end{aligned}
\end{equation*}
be a polynomial. Then
\begin{equation*}
\begin{aligned}
p'\left(x\right)=a_1 + 2a_2 x + 3a_3 x^2 + ... + n a_n x^{n-1}
\end{aligned}
\end{equation*}
is also a polynomial. So by induction, $p$ is $C^\infty$.
\end{eg}

\begin{thm}
Let $n\in \N$, $a\in \R$, let $f$ be $n$ times differentiable at $a$. Then $\exists \delta > 0$ and a function $R_n:\left(-\delta,\delta\right)\to\R$ s.t.
\begin{equation}\label{eq:1}
\begin{aligned}
f\left(a+h\right)&=f\left(a\right)+f'\left(a\right)h + \frac{f''\left(a\right)}{2}h^2 + ... + \frac{f^{\left(n\right)}\left(a\right)}{n!} + R_n\left(h\right)
\end{aligned}
\end{equation}
for all $h\in\left(-\delta,\delta\right)$, and $R_n\left(h\right) = o\left(h^n\right)$, i.e.
\begin{equation*}
\begin{aligned}
\frac{R_n\left(h\right)}{h^n} \to 0
\end{aligned}
\end{equation*}
as $h\to 0$.
\begin{proof}
By definition, $\exists \delta > 0$ s.t. $f$ is defined and is $n-1$ times differentiable on $\left(a-\delta,a+\delta\right)$, and
\begin{equation*}
\begin{aligned}
f^{\left(n\right)}\left(a\right) &= \lim_{h\to 0} \frac{f^{\left(n-1\right)} \left(a+h\right)-f^{\left(n-1\right)} \left(a\right)}{h}.
\end{aligned}
\end{equation*}
We define
\begin{equation*}
\begin{aligned}
R_n\left(n\right) &= f\left(a+h\right)-\sum_{k=0}^n \frac{f^{\left(k\right)} \left(a\right)}{k!}h^k, |h| < \delta
\end{aligned}
\end{equation*}
So (\ref{eq:1}) holds.\\
$\bullet n=1:$
\begin{equation*}
\begin{aligned}
\frac{R_1\left(h\right)}{h} = \frac{f\left(a+h\right)-f\left(a\right)}{h}-f'\left(a\right) \to 0
\end{aligned}
\end{equation*}
$\bullet n\geq 2:$ $R_n\left(h\right)$ is ($n-1$) times differentiable, and
\begin{equation*}
\begin{aligned}
R_n^{\left(k\right)}\left(h\right) &= f^{\left(k\right)}\left(a+h\right)\\
&=\sum_{l=k}^n \frac{f^{\left(l\right)}\left(a\right)}{l!} l\left(l-1\right)...\left(l-k+1\right)h^{l-k},\\
R_n^{\left(k\right)}\left(0\right) &= 0 \text{  }\forall k=0,1,...,n-1.
\end{aligned}
\end{equation*}
Now let
\begin{equation*}
\begin{aligned}
g\left(h\right)&=h^n,\\
g^{\left(k\right)}\left(h\right)&=n\left(n-1\right)...\left(n-k+1\right)h^{n-k},\\
g^{\left(k\right)}\left(0\right)&=0,\\
g^{\left(k\right)}\left(h\right)&\neq 0 \text{  }\forall 0<|h|<\delta.
\end{aligned}
\end{equation*}
Then
\begin{equation*}
\begin{aligned}
\frac{R_n^{\left(n-1\right)}\left(h\right)}{g^{\left(n-1\right)}\left(h\right)} &= \frac{f^{\left(n-1\right)}\left(a+h\right)-f^{\left(n-1\right)}\left(a\right)-hf^{\left(n\right)}\left(a\right)}{\left(n!\right)h}\\
&= \left(\frac{1}{n!}\right) \left[\frac{f^{\left(n-1\right)}\left(a+h\right)-f^{\left(n-1\right)}\left(a\right)}{h}-f^{\left(n\right)}\left(a\right)\right] \to 0
\end{aligned}
\end{equation*}
as $h\to 0$.\\
So apply L' H$\hat{o}$pital's rule $\left(n-1\right)$ times, we obtain
\begin{equation*}
\begin{aligned}
\frac{R_n\left(h\right)}{h^n}\to 0
\end{aligned}
\end{equation*}
as $h\to 0$.
\end{proof}
\end{thm}

Suppose $f$ is a $C^\infty$ function. Then the previous theorem applies for all $n$. So does
\begin{equation*}
\begin{aligned}
f\left(a+h\right) &= \sum_{k=0}^n \frac{f^{\left(k\right)}\left(a\right)}{k!}h^k + R_n\left(h\right)\\
\implies(?) f\left(a+h\right)&= \sum_{k=0}^\infty \frac{f^{\left(k\right)}\left(a\right)}{k!}h^k
\end{aligned}
\end{equation*}
(called the \emph{Taylor series}) on $\left(-\delta,\delta\right)$ on some $\delta >0$?

\begin{eg}
$f=\exp:\R \to \R$.\\
$f'=f$, so $f$ is $C^\infty$ and $f^{\left(n\right)} = f$ $\forall n$.\\
Taylor series at 0:
\begin{equation*}
\begin{aligned}
\sum_{k=0}^\infty \frac{1}{k!}h^k = \exp\left(h\right)
\end{aligned}
\end{equation*}
for all $h\in\R$...
\end{eg}

In general the answer is NO!
\begin{prob}
$R_n$ depends on $n$. What if $R_n\left(h\right)=n^{n+1}h^{n+1}$?\\
For all $n$, $\frac{R_n\left(h\right)}{h^n}\to 0$ as $h\to 0$.\\
For fixed $h\neq 0$, $R_n\left(h\right) \not\to 0$ as $n\to \infty$.
\end{prob}

\begin{thm}(Taylor's theorem with the Lagrange remainder)\\
Let $a\in\R$, $\delta>0$, $n\in\N$. Assume $f:\left(a-\delta,a+\delta\right)\to\R$ is $n$ times differentiable. Then $\forall h\in\left(-\delta,\delta\right)$, $\exists \theta\in\left(0,1\right)$ s.t.
\begin{equation*}
\begin{aligned}
f\left(a+h\right) = f\left(a\right)+\sum_{k=1}^{n-1} \frac{f^{\left(k\right)}\left(a\right)}{k!}
h^k + \frac{f^{\left(n\right)}\left(a+\theta h\right)}{n!}h^n.
\end{aligned}
\end{equation*}
\begin{rem}
$\bullet$ when $n=1$:
\begin{equation*}
\begin{aligned}
&f\left(a+h\right) = f\left(a\right) + f'\left(a+\theta h\right)h,\\
&\frac{f\left(a+h\right)-f\left(a\right)}{h} = f'\left(a+\theta h\right)
\end{aligned}
\end{equation*}
while $a+\theta h$ is between $a$ and $a+h$. So this is MVT!\\
$\bullet$ (\ref{eq:1}) says
\begin{equation*}
\begin{aligned}
&f\left(a+h\right)=f\left(a\right)+\sum_{k=1}^n \frac{f^{\left(k\right)}\left(a\right)}{k!}h^k +R_n\left(h\right),\\
&R_n\left(h\right)=\frac{h^n}{n!}\left(f^{\left(n\right)}\left(a+\theta h\right) - f^{\left(n\right)} \left(a\right)\right)
\end{aligned}
\end{equation*}
This is not obviously $o\left(h^n\right)$.
\end{rem}
\begin{proof}
For $n=1$ the theorem is just MVT.\\
For $n \geq 2$, fix $h\in\left(-\delta,\delta\right)$, WLOG $h\neq 0$. Choose $A\in \R$ s.t.
\begin{equation*}
\begin{aligned}
f\left(a+h\right) &= f\left(a\right)+\sum_{k=1}^{n-1} \frac{f^{\left(k\right)}\left(a\right)}{k!}h^k + \frac{A h^n}{n!}
\end{aligned}
\end{equation*}
want to prove:
\begin{equation*}
\begin{aligned}
A=f^{\left(n\right)} \left(a+\theta h\right)
\end{aligned}
\end{equation*}
for some $\theta \in \left(0,1\right)$.\\
Define
\begin{equation*}
\begin{aligned}
g\left(t\right)&=f\left(t\right)+\sum_{k=1}^{n-1}\frac{f^{\left(k\right)}\left(t\right)}{k!}\left(a+h-t\right)^k + \frac{A}{n!}\left(a+h-t\right)^n
\end{aligned}
\end{equation*}
For $t$ in the closed interval between $a$ and $a+b$.\\
$g$ is continuous differentiable on the open interval between $a$ and $a+h$.\\
Taylor expansion of $f$ about $t$:
\begin{equation*}
\begin{aligned}
f\left(t+u\right)=f\left(t\right) = \sum_{k=1}^{n-1} \frac{f^{\left(k\right)} \left(t\right)}{k!} u^k + \text{  error}
\end{aligned}
\end{equation*}
when $u=a+h-t$,
\begin{equation*}
\begin{aligned}
f\left(a+h\right)=f\left(t\right) + \sum_{k=1}^{n-1} \frac{f^{\left(k\right)} \left(t\right)}{k!}\left(a+h-t\right)^k + \text{  error}
\end{aligned}
\end{equation*}
Now
\begin{equation*}
\begin{aligned}
g\left(a\right)=f\left(a+h\right),\\
g\left(a+h\right)=f\left(a+h\right)
\end{aligned}
\end{equation*}
By Rolle's theorem, $\exists \theta \in \left(0,1\right)$ s.t. $g'\left(a+\theta h\right)=0$.
\begin{equation*}
\begin{aligned}
g'\left(t\right) &= f'\left(t\right) + \sum_{k=1}^{n-1} \left[-\frac{f^{\left(k\right)} \left(t\right)}{\left(k-1\right)!}\left(a+h-t\right)^{k-1} + \frac{f^{\left(k+1\right)}\left(t\right)}{k!}\left(a+h-t\right)^k\right] - A\frac{\left(a+h-t\right)^{n-1}}{\left(n-1\right)!}\\
&= \frac{f^{\left(n\right)} \left(t\right)}{\left(n-1\right)!}\left(a+h-t\right)^{n-1} - A \frac{\left(a+h-t\right)^{n-1}}{\left(n-1\right)!}
\end{aligned}
\end{equation*}
So $g'\left(a+\theta h\right) = 0$, which implies $A=f^{\left(n\right)} \left(a+\theta h\right)$.
\end{proof}
\end{thm}

\begin{eg}
Fix $\alpha \in \R$, $f:\left(-1,\infty\right)\to\R$.
\begin{equation*}
\begin{aligned}
f\left(x\right) &= \left(1+x\right)^\alpha = \exp\left(\alpha \log\left(1+x\right)\right)\\
f'\left(x\right) &= \alpha \left(1+x\right)^{\alpha-1}\\
f''\left(x\right) &= \alpha\left(\alpha-1\right)\left(1+x\right)^{\alpha-2}\\
f^{\left(n\right)}\left(x\right) &= \alpha\left(\alpha - 1\right)...\left(\alpha - n + 1\right)\left(1+x\right)^{\alpha-n}
\end{aligned}
\end{equation*}
So $f$ is $C^\infty$ on $\left(-1,\infty\right)$ and its Taylor series at 0 is
\begin{equation*}
\begin{aligned}
\sum_{n=0}^\infty {\alpha \choose n} x^n = 1+\alpha x + \frac{\alpha\left(\alpha-1\right)}{2}x^2+...
\end{aligned}
\end{equation*}
This converges to $f\left(x\right)$ $\forall x\in\left(-1,1\right)$ (binomial theorem).
\end{eg}
\begin{rem}
$\bullet$ $\alpha \in \Z, \alpha \geq 0$, then $\alpha^{\underline{n}} = 0 \forall n > \alpha$.(c.f. number and sets, falling power)\\
So
\begin{equation*}
\begin{aligned}
\sum_{n=0}^\infty {\alpha \choose n} x^n = \sum_{n=0}^\alpha {\alpha \choose n} x^n = \left(1+x\right)^n
\end{aligned}
\end{equation*}
$\bullet$ $\alpha = -1$:
\begin{equation*}
\begin{aligned}
\left(1+x\right)^{-1} = \frac{1}{1-\left(-x\right)} = \sum_{n=0}^\infty \left(-x\right)^n\\
\alpha^{\underline{n}} = \left(-1\right)\left(-2\right)...\left(-n\right) = \left(-1\right)^n n!
\end{aligned}
\end{equation*}
Proof for general $\alpha$ deferred until Chapter 6(integration).\\
We an give a proof for $|x| < \frac{1}{2}$.
From the previous theorem:
\begin{equation*}
\begin{aligned}
f\left(x\right) = \sum_{k=0}^n {\alpha \choose k} x^k + {\alpha \choose n+1} \left(1+\theta_n x\right)^{n+1} x^{n+1}
\end{aligned}
\end{equation*}
for some $\theta_n\in\left(0,1\right)$.
\begin{equation*}
\begin{aligned}
|\frac{x^{n+1}}{\left(n+1\right)!}\alpha^{\underline{n+1}}\left(1+\theta_n x\right)^{\alpha-n-1}| \leq C\cdot n^m \left(2|x|\right)^n
\end{aligned}
\end{equation*}
(for $m<\in \N, m>|\alpha|$)
\begin{equation*}
\begin{aligned}
|\frac{x}{1+\theta_n x} \leq \frac{|x|}{1-|\theta_n x|}\leq 2|x|
\end{aligned}
\end{equation*}
\end{rem}

\newpage

\section{Power series}
\begin{defi}
A series of the form
\begin{equation*}
\begin{aligned}
\sum_{n=0}^\infty a_n \left(z-a\right)^n
\end{aligned}
\end{equation*}
is a power series about $a$. Here $\left(a_n\right)_{n=0}^\infty$ is a complex sequence, $a,z\in\C$. Think of $a,\left(a_n\right)$ as fixed and $z$ as a variable.\\
Consider
\begin{equation*}
\begin{aligned}
D=\left\{z\in\C | \sum_{n=0}^\infty a_n \left(z-a\right)^n \text{  converges  } \right\}
\end{aligned}
\end{equation*}
and define $f:D\to \C$,
\begin{equation*}
\begin{aligned}
f\left(z\right)=\sum_{n=0}^\infty a_n \left(z-a\right)^n
\end{aligned}
\end{equation*}
\end{defi}

\begin{eg}
1) $\sum_{n=0}^\infty \frac{1}{n!}z^n$. $a_n = \frac{1}{n!}$, $a=0$.\\
Here $D=\C$ and $f=\exp$.\\
2) $\sum_{n=1}^\infty \frac{-1}{n}\left(1-z\right)^n$. $a=1$, $a_n = \frac{\left(-1\right)^{n-1}}{n}$, $a_0 = 0$ (note $\left(1-z\right)^n = \left(-1\right)^n \left(z-1\right)^n$).\\
3) $\sum_{n=1}^\infty n^n z^n$. $a_n = n$, $n\geq 1$, $a_0 = 0$, $a=0$.\\
D=$\left\{0\right\}$: given $z\neq 0$, $\exists N$ s.t. $|Nz|>1$. Then $\forall n\geq N$, $|\left(nz\right)^n| \geq 1$. So $\left(nz\right)^n \not\to 0$.
\end{eg}

\begin{notation}
$D\left(a,r\right)=\left\{z\in\C||z-a|<r\right\}$  the open disc with centre $a$, radius $r$.\\
$\bar{D}\left(a,r\right)=\left\{z\in\C||z-a|<r\right\}$  the closed disc with centre $a$, radius $r$.
\end{notation}

Note: $\left\{z|\sum a_n\left(z-a\right)^n \text{  converges  }\right\} = \left\{z|\sum a_n z^n \text{  converges  }\right\} + a$. So WLOG $a=0$.

\begin{thm}
Suppose $\sum_{n=0}^\infty a_n w^n$ converges. Then $\forall z\in\C$, if $|z| < |w|$, then $\sum_{n=0}^\infty a_n z^n$ converges absolutely.
\begin{proof}
Since $\sum a_n w^n$ converges, $a_n w^n \to 0$ as $n\to \infty$.\\
So $\exists N \leq \N$, $\forall n \geq N$, $|a_n w^n| \leq 1$. Then $\forall n\geq \N$,
\begin{equation*}
\begin{aligned}
|a_n z^n| = |a_n w^n \left(\frac{z}{w}\right)^n| \leq |\frac{z}{w}|^n
\end{aligned}
\end{equation*}
This converges as $|\frac{z}{w}|<1$ (geometric series). So by comparison test, $\sum a_n z^n$ converges absolutely.
\end{proof}
\end{thm}

\begin{conv}
1) Let $\left[0,\infty\right] = \left[0,\infty\right) \cup \left\{ \infty\right\}$.\\
Extend $\leq$ to $\left[0,\infty\right]$ by $x\leq \infty$ $\forall x\in\left[0,\infty\right]$, so $x<\infty$ $\forall x\in\left[0,\infty\right]$.\\
2) So $|z|<\infty$ $\forall z<\C$, and $\not\exists z\in\C, |z|>\infty$. Write $D\left(a,\infty\right) = \C$ by convention.\\
3) $A\subset\left[0,\infty\right)$, $a\neq \phi$. If $a$ is not bounded above then $\forall C \geq 0$, $\exists a\in A$ s.t. $a>C$. We define $\sup A = \infty$.
\end{conv}

\begin{thm}
For power series
\begin{equation*}
\begin{aligned}
\sum_{n=0}^\infty a_n\left(z-a\right)^n
\end{aligned}
\end{equation*}
There exists a unique $R\in\left[0,\infty\right]$ s.t. $\forall z\in \C$, 
\begin{equation*}
\begin{aligned}
&|z-a| < R \implies \sum_{n=0}^\infty a_n \left(z-a\right)^n \text{    converges absolutely},\\
&|z-a| > R \implies \sum_{n=0}^\infty a_n \left(z-a\right)^n \text{    diverges.}
\end{aligned}
\end{equation*}
\begin{proof}
WLOG let $a=0$.\\
$\bullet$ uniqueness: Suppose $R<S$ both work. Then fix $z\in\C$ with $R<|z|<S$.\\
Definition of $R$ $\implies$ $\sum a_n z^n$ diverges;\\
Definition of $S$ $\implies$ $\sum a_n z^n$ converges absolutely. Contradiction.\\
$\bullet$ existence: Let
\begin{equation*}
\begin{aligned}
A=\left\{|z| \mid \sum_{n=0}^\infty a_n z^n \text{   converges}\right\}
\end{aligned}
\end{equation*}
$A\neq \phi$ as $0\in A$. Let $R=\sup A$ (recall that $\sup A=\infty$ when $A$ is not bounded above).\\
Let $z\in\C$. If $|z|<R$, then $\exists w\in\C$ s.t. $|z|<|w|$ and $\sum_{n=0}^\infty a_n w^n$ converges. Hence by the previous theorem, $\sum_{n=0}^\infty a_n z^n$ converges absolutely.\\
If $|z|>R$, then $|z| \in A$, so $\sum_{n=0}^\infty a_n z^n$ diverges.
\end{proof}
\end{thm}

\begin{defi}
$R$ is called the \emph{radius of convergence} of $\sum_{n=0}^\infty a_n \left(z-a\right)^n$.
\end{defi}

\begin{rem}
This theorem says nothing about convergence or otherwise of the power series when $|z-a|=R$.\\
\end{rem}

\begin{eg}
$\bullet$ 1)
\begin{equation*}
\begin{aligned}
\sum_{n=0}^\infty z^n
\end{aligned}
\end{equation*}
converges if $|z|<1$ (to $\frac{1}{1-z}$).\\
When $|z| \geq 1$, then $|z^n|\geq 1$ $\forall n$, so $z^n \not\to 0$ and $\sum z^n$ is divergent.\\
It follows that $R=1$.\\
$\bullet$ 2)
\begin{equation*}
\begin{aligned}
\sum_{n=1}^\infty \frac{z^n}{n}
\end{aligned}
\end{equation*}
converges if $z=-1$ by the alternating series test. So $R\geq 1$.\\
On the other hand, this diverges when $z=1$ (harmonic series). So $R\leq 1$.\\
So $R=1$.\\
(in fact the series converges $\forall z$ s.t. $|z|=1, z\neq 1$).\\
$\bullet$ 3)
\begin{equation*}
\begin{aligned}
\sum_{n=1}^\infty \frac{z^n}{n^2}
\end{aligned}
\end{equation*}
Here
\begin{equation*}
\begin{aligned}
|z|\leq 1 \implies |\frac{z^n}{n^2}|\leq\frac{1}{n^2}
\end{aligned}
\end{equation*}
So by the comparison test, the series converges absolutely. So $R\geq 1$.\\
When $|z|>1$ then $|\frac{z^n}{n^2}|=\frac{|z^n|}{n^2} \to \infty$ as $n\to\infty$. So the series diverges. So $R\leq 1$.\\
So $R=1$ (the series converges absolutely for all $|z|=1$).
\end{eg}

\begin{thm}
Assume
\begin{equation*}
\begin{aligned}
\sum_{n=0}^\infty a_n \left(z-a\right)^n
\end{aligned}
\end{equation*}
has radius of convergence $R>0$.\\
Let $f: D\left(a,R\right) \to \C$ ($D\left(a,R\right)$ is the disc with centre $a$ and radius $R$) be defined by
\begin{equation*}
f\left(z\right) = \sum_{n=0}^\infty a_n \left(z-a\right)^n
\begin{aligned}
\end{aligned}
\end{equation*}
Then
\begin{equation*}
\begin{aligned}
\sum_{n=1}^\infty n a_n \left(z-a\right)^{n-1}
\end{aligned}
\end{equation*}
also has radius of convergence $R$.\\
Setting
\begin{equation*}
\begin{aligned}
&g:D\left(a,R\right) \to \C,\\
&g\left(z\right) = \sum_{n=1}^\infty n a_n \left(z-a\right)^{n-1}
\end{aligned}
\end{equation*}
we have $f$ is complex differentiable on $D\left(a,R\right)$, and
\begin{equation*}
\begin{aligned}
f'\left(z\right) = g\left(z\right) \forall z\in D\left(a,R\right).
\end{aligned}
\end{equation*}
\begin{rem}
So we can differentiate a power series term-by-term inside the radius of convergence. i.e.,
\begin{equation*}
\begin{aligned}
\frac{d}{dz}\sum_{n=0}^\infty = \sum_{n=0}^\infty \frac{d}{dz}
\end{aligned}
\end{equation*}
But this is dangerous in general.
\end{rem}
\end{thm}

\begin{coro}
A power series is infinitely complex differentiable inside the radius of convergence.
\end{coro}

\begin{eg}(non-examinable)\\
The previous theorem tells that $\exp$ is differentiable. We'll deduce that
\begin{equation*}
\begin{aligned}
\exp\left(z+w\right) = \exp z \cdot \exp w
\end{aligned}
\end{equation*}
\begin{proof}
Let
\begin{equation*}
\begin{aligned}
f\left(z\right) = \exp\left(z\right)\cdot\exp\left(-z\right).
\end{aligned}
\end{equation*}
So $f' \equiv 0$(?), i.e. $f \equiv 1$, so $\exp\left(-z\right) = \frac{1}{\exp\left(z\right)}$.\\
Then fix $w$, $g\left(z\right) = \exp\left(z+w\right)\exp\left(-z\right)$. So $g' \equiv 0$, $g\equiv \exp w$.\\
So $\forall z$,
\begin{equation*}
\begin{aligned}
\exp\left(z+w\right)\exp\left(-z\right) = \exp w,\\
\exp\left(z+w\right) = \exp\left(z\right) \exp\left(w\right)
\end{aligned}
\end{equation*}
\end{proof}
\end{eg}

\begin{thm}(non-examinable)\\
Suppose
\begin{equation*}
\begin{aligned}
f: D\left(a,R\right) \to \C
\end{aligned}
\end{equation*}
is complex differentiable. Then $\exists \left(a_n\right)_{n=0}^\infty$ in $\C$, s.t.
\begin{equation*}
\begin{aligned}
f\left(z\right) = \sum_{n=0}^\infty a_n \left(z-a\right)^n \forall z\in D\left(a,R\right)
\end{aligned}
\end{equation*}
\begin{proof}
WLOG let $a=0$. Fix $z\in D\left(0,R\right)$. Fix $\delta >0$ s.t. $|z|+\delta < R$ for $0<|h| < \delta$. Then
\begin{equation}\label{eq:2}
\begin{aligned}
|\frac{f\left(z+h\right)-f\left(z\right)}{h} - g\left(z\right)| &= \sum_{n=1}^\infty |a_n| |\left[\frac{\left(z+h\right)^n-z^n}{h} - nz^{n-1}\right]|
\end{aligned}
\end{equation}
Then look at the second term, i.e. (let $\delta = |h|$)
\begin{equation*}
\begin{aligned}
|\frac{\left(z+h\right)^n-z^n-hnz^{n-1}}{h}| &\leq \sum_{k=2}^n {n \choose k} |z|^{n-k} \delta^{k}\frac{|h|}{\delta^2}\\
&\leq \left(|z|+\delta\right)^n \frac{|h|}{\delta^2}
\end{aligned}
\end{equation*}
So (\ref{eq:2}) is at most
\begin{equation*}
\begin{aligned}
\left(\sum_{n=0}^\infty |a_n| \left(|z|+\delta\right)^n\right) \frac{|h|}{\delta^2}\to 0
\end{aligned}
\end{equation*}
as $h\to 0$.
\end{proof}
\end{thm}

\begin{coro}(non-examinable)\\
If
\begin{equation*}
\begin{aligned}
f: D\left(a,R\right) \to \C
\end{aligned}
\end{equation*}
is complex differentiable, then it is infinitely complex differentiable (holomorphic).\\
\begin{equation*}
\begin{aligned}
f^{\left(k\right)} \left(z\right) = \sum_{n=k}^\infty n\left(n-1\right)\left(n-2\right) ... \left(n-k+1\right) a_n \left(z-a\right)^{n-k}
\end{aligned}
\end{equation*}
So
\begin{equation*}
\begin{aligned}
f^{\left(k\right)} \left(a\right) = n! a_n
\end{aligned}
\end{equation*}
So
\begin{equation*}
\begin{aligned}
f\left(z\right) = \sum_{n=0}^\infty \frac{f^{\left(n\right)} \left(a\right)}{n!}\left(z-a\right)^n
\end{aligned}
\end{equation*}
is the taylor series!
\end{coro}

\begin{coro}(non-examinable)\\
Let
\begin{equation*}
\begin{aligned}
f,g: D\left(a,R\right) \to \C
\end{aligned}
\end{equation*}
be complex differentiable.\\
Suppose $\exists \delta > 0\left(\delta < R\right)$ s.t.
\begin{equation*}
\begin{aligned}
f \equiv g \text{  on  } D\left(a,\delta\right)
\end{aligned}
\end{equation*}
Then
\begin{equation*}
\begin{aligned}
f \equiv g \text{  on  } D\left(a,R\right).
\end{aligned}
\end{equation*}
\end{coro}

\newpage
\section{Integration}
Suppose $\left[a,b\right]$ is a closed bounded interval with $a\leq b$, and
\begin{equation*}
\begin{aligned}
f: \left[a,b\right] \to \R
\end{aligned}
\end{equation*}
is a bounded function, i.e. $\exists C$ s.t. $|f\left(t\right)| \leq C$ $\forall t\in \left[a,b\right]$.\\

A \emph{dissection of $\left[a,b\right]$} is a finite sequence
\begin{equation*}
\begin{aligned}
\D: a=x_0 < x_1 < x_2 < ... < x_n = b
\end{aligned}
\end{equation*}
The \emph{lower sum of $f$ w.r.t. $\D$} is
\begin{equation*}
\begin{aligned}
\S_{\D} \left(f\right) = \sum_{k=1}^n \left(x_k - x_{k-1}\right) \inf f\left[x_{k-1},x_k\right] 
\end{aligned}
\end{equation*}
The \emph{upper sum of $f$ w.r.t. $\D$} is
\begin{equation*}
\begin{aligned}
S_{\D} \left(f\right) = \sum_{k=1}^n \left(x_k - x_{k-1}\right) \sup f\left[x_{k-1},x_k\right] 
\end{aligned}
\end{equation*}
\begin{notation}
For $A\subset \left[a,b\right]$,
\begin{equation*}
\begin{aligned}
\sup_A f = \sup \left\{f\left(x\right)|x\in A\right\}
\end{aligned}
\end{equation*}
\end{notation}

Note that $\S_{\D}\left(f\right) \leq S_{\D}\left(f\right)$.
\begin{defi}
$\D'$ is a \emph{refinement} of $\D$ if it contains all the points in $\D$. Write $\D \leq \D'$.
\end{defi}

\begin{lemma}
Let $f:\left[a,b\right]\to \R$ be a bounded function. Let $\D,\D'$ be dissections of $[a,b]$ with $\D \leq \D'$. Then
\begin{equation*}
\begin{aligned}
\S_\D\left(f\right) \leq \S_{\D'}\left(f\right) \leq S_{\D'}\left(f\right) \leq S_\D\left(f\right).
\end{aligned}
\end{equation*}
\begin{proof}
Say $\D: a=x_0 < x_1 < ... < x_n = b$.\\
We may assume that $\D'$ has only one extra point $c$, then the rest can be done by induction.\\
Choose $k$ s.t. $x_{k-1} < c < x_k$. Then
\begin{equation*}
\begin{aligned}
\inf_{\left[x_{k-1},x_k\right]} f \cdot \left(x_k - x_{k-1}\right) &= \left(c-x_{k-1}\right) \cdot \inf_{\left[x_{k-1},x_k\right]} f + \left(x_k+c\right)\inf_{\left[x_{k-1},x_k\right]} f\\
&\leq \left(c-x_{k-1}\right) \inf_{\left[x_{k-1},c\right]}  f + \left(x_k-c\right) \inf_{\left[c,x_k\right]} f
\end{aligned}
\end{equation*}
We obtain
\begin{equation*}
\begin{aligned}
\S_\D\left(f\right) \leq \S_{\D'} \left(f\right).
\end{aligned}
\end{equation*}
Similarly,
\begin{equation*}
\begin{aligned}
S_{\D'}\left(f\right) \leq S_\D \left(f\right)
\end{aligned}
\end{equation*}
and we always have
\begin{equation*}
\begin{aligned}
\S_{\D'} \left(f\right) \leq S_{\D'} \left(f\right)
\end{aligned}
\end{equation*}
So done.
\end{proof}
\end{lemma}

\begin{coro}
If $\D_1$, $\D_2$ are two dissections of $\left[a,b\right]$, then
\begin{equation*}
\begin{aligned}
\S_{\D_1}\left(f\right) \leq S_{\D_2} \left(f\right)
\end{aligned}
\end{equation*}
Here $f$ is as in the previous lemma.
\begin{proof}
Let $D=\D_1 \cup \D_2$, the common refinement of $\D_1$ and $\D_2$, i.e. the union of the points of $\D_1$ and $\D_2$.\\
By the previous lemma,
\begin{equation*}
\begin{aligned}
\S_{\D_1}\left(f\right) \leq \S_{\D}\left(f\right) \leq S_{\D}\left(f\right) \leq S_{\D_2}\left(f\right).
\end{aligned}
\end{equation*}
\end{proof}
\end{coro}

\begin{defi}
Let $f:\left[a,b\right]\to \R$ be a bounded function.\\
The \emph{upper (Riemann) integral} of $f$ on $\left[a,b\right]$ is
\begin{equation*}
\begin{aligned}
\bar{\int_a^b} f = \inf_\D S_\D \left(f\right)
\end{aligned}
\end{equation*}
i.e. (the inf taken over all dissections $\D$ of $\left[a,b\right]$.\\
The \emph{lower (Riemann) integral} of $f$ on $\left[a,b\right]$ is
\begin{equation*}
\begin{aligned}
\underline{\int_a^b} f = \sup_\D \S_\D\left(f\right)
\end{aligned}
\end{equation*}
By the previous corollary, given a dissection $\D_1$, $\S_{\D_1} \left(f\right)$ is a lower bound of $\left\{S_\D \left(f\right)|\D \text{  any dissection of  }\left[a,b\right]\right\}$. So the upper integral exists and is at least $\S_{\D_1}\left(f\right)$.\\
Since $\D_1$ was arbitrary, $\bar{\int_a^b} f$ is an upper bound of $\left\{\S_\D \left(f\right)|\D \text{  any dissection of  }\left[a,b\right]\right\}$. Hence $\underline{\int_a^b} f$ exists and
\begin{equation*}
\begin{aligned}
\underline{\int_a^b} f \leq \bar{\int_a^b} f.
\end{aligned}
\end{equation*}
\end{defi}

\begin{defi} (Integrability)\\
Say $f$ is (Riemann) integrable on $\left[a,b\right]$ if it is bounded, and
\begin{equation*}
\begin{aligned}
\underline{\int_a^b} f = \bar{\int_a^b} f
\end{aligned}
\end{equation*}
We define the integral of $f$ on $\left[a,b\right]$ to be the common value of them, and denote it by
\begin{equation*}
\begin{aligned}
\int_a^b f
\end{aligned}
\end{equation*}
or
\begin{equation*}
\begin{aligned}
\int_a^b f\left(t\right) dt.
\end{aligned}
\end{equation*}
\end{defi}

\begin{prop}
Suppose $f:\left[a,b\right]\to\R$ is integrable and let
\begin{equation*}
\begin{aligned}
m=\inf_{\left[a,b\right]} f\\
M=\sup_{\left[a,b\right]} f
\end{aligned}
\end{equation*}
Then
\begin{equation*}
\begin{aligned}
m\left(b-a\right) \leq \int_a^b f \leq M\left(b-a\right)
\end{aligned}
\end{equation*}
and
\begin{equation*}
\begin{aligned}
|\int_a^b f| \leq \left(b-a\right) \sup_{\left[a,b\right]} |f|
\end{aligned}
\end{equation*}
\begin{proof}
Consider $\D:a<b$. Then
\begin{equation*}
\begin{aligned}
\S_\D \left(f\right) = m\left(b-a\right)\\
S_\D\left(f\right) = M\left(b-a\right)
\end{aligned}
\end{equation*}
In addition
\begin{equation*}
\begin{aligned}
M&\leq \sup_{\left[a,b\right]} |f|\\
m&\geq -\sup_{\left[a,b\right]} |f|
\end{aligned}
\end{equation*}
So the result follows.
\end{proof}
\end{prop}

\begin{eg}
$\bullet$ 1) If $f\left(x\right)=c$ $\forall x\in \left[a,b\right]$ then $f$ is integrable, and
\begin{equation*}
\begin{aligned}
\int_a^b f\left(t\right) dt = c\left(b-a\right)
\end{aligned}
\end{equation*}
Note that $m=M=c$. Then
\begin{equation*}
\begin{aligned}
c\left(b-a\right)=m\left(b-a\right) \leq \underline{\int_a^b}f\leq\bar{\int_a^b} f\leq M\left(b-a\right) = c\left(b-a\right)
\end{aligned}
\end{equation*}
So the upper and lower integral are equal and the value is $c\left(b-a\right)$.
\end{eg}

$\bullet$ 2) $f:\left[0,1\right] \to \R$, $f\left(x\right) = x$.\\
Consider
\begin{equation*}
\begin{aligned}
\D_n: 0\leq\frac{1}{n}\leq\frac{2}{n}\leq...\leq\frac{n}{n}=1
\end{aligned}
\end{equation*}
Then
\begin{equation*}
\begin{aligned}
\S_{\D_n}\left(f\right) = \sum_{k=1}^n \frac{1}{n}\frac{k-1}{n} = \frac{\left(n-1\right)n}{2n^2} = \frac{n-1}{2n} \to \frac{1}{2}
\end{aligned}
\end{equation*}
So
\begin{equation*}
\begin{aligned}
\underline{\int_0^1} f \geq \sup_n \S_{\D_n}\left(f\right) = \frac{1}{2}
\end{aligned}
\end{equation*}
Similarly,
\begin{equation*}
\begin{aligned}
S_{\D_n}\left(f\right) = \sum_{k=1}^n \frac{1}{n}\frac{k}{n} = \frac{\left(n+1\right)n}{2n^2} = \frac{n+1}{2n}\to \frac{1}{2}
\end{aligned}
\end{equation*}
So
\begin{equation*}
\begin{aligned}
\bar{\int_0^1}f \leq \inf_n S_{\D_n}\left(f\right) = \frac{1}{2}
\end{aligned}
\end{equation*}
So it follows that $f$ is integrable and is equal to $\frac{1}{2}$.

\begin{thm}
A bounded function $f:\left[a,b\right]\to\R$ is integrable if and only if
\begin{equation*}
\begin{aligned}
\forall \epsilon > 0 \exists \D \text{ s.t. } S_\D \left(f\right) - \S_\D\left(f\right) < \epsilon
\end{aligned}
\end{equation*}
\begin{proof}
$\bullet$ Forward: Since
\begin{equation*}
\begin{aligned}
\int_a^b = \inf_\D S_\D \left(f\right) = \sup_\D \S_\D\left(f\right)
\end{aligned}
\end{equation*}
we know that $\exists \D_1,\D_2$ with
\begin{equation*}
\begin{aligned}
S_{\D_1}\left(f\right) < \int_a^b f+\frac{\epsilon}{2},\\
\S_{\D_2}\left(f\right) > \int_a^b f-\frac{\epsilon}{2}.
\end{aligned}
\end{equation*}
Set $\D = \D_1 \cup \D_2$. Then
\begin{equation*}
\begin{aligned}
\int_a^b f-\frac{\epsilon}{2} < \S_{\D_2} \left(f\right) \leq \S_\D \left(f\right) \leq S_\D \left(f\right) \leq S_{\D_1}\left(f\right) < \int_a^b f + \frac{\epsilon}{2}
\end{aligned}
\end{equation*}

$\bullet$ Backward: Suppose
\begin{equation*}
\begin{aligned}
S_\D \left(f\right) - \S_\D \left(f\right) < \epsilon
\end{aligned}
\end{equation*}
Then
\begin{equation*}
\begin{aligned}
\bar{\int_a^b}f \leq S_\D\left(f\right) < \S_\D \left(f\right) + \epsilon \leq \underline{\int_a^b}f+\epsilon
\end{aligned}
\end{equation*}
So
\begin{equation*}
\begin{aligned}
\bar{\int_a^b}f \leq \underline{\int_a^b}f + \epsilon \forall \epsilon > 0
\end{aligned}
\end{equation*}
Hence
\begin{equation*}
\begin{aligned}
\bar{\int_a^b}f \leq \underline{\int_a^b}f
\end{aligned}
\end{equation*}
So $f$ is integrable.
\end{proof}
\end{thm}

\begin{coro}
A bounded function $f:\left[a,b\right]\to\R$ is integrable if and only if there exists a sequence $\D_n$ of dissections s.t.
\begin{equation*}
\begin{aligned}
S_{\D_n} \left(f\right) - \S_{\D_n} \left(f\right) \to 0
\end{aligned}
\end{equation*}
as $n\to\infty$. Then both $\S_{\D_n}\left(f\right)$ and $S_{\D_n}\left(f\right)$ converge to $\int_a^b f$.\\
Moreover, if
\begin{equation*}
\begin{aligned}
\D_n: a = x_0^{\left(n\right)}<x_1^{\left(n\right)}<...<x_{m_n}^{\left(n\right)}=b
\end{aligned}
\end{equation*}
and
\begin{equation*}
\begin{aligned}
\xi_k^{\left(n\right)} \in \left[x_{k-1}^{\left(n\right)},x_k^{\left(n\right)}\right]
\end{aligned}
\end{equation*}
For $k=1,2,...,m_n$. Then
\begin{equation*}
\begin{aligned}
\sum_{k=1}^{m_n} f\left(\xi_k^{\left(n\right)}\right)\left(x_k^{\left(n\right)}-x_{k-1}^{\left(n\right)}\right) \to \int_a^b f
\end{aligned}
\end{equation*}
as $n\to\infty$.
\begin{proof}
The first part is immediate from the previous theorem.\\
For the second part,
\begin{equation*}
\begin{aligned}
\int_a^b f &\leq S_{\D_n}\left(f\right)\\
&= \S_{\D_n} \left(f\right) + \left(S_{\D_n} \left(f\right) - \S_{\D_n}\left(f\right)\right)\\
&\leq \int_a^b f + \left(S_{\D_n} \left(f\right) - \S_{\D_n}\left(f\right)\right)
\end{aligned}
\end{equation*}
Hence
\begin{equation*}
\begin{aligned}
S_{\D_n}\left(f\right) \to \int_a^b f
\end{aligned}
\end{equation*}
as $n\to \infty$. Then
\begin{equation*}
\begin{aligned}
\S_{\D_n}\left(f\right) = S_{\D_n}\left(f\right) - \left(S_{\D_n}\left(f\right) - \S_{\D_n}\left(f\right)\right) \to \int_a^b f
\end{aligned}
\end{equation*}
\end{proof}
Together with
\begin{equation*}
\begin{aligned}
\inf_{\left[x_{k-1}^{\left(n\right)},x_k^{\left(n\right)}\right]} f \leq f\left(\xi_k^{\left(n\right)} \right) \leq \sup_{\left[x_{k-1}^{\left(n\right)},x_k^{\left(n\right)}\right]} f
\end{aligned}
\end{equation*}
Hence
\begin{equation*}
\begin{aligned}
\S_{\D_n}\left(f\right) \leq \sum_{k=1}^{m_n} f\left(\xi_k^{\left(n\right)}\right) \left(x_k^{\left(n\right)} - x_{k-1}^{\left(n\right)}\right) \leq S_{\D_n} \left(f\right)
\end{aligned}
\end{equation*}
\end{coro}

\begin{rem}
\emph{Darboux}: if $f$ is integrable and 
\begin{equation*}
\begin{aligned}
\D_n : a=x_0^{\left(n\right)}<x_1^{\left(n\right)}<...<x_{m_n}^{\left(n\right)} = b
\end{aligned}
\end{equation*}
is such that 
\begin{equation*}
\begin{aligned}
|\D_n| = \max_{1\leq k \leq m_n} \left(x_k^{\left(n\right)} - x_{k-1}^{\left(n\right)}\right) \to 0
\end{aligned}
\end{equation*}
Then
\begin{equation*}
\begin{aligned}
S_{\D_n} \left(f\right) - \S_{\D_n} \left(f\right) \to 0
\end{aligned}
\end{equation*}
\end{rem}

\begin{lemma}
Let $f,g:\left[a,b\right] \to \R$ be bounded functions. Assume there exists $k\geq 0$ such that
\begin{equation*}
\begin{aligned}
|f\left(x\right) - f\left(y\right)| \leq K|g\left(x\right) - g\left(y\right)| \forall x,y\in\left[a,b\right]
\end{aligned}
\end{equation*}
Then if $g$ is integrable, then $f$ is also integrable.
\begin{proof}
Given $\epsilon>0$, there exists $\D$ such that
\begin{equation*}
\begin{aligned}
S_\D \left(g\right) - \S_\D\left(g\right) < \epsilon
\end{aligned}
\end{equation*}
Now let 
\begin{equation*}
\begin{aligned}
\D: a=x_0 < x_1 < ... < x_n = b
\end{aligned}
\end{equation*}
And let $I=\left[x_{k-1},x_k\right]$. Then
\begin{equation*}
\begin{aligned}
\sup_I f - \inf_I f &= \sup_{x,y\in I} |f\left(x\right)-f\left(y\right)| \\
&\leq K\sup_{x,y\in I} |g\left(x\right)-g\left(y\right)|\\
&= K\left(\sup_I g - \inf_I g\right)
\end{aligned}
\end{equation*}
Multiply by $|I| = x_k - x_{k-1}$ and sum over $k$,
\begin{equation*}
\begin{aligned}
S_\D \left(f\right) - \S_\D \left(f\right) \leq K\left(S_\D\left(g\right) - \S_\D \left(g\right) \right) < k\epsilon
\end{aligned}
\end{equation*}
As $\epsilon$ is arbitrary, $f$ is integrable.
\end{proof}
\end{lemma}

\begin{thm}
Let $f,g: \left[a,b\right] \to \R$ be integrable functions. Then\\
1) $\lambda f + \mu g$ is integrable, and
\begin{equation*}
\begin{aligned}
\int_a^b\left(\lambda f+\mu y\right) = \lambda \int_a^b f + \mu \int_a^b g
\end{aligned}
\end{equation*}
2) If $f\leq g$, then 
\begin{equation*}
\begin{aligned}
\int_a^b f \leq \int_a^b g
\end{aligned}
\end{equation*}
3) $|f|$ is integrable, and
\begin{equation*}
\begin{aligned}
|\int_a^b f | \leq \int_a^b |f|
\end{aligned}
\end{equation*}
4) $\max\left(f,g\right)$, $\min\left(f,g\right)$ are integrable;\\
5) $f \cdot g$ is integrable, and (Cauchy-Schwarz inequality)
\begin{equation*}
\begin{aligned}
|\int_a^b fg | \leq \left(\int_a^b f^2 \right)^\frac{1}{2} \left(\int_a^b g^2 \right)^\frac{1}{2}
\end{aligned}
\end{equation*}
\begin{proof}
1) Enough to consider $f+g$, $\lambda f$ for $\lambda \geq 0$, and $-f$.\\
We know from a previous corollary that there exists a sequence $\D_n$ of dissections of $\left[a,b\right]$ s.t. $S_{\D_n} \left(f\right)$, $\S_{\D_n} \left(f\right)$ both converge to $\int_a^b f$, and$S_{\D_n} \left(g\right)$, $\S_{\D_n} \left(g\right)$ both converge to $\int_a^b g$.\\
For an interval $I\subset\left[a,b\right]$, we have
\begin{equation*}
\begin{aligned}
&\sup_I \left(f+g\right) \leq \sup_I f + \sup_I g\\
&\inf_I \left(f+g\right) \geq \inf_I f + \inf_I g\\
&\S_{\D_n} \left(f\right) + \S_{\D_n} \left(g\right) \leq \S_{\D_n} \left(f+g\right) \leq S_{\D_n} \left(f+g\right) \leq S_{\D_n} \left(f\right) + S_{\D_n} \left(g\right)
\end{aligned}
\end{equation*}
As $n\to \infty$, LHS and RHS both tend to $\int_a^b f + \int_a^b g$. So $f+g$ is integrable and
\begin{equation*}
\begin{aligned}
\int_a^b \left(f+g\right) = \int_a^b f + \int_a^b g
\end{aligned}
\end{equation*}
Also
\begin{equation*}
\begin{aligned}
\sup_I \left(\lambda f\right) = \lambda \sup_I f\\
\inf_I \left(\lambda f\right) = \lambda \inf_I f
\end{aligned}
\end{equation*}
Hence
\begin{equation*}
\begin{aligned}
S_{\D_n} \left(\lambda f\right) = \lambda S_{\D_n} \left(f\right) \to \lambda \int_a^b f\\
\S_{\D_n} \left(\lambda f\right) = \lambda \S_{\D_n} \left(f\right) \to \lambda \int_a^b f
\end{aligned}
\end{equation*}
By the previous corollary, $\lambda f$ is integrable and 
\begin{equation*}
\begin{aligned}
\int_a^b \left(\lambda f\right) = \lambda \int_a^b f
\end{aligned}
\end{equation*}
Finally,
\begin{equation*}
\begin{aligned}
\sup_I \left(-f\right) = -\inf_I f,\\
\inf_I \left(-f\right) = -\sup_I f
\end{aligned}
\end{equation*}
We get
\begin{equation*}
\begin{aligned}
S_{\D_n} \left(-f\right) = -\S_{\D_n} \left(f\right) \to -\int_a^b f,\\
\S_{\D_n} \left(-f\right) = -S_{\D_n} \left(f\right) \to -\int_a^b f
\end{aligned}
\end{equation*}
Hence $-f$ is integrable and
\begin{equation*}
\begin{aligned}
\int_a^b \left(-f\right) = -\int_a^b f.
\end{aligned}
\end{equation*}
2) If $f\leq g$ then $g-f\geq 0$. Hence
\begin{equation*}
\begin{aligned}
\int_a^b g - \int_a^b f = \int_a^b \left(g-f\right) \geq \left(b-a\right) \inf_{\left[a,b\right]} \left(g-f\right) \geq 0
\end{aligned}
\end{equation*}
3)Note that
\begin{equation*}
\begin{aligned}
||f\left(x\right)| - f\left(y\right)|| \leq |f\left(x\right) - f\left(y\right) |
\end{aligned}
\end{equation*}
for all $x,y\in\left[a,b\right]$. So by the previous lemma, together with the assumption that $f$ is integrable, we know that $|f|$ is integrable. Since
\begin{equation*}
\begin{aligned}
f\leq |f|, -f \leq |f|,
\end{aligned}
\end{equation*}
by 2) we know that
\begin{equation*}
\begin{aligned}
\int_a^b f \leq \int_a^b |f|,\\
-\int_a^b f = \int_a^b \left(-f\right) \leq \int_a^b |f|
\end{aligned}
\end{equation*}
Hence
\begin{equation*}
\begin{aligned}
|\int_a^b f| \leq \int_a^b |f|.
\end{aligned}
\end{equation*}
4) Given $s,t\in \R$,
\begin{equation*}
\begin{aligned}
\max\left(s,t\right) = \frac{s+t}{2} + \frac{|s-t|}{2}
\end{aligned}
\end{equation*}
So if $h=\max\left(f,g\right)$, then for all $x\in\left[a,b\right]$,
\begin{equation*}
\begin{aligned}
h\left(x\right) = \frac{f\left(x\right) + g\left(x\right)}{2} + \frac{|f\left(x\right) - g\left(x\right)|}{2}
\end{aligned}
\end{equation*}
Hence $h$ is integrable by 1) and 3).\\
($\min\left(f,g\right)$ can be proved similarly).\\
5) Let
\begin{equation*}
\begin{aligned}
M=\sup_{\left[a,b\right]} |f|
\end{aligned}
\end{equation*}
This exists since $f$ must be bounded on $\left[a,b\right]$. Then for all $x,y\in\left[a,b\right]$,
\begin{equation*}
\begin{aligned}
|f^2\left(x\right) - f^2\left(y\right)| = |f\left(x\right) - f\left(y\right)| \cdot |f\left(x\right) + f\left(y\right)| \leq 2M |f\left(x\right) - f\left(y\right)|
\end{aligned}
\end{equation*}
So by the previous lemma, $f^2$ is integrable. Hence by 1),
\begin{equation*}
\begin{aligned}
fg = \frac{1}{2}\left[\left(f+g\right)^2 - f^2 - g^2\right]
\end{aligned}
\end{equation*}
is integrable.\\
Next, we have
\begin{equation*}
\begin{aligned}
0\leq \int_a^b \left(f-\lambda g\right)^2 = \int_a^b f^2 + \lambda ^2 \int_a^b g^2 - 2\lambda \int_a^b fg
\end{aligned}
\end{equation*}
(using 2) and 1) respectively) for all $\lambda\in\R$.\\
Now put
\begin{equation*}
\begin{aligned}
\lambda = \frac{\int_a^b fg}{\int_a^b g^2}
\end{aligned}
\end{equation*}
After \emph{some algebra}, we obtain the required inequality.
\end{proof}
\end{thm}

\begin{prop}
1) Assume $h:\left[a,b\right]\to\R$ satisfies that $h\left(x\right)=0$ for all but finitely many $x$. Then $h$ is integrable, and
\begin{equation*}
\begin{aligned}
\int_a^b h = 0
\end{aligned}
\end{equation*}
2) Assume $f:\left[a,b\right] \to \R$ is integrable, and $g\left(x\right) = f\left(x\right)$ for all but finitely many $x$. Then $g$ is integrable, and
\begin{equation*}
\begin{aligned}
\int_a^b g = \int_a^b f
\end{aligned}
\end{equation*}
\begin{proof}
Choose
\begin{equation*}
\begin{aligned}
a=c_0 < c_1 < ... < c_n = b
\end{aligned}
\end{equation*}
s.t. $h\left(x\right) = 0$ $\forall x\in \left[a,b\right]\backslash\left\{c_0,c_1,...,c_n\right\}$ (it is not necessary for all of $f\left(c_i\right)$ to be non-zero).\\
If
\begin{equation*}
\begin{aligned}
M=\max_{0\leq i \leq n} |h\left(c_i\right)|
\end{aligned}
\end{equation*}
Then $|h\left(x\right)| \leq M$ $\forall x$, so $h$ is bounded.\\
Fix $\delta > 0$ s.t.
\begin{equation*}
\begin{aligned}
\delta < \frac{1}{2}\left(c_k - c_{k-1}\right)
\end{aligned}
\end{equation*}
for all $1\leq k \leq n$.\\
Now consider $\D: a,a+\delta, c_1-\delta, c_1+\delta, c_2-\delta, c_2+\delta,...,c_n-\delta,c_n=b$. We have
\begin{equation*}
\begin{aligned}
\inf_{\left[c_{k-1}+\delta, c_k-\delta\right]} h = \sup_{\left[c_{k-1}+\delta, c_k-\delta\right]} h = 0
\end{aligned}
\end{equation*}
for all $1\leq k \leq n$.\\
If $I=\left[c_k - \delta, c_k + \delta\right]$ ($1\leq k \leq n-1$) or $I=\left[a,a+\delta\right]$ or $I=\left[b-\delta,b\right]$, we have
\begin{equation*}
\begin{aligned}
\sup_I h \leq M\\
\inf_I h \geq -M
\end{aligned}
\end{equation*}
Hence
\begin{equation*}
\begin{aligned}
\S_\D \left(h\right) \geq \left(n-1\right) 2\delta \left(-M\right) + 2\delta \left(-M\right) = -2Mn\delta,\\
S_\D \left(h\right) \leq \left(n-1\right)2\delta M + 2\delta M = 2Mn \delta
\end{aligned}
\end{equation*}
Hence
\begin{equation*}
\begin{aligned}
-2Mn\delta \leq \underline{\int_a^b}h \leq \bar{\int_a^b} h \leq 2Mn\delta
\end{aligned}
\end{equation*}
But $\delta$ is arbitrary. So $h$ is integrable and the integral is 0.\\
2) $g=f+\left(g-f\right)$. By 1) $g-f$ is integrable, and
\begin{equation*}
\begin{aligned}
\int_a^b \left(g-f\right) = 0
\end{aligned}
\end{equation*}
So
\begin{equation*}
\begin{aligned}
\int_a^b g = \int_a^b f
\end{aligned}
\end{equation*}
\end{proof}
\end{prop}

\begin{thm}
Every continuous function is integrable.
\begin{proof}
Let $f:\left[a,b\right]\to \R$ be continuous. That means
\begin{equation*}
\begin{aligned}
\forall x\in\left[a,b\right] \forall \epsilon < 0 \exists \delta > 0 \forall y \in \left[a,b\right] |y-x|<\delta \implies |f\left(y\right) - f\left(x\right)|<\epsilon
\end{aligned}
\end{equation*}
We claim that
\begin{equation*}
\begin{aligned}
\forall \epsilon < 0 \exists \delta > 0 \forall x,y\in\left[a,b\right] |y-x|<\delta \implies |f\left(y\right)-f\left(x\right)|<\epsilon
\end{aligned}
\end{equation*}
(actually this is the definition of uniform convergence).\\
Proof of claim: Suppose otherwise. Then
\begin{equation*}
\begin{aligned}
\exists \epsilon > 0 \forall \delta > 0 \exists x,y\in\left[a,b\right] |y-x|<\delta, |f\left(y\right) - f\left(x\right)| \geq \epsilon
\end{aligned}
\end{equation*}
In particular, 
\begin{equation*}
\begin{aligned}
\forall n\in \N, \exists x_n, y_n \in \left[a,b\right] \text{  s.t.  } |x_n-y_n|<\frac{1}{n}, |f\left(x_n\right) - f\left(y_n\right) | \geq \epsilon
\end{aligned}
\end{equation*}
Now $\left(x_n\right)$ is bounded. So by B-W theorem, $\exists k_1<k_2<k_3<...$ in $\N$ s.t. $\left(x_{k_n}\right)_{n=1}^\infty$ converges to some $x\in \R$.\\
Since $a\leq x_{k_n} \leq b$ for all $n$, we have $x\in \left[a,b\right]$. Then since
\begin{equation*}
\begin{aligned}
|y_{k_n}-x_{k_n} | < \frac{1}{k_n} \leq \frac{1}{n} \to 0
\end{aligned}
\end{equation*}
as $n\to\infty$, so
\begin{equation*}
\begin{aligned}
y_{k_n} = x_{k_n} + \left(y_{k_n}-x_{k_n}\right) \to x
\end{aligned}
\end{equation*}
as $n\to \infty$.\\
As $f$ is continuous, 
\begin{equation*}
\begin{aligned}
\epsilon\leq|f\left(x_{k_n}\right) - f\left(y_{k_n}\right)| \to |f\left(x\right) - f\left(x\right)| = 0
\end{aligned}
\end{equation*}
Contradiction.\\
Now back to the main proof.\\
Given $n\in \N$, choose $\delta_n > 0$ s.t.
\begin{equation*}
\begin{aligned}
\forall x,y, |x-y|<\delta_n \implies |f\left(x\right) - f\left(y\right) | < \frac{1}{n}
\end{aligned}
\end{equation*}
Then choose a dissection $\D_n$ s.t.
\begin{equation*}
\begin{aligned}
|\D_n| < \delta_n\\
(\text{  if  } \D: a=x_0<x_1<...<x_m = b, \text{  then  } |\D| = \max_{1\leq_k\leq m} \left(x_k-x_{k-1}\right) )
\end{aligned}
\end{equation*}
If $I$ is an interval of $\D_n$ then 
\begin{equation*}
\begin{aligned}
\sup_I f - \inf_I f \leq \frac{1}{n}
\end{aligned}
\end{equation*}
Hence
\begin{equation*}
\begin{aligned}
S_{\D_n}\left(f\right) - \S_{\D_n} \left(f\right) \leq \frac{1}{n}\left(b-a\right) \to 0
\end{aligned}
\end{equation*}
as $n\to\infty$. So $f$ is integrable.
\end{proof}
\end{thm}

\begin{thm}
Monotonic functions are integrable.
\begin{proof}
Let $f:\left[a,b\right] \to \R$ be a monotonic function. WLOG let $f$ be increasing (otherwise look at $-f$).\\
Let
\begin{equation*}
\begin{aligned}
\D_n: a+\frac{k}{n}\left(b-a\right), 0\leq n (n\in \N)
\end{aligned}
\end{equation*}
Then
\begin{equation*}
\begin{aligned}
S_{\D_n}\left(f\right) - \S_{\D_n} \left(f\right) &= \sum_{k=1}^n \frac{b-a}{n}\left(\sup_{\left[a+\frac{k-1}{n}\left(b-a\right),a+\frac{k}{n}\left(b-a\right)\right]} f - \inf_{\left[a+\frac{k-1}{n}\left(b-a\right),a+\frac{k}{n}\left(b-a\right)\right]}\right)\\
&= \frac{b-a}{n}\sum_{k=1}^n \left(f\left(a+\frac{k}{n}\left(b-a\right)\right) - f\left(a+\frac{k-1}{n}\left(b-a\right)\right)\right)\\
&= \frac{b-a}{n} \left(f\left(b\right) - f\left(a\right)\right) \to 0
\end{aligned}
\end{equation*}
as $n \to \infty$. So $f$ is integrable by the previous corollary.
\end{proof}
\end{thm}

(A note on the proof of the integral form of Cauchy-Schwarz inequality:
\begin{equation*}
\begin{aligned}
0\leq \int_a^b \left(f-\lambda g\right)^2 = \int_a^b f^2 + \lambda^2 \int_a^b g^2 - s\lambda \int_a^b fg
\end{aligned}
\end{equation*}
for all $\lambda \in \R$.\\
Putting
\begin{equation*}
\begin{aligned}
\lambda = \frac{\int_a^b fg}{\int_a^b g^2}
\end{aligned}
\end{equation*}
yields the result, provided the denominator is not zero.\\
If $\int_a^b g^2 = 0$, then we get
\begin{equation*}
\begin{aligned}
2\lambda \int_a^b fg \leq \int_a^b f^2
\end{aligned}
\end{equation*}
for all $\lambda \in \R$.\\
Since $\lambda$ is arbitrary, this forces
\begin{equation*}
\begin{aligned}
\int_a^b fg = 0
\end{aligned}
\end{equation*}
so the result still holds.)

\begin{thm}
Let $a<b$, $f$ a bounded function on $\left[a,b\right]$.\\
1) If $a<c<b$ and $f$ is integrable on $\left[a,c\right]$ and $\left[c,b\right]$, then it's integrable on $\left[a,b\right]$, and
\begin{equation*}
\begin{aligned}
\int_a^b f = \int_a^c f + \int_c^b f
\end{aligned}
\end{equation*}
2) If $f$ is integrable on $\left[a,b\right]$, then $f$ is integrable on $\left[c,d\right]$ whenever $a\leq c<d\leq b$.
\begin{proof}
1) There exists sequences $\D_n'$ and $\D_n''$ of dissections of $\left[a,c\right]$ and $\left[c,b\right]$ respectively, such that
\begin{equation*}
\begin{aligned}
S_{D'_n} \left(f\right), \S_{\D'_n} \left(f\right) \to \int_a^c f\\
S_{D''_n} \left(f\right), \S_{\D''_n} \left(f\right) \to \int_c^b f
\end{aligned}
\end{equation*}
Then
\begin{equation*}
\begin{aligned}
\D_n = \D'_n \cup \D''_n
\end{aligned}
\end{equation*}
is a dissection of $\left[a,b\right]$, and
\begin{equation*}
\begin{aligned}
S_{\D_n} \left(f\right) = S_{\D'_n}\left(f\right) + S_{\D''_n} \left(f\right) \to \int_a^c f + \int_c^b f\\
\S_{\D_n} \left(f\right) = \S_{\D'_n}\left(f\right) + \S_{\D''_n} \left(f\right) \to \int_a^c f + \int_c^b f
\end{aligned}
\end{equation*}
So $f$ is integrable on $\left[a,b\right]$ and tends to the expected value.\\
2) Given $\epsilon>0$, there is a dissection $\D$ of $\left[a,b\right]$ s.t.
\begin{equation*}
\begin{aligned}
S_\D \left(f\right) - \S_\D \left(f\right) < \epsilon
\end{aligned}
\end{equation*}
WLOG we may assume that $c,d$ are in $\D$ (otherwise add them into $\D$ which refines it, and will make the difference between the upper and lower integral even smaller).\\
So
\begin{equation*}
\begin{aligned}
\D:a=x_0<x_1<...<x_n = b, c= x_{j-1}, d=x_k
\end{aligned}
\end{equation*}
for some $1\leq j \leq k \leq n$.\\
Then
\begin{equation*}
\begin{aligned}
D': x_{j-1} < x_j < ... < x_k
\end{aligned}
\end{equation*}
is a dissection of $\left[c,d\right]$. Then
\begin{equation*}
\begin{aligned}
S_{\D'} \left(f\right) - S_{D'} \left(f\right) &= \sum_{i=j}^k \left(x_i-x_{i-1}\right) \left(\sup_{\left[x_{i-1},x_i\right]} f - \inf_{\left[x_{i-1},x_i\right]} f\right)\\
&\leq \sum_{i=1}^n \left(x_i-x_{i-1}\right) \left(\sup_{\left[x_{i-1},x_i\right]} f - \inf_{\left[x_{i-1},x_i\right]} f\right)\\
&= S_\D \left(f\right) - \S_\D \left(f\right)\\
&<\epsilon
\end{aligned}
\end{equation*}
So $S$ is integrable on $\left[c,d\right]$.
\end{proof}
\end{thm}

\begin{coro}
Let $a,b,f$ be as in the previous theorem. Consider
\begin{equation*}
\begin{aligned}
a=c_0 < c_1 < ... < c_k = b
\end{aligned}
\end{equation*}
Then $f$ is integrable on $\left[a,b\right]$ if and only if $f$ is integrable on $\left[c_{j-1}, c_j\right]$, for all $1\leq j\leq k$, and then
\begin{equation*}
\begin{aligned}
\int_a^b f = \sum_{j=1}^k \int_{c_{j-1}}^{c_j} f
\end{aligned}
\end{equation*}
\end{coro}

\begin{coro}
Piecewise monotonic functions are integrable. $f:\left[a,b\right]\to \R$ is \emph{piecewise monotonic} if there exists $a=c_0 < c_1 < ... < c_k = b$ such that $f$ is monotonic on each $\left[c_{j-1},c_j\right]$.\\
\begin{proof}
By the theorem that monotonic functions are integrable, and the previous corollary.
\end{proof}
\end{coro}

\begin{thm}
If $a<b$, $f$ is a bounded function on $\left[a,b\right]$, continuous at all except finitely many points. Then $f$ is integrable.
\begin{proof}
Choose
\begin{equation*}
\begin{aligned}
a=c_0 < c_1 < ... < c_k = b
\end{aligned}
\end{equation*}
such that $f$ is continuous at $x$ if $x \not\in\left\{c_0, c_1, ..., c_k\right\}$.\\
Let $M = \sup_{\left[a,b\right]} |f|$. Choose $\delta > 0$ such that $\delta < \frac{1}{2}\left(c_j - c_{j-1}\right)$ for all $j$ and $4M\delta k < \frac{\epsilon}{2}$.\\
$f$ is continuous on $\left[c_{j-1}+\delta, c_j-\delta\right]$ for $1\leq j \leq k$. So it's integrable, so there exists a dissection $\D_j$ s.t.
\begin{equation*}
\begin{aligned}
S_{\D_j} \left(f\right) - \S_{\D_j} \left(f\right) < \frac{\epsilon}{2k}
\end{aligned}
\end{equation*}
Now consider
\begin{equation*}
\begin{aligned}
\D = \bigcup_{j=1}^k \D_j \bigcup \left\{a,b\right\}
\end{aligned}
\end{equation*}
which is a dissection of $\left[a,b\right]$. Then
\begin{equation*}
\begin{aligned}
S_\D \left(f\right) - \S_\D\left(f\right) &= \sum_{j=1}^k \left(S_{\D_j} \left(f\right) - \S_{\D_j} \left(f\right)\right) + \text{  contributions from the small segments around  } c_j's\\
&\leq k\cdot \frac{\epsilon}{2k} + 2\delta \cdot 2M \cdot \left(k-1\right) + \delta \cdot 2M \cdot 2\\
&= \frac{\epsilon}{2} + 4M\delta k\\
&< \epsilon
\end{aligned}
\end{equation*}
\end{proof}
\end{thm}

\begin{eg} (a function that is not integrable)\\
Let
\begin{equation*}
\begin{aligned}
f\left(x\right) = \left\{
\begin{array}{ll}
0 & x\in \Q\\
1 & x\not\in \Q
\end{array}
\right.
\end{aligned}
\end{equation*}
for any interval $I\subset \left[0,1\right]$ of positive length, we have
\begin{equation*}
\begin{aligned}
\sup_I f = 1, \inf_I f = 0
\end{aligned}
\end{equation*}
Then for all dissection $\D$, $S_\D\left(f\right)=1$, $\S_\D\left(f\right)=0$. So
\begin{equation*}
\begin{aligned}
\bar{\int_0^1}f = 1 \neq 0 = \underline{\int_0^1} f
\end{aligned}
\end{equation*}
So $f$ is not integrable.
\end{eg}

\begin{defi}
Given $a<b$, $f$ integrable on $\left[a,b\right]$, we \emph{define}
\begin{equation*}
\begin{aligned}
\int_b^a = -\int_a^b f
\end{aligned}
\end{equation*}
So if $f$ is integrable on some closed, bounded interval containing $a,b,c$ (in any order), then
\begin{equation*}
\begin{aligned}
\int_a^b f = \int_a^c f + \int_c^b f
\end{aligned}
\end{equation*}
This comes from the theorem about integrals on union of intervals and this definitions (a few cases for signs).\\
(eg if $c<b<a$ then
\begin{equation*}
\begin{aligned}
\int_c^a f = \int_c^b f + \int_b^a f\\
-\int_a^c f = \int_c^b f - \int_a^b f
\end{aligned}
\end{equation*}
So consistent.)\\
Note:
\begin{equation*}
\begin{aligned}
|\int_a^b f| \leq |b-a| \sup_{\left[a,b\right]} |f|
\end{aligned}
\end{equation*}
Since if $a<b$, this holds by proposition 3;\\
if $b<a$ then
\begin{equation*}
\begin{aligned}
|\int_a^b f| &= |-\int_b^a f| = |\int_b^a f|\\
&\leq \left(a-b\right) \sup |f| = |b-a| \sup |f|.
\end{aligned}
\end{equation*}
\end{defi}

\begin{defi} (indefinite integral)\\
Suppose $a<b$, $f$ is integrable on $\left[a,b\right]$ and $c\in \left[a,b\right]$. The function
\begin{equation*}
\begin{aligned}
F\left(x\right) = \int_c^x f\left(t\right)dt, x\in\left[a,b\right]
\end{aligned}
\end{equation*}
is called \emph{\underline{an} indefinite integral of $f$ on $\left[a,b\right]$} (since this depends on c).\\
Note:
\begin{equation*}
\begin{aligned}
F\left(y\right) - F\left(x\right) = \int_x^y f\left(t\right) dt
\end{aligned}
\end{equation*}
This does not depend on $c$.
\end{defi}

\begin{thm}
If $a<b$, $f$ integrable on $\left[a,b\right]$, $F$ is an indefinite integral of $f$ on $\left[a,b\right]$, then $F$ is continuous. In fact, there exists some $k\geq 0$ such that
\begin{equation*}
\begin{aligned}
|F\left(y\right) - F\left(x\right)| \leq k |y-x|
\end{aligned}
\end{equation*}
\begin{proof}
Let $K=\sup_{\left[a,b\right]} |f|$. Then
\begin{equation*}
\begin{aligned}
|F\left(y\right) - F\left(x\right)| &= |\int_x^y f\left(t\right) dt|\\
&\leq |y-x| \cdot \sup_{\left[a,b\right]} |f|\\
&=K|y-x|
\end{aligned}
\end{equation*}
So the second part is done.\\
Now given $x\in\left[a,b\right], \epsilon > 0$, letting $\delta = \frac{\epsilon}{k}$, we have
\begin{equation*}
\begin{aligned}
\forall y\in\left[a,b\right], |y-x|<\delta \implies |F\left(y\right) - F\left(x\right)| < \epsilon.
\end{aligned}
\end{equation*}
So $F$ is continuous.
\end{proof}
\end{thm}

\begin{thm} (Fundamental theorem of Calculus)\\
Let $a,b,f,F$ be as in the previous theorem. If $c\in\left[a,b\right]$ and $f$ is continuous at $c$, then $F$ is differentiable at $c$, and
\begin{equation*}
\begin{aligned}
F'\left(c\right) = f\left(c\right)
\end{aligned}
\end{equation*}
Note: if $c=a$,
\begin{equation*}
\begin{aligned}
F'\left(a\right) = \lim_{h\to 0^+} \frac{F\left(a+h\right) - F\left(a\right)}{h}
\end{aligned}
\end{equation*}
and if $c=b$,
\begin{equation*}
\begin{aligned}
F'\left(b\right) = \lim_{h\to 0^-} \frac{F\left(b+h\right) - F\left(b\right)}{h}.
\end{aligned}
\end{equation*}
\begin{proof}
Given $\epsilon>0$, there is a $\delta>0$ s.t. 
\begin{equation*}
\begin{aligned}
\forall t\in\left[a,b\right] |t-c|<\delta \implies |f\left(t\right) - f\left(c\right)| < \epsilon
\end{aligned}
\end{equation*}
Now we have
\begin{equation*}
\begin{aligned}
|\frac{F\left(c+h\right)-F\left(c\right)}{h}-f\left(c\right)| &= |\frac{1}{h}\int_c^{c+h} f\left(t\right)dt-f\left(c\right)|\\
&=|\frac{1}{h} \int_c^{c+h} \left(f\left(t\right)-f\left(c\right)\right) dt|\\
&\leq \frac{1}{|h|}|h| \cdot \sup \left\{|f\left(t\right)-f\left(c\right)|: c \leq t \leq c+h\right\}\\
&\leq \epsilon
\end{aligned}
\end{equation*}
Whenever $0<|h|<\delta$ (provided $c+h\in\left[a,b\right]$).
\end{proof}
\end{thm}

Let $a<b$, $f,F$ be functions on $\left[a,b\right]$. We say $F$ is an \emph{antiderivative of $f$ on $\left[a,b\right]$} if $F$ is differentiable on $\left[a,b\right]$ and $F'\left(x\right) = f\left(x\right)$ for all $x\in\left[a,b\right]$.

\begin{coro}
Let $a<b$, $f$ a continuous function on $\left[a,b\right]$. Then $f$ has an antiderivative $F$ on $\left[a,b\right]$. Moreover, if $G$ is any antiderivative of $f$ on $\left[a,b\right]$, then
\begin{equation*}
\begin{aligned}
\int_a^b f\left(t\right)dt = G\left(b\right) - g\left(a\right)
\end{aligned}
\end{equation*}
\begin{proof}
For the first part, just take $F$ to be an indefinite integral of $f$ on $\left[a,b\right]$.\\
For the second part, we have
\begin{equation*}
\begin{aligned}
\left(F-G\right)'\left(x\right) = F'\left(x\right) - G'\left(x\right) = f\left(x\right) - f\left(x\right) = 0
\end{aligned}
\end{equation*}
for all $x$.\\
So (by mean value theorem) $F-G$ is a constant. Hence
\begin{equation*}
\begin{aligned}
G\left(b\right) - G\left(a\right) = F\left(b\right) - F\left(a\right) = \int_a^b f\left(t\right) dt
\end{aligned}
\end{equation*}
by definition of indefinite integrals.
\end{proof}
\end{coro}

\begin{rem}
$\bullet$ this corollary shows that the differential equation
\begin{equation*}
\begin{aligned}
\frac{dy}{dt}=f
\end{aligned}
\end{equation*}
has a solution when $f$ is continuous, and it is unique up to a constant.\\
So given $y_0\in \R$, the initial value problem
\begin{equation*}
\begin{aligned}
\left\{\begin{array}{ll}
\frac{dy}{dt}=f\\
y\left(a\right) = y
\end{array}
\right.
\end{aligned}
\end{equation*}
has a unique solution.\\
$\bullet$ this corollary provides a way for computing
\begin{equation*}
\begin{aligned}
\int_a^b f\left(t\right) dt
\end{aligned}
\end{equation*}
when $f$ is continuous.
\end{rem}

\begin{thm}
Let $a<b$, $f$ integrable on $\left[a,b\right]$. Assume $f$ has an antiderivative $G$. Then
\begin{equation*}
\begin{aligned}
\int_a^b f\left(t\right) dt = G\left(b\right) - G\left(a\right)
\end{aligned}
\end{equation*}
\begin{proof}
By a previous corollary(5) we know that there exists a sequence $\D_n$ of dissections of $\left[a,b\right]$ such that $S_{\D_n} \left(f\right)$ and $\S_{\D_n} \left(f\right)$ both converge to $\int_a^b f\left(t\right) dt$.\\
Say
\begin{equation*}
\begin{aligned}
\D_n: a=x_0^{\left(n\right)} < x_1^{\left(n\right)} < ... < x_{m_n}^{\left(n\right)} = b
\end{aligned}
\end{equation*}
for some $m_n \in \N$.\\
Apply mean value theorem to $G$ on $\left[x_{k-1}^{\left(n\right)},x_k^{\left(n\right)}\right]$ we get, there exists $\xi_k^{\left(n\right)} \in \left(x_{k-1}^{\left(n\right)},x_k^{\left(n\right)}\right)$ such that
\begin{equation*}
\begin{aligned}
\frac{G\left(x_k^{\left(n\right)}\right) - G \left(x_{k-1}^{\left(n\right)}\right)}{x_k^{\left(n\right)-x_{k-1}^{\left(n\right)}}}=G'\left(\xi_k^{\left(n\right)}\right) = f\left(\xi_k^{\left(n\right)}\right)
\end{aligned}
\end{equation*}
So
\begin{equation*}
\begin{aligned}
\sum_{k=1}^{m_n} f\left(\xi_k^{\left(n\right)}\right) \left(x_k^{\left(n\right)}-x_{k-1}^{\left(n\right)}\right)
=\sum_{k=1}^{m_n} \left(G\left(x_k^{\left(n\right)}\right) - G \left(x_{k-1}^{\left(n\right)}\right)\right) = G\left(b\right) - G\left(a\right)
\end{aligned}
\end{equation*}
Then by corollary 5, $LHS \to \int_a^b f\left(t\right)dt$ as $n\to\infty$.
\end{proof}
\end{thm}

\begin{rem}
Let $f,G$ be as in the previous theorem. Then
\begin{equation*}
\begin{aligned}
\int_a^x f\left(t\right) dt = G\left(x\right) - G\left(a\right)
\end{aligned}
\end{equation*}
for all $x\in\left[a,b\right]$.\\
So any indefinite integral of $f$ must be differentiable.
\end{rem}

\begin{coro}
Let $a<b$, $f,g$ be integrable functions on $\left[a,b\right]$. Assume $F,G$ are antiderivatives of $f,g$ respectively on $\left[a,b\right]$. (eg this happens if $f,g$ are continuous)\\
Then
\begin{equation*}
\begin{aligned}
\int_a^b fG = G\left(b\right) F\left(b\right) - G\left(a\right)F\left(a\right) - \int_a^b Fg
\end{aligned}
\end{equation*}
\begin{proof}
Let
\begin{equation*}
\begin{aligned}
H\left(x\right) = F\left(x\right)G\left(x\right)
\end{aligned}
\end{equation*}
for $x\in\left[a,b\right]$. By product rule, $H$ is differentiable, and
\begin{equation*}
\begin{aligned}
H'\left(x\right) = f\left(x\right)G\left(x\right) + F\left(x\right) g\left(x\right)
\end{aligned}
\end{equation*}
RHS is integrable. Thus
\begin{equation*}
\begin{aligned}
H\left(b\right)-H\left(a\right) = \int_a^b \left(fG+Fg\right)
\end{aligned}
\end{equation*}
rearrange to get the desired result.
\end{proof}
\end{coro}

\begin{coro} (Change of variable)\\
Let $a<b$, $\varphi:\left[a,b\right] \to \R$ be continuously differentiable. Let $f$ be a continuous function on the closed bounded interval $\varphi\left(\left[a,b\right]\right)$. Then
\begin{equation*}
\begin{aligned}
\int_{\varphi\left(a\right)}^{\varphi\left(b\right)} = \int_a^b f\left(\varphi\left(t\right)\right)\varphi'\left(t\right) dt
\end{aligned}
\end{equation*}

\begin{rem}
As $\varphi$ is continuous, there exists $c,d\in\left[a,b\right]$ such that $\varphi\left(c\right) \leq \varphi\left(x\right) \leq\varphi\left(d\right)$ for all $x\in\left[a,b\right]$. Then by IVT, 
\begin{equation*}
\begin{aligned}
\varphi\left(\left[a,b\right]\right) = \left[\varphi\left(c\right),\varphi\left(d\right)\right]
\end{aligned}
\end{equation*}
We do not assume that $\varphi\left(a\right)$, $\varphi\left(b\right)$ are the end points of this interval.
\end{rem}
\begin{proof}
Let $F$ be an antiderivative of $f$ on $\varphi\left(\left[a,b\right]\right)$ (this exists by a previous corollary(17)). By chain rule,
\begin{equation*}
\begin{aligned}
\left(F\circ\varphi\right)'\left(t\right) = F'\left(\varphi\left(t\right)\right)\varphi'\left(t\right) = f\left(\varphi\left(t\right)\right) \varphi'\left(t\right)
\end{aligned}
\end{equation*}
ans is continuous. By corollary 17,
\begin{equation*}
\begin{aligned}
\int_a^b f\left(\varphi\left(t\right)\right) \varphi'\left(t\right) dt = F\left(\varphi\left(b\right)\right) - F\left(\varphi\left(a\right)\right) = \int_{\varphi\left(a\right)}^{\varphi\left(b\right)} f\left(y\right) dy
\end{aligned}
\end{equation*}
\end{proof}
\end{coro}

Note that this corollary remains true if $\varphi$ is differentiable, $\varphi'$ is integrable, $f$ is integrable, and $f$ has antiderivative.\\
(need: $\varphi$ continuous, $f$ integrable $\implies$ $f\circ\varphi$ integrable).

\begin{thm} (Taylor's theorem with the integral remainder)\\
Assume $a,\delta \in \R$, $\delta>0$, $f:\left(a-\delta, a+\delta\right) \to \R$ is $n$ times continuously differentiable. Then for all $h\in\left(-\delta,\delta\right)$,
\begin{equation*}
\begin{aligned}
f\left(a+h\right)=\sum_{k=0}^{n-1} \frac{f^{\left(k\right)} \left(a\right)}{k!} h^k + \frac{1}{\left(n-1\right)!}\int_0^h \left(h-t\right)^{n-1} f^{\left(n\right)} \left(a+t\right) dt
\end{aligned}
\end{equation*}
\begin{proof}
Induction on $n$:\\
$n=1$:
\begin{equation*}
\begin{aligned}
RHS &= f\left(a\right) + \int_0^h f'\left(a+t\right) dt\\
&= f\left(a\right) + f\left(a+h\right) - f\left(a\right)\\
&= LHS.
\end{aligned}
\end{equation*}
$n\geq 1$: assume result for $n$. Then
\begin{equation*}
\begin{aligned}
&\frac{1}{n!}\int_0^h \left(h-t\right)^n f^{\left(n+1\right)} \left(a+t\right) dt\\
&= \left[\frac{\left(h-t\right)^n}{n!}f^{\left(n\right)} \left(a+t\right)\right]_0^h + \frac{1}{n!} \int_0^h n\left(h-t\right)^{n-1} \cdot f^{\left(n\right)} \left(a+t\right) dt\\
&= -\frac{h^n}{n!} f^{\left(n\right)} \left(a\right) + \frac{1}{\left(n-1\right)!} \int_0^h \left(h-t\right)^{n-1} f^{\left(n\right)} \left(a+t\right)dt\\
&=-\frac{h^n}{n!} f^{\left(n\right)} \left(a\right) + \left(f\left(a+h\right) - \sum_{k=0}^{n-1} \frac{f^{\left(k\right)} \left(a\right)}{k!} h^k\right)
\end{aligned}
\end{equation*}
rearrange to get the desired results.
\end{proof}
\end{thm}


\end{document}