\documentclass[a4paper]{article}

\setlength{\parindent}{0pt}
\setlength{\parskip}{1em}

\pagestyle{headings}

\usepackage{amssymb}
\usepackage{amsmath}
\usepackage{amsthm}
\usepackage{mathtools}
\usepackage{graphicx}
\usepackage{hyperref}
\usepackage{color}
\usepackage{microtype}
\usepackage{tikz}
\usepackage{pgfplots}
\usepackage{pgfplotstable}

\newcommand{\N}{\mathbb{N}}
\newcommand{\Q}{\mathbb{Q}}
\newcommand{\Z}{\mathbb{Z}}
\newcommand{\R}{\mathbb{R}}
\newcommand{\C}{\mathbb{C}}
\newcommand{\D}{\mathcal{D}}
\renewcommand{\S}{\mathcal{S}}
\renewcommand{\P}{\mathbb{P}}
\newcommand{\F}{\mathbb{F}}
\newcommand{\E}{\mathbb{E}}
\newcommand{\bra}{\langle}
\newcommand{\ket}{\rangle}


\graphicspath{{Image/}}

\hypersetup{
    colorlinks=true,
    linktoc=all,
    linkcolor=blue
}

\theoremstyle{definition}
\newtheorem*{axiom}{Axiom}
\newtheorem*{claim}{Claim}
\newtheorem*{conv}{Convention}
\newtheorem*{coro}{Corollary}
\newtheorem*{defi}{Definition}
\newtheorem*{eg}{Example}
\newtheorem*{lemma}{Lemma}
\newtheorem*{notation}{Notation}
\newtheorem*{prob}{Problem}
\newtheorem*{post}{Postulate}
\newtheorem*{prop}{Proposition}
\newtheorem*{rem}{Remark}
\newtheorem*{thm}{Theorem}

\DeclareMathOperator{\vdiv}{div}
\DeclareMathOperator{\grad}{grad}
\DeclareMathOperator{\curl}{curl}
\DeclareMathOperator{\Ann}{Ann}
\DeclareMathOperator{\Fit}{Fit}
\DeclareMathOperator{\Diag}{Diag}
\DeclareMathOperator{\tr}{tr}
\DeclareMathOperator{\im}{im}
\DeclareMathOperator{\Mat}{Mat}
\DeclareMathOperator{\Log}{Log}
\DeclareMathOperator{\Isom}{Isom}
\DeclareMathOperator{\Mesh}{Mesh}
\DeclareMathOperator{\Sym}{Sym}
\DeclareMathOperator{\Aut}{Aut}
\DeclareMathOperator{\cosech}{cosech}
\DeclareMathOperator{\Card}{Card}
\DeclareMathOperator{\Gal}{Gal}


\setcounter{section}{-2}

\begin{document}

\title{Number Fields}

\maketitle

\newpage

\tableofcontents

\newpage

\section{Miscellaneous}

Book: Number Fields, Marcus

Course notes: www.dpmms.ac.uk/~jat58/nfl2018

\newpage

\section{Motivation}
\begin{thm}
If $p$ is an odd prime, then $p=a^2+b^2$ for $a,b \in \Z \iff p \equiv 1 \pmod 4$.
\begin{proof}
If $p=a^2+b^2$, then $p\equiv 0,1,2 \pmod 4$. So this condition on $p$ is necessary.\\
Suppose instead $p \equiv 1 \pmod 4$. Then $\left(\frac{-1}{p}\right) = 1$. Thus $\exists a \in \Z$ such that $a^2 \equiv -1 \pmod p$, or $p|a^2+1$. We can factor $a^2+1=(a+i)(a-i)$ in the ring $\Z[i]$. Here we introduce a notation: if $R \subseteq S$ are rings and $\alpha \in S$, then $$R[\alpha] = \{\sum_{i=0}^n a_i \alpha^i \in S | a_i \in R\}$$, the smallest subring of $S$ containing both $R$ and $\alpha$.

We know from IB GRM that $\Z[i]$ is a UFD. Now $p|(a+i)(a-i)$. If $p$ is irreducible in $\Z[i]$ then $p|a+i$ or $p|a-i$, contradiction. Thus $p$ is reducible in $\Z[i]$, hence $p=z_1z_2$ with $z_1,z_2 \in \Z[i]$. If $z_1 = A+Bi$, $A,B\in \Z$, then $A^2+B^2 = p$.
\end{proof}
\end{thm}

Another example is when $p$ is an odd prime. Does the equation $$x^p+y^p=z^p$$ have solutions with $x,y,z\in \Z$ and $xyz \neq 0$?

\begin{thm} (Kummer, 1850)\\
If $\Z[e^{2\pi i/p}]$ is a UFD, then there are no solutions.\\
Strategy: factor $x^p+y^p = \prod_{j=0}^{p-1} (x+e^{2\pi ij/p}y)$ in $\Z[e^{2\pi i/p}]$.
\end{thm}

However, we now know $\Z[e^{2\pi i/p}]$ is a UFD $\iff$ $p \leq 19$.

\begin{thm} (Kummer, 1850)\\
If $p$ is a \emph{regular} prime, then there are no solutions.\\
If $p<100$, then $p$ is regular $\iff$ $p \neq 37,59,67$.
\end{thm}

We have seen various examples such as $\Z \subseteq \Q$, $\Z[i] \subseteq \Q[i]$, $\Z[e^{2\pi i/p}] \subseteq \Q[e^{2\pi i/p}]$, or in general, $\mathcal{O}_L\subseteq L$, where a ring of "integers" lies in a number field.

\newpage

\section{Ring of integers}
Recall: A field extension $L/K$ is an inclusion $K \leq L$ of fields. The degree of $L/K$ is $[L:K] = \dim_K L$. We say $L/K$ is finite if $[L:K]<\infty$.

\begin{defi} (1.1)\\
A number field is a finite extension $L/\Q$. Here are two ways to construct number fields:\\
(1) Let $\alpha \in \C$ be an algebraic number. Then $L=\Q(\alpha)$ is a number field;\\
(2) Let $K$ be a number field, and let $f(X) \in K[X]$ be an irreducible polynomial. Then $L=K[X]/(f(X))$ is a number field.\\
(Recall Tower Law: $[L:Q] = [L:K][K:Q] < \infty$).
\end{defi}

\begin{defi} (1.2)\\
(1) Let $L/K$ be a field extension. Then we say $\alpha \in L$ is algebraic over $K$ if there exists a monic $f(X) \in K[X]$ such that $f(\alpha) = 0$;\\
(2) Let $L/\Q$ be a field extension. Then we say $\alpha \in L$ is an algebraic integer if there exists a monic $f(X) \in Z[X]$ such that $f(\alpha) = 0$.
\end{defi}

\begin{defi} (1.3)\\
Let $L/K$ be a field extension, and let $\alpha \in L$ be algebraic over $K$. We call the minimal polynomial of $\alpha$ over $K$ the monic polynomial $f_\alpha(X) \in K[X]$ of least degree such that $f_\alpha(\alpha)=0$.
\end{defi}

We recall why $f_\alpha(X)$ is well-defined: there exists some monic $f(X) \in K[X]$ with $f(\alpha)=0$ as $\alpha$ is algebraic. If $f_\alpha(\alpha),f'_\alpha(\alpha) \in K[X]$ both satisfy the definition of minimal polynomial, then we apply the polynomial division algorithm to write $$f_\alpha(X) = p(X) f'_\alpha(X) + r(X)$$ where $p(X),r(X) \in K[X]$, and $\deg r < \deg f'_\alpha$. Evaluate at $X=\alpha$, we have $0=f_\alpha(\alpha) = p(\alpha)f'_\alpha(\alpha)+r(\alpha)=r(\alpha)$. By minimality of $\deg f'_\alpha$, we must have $r=0$. Then $\deg f_\alpha = \deg f'_\alpha$, and $f_\alpha(X),f'(\alpha)$ are both monic, i.e. $p(X) = 1$ and $f_\alpha(X) = f'_\alpha(X)$.

\begin{lemma}(1.4)\\
Let $L/\Q$ be a field extension, and let $\alpha \in L$ be an algebraic integer. Then:\\
(1) The minimal polynomail $f_\alpha(X)$ of $\alpha$ over $\Q$ lies in $\Z[X]$;\\
(2) If $g(X) \in \Z[X]$ satisfies $g(\alpha) = 0$, then there exists $q(X) \in \Z[X]$ such that $g(X) = f_\alpha(X) q(X)$;\\
(3) The kernel of the ring homomorphism $\Z[X] \to L$ by $f(X) \to f(\alpha)$ equals $(f_\alpha(X))$, the ideal generated by $f_\alpha(X)$.
\begin{proof}
(1) Recall that if $f(X) = a_n X^n + ... + a_0 \in \Z[X]$, then we define from GRM, the content $c(f) = gcd(a_n,...,a_0)$. Recall Gauss' Lemma: If $f(X), g(X) \in \Z[X]$, then $c(fg) = c(f)c(g)$. Since $\alpha \in L$ is an algebraic integer, there exists monic $f(X) \in \Z[X]$ such that $f(\alpha) = 0$, i.e. $c(f) = 1$. Apply polynomial division in $\Q[X]$ to get $f(X) = p(X) f_\alpha(X) +r(X)$, where $p(X),r(X) \in \Q[X]$, $\deg r < \deg f_\alpha$. The definition of $f_\alpha(X)$ implies that $r(X) = 0$, hence $f(X) = p(X) f_\alpha(X)$. Now choose integers $n,m \geq 1$ such that $np(X) \in \Z[X]$, $c(np) = 1$, and $mf_\alpha(X) \in \Z[x]$, $c(mf_\alpha) = 1$. Then $nmf(x) = (np(x))(mf_\alpha(x)) \implies c(nmf(x)) = nm = 1$. So $n=m=1$, hence $f_\alpha(x) \in \Z[X]$.\\
(2) Let $g(X) \in \Z[X]$ be such that $g(\alpha) =0 $. WLOG $g(x) \neq 0$ and $c(g) = 1$. Now apply polynomial division to write $g(x) = q(x) f_\alpha(x) + s(x)$ where $q(x),s(x) \in \Q[x]$, $\deg s < \deg f_\alpha$. Again by definition we have $s(x) = 0$. Choose an integer $k \geq 1$ such that $kq(x) \in Z[x]$ and $c(kq) =1$. Then $kg(x) = kq(x) f_\alpha(x) \implies k=c(kg) = c(kq) c(f_\alpha) = 1$. So $k=1$, hence $q(x) \in \Z[x]$.\\
(3) is a reformulation of (2).
\end{proof}
\end{lemma}




\end{document}
