\documentclass[a4paper]{article}

\setlength{\parindent}{0pt}
\setlength{\parskip}{1em}

\pagestyle{headings}

\usepackage{amssymb}
\usepackage{amsmath}
\usepackage{amsthm}
\usepackage{mathtools}
\usepackage{graphicx}
\usepackage{hyperref}
\usepackage{color}
\usepackage{microtype}
\usepackage{tikz}
\usepackage{pgfplots}
\usepackage{pgfplotstable}

\newcommand{\N}{\mathbb{N}}
\newcommand{\Q}{\mathbb{Q}}
\newcommand{\Z}{\mathbb{Z}}
\newcommand{\R}{\mathbb{R}}
\newcommand{\C}{\mathbb{C}}
\newcommand{\D}{\mathcal{D}}
\renewcommand{\S}{\mathcal{S}}
\renewcommand{\P}{\mathbb{P}}
\newcommand{\F}{\mathbb{F}}
\newcommand{\E}{\mathbb{E}}

\graphicspath{{Image/}}

\hypersetup{
    colorlinks=true,
    linktoc=all,
    linkcolor=blue
}

\theoremstyle{definition}
\newtheorem*{axiom}{Axiom}
\newtheorem*{claim}{Claim}
\newtheorem*{conv}{Convention}
\newtheorem*{coro}{Corollary}
\newtheorem*{defi}{Definition}
\newtheorem*{eg}{Example}
\newtheorem*{lemma}{Lemma}
\newtheorem*{notation}{Notation}
\newtheorem*{prob}{Problem}
\newtheorem*{post}{Postulate}
\newtheorem*{prop}{Proposition}
\newtheorem*{rem}{Remark}
\newtheorem*{thm}{Theorem}

\DeclareMathOperator{\vdiv}{div}
\DeclareMathOperator{\grad}{grad}
\DeclareMathOperator{\curl}{curl}
\DeclareMathOperator{\Ann}{Ann}
\DeclareMathOperator{\Fit}{Fit}
\DeclareMathOperator{\Diag}{Diag}
\DeclareMathOperator{\tr}{tr}
\DeclareMathOperator{\im}{im}
\DeclareMathOperator{\Mat}{Mat}
\DeclareMathOperator{\Log}{Log}
\DeclareMathOperator{\Isom}{Isom}
\DeclareMathOperator{\Mesh}{Mesh}
\DeclareMathOperator{\Sym}{Sym}
\DeclareMathOperator{\Aut}{Aut}
\DeclareMathOperator{\cosech}{cosech}
\DeclareMathOperator{\Card}{Card}
\DeclareMathOperator{\Gal}{Gal}


\setcounter{section}{-1}

\begin{document}

\title{Logic and Set Theory}

\maketitle

\newpage

\tableofcontents

\newpage

\section{Miscellaneous}

Some introductory speech

\newpage

\section{Propositional logic}
Let $P$ denote a set of \emph{primitive proposition}, unless otherwise stated, $P=\{p_1,p_2,...\}$.

The \emph{language} or \emph{set of propositions} $L=L(P)$ is defined inductively by:\\
1) $p \in L$ $\forall p \in P$;\\
2) $\perp in L$, where $\perp$ is read as 'false';\\
3) If $p,q \in L$, then $(p \implies q) \in L$. For example, $(p_1 \implies L)$, $((p_1 \implies p_2) \implies (p_1 \implies p_3))$.

Note that at this point, each proposition is only a finite string of symbols from the alphabet $(,),\implies,\perp,p_1,p_2,...$ and do not really mean anything (until we define so).\\
By \emph{inductively define}, we mean more precisely that we set $L_1 = P \cup \{\perp\}$, and $L_{n+1} = L_n \cup \{(p \implies q):p,q \in L_n\}$, and then put $L = L_1 \cup L_2 \cup ...$.

Each proposition is built up \emph{uniquely} from 1) and 2) using 3). For example, $((p_1 \implies p_2) \implies (p_1 \implies p_3))$ came from $(p_1 \implies p_2)$ and $(p_1 \implies p_3)$. We often omit outer brackets or use different brackets for clarity.

Now we can define some useful things:\\
$\bullet$ $\neg p$ (not $p$), as an abbreviation for $p \implies L$;\\
$\bullet$ $p \vee q$ ($p$ or $q$), as an abbreviation for $(\neg p) \implies q$;\\
$\bullet$ $p \wedge q$ ($p$ and $q$), as an abbreviation for $(p \implies (\neg q))$.

Note that these definitions 'make sense' in the way that we expect them to.


\end{document}
