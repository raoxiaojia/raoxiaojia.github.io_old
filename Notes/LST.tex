\documentclass[a4paper]{article}

\setlength{\parindent}{0pt}
\setlength{\parskip}{1em}

\pagestyle{headings}

\usepackage{amssymb}
\usepackage{amsmath}
\usepackage{amsthm}
\usepackage{mathtools}
\usepackage{graphicx}
\usepackage{hyperref}
\usepackage{color}
\usepackage{microtype}
\usepackage{tikz}
\usepackage{pgfplots}
\usepackage{pgfplotstable}

\newcommand{\N}{\mathbb{N}}
\newcommand{\Q}{\mathbb{Q}}
\newcommand{\Z}{\mathbb{Z}}
\newcommand{\R}{\mathbb{R}}
\newcommand{\C}{\mathbb{C}}
\newcommand{\D}{\mathcal{D}}
\renewcommand{\S}{\mathcal{S}}
\renewcommand{\P}{\mathbb{P}}
\newcommand{\F}{\mathbb{F}}
\newcommand{\E}{\mathbb{E}}
\newcommand{\bra}{\langle}
\newcommand{\ket}{\rangle}


\graphicspath{{Image/}}

\hypersetup{
    colorlinks=true,
    linktoc=all,
    linkcolor=blue
}

\theoremstyle{definition}
\newtheorem*{axiom}{Axiom}
\newtheorem*{claim}{Claim}
\newtheorem*{conv}{Convention}
\newtheorem*{coro}{Corollary}
\newtheorem*{defi}{Definition}
\newtheorem*{eg}{Example}
\newtheorem*{lemma}{Lemma}
\newtheorem*{notation}{Notation}
\newtheorem*{prob}{Problem}
\newtheorem*{post}{Postulate}
\newtheorem*{prop}{Proposition}
\newtheorem*{rem}{Remark}
\newtheorem*{thm}{Theorem}

\DeclareMathOperator{\vdiv}{div}
\DeclareMathOperator{\grad}{grad}
\DeclareMathOperator{\curl}{curl}
\DeclareMathOperator{\Ann}{Ann}
\DeclareMathOperator{\Fit}{Fit}
\DeclareMathOperator{\Diag}{Diag}
\DeclareMathOperator{\tr}{tr}
\DeclareMathOperator{\im}{im}
\DeclareMathOperator{\Mat}{Mat}
\DeclareMathOperator{\Log}{Log}
\DeclareMathOperator{\Isom}{Isom}
\DeclareMathOperator{\Mesh}{Mesh}
\DeclareMathOperator{\Sym}{Sym}
\DeclareMathOperator{\Aut}{Aut}
\DeclareMathOperator{\cosech}{cosech}
\DeclareMathOperator{\Card}{Card}
\DeclareMathOperator{\Gal}{Gal}


\setcounter{section}{-1}

\begin{document}

\title{Logic and Set Theory}

\maketitle

\newpage

\tableofcontents

\newpage

\section{Miscellaneous}

Some introductory speech

\newpage

\section{Propositional logic}
Let $P$ denote a set of \emph{primitive proposition}, unless otherwise stated, $P=\{p_1,p_2,...\}$.

\begin{defi}
The \emph{language} or \emph{set of propositions} $L=L(P)$ is defined inductively by:\\
(1) $p \in L$ $\forall p \in P$;\\
(2) $\perp \in L$, where $\perp$ is read as 'false';\\
(3) If $p,q \in L$, then $(p \implies q) \in L$. For example, $(p_1 \implies L)$, $((p_1 \implies p_2) \implies (p_1 \implies p_3))$.
\end{defi}

Note that at this point, each proposition is only a finite string of symbols from the alphabet $(,),\implies,\perp,p_1,p_2,...$ and do not really mean anything (until we define so).\\
By \emph{inductively define}, we mean more precisely that we set $L_1 = P \cup \{\perp\}$, and $L_{n+1} = L_n \cup \{(p \implies q):p,q \in L_n\}$, and then put $L = L_1 \cup L_2 \cup ...$.

Each proposition is built up \emph{uniquely} from 1) and 2) using 3). For example, $((p_1 \implies p_2) \implies (p_1 \implies p_3))$ came from $(p_1 \implies p_2)$ and $(p_1 \implies p_3)$. We often omit outer brackets or use different brackets for clarity.

Now we can define some useful things:\\
$\bullet$ $\neg p$ (not $p$), as an abbreviation for $p \implies L$;\\
$\bullet$ $p \vee q$ ($p$ or $q$), as an abbreviation for $(\neg p) \implies q$;\\
$\bullet$ $p \wedge q$ ($p$ and $q$), as an abbreviation for $(p \implies (\neg q))$.

These definitions 'make sense' in the way that we expect them to.

\begin{defi}
A \emph{valuation} is a function $v:L \to \{0,1\}$ s.t.\\
(1) $v(\perp) = 0$;
(2) 
\begin{equation*}
\begin{aligned}
v(p \implies q) = \left\{ \begin{array}{ll}
0 & v(p) =1, v(q) = 0\\
1 & else
\end{array}
\right. \forall p,q \in L
\end{aligned}
\end{equation*}
\end{defi}

\begin{rem}
On $\{0,1\}$, we could define a constant $\perp$ by $\perp = 0$, and an operation $\implies$ by $a \implies b = 0$ if $a=1, b=0$ and $1$ otherwise. Then a valuation is a function $L \to \{0,1\}$ that preserves the structure ($\perp$ and $\implies$), i.e. a homomorphism.
\end{rem}

\begin{prop}
(1) If $v,v'$ are valueations with $v(p) = v'(p)$ $\forall p \in P$, then $v=v'$ (on $L$).\\
(2) For any $w:P \to \{0,1\}$, there exists a valuation $v$ with $v(p) = w(p)$ $\forall p \in P$.\\
In short, a valuation is defined by its value on $p$, and any values will do.
\begin{proof}
(1) We have $v(p)=v'(p)$ $\forall p \in L_1$. However, if $v(p) = v'(p)$ and $v(q) = v'(q)$ then $v(p \implies q) = v'(p \implies q)$, so $v=v'$ on $L_2$. Continue inductively we have $v=v'$ on $L_n \forall n$.\\
(2) Set $v(p) = w(p)$ $\forall p \in P$ and $v(\perp) = 0$: this defines $v$ on $L_1$. Having defined $v$ on $L_n$, use the rules for valuaton to inductively define $v$ on $L_{n+1}$ so we can extend $v$ to $L$.
\end{proof}
\end{prop}

\begin{defi}
We say $p$ is a \emph{tautology}, written $\vDash p$, if $v(p) = 1$ $\forall$ valuations $v$. Some examples:\\
(1) $p \implies (q \implies p)$: a true statement is implies by anything. We can verify this by:
\begin{equation*}
\begin{aligned}
\begin{matrix}
v(p) & v(q) & v(q \implies p) & v(p \implies (q \implies p))\\
1 & 1 & 1 & 1\\
1 & 0 & 1 & 1\\
0 & 1 & 0 & 1\\
0 & 0 & 1 & 1
\end{matrix}
\end{aligned}
\end{equation*}

So we see that this is indeed a tautology;\\
(2) $(\neg\neg p) \implies p$, i.e. $((p \implies \perp) \implies \perp) \implies p$, called the "law of excluded middle";

(3) $[p \implies (q \implies r)] \implies [(p \implies q) \implies (p \implies r)]$.\\
Indeed, if not then we have some $v$ with $v(p\implies(q \implies r)) = 1$, $v(\implies (p \implies q) \implies (p \implies r)) = 0$. So $v(p\implies q) = 1$, $v(p \implies r) =0$. This happens when $v(p) = 1$, $v(r) = 0$, so also $v(q) = 1$. But then $v(q \implies r)=0$, so $v(p \implies (q \implies r)) = 0$.
\end{defi}

\begin{defi}
For $S \subset L$, $t \in L$, say $S$ \emph{entails} or \emph{semantically implies} $t$, written $S \vDash t$ if $v(s) = 1 \forall s \in S \implies v(t) = 1$, each valuation $v$.\\
("Whenever all of $S$ is true, $t$ is true as well.")

For example, $\{p \implies q, q \implies r\} \vDash (p \implies r)$. To prove this, suppose not: so we have $v$ with $v(p\implies q) = v(q\implies r) = 1$ but $v(p\implies r) = 0$. So $v(p) = 1$, $v(r) = 0$, so $v(q) = 0$, but then $v(p \implies q)$ = 0.

If $v(t) = 1$ we say$t$ is true in $v$ or that $v$ is a model of $t$.

For $S \subset L$, $v$ is a model of $S$ if $v(s) = 1$ $\forall s \in S$. So $S \vDash t$ says that every model of $S$ is a model of $t$. For example, in fact $\vDash t$ is the same as $\phi \vDash t$.
\end{defi}

\newpage

\section{Syntactic implication}

For a notion of 'proof', we will need axioms and deduction rules. As axioms, we'll take:\\
1. $p \implies (q \implies p)$ $\forall p,q \in L$;\\
2. $[p\implies (q \implies r)] \implies [(p \implies q) \implies (p \implies r)]$ $\forall p,q,r \in L$;
3. $(\neg\neg p) \implies p$ $\forall p \in L$.

Note: these are all tautologies. Sometimes we say they are 3 axiom-schemes, as all of these are infinite sets of axioms.

As deduction rules, we'll take just \emph{modus ponens}: from $p$, and $p\implies q$, we can deduce $q$.

For $S \subset L$, $t \in L$, a \emph{proof} of $t$ from $S$ cosists of a finite sequence $t_1,...,t_n$ of propositions, with $t_n = t$, s.t. $\forall i$ the proposition $t_i$ is an axiom, or a member of $S$, or there exists $j,k < i$ with $t_j=(t_k \implies t_i)$.

We say $s$ is the \emph{hypotheses} or \emph{premises} and $t$ is the \emph{conclusion}.

If there exists a proof of $t$ from $S$, we say $S$ \emph{proves} or \emph{syntactically implies} $t$, written $s \vdash t$.

If $\phi \vdash t$, we say $t$ is a \emph{theorem}, written $\vdash t$.

\iffalse
\begin{equation*}
\begin{aligned}

\end{aligned}
\end{equation*}
\fi


\end{document}
