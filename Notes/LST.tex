\documentclass[a4paper]{article}

\setlength{\parindent}{0pt}
\setlength{\parskip}{1em}

\pagestyle{headings}

\usepackage{amssymb}
\usepackage{amsmath}
\usepackage{amsthm}
\usepackage{mathtools}
\usepackage{graphicx}
\usepackage{hyperref}
\usepackage{color}
\usepackage{microtype}
\usepackage{tikz}
\usepackage{pgfplots}
\usepackage{pgfplotstable}

\newcommand{\N}{\mathbb{N}}
\newcommand{\Q}{\mathbb{Q}}
\newcommand{\Z}{\mathbb{Z}}
\newcommand{\R}{\mathbb{R}}
\newcommand{\C}{\mathbb{C}}
\newcommand{\D}{\mathcal{D}}
\renewcommand{\S}{\mathcal{S}}
\renewcommand{\P}{\mathbb{P}}
\newcommand{\F}{\mathbb{F}}
\newcommand{\E}{\mathbb{E}}
\newcommand{\bra}{\langle}
\newcommand{\ket}{\rangle}


\graphicspath{{Image/}}

\hypersetup{
    colorlinks=true,
    linktoc=all,
    linkcolor=blue
}

\theoremstyle{definition}
\newtheorem*{axiom}{Axiom}
\newtheorem*{claim}{Claim}
\newtheorem*{conv}{Convention}
\newtheorem*{coro}{Corollary}
\newtheorem*{defi}{Definition}
\newtheorem*{eg}{Example}
\newtheorem*{lemma}{Lemma}
\newtheorem*{notation}{Notation}
\newtheorem*{prob}{Problem}
\newtheorem*{post}{Postulate}
\newtheorem*{prop}{Proposition}
\newtheorem*{rem}{Remark}
\newtheorem*{thm}{Theorem}

\DeclareMathOperator{\vdiv}{div}
\DeclareMathOperator{\grad}{grad}
\DeclareMathOperator{\curl}{curl}
\DeclareMathOperator{\Ann}{Ann}
\DeclareMathOperator{\Fit}{Fit}
\DeclareMathOperator{\Diag}{Diag}
\DeclareMathOperator{\tr}{tr}
\DeclareMathOperator{\im}{im}
\DeclareMathOperator{\Mat}{Mat}
\DeclareMathOperator{\Log}{Log}
\DeclareMathOperator{\Isom}{Isom}
\DeclareMathOperator{\Mesh}{Mesh}
\DeclareMathOperator{\Sym}{Sym}
\DeclareMathOperator{\Aut}{Aut}
\DeclareMathOperator{\cosech}{cosech}
\DeclareMathOperator{\Card}{Card}
\DeclareMathOperator{\Gal}{Gal}


\setcounter{section}{-1}

\begin{document}

\title{Logic and Set Theory}

\maketitle

\newpage

\tableofcontents

\newpage

\section{Miscellaneous}

Some introductory speech

\newpage

\section{Propositional logic}
Let $P$ denote a set of \emph{primitive proposition}, unless otherwise stated, $P=\{p_1,p_2,...\}$.

\begin{defi}
The \emph{language} or \emph{set of propositions} $L=L(P)$ is defined inductively by:\\
(1) $p \in L$ $\forall p \in P$;\\
(2) $\perp \in L$, where $\perp$ is read as 'false';\\
(3) If $p,q \in L$, then $(p \implies q) \in L$. For example, $(p_1 \implies L)$, $((p_1 \implies p_2) \implies (p_1 \implies p_3))$.
\end{defi}

Note that at this point, each proposition is only a finite string of symbols from the alphabet $(,),\implies,\perp,p_1,p_2,...$ and do not really mean anything (until we define so).\\
By \emph{inductively define}, we mean more precisely that we set $L_1 = P \cup \{\perp\}$, and $L_{n+1} = L_n \cup \{(p \implies q):p,q \in L_n\}$, and then put $L = L_1 \cup L_2 \cup ...$.

Each proposition is built up \emph{uniquely} from 1) and 2) using 3). For example, $((p_1 \implies p_2) \implies (p_1 \implies p_3))$ came from $(p_1 \implies p_2)$ and $(p_1 \implies p_3)$. We often omit outer brackets or use different brackets for clarity.

Now we can define some useful things:\\
$\bullet$ $\neg p$ (not $p$), as an abbreviation for $p \implies \perp$;\\
$\bullet$ $p \vee q$ ($p$ or $q$), as an abbreviation for $(\neg p) \implies q$;\\
$\bullet$ $p \wedge q$ ($p$ and $q$), as an abbreviation for $\neg (p \implies (\neg q))$.

These definitions 'make sense' in the way that we expect them to.

\begin{defi}
A \emph{valuation} is a function $v:L \to \{0,1\}$ s.t.\\
(1) $v(\perp) = 0$;
(2) 
\begin{equation*}
\begin{aligned}
v(p \implies q) = \left\{ \begin{array}{ll}
0 & v(p) =1, v(q) = 0\\
1 & else
\end{array}
\right. \forall p,q \in L
\end{aligned}
\end{equation*}
\end{defi}

\begin{rem}
On $\{0,1\}$, we could define a constant $\perp$ by $\perp = 0$, and an operation $\implies$ by $a \implies b = 0$ if $a=1, b=0$ and $1$ otherwise. Then a valuation is a function $L \to \{0,1\}$ that preserves the structure ($\perp$ and $\implies$), i.e. a homomorphism.
\end{rem}

\begin{prop}
(1) If $v,v'$ are valuations with $v(p) = v'(p)$ $\forall p \in P$, then $v=v'$ (on $L$).\\
(2) For any $w:P \to \{0,1\}$, there exists a valuation $v$ with $v(p) = w(p)$ $\forall p \in P$.\\
In short, a valuation is defined by its value on $p$, and any values will do.
\begin{proof}
(1) We have $v(p)=v'(p)$ $\forall p \in L_1$. However, if $v(p) = v'(p)$ and $v(q) = v'(q)$ then $v(p \implies q) = v'(p \implies q)$, so $v=v'$ on $L_2$. Continue inductively we have $v=v'$ on $L_n \forall n$.\\
(2) Set $v(p) = w(p)$ $\forall p \in P$ and $v(\perp) = 0$: this defines $v$ on $L_1$. Having defined $v$ on $L_n$, use the rules for valuation to inductively define $v$ on $L_{n+1}$ so we can extend $v$ to $L$.
\end{proof}
\end{prop}

\begin{defi}
We say $p$ is a \emph{tautology}, written $\vDash p$, if $v(p) = 1$ $\forall$ valuations $v$. Some examples:\\
(1) $p \implies (q \implies p)$: a true statement is implies by anything. We can verify this by:
\begin{equation*}
\begin{aligned}
\begin{matrix}
v(p) & v(q) & v(q \implies p) & v(p \implies (q \implies p))\\
1 & 1 & 1 & 1\\
1 & 0 & 1 & 1\\
0 & 1 & 0 & 1\\
0 & 0 & 1 & 1
\end{matrix}
\end{aligned}
\end{equation*}

So we see that this is indeed a tautology;\\
(2) $(\neg\neg p) \implies p$, i.e. $((p \implies \perp) \implies \perp) \implies p$, called the "law of excluded middle";

(3) $[p \implies (q \implies r)] \implies [(p \implies q) \implies (p \implies r)]$.\\
Indeed, if not then we have some $v$ with $v(p\implies(q \implies r)) = 1$, $v(\implies (p \implies q) \implies (p \implies r)) = 0$. So $v(p\implies q) = 1$, $v(p \implies r) =0$. This happens when $v(p) = 1$, $v(r) = 0$, so also $v(q) = 1$. But then $v(q \implies r)=0$, so $v(p \implies (q \implies r)) = 0$.
\end{defi}

\begin{defi}
For $S \subset L$, $t \in L$, say $S$ \emph{entails} or \emph{semantically implies} $t$, written $S \vDash t$ if $v(s) = 1 \forall s \in S \implies v(t) = 1$, for each valuation $v$.\\
("Whenever all of $S$ is true, $t$ is true as well.")

For example, $\{p \implies q, q \implies r\} \vDash (p \implies r)$. To prove this, suppose not: so we have $v$ with $v(p\implies q) = v(q\implies r) = 1$ but $v(p\implies r) = 0$. So $v(p) = 1$, $v(r) = 0$, so $v(q) = 0$, but then $v(p \implies q)$ = 0.

If $v(t) = 1$ we say $t$ is true in $v$ or that $v$ is a model of $t$.

For $S \subset L$, $v$ is a model of $S$ if $v(s) = 1$ $\forall s \in S$. So $S \vDash t$ says that every model of $S$ is a model of $t$. For example, in fact $\vDash t$ is the same as $\phi \vDash t$.
\end{defi}

\newpage

\section{Syntactic implication}

For a notion of 'proof', we will need axioms and deduction rules. As axioms, we'll take:\\
1. $p \implies (q \implies p)$ $\forall p,q \in L$;\\
2. $[p\implies (q \implies r)] \implies [(p \implies q) \implies (p \implies r)]$ $\forall p,q,r \in L$;\\
3. $(\neg\neg p) \implies p$ $\forall p \in L$.

Note: these are all tautologies. Sometimes we say they are 3 axiom-schemes, as all of these are infinite sets of axioms.

As deduction rules, we'll take just \emph{modus ponens}: from $p$, and $p\implies q$, we can deduce $q$.

For $S \subset L$, $t \in L$, a \emph{proof} of $t$ from $S$ cosists of a finite sequence $t_1,...,t_n$ of propositions, with $t_n = t$, s.t. $\forall i$ the proposition $t_i$ is an axiom, or a member of $S$, or there exists $j,k < i$ with $t_j=(t_k \implies t_i)$.

We say $S$ is the \emph{hypotheses} or \emph{premises} and $t$ is the \emph{conclusion}.

If there exists a proof of $t$ from $S$, we say $S$ \emph{proves} or \emph{syntactically implies} $t$, written $S \vdash t$.

If $\phi \vdash t$, we say $t$ is a \emph{theorem}, written $\vdash t$.

\begin{eg}
$\{p \implies q, q \implies r\} \vdash p \implies r$.\\
we deduce by the following:\\
(1) $[p \implies (q \implies r)] \implies [(p \implies q) \implies (p \implies r)]$; (axiom 2)\\
(2) $q \implies r$; (hypothesis)\\
(3) $(q \implies r) \implies (p \implies (q \implies r))$; (axiom 1)\\
(4) $p \implies (q \implies r)$; (mp on 2,3)\\
(5) $(p \implies q) \implies (p \implies r)$ (mp on 1,4);\\
(6) $p \implies q$; (hypothesis)\\
(7) $p \implies r$. (mp on 5,6)
\end{eg}

\begin{eg}
Let's now try to prove $\vdash p \implies p$. Axiom 1 and 3 probably don't help so look at axiom 2; if we make $(p \implies q)$ and $p\implies (q \implies r)$ something that's a theorem, and make $p \implies r$ to be $p \implies p$ then we are done. So we need to take $p=p,q=(p \implies p), r=p$. Now:\\
(1) $[p \implies ((p \implies p) \implies p)] \implies [( p \implies (p \implies p)) \implies (p \implies p)]$; (axiom 2)\\
(2) $p \implies ((p \implies p) \implies p)$; (axiom 1)\\
(3) $(p \implies (p \implies p)) \implies (p \implies p)$; (mp on 1,2)\\
(4) $p \implies (p \implies p)$; (axiom 1)\\
(5) $p \implies p$. (mp on 3,4)
\end{eg}

Proofs are made easier by:
\begin{prop} (2, deduction theorem)\\
Let $S \subset L$, $p,q \in L$. Then $S \vdash (p \implies q)$ if and only if $(S \cup \{p\}) \vdash q$.
\begin{proof}
Forward: given a proof of $p \implies q$ from $S$, add the lines $p$ (hypothesis), $q$ (mp) to optaion a proof of $q$ from $S\cup\{p\}$.\\
Backward: if we have proof $t_1,...,t_n=q$ of $q$ from $S \cup \{p\}$. We'll show that $S \vdash (p \implies t_i) \forall i$, so $p \implies t_n=q$.\\
If $t_i$ is an axiom, then we have $\vdash t_i \implies (p \implies t_i)$, so $\vdash p \implies t_i$;\\
If $t_i \in S$, write down $t_i$, $t_i \implies (p \implies t_i)$, $p \implies t_i$ we get a proof of $p \implies t_i$ from $S$;\\
If $t_i = p$: we know $\vdash (p \implies p)$, so done;\\
If $t_i$ obtained by mp: in that case we have some earlier lines $t_j$ and $t_j \implies t_i$. By induction, we may assume $S \vdash (p \implies t_j)$ and $S \vdash (p \implies (t_j \implies t_i))$. Now we can write down $[p \implies (t_j \implies t_i)] \implies [(p \implies t_j) \implies (t_i)]$ by axiom 2, $p \implies (t_j \implies t_i)$, $p \implies t_j) \implies (p \implies t_i)$ (mp), $p \implies t_j$, $p \implies t_i$ (mp) to obtain $S \vdash (p \implies t_i)$.

These are all of the cases. So $S \vdash (p \implies q)$.
\end{proof}
This is why we chose axiom 2 as we did -- to make this proof work.
\end{prop}

\begin{eg}
To show $\{p \implies q, q \implies r\} \vdash (p \implies r)$, it's enough to show that $\{p \implies q, q \implies r, p\} \vdash r$, which is trivial by mp.
\end{eg}

Now, how are $\vdash$ and $\vDash$ related? We are going to prove the \emph{completeness theorem}: $S \vdash t \iff S \vDash t$.

This ensures that our proofs are sound, in the sense that everything it can prove is not absurd ($S \vdash t$ then $S \vDash t$), and are adequate, i.e. our axioms are powerful enough to define every semantic consequence of $S$, which is not obvious ($S \vDash t$ then $S \vdash t$).

\begin{prop} (3)\\
Let $S \subset L$, $t \in L$. Then $S \vdash t \implies S \vDash t$.
\begin{proof}
Given a valuation $v$ with $v(s) =1$ $\forall s \in S$, we want $v(t) = 1$.\\
We have $v(p) = 1$ $\forall p$ axiom as our axioms are all tautologies (proven earier); $v(p)=1$ $\forall p \in S$ by definition of $v$; also if $v(p) = 1$ and $v(p \implies q) = 1$, then also $v(q) = 1$ (by definition of $\implies$). So $v(p) = 1$ for each line $p$ of our proof of $t$ from $S$.
\end{proof}
\end{prop}

We say $S \subset L$ consistent if $S \not\vdash \perp$. One special case of adequacy is: $S \vDash \perp \implies S \vdash \perp$, i.e. if $S$ has no model then $S$ inconsistent, i.e. if $S$ is consistent then $S$ has a model. This implies adequacy: given $S \vDash t$, we have $S \cup \{\neg t\} \vDash \perp$, so by our special case we have $S \cup \{\neg t\} \vdash \perp$, i.e. $S \vdash ((\neg t) \implies t)$ by deduction theorem, so $S \vdash \neg\neg t$. But $S \vdash ((\neg\neg t) \implies t)$ by axiom 3, so $S \vdash t$ (mp).

\begin{thm} (4)\\
Let $S \subset L$ be consistent, then $S$ has a model.\\
The idea is that we would like to define valuation $v$ by $v(p) = 1 \iff p \in S$, or more sensibly, $v(p)=1 \iff S \vdash p$.\\
But maybe $S \not \vdash p_3, S \not\vdash \neg p_3$, but a valuation maps half of $L$ to 1, so we want to 'grow' $S$ to contain one of $p$ or $\neg p$ for each $p \in L$, while keeping consistency.
\begin{proof}
Claim: for any consistent $S \subset L$, $p \in L$, $S \cup \{p\}$ or $S \cup \{\neg p\}$ consistent.
\emph{Proof of claim.} If not, then $S \cup \{p\} \vdash \perp$ and $S \cup \{\neg p\} \vdash \perp$, then $S \vdash (p \implies \perp)$ (deduction theorem), i.e. $S \vdash \not p$, so $S \vdash \perp$ contradiction.

Now $L$ is countable as each $L_n$ is countable, so we can list $L$ as $t_1,t_2,...$. Put $S_0 = S$; set $S_1 = s_0 \cup \{t_1\}$ or $s_0 \cup (\neg t_1\}$ so that $S_1$ is consistent. Then set $S_2 = S_1 \cup \{t_2\}$ or $S_1 \cup \{\neg t_2\}$ so that $S_2$ is consistent, and continue likewise. Set $\bar{S} = S_0 \cup S_1 \cup S_2 \cup ...$ Then $\bar{S} \supset S$, and $\bar{S}$ is consistent (as each $S_n$ is, and each proof is finite). $\forall p \in L$, we have either $p \in S$ or $(\neg p) \in S$. Also, $\bar{S}$ is \emph{deductively closed}, meaning that is $\bar{S} \vdash p$ then $p \in \bar{S}$: if $p \not\in \bar{S}$ then $(\neg p) \in \bar{S}$, so $\bar{S} \vdash p$, $\bar{S} \vdash (\not p)$ so $\bar{S} \vdash \perp$ contradiction.\\
Define $v:L \to \{0,1\}$ by $p \to 1$ if $p \in \bar{S}$, $0$ otherwise. Then $v$ is a valuation: $v(\perp) = 0$ as $\perp \not\in \bar{S}$; for $v(p \implies q)$:\\
If $v(p) = 1$, $v(q) = 0$: We have $p \in \bar{S}$, $q \not\in \bar{S}$, and want $v(p \implies q) = 0$, i.e. $(p \implies q \not\in \bar{S}$. But if $9p \implies q) \in \bar{S}$ then $\bar{S} \vdash q$ contradiction;\\
If $v(q) = 1$: have $q \in \bar{S}$, and want $v(p \implies q) =1$, i.e. $(p \implies q) \int \bar{S}$. But $\vdash q \implies (p \implies q)$ so $\bar{S} \vdash (p \implies q)$;\\
If $v(p) = 0$: have $p \not\in \bar{S}$, i.e. $(\neg p) \in \bar{S}$ and want $(p \implies q) \in \bar{S}$. So we need $(p \implies \perp) \vdash (p \implies q)$, i.e. $p \implies \perp,p \vdash q$ (deduction theorem). Thus it's enough to show that $\perp \vdash q$. But $(\neg\neg q) \implies q$, and $\vdash (\perp \implies (\neg\neg q))$ (axiom 3 and 1 -- to see the second one, write $\neg$ explicitly using $\implies$ and $\perp$), so $\vdash(\perp \implies q)$, i.e. $\perp\vdash q$.
\end{proof}
\end{thm}

\begin{rem}
Sometimes this is called 'completeness theorem'. The proof used $P$ being countable to get $L$ countable; in fact, result still holds if $P$ is uncountable (see chapter 3).
\end{rem}

By remark before theorem 4, we have
\begin{coro} (5, adequacy)\\
Let $S \subset L$, $t \in L$. Then if $S \vDash t$ then $S \vdash t$.
\end{coro}

And hence,
\begin{thm} (6, completeness theorem)\\
Let $S \subset L$, $t \in L$. Then $S \vdash t \iff S \vDash t$.
\end{thm}

Some consequences:
\begin{coro} (7, compactness theorem)\\
Let $S \subset L$, $t \in L$ with $S \vDash t$. Then $\exists$ finite $S' \subset S$ with $S' \vDash t$.\\
This is trivial if we replace $\vDash$ by $\vdash$ (as proofs are finite).
\end{coro}

Special case for $t=\perp$: If $S$ has no model then some finite $S' \subset S$ has no model. Equivalently,
\begin{coro} (7', compactness theorem, equivalent form)\\
Let $S \subset L$. If every finite subset of $S$ has a model then $S$ has a model.\\
This \emph{isi} equivalent to corollary 7 because $S \vDash t$ $\iff S \cup \{\neg t\}$ has no model and $S' \vDash t \iff S' \cup (\neg t)$ has no model.
\end{coro}

\begin{coro} (8, decidability theorem)\\
There is an algorithm to determine (in finite time) whether or not, for a given finite $S \subset L$ and $t \in L$, we have $S \vdash t$.\\
This is highly non-obviuos; however it's trivial to decide if $S \vDash t$ just by drawing a truth table, and $\vDash \iff \vdash$.
\end{coro}

\newpage

\section{Well-Orderings and Ordinals}

\begin{defi}
A \emph{total order} or \emph{linear order} on a set $X$ is a relation $<$ on $X$, such that\\
(1) Irreflexive: Not $x<x$ $\forall x \in X$;\\
(2) Transitive: $x<y,y<z\implies x<z$ $\forall x,y,z \in X$;\\
(3) Trichotomous: $x<y$ or $x=y$ or $y<x$ $\forall x,y \in X$.\\
Note: two of (iii) cannot hold: if $x<y$, $y<x$ then $x<x$ by transitivity.\\
Write $x \leq y$ if $x<y$ or $x=y$, and $y>x$ if $x<y$.

We can also define total order in terms of $\leq$:\\
(1) Reflexive: $x \leq x$ $\forall x \in X$;\\
(2) Transitive: $x\leq y, y \leq z \implies x \in z$ $\forall x,y,z \in X$;\\
(3) Antisymmetric: $x \leq y,y \leq x \implies x=y$ $\forall x,y \in X$;\\
(4) 'Tri'chotomous (although it's only two): $x\leq y$ or $y \leq x$ $\forall x,y \in X$.
\end{defi}

\begin{eg}
$\N,\Q,\R$ with the usual orders are all total orders.\\
$\N^+$ the relation 'divides' is not a total order: for example we don't have any of $2|3,3|2$ or $2=3$.\\
$\mathcal{P}(S)$ for some $S$ (with $|S|\geq 2$ to be rigorous), with $x \leq y$ if $x\subseteq y$ is not a total order for the same reason.
\end{eg}

A total order is a \emph{well-ordering} if every (non-empty) subset has a least element, i.e. $\forall S \subset X, S \neq \phi \implies \exists x \in S, x \leq y \forall y \in S$.

\begin{eg}
1.$\N$ with the usual $<$ is a well ordering.\\
2.$\Z,\Q,\R$ with the usual $<$ are not well orderings.\\
3.$\Q^+ \cup \{0\}$ with the usual $<$ is not a well ordering (e.g. $(0,\infty) \subset \Q^+\cup \{0\}$).\\
4.The set $\{1-\frac{1}{n} : n=2,3,...\}$ as a subset of $\R$ with the usual ordering is a well ordering.
5.The set $\{1-\frac{1}{n} : n=2,3,...\} \cup \{1\}$ as a subset of $\R$ with the usual ordering is a well ordering.
6.The set $\{1-\frac{1}{n} : n=2,3,...\} \cup \{2-\frac{1}{n} : n=2,3,...\}$ (same assumption) is a well ordering.
\end{eg}

\begin{rem}
$X$ is well-ordered iff there is no $x_1>x_2>x_3>...$ in $X$.\\
Clearly if there is such a sequence then $S=\{x_1,x_2,...\}$ has no least element. Conversely, if $S \subset X$ has no least element, then for each element $x \in S$ there exists a $x' \in S$ with $x' < x$, so we can just pick $x,x',...$ inductively.
\end{rem}

\begin{defi}
We say total orders $X,Y$ are \emph{isomorphic} if there exists a bijection $f:X \to Y$ that is order-preserving, i.e. $x<y \iff f(x) <f(y)$.\\
For example, 1 and 4 above are isomorphic; 5 and 6 are isomorphic; 4 and 5 are not isomorphic (one has a greatest element, and the other doesn't).
\end{defi}

Here comes the first reason why well orderings are useful:
\begin{prop} (1, Proof by induction)\\
Let $X$ be well-ordered, and let $S \subset X$ be s.t. if $y \in S$ $\forall y < x$ then $x \in S$ (each $x \in X$). Then $S=X$.\\
Equivalently, if $p(x)$ is a property s.t. $\forall x$: if $p(y) \forall y < x$ then $p(x)$, then $p(x) \forall x$.
\begin{proof}
If $S \neq X$ then let $x$ be the least element of $X \setminus S$. Then $x \not\in S$. But $y \in S$ $\forall y<x$, contradiction.
\end{proof}
\end{prop}

A typical use:

\begin{prop}
Let $X,Y$ be isomorphic well-orderings. Then there is a \emph{unique} isomorphism from $X$ to $Y$.
\begin{proof}
Let $f,g$ be isomorphisms. We'll show $f(x) = g(x)$ $\forall x$ by induction. Thus we may assume $f(y) = g(y)$ $\forall y < x$, and want $f(x) = g(x)$. Let $a$ be the least element of $Y\setminus \{f(y):y<x\}$. Then we must have $f(x) = a$: if $f(x) > a$, then some $x'>x$ has $f(x') = a$ by surjectivity, contradiction. The same shows $g(x) = $least element of $Y\setminus \{g(y):y<x\}$, but this is the same as $a$. So $f(x) = g(x)$.
\end{proof}
\end{prop}

\begin{rem}
This is false for total orders in general. One example is, consider from $\Z to \Z$, we could either take identity, or $x \to x-5$; or from $\R$ to $\R$ we could take identity or $x \to x-5$ or $x\to x^3$...
\end{rem}

\begin{defi}
In a total order $X$, an \emph{initial segment} $I$ is a subset of $X$ such that $x \in I, y < x \implies y \in I$.
\end{defi}

\begin{eg}
For any $x \in X$, set $I(x) = \{y\in X : y < x\}$. Then this is an initial segment.\\
Obviously, not every initial segment is of this form: for example, in $\R$ we can take $\{x:x \leq 3\}$; or in $\Q$, take $\{x:x^2 < 2\} \cup \{x<0\}$ (this cannot be written as above form as $\sqrt{2} \not\in \Q$.\\
Note: in a well-ordering, every proper initial segment \emph{is} of the above form: let $x$ be the least elemnt of $X \setminus I$. Then $y<x \implies y \in I$. Conversely, if $y \in I$, then we must have $y < x$: otherwise $x \in I$, contradiction.
\end{eg}

\iffalse
\begin{equation*}
\begin{aligned}

\end{aligned}
\end{equation*}
\fi


\end{document}
