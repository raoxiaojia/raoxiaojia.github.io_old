\documentclass[a4paper]{article}

\setlength{\parindent}{0pt}
\setlength{\parskip}{1em}

\pagestyle{headings}

\usepackage{amssymb}
\usepackage{amsmath}
\usepackage{amsthm}
\usepackage{mathtools}
\usepackage{graphicx}
\usepackage{hyperref}
\usepackage{color}
\usepackage{microtype}
\usepackage{tikz}
\usepackage{pgfplots}
\usepackage{pgfplotstable}

\newcommand{\N}{\mathbb{N}}
\newcommand{\Q}{\mathbb{Q}}
\newcommand{\Z}{\mathbb{Z}}
\newcommand{\R}{\mathbb{R}}
\newcommand{\C}{\mathbb{C}}
\newcommand{\D}{\mathcal{D}}
\renewcommand{\S}{\mathcal{S}}
\renewcommand{\P}{\mathbb{P}}
\newcommand{\F}{\mathbb{F}}
\newcommand{\E}{\mathbb{E}}
\newcommand{\bra}{\langle}
\newcommand{\ket}{\rangle}


\graphicspath{{Image/}}

\hypersetup{
    colorlinks=true,
    linktoc=all,
    linkcolor=blue
}

\theoremstyle{definition}
\newtheorem*{axiom}{Axiom}
\newtheorem*{claim}{Claim}
\newtheorem*{conv}{Convention}
\newtheorem*{coro}{Corollary}
\newtheorem*{defi}{Definition}
\newtheorem*{eg}{Example}
\newtheorem*{lemma}{Lemma}
\newtheorem*{notation}{Notation}
\newtheorem*{prob}{Problem}
\newtheorem*{post}{Postulate}
\newtheorem*{prop}{Proposition}
\newtheorem*{rem}{Remark}
\newtheorem*{thm}{Theorem}

\DeclareMathOperator{\vdiv}{div}
\DeclareMathOperator{\grad}{grad}
\DeclareMathOperator{\curl}{curl}
\DeclareMathOperator{\Ann}{Ann}
\DeclareMathOperator{\Fit}{Fit}
\DeclareMathOperator{\Diag}{Diag}
\DeclareMathOperator{\tr}{tr}
\DeclareMathOperator{\im}{im}
\DeclareMathOperator{\Mat}{Mat}
\DeclareMathOperator{\Log}{Log}
\DeclareMathOperator{\Isom}{Isom}
\DeclareMathOperator{\Mesh}{Mesh}
\DeclareMathOperator{\Sym}{Sym}
\DeclareMathOperator{\Aut}{Aut}
\DeclareMathOperator{\cosech}{cosech}
\DeclareMathOperator{\Card}{Card}
\DeclareMathOperator{\Gal}{Gal}


\setcounter{section}{-1}

\begin{document}

\title{Logic and Set Theory}

\maketitle

\newpage

\tableofcontents

\newpage

\section{Miscellaneous}

Some introductory speech

\newpage

\section{Propositional logic}
Let $P$ denote a set of \emph{primitive proposition}, unless otherwise stated, $P=\{p_1,p_2,...\}$.

\begin{defi}
The \emph{language} or \emph{set of propositions} $L=L(P)$ is defined inductively by:\\
(1) $p \in L$ $\forall p \in P$;\\
(2) $\perp \in L$, where $\perp$ is read as 'false';\\
(3) If $p,q \in L$, then $(p \implies q) \in L$. For example, $(p_1 \implies L)$, $((p_1 \implies p_2) \implies (p_1 \implies p_3))$.
\end{defi}

Note that at this point, each proposition is only a finite string of symbols from the alphabet $(,),\implies,\perp,p_1,p_2,...$ and do not really mean anything (until we define so).\\
By \emph{inductively define}, we mean more precisely that we set $L_1 = P \cup \{\perp\}$, and $L_{n+1} = L_n \cup \{(p \implies q):p,q \in L_n\}$, and then put $L = L_1 \cup L_2 \cup ...$.

Each proposition is built up \emph{uniquely} from 1) and 2) using 3). For example, $((p_1 \implies p_2) \implies (p_1 \implies p_3))$ came from $(p_1 \implies p_2)$ and $(p_1 \implies p_3)$. We often omit outer brackets or use different brackets for clarity.

Now we can define some useful things:\\
$\bullet$ $\neg p$ (not $p$), as an abbreviation for $p \implies \perp$;\\
$\bullet$ $p \vee q$ ($p$ or $q$), as an abbreviation for $(\neg p) \implies q$;\\
$\bullet$ $p \wedge q$ ($p$ and $q$), as an abbreviation for $\neg (p \implies (\neg q))$.

These definitions 'make sense' in the way that we expect them to.

\begin{defi}
A \emph{valuation} is a function $v:L \to \{0,1\}$ s.t.\\
(1) $v(\perp) = 0$;
(2) 
\begin{equation*}
\begin{aligned}
v(p \implies q) = \left\{ \begin{array}{ll}
0 & v(p) =1, v(q) = 0\\
1 & else
\end{array}
\right. \forall p,q \in L
\end{aligned}
\end{equation*}
\end{defi}

\begin{rem}
On $\{0,1\}$, we could define a constant $\perp$ by $\perp = 0$, and an operation $\implies$ by $a \implies b = 0$ if $a=1, b=0$ and $1$ otherwise. Then a valuation is a function $L \to \{0,1\}$ that preserves the structure ($\perp$ and $\implies$), i.e. a homomorphism.
\end{rem}

\begin{prop}
(1) If $v,v'$ are valuations with $v(p) = v'(p)$ $\forall p \in P$, then $v=v'$ (on $L$).\\
(2) For any $w:P \to \{0,1\}$, there exists a valuation $v$ with $v(p) = w(p)$ $\forall p \in P$.\\
In short, a valuation is defined by its value on $p$, and any values will do.
\begin{proof}
(1) We have $v(p)=v'(p)$ $\forall p \in L_1$. However, if $v(p) = v'(p)$ and $v(q) = v'(q)$ then $v(p \implies q) = v'(p \implies q)$, so $v=v'$ on $L_2$. Continue inductively we have $v=v'$ on $L_n \forall n$.\\
(2) Set $v(p) = w(p)$ $\forall p \in P$ and $v(\perp) = 0$: this defines $v$ on $L_1$. Having defined $v$ on $L_n$, use the rules for valuation to inductively define $v$ on $L_{n+1}$ so we can extend $v$ to $L$.
\end{proof}
\end{prop}

\begin{defi}
We say $p$ is a \emph{tautology}, written $\vDash p$, if $v(p) = 1$ $\forall$ valuations $v$. Some examples:\\
(1) $p \implies (q \implies p)$: a true statement is implies by anything. We can verify this by:
\begin{equation*}
\begin{aligned}
\begin{matrix}
v(p) & v(q) & v(q \implies p) & v(p \implies (q \implies p))\\
1 & 1 & 1 & 1\\
1 & 0 & 1 & 1\\
0 & 1 & 0 & 1\\
0 & 0 & 1 & 1
\end{matrix}
\end{aligned}
\end{equation*}

So we see that this is indeed a tautology;\\
(2) $(\neg\neg p) \implies p$, i.e. $((p \implies \perp) \implies \perp) \implies p$, called the "law of excluded middle";

(3) $[p \implies (q \implies r)] \implies [(p \implies q) \implies (p \implies r)]$.\\
Indeed, if not then we have some $v$ with $v(p\implies(q \implies r)) = 1$, $v(\implies (p \implies q) \implies (p \implies r)) = 0$. So $v(p\implies q) = 1$, $v(p \implies r) =0$. This happens when $v(p) = 1$, $v(r) = 0$, so also $v(q) = 1$. But then $v(q \implies r)=0$, so $v(p \implies (q \implies r)) = 0$.
\end{defi}

\begin{defi}
For $S \subset L$, $t \in L$, say $S$ \emph{entails} or \emph{semantically implies} $t$, written $S \vDash t$ if $v(s) = 1 \forall s \in S \implies v(t) = 1$, for each valuation $v$.\\
("Whenever all of $S$ is true, $t$ is true as well.")

For example, $\{p \implies q, q \implies r\} \vDash (p \implies r)$. To prove this, suppose not: so we have $v$ with $v(p\implies q) = v(q\implies r) = 1$ but $v(p\implies r) = 0$. So $v(p) = 1$, $v(r) = 0$, so $v(q) = 0$, but then $v(p \implies q)$ = 0.

If $v(t) = 1$ we say $t$ is true in $v$ or that $v$ is a model of $t$.

For $S \subset L$, $v$ is a model of $S$ if $v(s) = 1$ $\forall s \in S$. So $S \vDash t$ says that every model of $S$ is a model of $t$. For example, in fact $\vDash t$ is the same as $\phi \vDash t$.
\end{defi}

\newpage

\section{Syntactic implication}

For a notion of 'proof', we will need axioms and deduction rules. As axioms, we'll take:\\
1. $p \implies (q \implies p)$ $\forall p,q \in L$;\\
2. $[p\implies (q \implies r)] \implies [(p \implies q) \implies (p \implies r)]$ $\forall p,q,r \in L$;\\
3. $(\neg\neg p) \implies p$ $\forall p \in L$.

Note: these are all tautologies. Sometimes we say they are 3 axiom-schemes, as all of these are infinite sets of axioms.

As deduction rules, we'll take just \emph{modus ponens}: from $p$, and $p\implies q$, we can deduce $q$.

For $S \subset L$, $t \in L$, a \emph{proof} of $t$ from $S$ cosists of a finite sequence $t_1,...,t_n$ of propositions, with $t_n = t$, s.t. $\forall i$ the proposition $t_i$ is an axiom, or a member of $S$, or there exists $j,k < i$ with $t_j=(t_k \implies t_i)$.

We say $S$ is the \emph{hypotheses} or \emph{premises} and $t$ is the \emph{conclusion}.

If there exists a proof of $t$ from $S$, we say $S$ \emph{proves} or \emph{syntactically implies} $t$, written $S \vdash t$.

If $\phi \vdash t$, we say $t$ is a \emph{theorem}, written $\vdash t$.

\begin{eg}
$\{p \implies q, q \implies r\} \vdash p \implies r$.\\
we deduce by the following:\\
(1) $[p \implies (q \implies r)] \implies [(p \implies q) \implies (p \implies r)]$; (axiom 2)\\
(2) $q \implies r$; (hypothesis)\\
(3) $(q \implies r) \implies (p \implies (q \implies r))$; (axiom 1)\\
(4) $p \implies (q \implies r)$; (mp on 2,3)\\
(5) $(p \implies q) \implies (p \implies r)$ (mp on 1,4);\\
(6) $p \implies q$; (hypothesis)\\
(7) $p \implies r$. (mp on 5,6)
\end{eg}

\begin{eg}
Let's now try to prove $\vdash p \implies p$. Axiom 1 and 3 probably don't help so look at axiom 2; if we make $(p \implies q)$ and $p\implies (q \implies r)$ something that's a theorem, and make $p \implies r$ to be $p \implies p$ then we are done. So we need to take $p=p,q=(p \implies p), r=p$. Now:\\
(1) $[p \implies ((p \implies p) \implies p)] \implies [( p \implies (p \implies p)) \implies (p \implies p)]$; (axiom 2)\\
(2) $p \implies ((p \implies p) \implies p)$; (axiom 1)\\
(3) $(p \implies (p \implies p)) \implies (p \implies p)$; (mp on 1,2)\\
(4) $p \implies (p \implies p)$; (axiom 1)\\
(5) $p \implies p$. (mp on 3,4)
\end{eg}

Proofs are made easier by:
\begin{prop} (2, deduction theorem)\\
Let $S \subset L$, $p,q \in L$. Then $S \vdash (p \implies q)$ if and only if $(S \cup \{p\}) \vdash q$.
\begin{proof}
Forward: given a proof of $p \implies q$ from $S$, add the lines $p$ (hypothesis), $q$ (mp) to optaion a proof of $q$ from $S\cup\{p\}$.\\
Backward: if we have proof $t_1,...,t_n=q$ of $q$ from $S \cup \{p\}$. We'll show that $S \vdash (p \implies t_i) \forall i$, so $p \implies t_n=q$.\\
If $t_i$ is an axiom, then we have $\vdash t_i \implies (p \implies t_i)$, so $\vdash p \implies t_i$;\\
If $t_i \in S$, write down $t_i$, $t_i \implies (p \implies t_i)$, $p \implies t_i$ we get a proof of $p \implies t_i$ from $S$;\\
If $t_i = p$: we know $\vdash (p \implies p)$, so done;\\
If $t_i$ obtained by mp: in that case we have some earlier lines $t_j$ and $t_j \implies t_i$. By induction, we may assume $S \vdash (p \implies t_j)$ and $S \vdash (p \implies (t_j \implies t_i))$. Now we can write down $[p \implies (t_j \implies t_i)] \implies [(p \implies t_j) \implies (t_i)]$ by axiom 2, $p \implies (t_j \implies t_i)$, $p \implies t_j) \implies (p \implies t_i)$ (mp), $p \implies t_j$, $p \implies t_i$ (mp) to obtain $S \vdash (p \implies t_i)$.

These are all of the cases. So $S \vdash (p \implies q)$.
\end{proof}
This is why we chose axiom 2 as we did -- to make this proof work.
\end{prop}

\begin{eg}
To show $\{p \implies q, q \implies r\} \vdash (p \implies r)$, it's enough to show that $\{p \implies q, q \implies r, p\} \vdash r$, which is trivial by mp.
\end{eg}

Now, how are $\vdash$ and $\vDash$ related? We are going to prove the \emph{completeness theorem}: $S \vdash t \iff S \vDash t$.

This ensures that our proofs are sound, in the sense that everything it can prove is not absurd ($S \vdash t$ then $S \vDash t$), and are adequate, i.e. our axioms are powerful enough to define every semantic consequence of $S$, which is not obvious ($S \vDash t$ then $S \vdash t$).

\begin{prop} (3)\\
Let $S \subset L$, $t \in L$. Then $S \vdash t \implies S \vDash t$.
\begin{proof}
Given a valuation $v$ with $v(s) =1$ $\forall s \in S$, we want $v(t) = 1$.\\
We have $v(p) = 1$ $\forall p$ axiom as our axioms are all tautologies (proven earier); $v(p)=1$ $\forall p \in S$ by definition of $v$; also if $v(p) = 1$ and $v(p \implies q) = 1$, then also $v(q) = 1$ (by definition of $\implies$). So $v(p) = 1$ for each line $p$ of our proof of $t$ from $S$.
\end{proof}
\end{prop}

We say $S \subset L$ consistent if $S \not\vdash \perp$. One special case of adequacy is: $S \vDash \perp \implies S \vdash \perp$, i.e. if $S$ has no model then $S$ inconsistent, i.e. if $S$ is consistent then $S$ has a model. This implies adequacy: given $S \vDash t$, we have $S \cup \{\neg t\} \vDash \perp$, so by our special case we have $S \cup \{\neg t\} \vdash \perp$, i.e. $S \vdash ((\neg t) \implies t)$ by deduction theorem, so $S \vdash \neg\neg t$. But $S \vdash ((\neg\neg t) \implies t)$ by axiom 3, so $S \vdash t$ (mp).

\begin{thm} (4)\\
Let $S \subset L$ be consistent, then $S$ has a model.\\
The idea is that we would like to define valuation $v$ by $v(p) = 1 \iff p \in S$, or more sensibly, $v(p)=1 \iff S \vdash p$.\\
But maybe $S \not \vdash p_3, S \not\vdash \neg p_3$, but a valuation maps half of $L$ to 1, so we want to 'grow' $S$ to contain one of $p$ or $\neg p$ for each $p \in L$, while keeping consistency.
\begin{proof}
Claim: for any consistent $S \subset L$, $p \in L$, $S \cup \{p\}$ or $S \cup \{\neg p\}$ consistent.
\emph{Proof of claim.} If not, then $S \cup \{p\} \vdash \perp$ and $S \cup \{\neg p\} \vdash \perp$, then $S \vdash (p \implies \perp)$ (deduction theorem), i.e. $S \vdash \not p$, so $S \vdash \perp$ contradiction.

Now $L$ is countable as each $L_n$ is countable, so we can list $L$ as $t_1,t_2,...$. Put $S_0 = S$; set $S_1 = s_0 \cup \{t_1\}$ or $s_0 \cup (\neg t_1\}$ so that $S_1$ is consistent. Then set $S_2 = S_1 \cup \{t_2\}$ or $S_1 \cup \{\neg t_2\}$ so that $S_2$ is consistent, and continue likewise. Set $\bar{S} = S_0 \cup S_1 \cup S_2 \cup ...$ Then $\bar{S} \supset S$, and $\bar{S}$ is consistent (as each $S_n$ is, and each proof is finite). $\forall p \in L$, we have either $p \in S$ or $(\neg p) \in S$. Also, $\bar{S}$ is \emph{deductively closed}, meaning that is $\bar{S} \vdash p$ then $p \in \bar{S}$: if $p \not\in \bar{S}$ then $(\neg p) \in \bar{S}$, so $\bar{S} \vdash p$, $\bar{S} \vdash (\not p)$ so $\bar{S} \vdash \perp$ contradiction.\\
Define $v:L \to \{0,1\}$ by $p \to 1$ if $p \in \bar{S}$, $0$ otherwise. Then $v$ is a valuation: $v(\perp) = 0$ as $\perp \not\in \bar{S}$; for $v(p \implies q)$:\\
If $v(p) = 1$, $v(q) = 0$: We have $p \in \bar{S}$, $q \not\in \bar{S}$, and want $v(p \implies q) = 0$, i.e. $(p \implies q \not\in \bar{S}$. But if $9p \implies q) \in \bar{S}$ then $\bar{S} \vdash q$ contradiction;\\
If $v(q) = 1$: have $q \in \bar{S}$, and want $v(p \implies q) =1$, i.e. $(p \implies q) \int \bar{S}$. But $\vdash q \implies (p \implies q)$ so $\bar{S} \vdash (p \implies q)$;\\
If $v(p) = 0$: have $p \not\in \bar{S}$, i.e. $(\neg p) \in \bar{S}$ and want $(p \implies q) \in \bar{S}$. So we need $(p \implies \perp) \vdash (p \implies q)$, i.e. $p \implies \perp,p \vdash q$ (deduction theorem). Thus it's enough to show that $\perp \vdash q$. But $(\neg\neg q) \implies q$, and $\vdash (\perp \implies (\neg\neg q))$ (axiom 3 and 1 -- to see the second one, write $\neg$ explicitly using $\implies$ and $\perp$), so $\vdash(\perp \implies q)$, i.e. $\perp\vdash q$.
\end{proof}
\end{thm}

\begin{rem}
Sometimes this is called 'completeness theorem'. The proof used $P$ being countable to get $L$ countable; in fact, result still holds if $P$ is uncountable (see chapter 3).
\end{rem}

By remark before theorem 4, we have
\begin{coro} (5, adequacy)\\
Let $S \subset L$, $t \in L$. Then if $S \vDash t$ then $S \vdash t$.
\end{coro}

And hence,
\begin{thm} (6, completeness theorem)\\
Let $S \subset L$, $t \in L$. Then $S \vdash t \iff S \vDash t$.
\end{thm}

Some consequences:
\begin{coro} (7, compactness theorem)\\
Let $S \subset L$, $t \in L$ with $S \vDash t$. Then $\exists$ finite $S' \subset S$ with $S' \vDash t$.\\
This is trivial if we replace $\vDash$ by $\vdash$ (as proofs are finite).
\end{coro}

Special case for $t=\perp$: If $S$ has no model then some finite $S' \subset S$ has no model. Equivalently,
\begin{coro} (7', compactness theorem, equivalent form)\\
Let $S \subset L$. If every finite subset of $S$ has a model then $S$ has a model.\\
This \emph{isi} equivalent to corollary 7 because $S \vDash t$ $\iff S \cup \{\neg t\}$ has no model and $S' \vDash t \iff S' \cup (\neg t)$ has no model.
\end{coro}

\begin{coro} (8, decidability theorem)\\
There is an algorithm to determine (in finite time) whether or not, for a given finite $S \subset L$ and $t \in L$, we have $S \vdash t$.\\
This is highly non-obviuos; however it's trivial to decide if $S \vDash t$ just by drawing a truth table, and $\vDash \iff \vdash$.
\end{coro}

\newpage

\section{Well-Orderings and Ordinals}

\begin{defi}
A \emph{total order} or \emph{linear order} on a set $X$ is a relation $<$ on $X$, such that\\
(1) Irreflexive: Not $x<x$ $\forall x \in X$;\\
(2) Transitive: $x<y,y<z\implies x<z$ $\forall x,y,z \in X$;\\
(3) Trichotomous: $x<y$ or $x=y$ or $y<x$ $\forall x,y \in X$.\\
Note: two of (iii) cannot hold: if $x<y$, $y<x$ then $x<x$ by transitivity.\\
Write $x \leq y$ if $x<y$ or $x=y$, and $y>x$ if $x<y$.

We can also define total order in terms of $\leq$:\\
(1) Reflexive: $x \leq x$ $\forall x \in X$;\\
(2) Transitive: $x\leq y, y \leq z \implies x \in z$ $\forall x,y,z \in X$;\\
(3) Antisymmetric: $x \leq y,y \leq x \implies x=y$ $\forall x,y \in X$;\\
(4) 'Tri'chotomous (although it's only two): $x\leq y$ or $y \leq x$ $\forall x,y \in X$.
\end{defi}

\begin{eg}
$\N,\Q,\R$ with the usual orders are all total orders.\\
$\N^+$ the relation 'divides' is not a total order: for example we don't have any of $2|3,3|2$ or $2=3$.\\
$\mathcal{P}(S)$ for some $S$ (with $|S|\geq 2$ to be rigorous), with $x \leq y$ if $x\subseteq y$ is not a total order for the same reason.
\end{eg}

A total order is a \emph{well-ordering} if every (non-empty) subset has a least element, i.e. $\forall S \subset X, S \neq \phi \implies \exists x \in S, x \leq y \forall y \in S$.

\begin{eg}
1.$\N$ with the usual $<$ is a well ordering.\\
2.$\Z,\Q,\R$ with the usual $<$ are not well orderings.\\
3.$\Q^+ \cup \{0\}$ with the usual $<$ is not a well ordering (e.g. $(0,\infty) \subset \Q^+\cup \{0\}$).\\
4.The set $\{1-\frac{1}{n} : n=2,3,...\}$ as a subset of $\R$ with the usual ordering is a well ordering.
5.The set $\{1-\frac{1}{n} : n=2,3,...\} \cup \{1\}$ as a subset of $\R$ with the usual ordering is a well ordering.
6.The set $\{1-\frac{1}{n} : n=2,3,...\} \cup \{2-\frac{1}{n} : n=2,3,...\}$ (same assumption) is a well ordering.
\end{eg}

\begin{rem}
$X$ is well-ordered iff there is no $x_1>x_2>x_3>...$ in $X$.\\
Clearly if there is such a sequence then $S=\{x_1,x_2,...\}$ has no least element. Conversely, if $S \subset X$ has no least element, then for each element $x \in S$ there exists a $x' \in S$ with $x' < x$, so we can just pick $x,x',...$ inductively.
\end{rem}

\begin{defi}
We say total orders $X,Y$ are \emph{isomorphic} if there exists a bijection $f:X \to Y$ that is order-preserving, i.e. $x<y \iff f(x) <f(y)$.\\
For example, 1 and 4 above are isomorphic; 5 and 6 are isomorphic; 4 and 5 are not isomorphic (one has a greatest element, and the other doesn't).
\end{defi}

Here comes the first reason why well orderings are useful:
\begin{prop} (1, Proof by induction)\\
Let $X$ be well-ordered, and let $S \subset X$ be s.t. if $y \in S$ $\forall y < x$ then $x \in S$ (each $x \in X$). Then $S=X$.\\
Equivalently, if $p(x)$ is a property s.t. $\forall x$: if $p(y) \forall y < x$ then $p(x)$, then $p(x) \forall x$.\\
(I think we must assert $S$ to be non-empty here, but the lecturer didn't agree with me; need to check later.)
\begin{proof}
If $S \neq X$ then let $x$ be the least element of $X \setminus S$. Then $x \not\in S$. But $y \in S$ $\forall y<x$, contradiction.
\end{proof}
\end{prop}

A typical use:

\begin{prop}
Let $X,Y$ be isomorphic well-orderings. Then there is a \emph{unique} isomorphism from $X$ to $Y$.
\begin{proof}
Let $f,g$ be isomorphisms. We'll show $f(x) = g(x)$ $\forall x$ by induction. Thus we may assume $f(y) = g(y)$ $\forall y < x$, and want $f(x) = g(x)$. Let $a$ be the least element of $Y\setminus \{f(y):y<x\}$. Then we must have $f(x) = a$: if $f(x) > a$, then some $x'>x$ has $f(x') = a$ by surjectivity, contradiction. The same shows $g(x) = $least element of $Y\setminus \{g(y):y<x\}$, but this is the same as $a$. So $f(x) = g(x)$.
\end{proof}
\end{prop}

\begin{rem}
This is false for total orders in general. One example is, consider from $\Z \to \Z$, we could either take identity, or $x \to x-5$; or from $\R$ to $\R$ we could take identity or $x \to x-5$ or $x\to x^3$...
\end{rem}

\begin{defi}
In a total order $X$, an \emph{initial segment} $I$ is a subset of $X$ such that $x \in I, y < x \implies y \in I$.
\end{defi}

\begin{eg}
For any $x \in X$, set $I(x) = \{y\in X : y < x\}$. Then this is an initial segment.\\
Obviously, not every initial segment is of this form: for example, in $\R$ we can take $\{x:x \leq 3\}$; or in $\Q$, take $\{x:x^2 < 2\} \cup \{x<0\}$ (this cannot be written as above form as $\sqrt{2} \not\in \Q$.\\
Note: in a well-ordering, every proper initial segment \emph{is} of the above form: let $x$ be the least elemnt of $X \setminus I$. Then $y<x \implies y \in I$. Conversely, if $y \in I$, then we must have $y < x$: otherwise $x \in I$, contradiction.
\end{eg}

Our aim is to show that every subset of a well-ordered $X$ is isomorphic to an initial segment.\\
Note: this is very false for total orders: e.g. $\{1,5,9\} \subset \Z$, or $\Q \subset \R$. If we have $S \subset X$, Wwe would like to define $f:S \to X$ that sends the smallest of $S$ to the smallest of $X$, then remove them from both sets and send the smallest of the remaining to the smallest of the remaining, etc... But to do this we need a theorem.

\begin{thm} (3, definition by recursion)\\
Let $X$ be well-ordered, $Y$ be a set, and $G:\mathcal{P}(X\times Y) \to Y$. Then $\exists f : X \to Y$ s.t. $f(x) = G(f|_{I_x})$ for all $x \in X$. Moreover, such $f$ is unique.\\
Here we define the restriction as: for $f:A \to B$, and $C \subset A$, the restriction of $f$ to $C$ is $f|_C = \{(x,f(x)) : x \in C\}$. (I think the lecturer is regarding a function as subset of a cartesian product)\\
In defining $f(x)$, make use of $f|_{I_x}$, i.e. the values of $f(y), y<x$.\\
\begin{proof}
Existence: define '$h$ is an attempt' to mean: $h:I \to Y$, some initial segment $I$ of $X$, and $\forall x \in I$ we have $h(x) = G(h|_{I_X})$. Note that is $h,h'$ are attempts, both defined at $x$, then $h(x) = h'(x)$ by induction on $x$. Since if $h(y) = h'(y) \forall y < x $ then $h(x) = h'(x)$.\\
Also, $\forall x \in X$ there exists an attempt defined at $x$ by induction on $x$: we want attempt definde at $x$, given $\forall y < x$ there exists attempt defined at $y$. For each $y<x$, we have unique attempt $h_y$ defined on $\{z:z\leq y\}$ (unique by what we just showed).\\
Let $h = \cup_{y < x} h_y$: an attempt defined on $I_x$. This is single-valued by uniqueness, so is indeed a function.\\
So $h'=h \cup \{(x,G(h))\}$ is an attempt defined at $x$.\\
Now set $f(x) = y$ if $\exists$ attempt $h$, defined at $x$, with $h(x) = y$ (single-valued).\\
Uniqueness: if $f,f'$ suitable then $f(x) = f'(x) \forall x \in X$ (induction on $X$) -- since if $f(y) = f'(y) \forall y < x$ then $f(x) = f'(x)$.
\end{proof}
\end{thm}

A typical application:

\begin{prop} (4, subset collapse)\\
Let $X$ be well-ordered, $Y \subset X$. Then $Y$ is isomorphic to an initial segment of $X$. Moreover, such initial segment is unique.\\
\begin{proof}
To have $f$ an isomorphism from $y$ to an initial segment of $X$, we need precisely that $\forall x \in Y: f(x) = \min X \setminus \{f(y):y<x\}$. So done (existence and uniqueness) by theorem 3.\\
Note that $X \setminus \{f(y):y<x\} \neq \phi$, e.g. because $f(y) \leq y$ $\forall y$ (induction), so $x \not\in \{f(y):y<x\}$.
\end{proof}
\end{prop}

In particular, a well-ordered $X$ cannot be isomorphic to a proper initial segment of $X$ -- by uniqueness in subset collapse, as $X$ is isomorphic to $X$.

How do different well-orderings relate to each other?

We say $X \leq Y$ if $X$ is isomorphic to an initial segment of $Y$. For example, $\N \leq \{1-\frac{1}{n}:n=2,3,...\}\cup\{1\}$.

\begin{thm} (5)\\
Let $X,Y$ be well-orderings. Then $X\leq Y$ or $Y \leq X$.
\begin{proof}
Suppose $Y \not\leq X$. To obtain $f:X \to Y$ that is an isomorphism with an initial segment of $Y$, need $\forall x \in X: f(x) = \min Y \setminus \{f(y):y<x\}$. So we are done by theorem 3.\\
Note that we cannot have $\{f(y):y<x\} = X$, as then $Y$ is isomorphic to $I_x$.
\end{proof}
\end{thm}

\begin{prop} (6)\\
Let $X,Y$ be well-orderings with $X \leq Y$ and $Y \leq X$. Then $X$ and $Y$ are isomorphic.
\begin{proof}
We have isomorphism $f$ from $X$ to an isomorphism of $Y$, and $g$ the other way round. Then $g \circ f: X \to X$ is an isomorphism from $X$ to an initial segment of $X$ (i.s. of i.s. is i.s.), but that is impossible unless the initial segment is $X$ itself. So $g\circ f$ is identity (by uniqueness in subset collapse). Similarly, $f \circ g$ is identity on $Y$.
\end{proof}
\end{prop}

New well-orderings from old:

Write $X<Y$ if $X \leq Y$ but $X$ not isomorphic to $Y$. Equivalently, $X < Y$ iff $X$ is isomorphic to a proper initial segment of $Y$. For example, if $X = \N$, $Y = \{1-\frac{1}{n}\} \cup \{1\}$ then $X < Y$.

Make a bigger one: given well-ordered $X$, choose $x \not\in X$, and set $x>y$ for all $y \in X$. This is a well-ordering on $X \cup \{x\}$: written $X^+$. Clearly $X < X^+$.

Put some together:\\
Let $(X,<_X)$ and $(Y,<_Y)$ be well-orderings. Say $Y$ extends $X$ if $X \subset Y$, and $<_X$, $<_Y$ agree on $X$, and $X$ an initial segment of $(Y,<_Y)$.\\
Well-orderings $(X_i: i \in I)$ are nested if $\forall i,j \in I:$ $X_i$ extends $X_j$ or $X_j$ extends $X_i$.

\begin{prop} (7)\\
Let $(X_i:i \in I)$ be a nested family of well-orderings. Then there exist well-ordering $X$ with $X \geq X_i$ $\forall i$.
\begin{proof}
Let $X = \cup_{i \in I} X_i$, with $x<y$ if $\exists i$ with $x,y \in X_i$ and $x <_i y$, Then $<$ is a well-defined total order on $X$. given $S \subset X$, $S \neq \phi$, choose $i$ with $S\cap X_i \neq \phi$. Then $S \cap X_i$ has a minimal element (as $X_i$ is well-ordered), which must also be a minimal element of $S$ (as $X_i$ an i.s. of $X$). Also, $X \geq X_i \forall i$.
\end{proof}
\end{prop}

\newpage

\section{Ordinals}

Are the well-orderings themselves well-ordered?

An ordinal is a well-ordered set, with two sell-ordered sets regarded as the same if they are isomorphic. (Just as a rational is an expression $\frac{M}{N}$, with $\frac{M}{N}$, $\frac{M'}{N'}$ regarded as the same if $MN' = M'N$. But, unlike for $\Q$, we cannot formalise by equivalence classes -- see later).

If $X$ is a well-ordering corresponding to ordinal $X$, say $X$ has order-type $\alpha$.

\begin{eg}
For each $k \in \N$, write $k$ for the order-type of the (unique) well-ordering of a set of size $k$, and write $\omega$ for order-type of $\N$. So, in $\R$, $\{1,3,7\}$ has order-type 3. $\{1-\frac{1}{n}:n=2,3,...\}$ has order-type $\omega$. For $X$ of o-t $\alpha$ and $Y$ of o-t $\beta$, write $\alpha \leq \beta$ if $X \leq Y$ (this is independent of choice of $X$,$Y$).\\
Similarly for $\alpha<\beta$ etc.
\end{eg}

We know: $\forall \alpha,\beta$, $\alpha\leq \beta$ or $\beta\leq\alpha$, and if $\alpha\leq \beta$, $\beta\leq\alpha$ then $\alpha=\beta$.

\begin{thm}
Let $\alpha$ be an ordinal. Then the ordinals $<\alpha$ form a well-ordered set of order-type $\alpha$. e.g. the ordinals $<\omega$ are $0,1,2,3,...$.
\begin{proof}
Let $X$ have o-t $\alpha$. the well-orderings $<X$ are precisely (up to isomorphism) the proper initial segments of $X$, i.e. the $I_{x},x \in X$.\\
But these are isomorphic to $X$ itself, via $x \to I_x$.
\end{proof}
\end{thm}

We often write $I_\alpha$ to be the set of ordinals less than $\alpha$.

\begin{prop} (9)\\
Let $S$ be a non-empty set of ordinals. Then $S$ has a least element.
\begin{proof}
Choose $\alpha \in S$. If $\alpha$ minimal in $S$ then done. If not, then $S \cap I_\alpha \neq \phi$, so have a minimal element of $S\cap I_\alpha$, which is therefore minimal in $S$.
\end{proof}
\end{prop}

\begin{thm} (10, Burali-Forti paradox):\\
The ordinals do not form a set.
\begin{proof}
Suppose not, let $X$ be set of all ordinals. Then $X$ is a well-orderings, say order-type $\alpha$. So $X$ is isomorphic to $I_\alpha$. But $I_\alpha$ is a proper i.s. of $X$.
\end{proof}
\end{thm}

Given $\alpha$, we have $\alpha^+ > \alpha$. Also, if $\{\alpha_i:i \in I\}$ is a set of ordinals, then there exists $\alpha$ with $\alpha \geq \alpha_i \forall i$ (by applying prop 7 to the nested family of $I_{\alpha_i};i \in I$).

In fact, there is therefore a least upper bound for $\{\alpha_i: i \in I\}$ by applying prop 9 to the set $\{\beta \leq \alpha: \beta$ an upper bound for the $\alpha_i\}$. This is written $\sup\{\alpha_i:i \in I\}$, e.g. $\sup\{2,4,6,8,...\} = \omega$.

Some ordinals: $0,1,2,...,\omega,\omega+1$(officially $\omega^+$),$\omega+2$,...,\\
$\omega+\omega =\omega2 = \sup\{\omega+1,\omega+2,...,\}$, $\omega^2+1,\omega^2+2,...$,\\
$\omega3,...,\omega4,...,...,\omega\omega = \omega^2 = \sup\{\omega,\omega2,\omega3,...\}$,\\
$\omega^2+1,...,\omega^2+\omega,\omega^2+\omega+1,...\omega^2+\omega2,...,\omega^2+\omega^2=\omega^2 2,...,\omega^2 3,..., \omega^2 4,..., \omega^2 5,..., \omega^2 \omega = \omega^3, ...\omega^3 2,...,\omega^4,...,\omega^\omega = \sup\{\omega,\omega^2,\omega^3,...\}$,\\
$\omega^\omega+1,...,\omega^\omega 2,...,\omega^\omega \omega = \omega^{\omega+1}$,\\
$\omega^{\omega+2},...,\omega^{\omega+3},...,\omega^{\omega^2},...,\omega^{\omega^3},...,\omega^{\omega^\omega}$,...\\
And as expected we have $\omega^{\omega^{\omega^{\omega^{...}}}} = \sup\{\omega,\omega^2,\omega^3,...\} := \varepsilon_0$, and then $\varepsilon_0+1,...$, and then the whole thing again until $\varepsilon_1 = \varepsilon_0^{\varepsilon_0^{...}}$.

However, although this thing looks quite magnificent, they are all just countable (as we have just done it). Is there an uncoutnable ordinal? In other words, is there an uncountable well-ordered set?

\begin{thm} (11)\\
There is an uncountable ordinal.
\begin{proof}
\[IDEA: take sup of all countable ordinals. However, this might not be a set.\]\\
Let $R=\{A \in \mathcal{P}(\N \times \N)\}$ s.t. $A$ is a well-ordering of a subset of $\N$. Let $S$ be image of $R$ under 'order-type', i.e. $S$ is the set of all order-types of well-orderings of some subset of $\N$. Then $S$ is the set of all countable ordinals. Let $\omega_1$ be $\sup S$. Then $\omega_1$ is uncountable: otherwise, then $\omega_1 \in S$, so $\omega_1$ would be the greatest member of $S$. But then $\omega_1+1$ is also in $S$.
\end{proof}
\end{thm}

Note that, by contradiction, $\omega_1$ is the \emph{least} uncountable ordinal. $\omega_1$ has some strange properties, e.g.\\
1. $\omega_1$ is uncountable, but for any $\alpha < \omega_1$, we have $\{\beta:\beta < \alpha\}$ countable.\\
2. If $\alpha_1,\alpha_2,... < \omega_1$ is any sequence, then it is bounded in $\omega_1$: $\sup\{\alpha_1,...,\alpha_2\}$ is countable, so is less than $\omega_1$.

Similarly we have

\begin{thm} (11', Hartogs' lemma)\\
For any set $X$, there is an ordinal that  does not inject into $X$.\\
To see that, just replace $\mathcal{P}(\N\times\N)$ by $\mathcal{P}(X \times X)$ in the previous proof.\\
\end{thm}

Write $\gamma(X)$ for the least such ordinal -- e.g. $\gamma(\omega) = \omega_1$.

\newpage

\subsection{Successors and limits}
Given ordinal $\alpha$, does $\alpha$ (any set of order-type $\alpha$, e.g. $I_\alpha$) have a greatest element?\\
If yes: say $\beta$ is that greatest element. Then $\gamma<\beta$ or $\gamma =\beta$ $\implies$ $\gamma<\alpha$, and $\gamma<\alpha \implies \gamma < \beta$ or $\gamma = \beta$ (as we can't have $\gamma > \beta$). In other words, $\alpha = \beta^+$. In that case, we call $\alpha$ a \emph{successor};\\
If not: then $\forall \beta < \alpha$, $\exists \gamma < \alpha$ s.t. $\gamma > \beta$. So $\alpha = \sup\{\beta:\beta < \alpha\}$. (this is false in general, e.g. $\omega+5$). We call $\alpha$ a \emph{limit}.

For example, $5$ is a successor, $\omega+5$ is a successor, $\omega$ is a limit, $\omega+\omega$ is a limit. ($0$ is a limit as well).

For ordinals $\alpha,\beta$, define $\alpha+\beta$ by recursion on $\beta$ ($\alpha$ fixed) by: $\alpha+0=\alpha$, $\alpha+\beta^+ = (\alpha+\beta)^+$, $\alpha+\lambda = \sup \{\alpha+\gamma: \gamma < \lambda\}$ for $\lambda$ a non-zero limit.

For example, $\omega+1 = (\omega+0)^+ = \omega^+$, $\omega+2 = \omega^{++}$, $1+\omega = \sup \{1+\gamma:\gamma < \omega\} =\omega$ -- so addition is not commutative.

Officially, by 'recursion on the ordinals', we mean: define $\alpha+\gamma$ on $\{\gamma:\gamma \leq \beta\}$ (a set) recursively, plus uniqueness. Similarly for induction: if know $p(\beta) \forall \beta < \alpha \implies p(\alpha)$ (for each $\alpha$), then must have $p(\alpha) \forall \alpha$. If not, say $p(\alpha)$ false: then look at $\{\beta \leq \alpha : p(\beta)$ false $\}$.

Note that $\beta \leq \gamma \implies \alpha + \beta \leq \alpha +\gamma$ (induction on $\gamma$). Also, $\beta <\gamma \implies \alpha+\beta<\alpha+\gamma$. Indeed, $\gamma \geq \beta^+$, so $\alpha+\gamma\geq \alpha+\beta^+ = (\alpha+\beta)^+ > \alpha+\beta$. However, $1<2$, but $1+\omega = 2+\omega$.

\begin{prop} (12)\\
$\alpha+(\beta+\gamma) = (\alpha+\beta)+\gamma \forall \alpha,\beta,\gamma$ ordinals.
\begin{proof}
Induction on $\gamma$:\\
$0$: $\alpha+(\beta+0) = \alpha+\beta = (\alpha+\beta)+0$.\\
Successors: $(\alpha+\beta)+\gamma^+ = ((\alpha+\beta)+\gamma)^+ = (\alpha+(\beta+\gamma))^+ = \alpha+(\beta+\gamma)^+ = \alpha+(\beta+\gamma^+)$.\\
$\lambda$ a non-zero limit: $(\alpha+\beta)+\lambda = \sup\{(\alpha+\beta)+\gamma: \gamma<\lambda\} = \sup\{\alpha+(\beta+\gamma):\gamma<\lambda\}$.

Claim: $\beta+\lambda$ is a limit.\\
Proof of claim: We have $\beta + \gamma = \sup\{\beta+\gamma:\gamma<\lambda\}$. But $\gamma<\lambda \implies \exists \gamma' <\lambda$ with $\gamma <\gamma'$ $\implies $ $\beta+\gamma < \beta+\gamma'$. So $\{\beta+\gamma : \gamma<\lambda\}$ does not have a greatest element.

Back to the main proof, now $\alpha+(\beta+\gamma) = \sup\{\alpha+\delta:\delta<\beta+\lambda\}$. So want $\sup\{\alpha+(\beta+\gamma):\gamma<\lambda\{ = \sup\{\alpha+\delta:\delta <\beta+\lambda\}$. \\
$\leq$: $\gamma < \lambda \implies \beta+\gamma<\beta+\lambda$, so LHS $\subset$ RHS;\\
$\geq$: $\delta < \beta+\lambda \implies \delta<\beta+\gamma$, some $\gamma<\lambda$ (definition of $\beta+\lambda$). So $\alpha+\delta \leq \alpha+(\beta+\gamma)$.
\end{proof}
\end{prop}

Alternative viewpoint:\\
Above is the 'inductive' definition of $+$. Thereis also a synthetic definition: $\alpha+\beta$ is the order-type of $\alpha \sqcup \beta$ ($\alpha$ disjoint union $\beta$), with all of $\alpha$ coming before all of $\beta$.

Clearly we have $\alpha+(\beta+\gamma) = (\alpha+\beta)+\gamma$ with this definition (same order-type). We need:

\begin{prop} (13)\\
The synthetic and inductive definition of $+$ coincide.
\begin{proof}
Write $\alpha+\beta$ for inductive, $\alpha +' \beta$ for synthetic. Do induction on $\beta$ ($\alpha$ fixed).\\
$0$: $\alpha+0=\alpha=\alpha+'0$:\\
Successors: $\alpha+'\beta^+ = (\alpha+'\beta)^+ = (\alpha+\beta)^+=\alpha+\beta^+$;\\
$\lambda$ a non-zero limit: $\alpha+'\gamma$ = order-type of $\alpha \sqcup \lambda$ = $\sup$ of order-type of $\alpha\sqcup \gamma$, $\gamma <\lambda$ (nest union, so order-type of union = $\sup$ -- this was proved before) = $\sup(\alpha+'\gamma:\gamma<\lambda) = \sup(\alpha+\gamma:\gamma<\lambda) = \alpha+\lambda$.
\end{proof}
\end{prop}

Normally we prefer to use synthetic than inductive, \emph{if} we do have a synthetic definition available.

Ordinal multiplication:\\
Define $\alpha \beta$ recursively by:\\
$\alpha 0 = 0$, $\alpha(\beta^+) = \alpha\beta + \alpha$, $\alpha\lambda = \sup\{\alpha\gamma:\gamma <\lambda \}$ for $\lambda$ a non-zero limit. e.g:\\
$\omega 1 = \omega 0 + \omega = 0 + \omega = \omega$;\\
$\omega 2 = \omega 1 + \omega = \omega+\omega$;\\
$\omega\omega = \sup\{0,\omega,\omega+\omega,\omega+\omega+\omega,...\}$ (as in our big picture)\\
$2\omega = \sup\{2 \gamma:\gamma < \omega\} = \omega$, so multiplication is not commutative.

Similarly, this also has a synthetic definition: $\alpha\beta$ is the order-type of $\alpha\times\beta$, with $(x,y) < (z,t)$ if either $y<t$ or $y=t$ and $x<z$. We can check that these coincide on the previous examples. Also we can see $\alpha(\beta\gamma) = (\alpha\beta)\gamma$ etc.

We can define ordinal exponentiation, powers, etc. Similarly. For example, let's define exponentiation:\\
$\alpha^0 = 1$, $\alpha^{\beta^+} = \alpha^\beta \cdot \alpha$, $\alpha^\lambda = \sup\{\alpha^\gamma:\gamma<\lambda\}$ for $\lambda$ a non-zero limit.

Note that $\omega^1 = \omega$, $\omega^2 = \omega \cdot \omega$, and $2^\omega = \sup\{2^\gamma: \gamma<\omega\} = \omega$ (and is countable). This is different to what we expect from cardinality, but the notation in cardinality and here is different.

\newpage

\section{Posets and Zorn's lemma}
A \emph{Partially ordered} set or {poset} is a pair $(X,\leq)$ where $X$ is a set and $\leq$ is a relation on $X$ that is reflexive,transitive and antisymmetric. Write $x<y$ if $x\leq y ,x \neq y$. In terms of $<$, a poset is irreflexive and transitive.

For example, any total order is a partial order; $\N^+$ with divides; for any set $S$, $\mathcal{P}(S)$, wiith $x \leq y$ if $x \subset y$; for any $X \subset \mathcal{P}(S)$, with same relation of $x \leq y$ if $x \subset y$ (e.g. all subspaces of a given vector space).

In general, a hasse diagram for a poset $X$ consists of a drawing of the posets of $X$, with an upward line from $x$ to $y$ if $y$ \emph{covers} x, i.e. $y>x$, but no $z$ that $y>z>x$.

Hasse diagrams can be useful to visualize a poset (e.g. $\N$, usual order), or useless (e.g. $\Q$, usual order).

In a poset $X$, a \emph{chain} is a set $S \subset X$ that is totally ordered ($\forall x,y\in S: x \leq y$ or $y \leq x$). 

Note: chains can be uncountable, e.g. in $(\R,\leq)$ take $\R$.

We say $S \subset X$ is an antichain if no two elmeent are related.

For $S \subset X$, an \emph{upper bound} for $S$ is an $x \in X$ s.t. $x \geq y$ $\forall y \in S$.

Say $X$ is a \emph{least upper bound}, or \emph{supremum} for $S$, if $x$ is an upper bound for $S$, and $x \leq y$ for every upper bound $y$ of $S$.

Write $x = \sup S$ or $x = \vee S$.

e.g. In $\R$, $\{x:x^2<2\}$ has 7 as least upper bound, and $\sup = \sqrt{2}$ (so $\sup S$ need not be in $S$). In $\R$, $\Z$ has no upper bound. In $\Q$, $\{x:x^2 < 2\}$ has 7 as an upper bound, but no least upper bound.

We say a poset is \emph{complete} if every subset has a sup.

e.g. $(\R,\leq)$ is not complete: $\Z$ has no sup (so different to notion of 'completeness' from analysis);\\
$[0,1]$ is complete; $(0,1)$ is not complete: itself has no sup;\\
$\P(S)$ is always complete: $\{A_i: i \in I\}$ has sup $\cup_{i \in I} A_i$.

A function $f:X \to X$, where $X$ is any poset, is order-preserving if $f(x) \leq f(y)$ $\forall x \leq y$.

e.g. on $\N$ : $f(x) = x+1$; on $[0,1]: f(x) = \frac{1+x}{2}$ (halve the distance to 1); on $\P(S)$: $f(A) = A \cup \{i\}$ for some fixed $i \in S$.

not every order-preserving $f$ has a fixed point ($f(x) = x$), e.g. $f(x) = x+1$ on $\N$.

\begin{thm} (1, Knaster-Tarski fixed point theorem):\\
Let $X$ be a complete poset. Then every order-preserving function $f:X \to X$ has a fixed point.
\begin{proof}
Let $E = \{x \in X: x \leq f(x)\}$, and put $s = \sup E$. To show $f(s) = s$, we'll show that $s \leq f(s)$ and $s \geq f(s)$.\\
$s \leq f(S)$: Enough to show $f(s)$ is an upper bound for $E$ (as $s$ the \emph{least} upper bound). But $x \in E$ $\implies x \leq s$ $\implies f(x) \leq f(s) \implies x \leq f(x) \leq f(s)$.\\
$s \geq f(s)$: Enough to show $f(s) \in E$ (as $s$ an upper bound). We know $s \leq f(s)$, and want $f(s) \leq f(f(s))$. But that's true because $f$ is order preserving.
\end{proof}
\end{thm}

Note: in any complete poset $X$, we have a greatest element ($x s.t. x \geq y \forall y$), namely $\sup X$. A typical application of knaster-tarski:

\begin{thm} (2, schr$\ddot{o}$der-bernstein theorem)\\
Let $a,B$ be sets s.t. there exists injection $f:A \to B$ and an injection $g:B \to A$. Then there exists an bijection from $A$ to $B$.
\begin{proof}
Seek partition $A = P \sqcup Q,B = R \sqcup S$ s.t. $f(P) = R$ and $g(S) = Q$. Then we are done: set $h$ to be $f$ on $P$, $y^{-1}$ on $Q$, then $h:A \to B$ is a bijection.\\
i.e. we seek $P \subset A$ s.t. $A \setminus g(B\setminus f(P)) = P$. Define $\theta: \mathcal{P}(A) \to \mathcal{P}(A)$ via $P \to A \setminus g(B\setminus f(P))$. Then since $\mathcal{P}(A)$ is complete, $\theta$ order-preserving, there is a fixed point by K-T theorem.
\end{proof}
\end{thm}

\subsection{Zorn's Lemma}
An element $x$ in poset $X$ is \emph{Maximal} if no $y \in X$ has $y >x$.

Posets need not have a maximal element, for example $\Z,\Q,\R$.

\begin{thm} (3, Zorn's lemma)\\
Let $X$ be a non-empty poset in which every chain has an u.b.. Then $X$ has a maximal element.
\begin{proof}
Suppose not. Then for each $x \in X$ there is some $x' \in X$ with $x' > x$. Also, for any chain $C$ we have an upper bound $u(C)$. Pick $x \in X$. Define $x_\alpha \in X$, each $\alpha < \gamma(x)$ ($\gamma(x)$ is the u.b.?) recursively by: $x_0 = x$, $x_{\alpha+1} = x'_\alpha$, $x_\lambda = u(\{x_\alpha: \alpha < \lambda\})$ for $\lambda$ a non-zero limit (this is a chain by induction). Then $\alpha \to x_\alpha$ is an injection from $\gamma(X) to X$.
\end{proof}
\end{thm}

A typical application of Zorn: does every vecotr space have a basis? Recall that a basis is a LI spanning set.

e.g. $V=$ space of all real polynomials. We can take $1,x,x^2,...$\\
Let $V$ now be all real sequences. But $l_1 = (1,0,0,0,...)$, $l_2 = (0,1,0,0,...)$, then $l_1,l_2$ LI but not spanning! (recall span must be a finite linear combination!) It's easy to check that there is no countable basis. Also, it turns out that there is no \emph{explicit} basis.\\
$\R$ as a vector space over $\Q$. Basis is called a Hamel basis.

\begin{thm} (4)
Every vector space $V$ has a basis.
\begin{proof}
Let $X = \{A\subset V: A$ is LI$\}$, ordered by $\subset$. We seek a maximal element $M$ of $X$ (then we are done: if $M$ does not span then choose $x \not\in \bra M\ket$, and now $M \cup \{x\}$ is LI, contradiction.\\
We have $X\neq\phi$, as $\phi \in X$.\\
Given a chain $\{A_i: i \in I\}$ in $X$, put $A = \cup_{i \in I} A_i$, then $A > A_i$ $\forall i$, so just need $A \in X$, i.e. $A$ LI. Suppose $A$ is not LI, hten $\sum_{i=1}^n \lambda_i x_i = 0$ for some $x_1,...,x_n \in A$, and $\lambda_i$ scalars not all zero. We have $x_i \in A_{i_1},...,x_n \in A_{i_n}$ for some $i_1,...,i_n \in I$. But $A_{i_1},...,A_{i_n} \in A_{i_k}$, some $k$ (as they are nested), contradicting $A_{i_k}$ being LI.
\end{proof}
\end{thm}

Note: the only actualy maths (i.e. linear alebra) in the proof was the 'then done' part.

Another application: completeness theorem when proposition language uncountable.

\begin{thm} (5)\\
Let $S \subset L(P)$, where $P$ is any set. Then $S$ consistent implies that $S$ has a model.
\begin{proof}
We seek a maximal consistent $\bar{S} \supset S$. Then done: for each $t \in L(p)$ we have $\bar{S} \cup \{t\}$ or $\bar{S} \cup \{\neg t\}$ consistent (see chapter 1), hence $t \in \bar{S}$ or $\neg t \in \bar{S}$ by maximality of $\bar{S}$.  Now define $v(t) = 1$ if $t \in \bar{S}$, $0$ otherwise (as in chapter 1). Let $X$ be the set of all consistent subsets of $L(P)$, ordered by $\subset$. Then $X \neq \phi$, as $S \in X$. Given a non-empty chain $(T_i:i \in I)$ in $X$, put $T = \cup_{i \in I} T_i$. Then $T \supset T_i$ for each $i$, so we just need $T \in X$. We have $S \subset T$ as $T \neq \phi$. Also $T$ is consistent: if $T \vdash \perp$, then $\{t_1,...,t_n\} \vdash \perp$ for some $t_1,...,t_n \in T$. We have $t_1 \in T_{i_1},...,t_n \in T_{i_n}$ for some $i_1,...,i_n \in I$. But $T_{i_1},...,T_{i_n} \subset T_{i_k}$ for some $k$ (nested), contradicting $T_{i_k}$ being consistent.
\end{proof}
\end{thm}

One more:

\begin{thm} (6, well-ordering principle)\\
Every set $S$ can be well-ordered.\\
Note that this is very surprising for e.g $S=\R$.
\begin{proof}
Let $X = \{(A,R):A \subset S$ and $R$ is a well-ordering of $A\}$. We order this by: $(A,R) \leq (A',R')$ if $(A',R')$ extends $(A,R)$. Then $X \neq \phi$, as $(\phi,\phi) \in X$. Given a chain $((A_i,R_i) : i \in I)$, we have $(\cup_{i \in I} A_i, \cup_{i \in I} R_i) \in X$, and extends each $(A_i,R_i)$ from chapter 2. So by Zorn's lemma, $X$ has a maximal element $(A,R)$. We must have $A = S$: otherwise choose $x \in S\setminus A$ and take 'successor': well-order $A\cup\{x\}$ by putting $x > a$ $\forall a \in A$, contradicting maximality of $(A,R)$.
\end{proof}
\end{thm}

\begin{rem}
Proof of zorn was easy, but we used a lot of machinery there (ordinals, recursion, hartog's lemma).
\end{rem}

\subsection{Zorn's lemma and the axiom of choice}
In proof of Zorn's kemma, we chose, for each $x \in X$, and $x' \supset x$, i.e. we made infinitely many arbitrary choices, even by time we get to $x_\omega$. We did the same in part IA, to prove that a countable union of countable sets is countable. This is appealing to the axiom of choice, saying that we may choose an element of each set in a family of non-empty sets.

More precisely, the axiom of choice states that, if $(A_i : i \in I)$ is a family of sets, we have a choice function, meaning a function $f:I \to \cup_{i \in I} A_i$ s.t. $f(i) \in A_i$ $\forall i$. This is of a different characterto the other set-building rules in that the object whose existence is asserted is not uniquely specified by its properties (unlike ,e.g., $A \cup B$).\\
So often one points out when one has used axiom of choice.

Note that AC is trivial $|I| = 1$ ($A \neq \phi$ means $\exists x \in A$). Similarly for $I$ finite by induction. However, there is no derivation of AC from the other set-building rules for general $I$.

Also, we cannot prove ZL without AC because we can deduce AC from ZL:\\
Given family $(A_i:i \in I)$ of non-empty sets, a partial choice function is an $f:J \to \cup_{i \in I} A_i$ for some $J \subset I$, s.t. $f(j) \in A_j \forall j \in J$. Put $(J,f) \leq (J',f')$ if $J \subset J'$ and $f'|J = f$. This poset is not empty. Also, given a chain we have an upper bound being the union of them. So by ZL, there is a maximal of such. We must have $J=I$ in that case, as if not we can choose (???) $i \in I \setminus J$, $x \in A_i$ and put $J' = J \cup \{i\}$, $f' = f \cup \{(i,x)\}$. Contradiction.

Conclusion: ZL $\iff$ AC (in presence of the other set-building rules).

Also, we had $ZL \implies WO$, and $WO \implies AC$ trivially (well order $\cup i \in I A_i$ and let $f(i)$ be the least element of $A_i$). So we get $ZL \iff AC \iff WO$.

\subsection{The Bourbaki-Witt theorem}
Poset $X$ is \emph{chain-complete} if $X \neq \phi$ and every non-empty chain has a sup.\\
For example, any complete poset is chain-complete; any finite poset is chain-complete; and $\{A \subset V: A$ is LI$\}$, for a vector space $V$ is also.

We say $f:X \to X$ is \emph{inflationary} if $f(x) \geq x$ $\forall x$.

\begin{thm} (Bourbaki-Witt)\\
$X$ chain-complete, $f:X \to X$ inflationary. Then $f$ has a fixed point.\\
Note that BW follows instantly from ZL: take maximal $x$, and now $f(x) \geq x$ $\implies f(x) = x$.\\
However, we can prove BW without AC: we pick some $x_0 \in X$, then let $x_1 = f(x_0)$, $x_2 = f(x_1)$, ..., and let $x_\omega$ be the sup of them.

In chapter 2, we did not use AC, except in remark that well-ordering $\iff$ no decreasing sequence, and that $\omega_1$ is not a countable sup.
\end{thm}

In fact, it's easy to deduce ZL from BW (using AC). So we can view BW as the choice-free version of ZL.

\newpage
\section{Predicate Logic}
Recall that a group is a set equipped with functions:\\
$M:A^2 \to A$ ('arity' (slots) 2) and inverse $iA \to A$ ('arity' 1), and a constant $e \in A$ (kind of 'arity' 0), s.t.
\begin{equation*}
\begin{aligned}
& (\forall x,y,z \in A) (M(x,M(y,z)) = M(M(x,y),z)),\\
(\forall x \in A) (M(x,e) = x \wedge M(e,x) = x),\\
(\forall x \in A) (M(x,i(x)) = e \wedge M(i(x),x) = e)
\end{aligned}
\end{equation*}

And a poset is a set $A$ equipped with a predicate (relation) $\leq$ (arity 2) $\subset A^2$ s.t\\
\begin{equation*}
\begin{aligned}
(\forall x \in A) (x \leq x),\\
(\forall x,y,z \in A) ((x \leq y) \wedge (y \leq z) \implies x \leq z),\\
(\forall x,y \in A) ((x \leq y \wedge y \leq x) \implies x = y)
\end{aligned}
\end{equation*}

We try to establish these correspondence between propositional logic and predicate logic:
Language $\to$ e.g. language of groups (thinks like the definitions above);\\
Valuation $\to$ structure (set equipped with functions and relations of given arities);\\
Model of $S$ (valuation making each $s \in S$ true) $\to$ model of $S$ (structure in which each $s \in S$ holds);\\
$S \vDash t$ $\to $ same (e.g. In language of groups, should have the above 3 definitions $\vDash M(e,e) = e$ etc);\\
$S \vdash t$ $\to $ same (but a bit more complicated).

Let $\Omega$ (function symbols) and $\Pi$(relation symbols) be disjoint sets, and $\alpha$ (arity) : $\Omega \cup \Pi \to \N$. The \emph{language} $L=L(\Omega,\Pi,\alpha)$ is the set of \emph{formulae}, defined by:\\
$\bullet$ variables: $x_1,x_2,x_3,...$ (can use $x,y$, etc);\\
$\bullet$ terms: defined inductively by:\\
(i) each variable is a term;\\
(ii) If $f \in \Omega$, $\alpha(f) = n$, and $t_1,...,t_n$ are terms, then $ft_1...t_n$ is a term (and as always, we can add brackets, commas, etc).
For example, in the language of groups: $\Omega = \{m,i,e\}$ of arities $2,1,0$, $\Pi = \phi$. Some terms: $x_1,m(x_1,x_2),e,m(e,e),m(x_1,i(x_1))$, etc.\\
$\bullet$ Atomic formulae, consists of:\\
(i) $\perp$;\\
(ii) $(s=t)$, any terms $s,t$;\\
(iii) $\phi(t_1,...,t_n)$, any $\phi \in \Pi$, $\alpha(\phi) = n$, and terms $t_1,...,t_n$.\\
Again use the language of groups as example: $m(x,y) = m(y,x)$, $m(x,i(x)) = e$;\\
In language of posets: $\Omega = \phi$, $\Pi = \{\leq\}$ of arity 2. We could take $x=y,x\leq y,x \leq x$.\\
$\bullet$ Formulae: defined inductively by:\\
(i) Each atomic formula is a formula;\\
(ii) If $p,q$ are formulae, then so is $(p \implies q)$;\\
(iii) If $p$ is a formulae, $x$ is a variable, then $(\forall x) p$ is a formula.\\
e.g. in language of groupsL $(\forall x)(m(x,x) = e)$, $(\forall x)((m(x,x) = e) \implies (\exists y)(m(y,y) = x))$ (note that we have not talked about $\exists$ yet; we'll do that later).\\
In language of posets: $(\forall x)(x \leq x)$.\\

Notes:\\
1. A formula is just a string of symbols.\\
2. We can now write $\neg p$ for $p \implies \perp$, and similarly for $p \wedge q$, $p \vee q$ etc, and $(\exists x) p$ for $\neg(\forall x)(\neg p)$.

A term is \emph{closed} if it contains no variables. For example, $e,m(e,e),m(e,m(e,e))$. However, $m(x,i(x))$ is \emph{not} closed.\\
An occurrence of variable $x$ in formular $p$ is \emph{bound} if it is inside the brackets of '$\forall x$' quantifier. Otherwise, it is \emph{free}.\\
For example, in $m(x,x) = e \implies (\exists y) (m(y,y) = x)$, each $x$ is free and each $y$ is bound.\\
Note that in some cases we can make a variable both free and bound: $(m(x,x) = e) \implies (\forall x)(\forall y) (m(x,y) = m(y,x))$. We see that $x$ in LHS is free, but in RHS is bound (although it's not a very helpful expression).

A \emph{sentence} is a formula without free variables: e.g., $(\forall x) (m(x,e) = x)$. For formula $p$, variable $x$, term $t$, the \emph{substitution} $p[t/x]$ is obtained by replacing each free occurence of $x$ with $t$.\\
For example, if $p$ is $(\exists y) (m(y,y) = x)$, then $p[e/x]$ is $(\exists y) (m(y,y) = e)$.

\emph{Semantic entailment}:
An \emph{$L$-structure} consists of a non-empty (see later wfor why) set $A$ equipped with, for each $f \in \Omega$ with $\alpha(f) = m$, a function $f_A:A^m \to A$, and for each $\phi \in \Pi$, with $\alpha(\phi) = n$, a relation $\phi_A \subset A^n$.

For example, let $L$ be the language of groups: an $L$-structure is a set $A$ with functions $m_A:A^2 \to A$, $i_A: A \to A$, $e_A$ an element of $A$ (need not be a group! These have no 'meaning' yet).\\
Another example: $L$ be the language of posets: an $L$-structure is a set $A$ with a relation $\leq_A \subset A^2$.

We want to define the \emph{interpretation} $p_A \in \{0,1\}$ of a sentence $p$ in structure $A$, e.g. $(\forall x)(m(x,x) = e)$ shold be 'true in $A$' if $\forall a \in A: m_A(a,a) = e_A$.\\
So: 'insert $\in A$ subsubscript $A$ and say it aloud'.

\emph{Formal bit}:
For $L$-structure $A$, define \emph{interpretation} of a closed term $t$ to be $t_A \in A$, defined inductively by:\\
$(ft_1...t_n)_A = f_A({t_1}_A,...,{t_n}_A)$ for any $f \in \Omega$, $\alpha(f) = n$, closed terms $t_1,...,t_n$.\\
e.g. $m(e,i(e))_A =m_A (e_A, i_A (e_A))$ (and $e_A$ already defined).\\

Atomic formulae: define $p_A \in \{9,1\}$ for $p$ atomic by:\\
(i) $\perp_A = 0$;\\
(ii) 
\begin{equation*}
\begin{aligned}
(s=t)_A = \left\{\begin{array}{ll}
1 & s_A = t_A\\
0 & else
\end{array}
\right.
\end{aligned}
\end{equation*}
for $s,t$ closed terms;\\
(iii) 
\begin{equation*}
\begin{aligned}
\phi(t_1...t_n)_A = \left\{\begin{array}{ll}
1 & ({t_1}_A,...,{t_n}_A) \in \phi_A\\
0 & else
\end{array}
\right.
\end{aligned}
\end{equation*}
for $\phi \in \Pi$, $\alpha(\phi) = n$, closed terms $t_1,...,t_n$.

Sentences: $p_A$ defined inductively by:\\
(i)
\begin{equation*}
\begin{aligned}
(p \implies q)_A =\left\{\begin{array}{ll}
0 & p_A=1,q_A=0\\
1 & else
\end{array}
\right.
\end{aligned}
\end{equation*}
(ii)
\begin{equation*}
\begin{aligned}
((\forall i)_p)_A =\left\{\begin{array}{ll}
1 & p[\bar{a}/x]_A = 1 \text{ for all } a \in A\\
0 & else
\end{array}
\right.
\end{aligned}
\end{equation*}
where, for any $a \in A$, add constant symbol $\bar{a}$ to $L$, obtaining $L'$, and make $A$ an $L'$-structure by setting $\bar{a}_A = a$.

If $p$ has free variables, we can define $p_A \subset A^{\text{number of free variables of } p}$.\\
e.g. if $p$ is $(\exists y)(m(y,y) = x)$, then $p_A = \{a \in A: \exists b \in A$ with $m_A(b,b) = a\}$.

If $p_A = 1$, say $p$ \emph{true} in $A$, or $p$ holds in $A$, or $A$ is a \emph{model} of $p$. For $T$ a theoy (set of sentences), say $T$ semantically entails $p$, written $T \vDash p$, if every model of $T$ is a model of $p$.

$p$ is a \emph{tautology} if $\phi \vDash p$ (or just $\vDash p$), i.e. $p$ holds in every $L$-structure. For example, $\vDash (\forall x)(x=x)$.

Examples: theory of groups: $\Omega = (m,i,e)$, $\Pi = \phi$. Let
\begin{equation*}
\begin{aligned}
T = \{(\forall x)(\forall y)(\forall z)(m(x,m(y,z)) = m(m(x,y),z),(\forall x)(m(x,e) = x \wedge m(e,x) = x),(\forall x)(m(x,i(x)) = e \wedge m(i(x),x) = e)\}
\end{aligned}
\end{equation*}
Then an $L$-structure is a model of $T$ $\iff$ it is a group.

Say $T$ 'axiomatises' the class of groups or 'axiomatises the theory of groups'.

Sometimes call the elements of $T$ the 'axioms' of $T$.

Theory of fields: $\Omega = \{+,\times,-,0,1\}$. $T$ is: abelian group under $(+,-,0)$; $X$ is commutative, associative, distributive under $+$; $(\forall x)(1x=x)$, $\neg(1=0)$, $(\forall x)((\neg(x=0)) \implies (\exists y) (xy = 1))$. Then $T$ axiomatises the class of fields. E.g., $T\vDash$ inverses are unique: $(\forall x) ((\neg(x \neq 0)) \implies ((\forall y)(\forall x)((yx = 1 \wedge zx=1) \implies y=z))$.

Theory of posets: $\Omega = \phi, \Pi = \{\leq\}$.

$T$ is: $(\forall x)(x \leq x)$, $(\forall x)(\forall y)(\forall z)((x \leq y \wedge y \leq z) \implies x \leq z)$, $(\forall x)(\forall y)((x \leq y \wedge y \leq x) \implies x=y)$.

Theory of graphs: $\Omega = \phi$, $\Pi =\{a\}$ ('is adjacent to').

$T$ is $(\forall x)(\neg a(x,x))$, $(\forall x)(\forall y)(a(x,y) \implies a(y,x))$.

Proofs:

Logical axioms:\\
(1) $p \implies (q \implies p)$ (any formulae $p,q$);\\
(2) $p \implies (q \implies r)) \implies ((p \implies q) \implies (p \implies r))$ (any formulae $p,q,r$);\\
(3) $(\neg\neg p) \implies p$ (any formula $p$);\\
(4) $(\forall x) (x=x)$; (any variable $x$);\\
(5) $(\forall x)(\forall y)(x=y) \implies (p\implies p[y/x]))$ (any variables $x,y$, formula $p$ where $y$ is a bound);\\
(6) $((\forall x) p) \implies p [t/x]$ (any variable $x$, term $t$, formula $p$ with no variable in $t$ occuring bound in $p$)\\
(7) $((\forall x)(p \implies q)) \implies (p \implies (\forall x) q)$ (any variable $x$, formulae $p,q$ with $x$ not occurring free in $p$).

As rules of deduction, we take:\\
\emph{Modus Ponens}: From $p,p \implies q$ can deduce $q$;\\
\emph{Generalisation}: From $p$ can deduce $(\forall x)p$, if $x$ does not occur free in any premise used to prove $p$.

For $S \subset L$, $p \in L$, a proof of $p$ from $S$ is a finite sequence of formulae, ending with $p$, s.t. each line is a logical axiom, or a member of $S$, or follows from earlier lines by MP or GEN. Write $S\vdash p$ ('$S$ proves $P$') if there exists a proof of $p$ from $S$.

Example: $\{x=y,x=z\} \vdash\{y=z\}$ (use axiom 5, with $p$ being $'x=z')$.

1. $(\forall x)(\forall y) (x=y \implies (x=z \implies y=z))$ (axiom 5);\\
2. $(\forall x)(\forall y)(x=y \implies (x=z \implies y=z)) \implies (\forall y)(x=y \implies (x=z \implies y=z))$ (axiom 6, $t='x'$);\\
3. $(\forall y)(x=y \implies (x=z \implies y=z))$ (MP on 1,2);\\
4. $(\forall y) (x=y \implies (x=z \implies y=z)) \implies (x=y \implies (x=z \implies y=z))$ (axiom 6);\\
5. $x=y \implies (x=z \implies y=z)$ (MP on 3,4);\\
6. $x=y$ (hypothesis)\\
7. $x=y \implies y=z$ (mp on 5,6)\\
8. $x \implies z$ (hypothesis)\\
9. $y=z$ (mp on 7,8).

Aim: $T \vdash p$ $\iff$ $T \vDash p$.

e.g. if $p$ holds in every group then $p$ can be proved from the three group axioms (completely obvious).

\begin{prop} (1, deduction theorem)\\
Let $S \subset L$, $p,q \in L$. Then $S \vdash (p\implies q) \iff S \cup \{p\} \vdash q$.
\begin{proof}
Forward: as for propositional logic, from $p \implies q$ write down $p$ and apply MP to obtain $S \cup \{p\} \vdash q$;\\
Backward: as for propositional logic: the only new case is 'generalisation'. So in proof of $q$ from $S \cup \{p\}$ we have something like $r$ then $(\forall x) r$ (Gen), and have a proof of $p \implies r$ from $S$ (induction), and we want $S \vdash p \implies (\forall x) r$. In proof of $r$ from $S \cup \{p\}$, no premise had $x$ free. So in proof of $p \implies r$ from $S$, no premise had $x$ free. Hence $S \vdash (\forall x)(p \implies r$ (gen).\\
$\bullet$ If $x$ does not occur free in $p$: we have $S \vdash p \implies (\forall x) r$ by axiom 6 and MP;\\
$\bullet$ If $x$ does occur free in $p$: proof of $r$ from $S \cup \{p\}$ cannot have used $p$. So in fact $S \vdash (\forall x) r$ whence $S \vdash (p \implies (\forall x) r)$ by axiom 1.
\end{proof}
\end{prop}

\begin{prop} (2, soundness)\\
Let $S$ be a set of sentences, $p$ a sentence. Then if $S \vdash p$ then $S \vDash p$.
\begin{proof}
We have proof of $p$ from $S$, and a model $A$ of $S$, and we want $p_x = 1$. This is an induction down the lines of the proof.
\end{proof}
\end{prop}

For adequacy, we want if $S \vDash p$, i.e. that if $S \cup \{\neg p\} \vDash \perp$, then $S \cup \{\neg p\} \vdash \perp$.

\begin{thm} (3, model existence lemma, or completeness theorem)\\
Let $S \subset L$ be a set of setences. Then $S$ consistent implies that $S$ have a model.\\
Ideas:\\
$\bullet$ 1. Build model out of language: let $A$ be the set of closed terms of $L$, with operation line $(1+1) +_A (1+1) = (1+1) +(1+1)$;\\
$\bullet$ 2. Say for $S$ be the theory of fields: $(1+1)+1 \neq 1+(1+1)$, but $S \vdash (1+1)+1 = 1+(1+1)$. So quotient out by $s \sim t$ if $S \vdash s = t$;\\
$\bullet$ 3. Suppose $s$ is the fields of characteristic $2$ or $3$, i.e. field axioms, and the statement $1+1=0 \vee 1+1+1=0$. Then $S \not\vdash 1+1=0$. So $[1+1] \neq [0]$, where $[\cdot]$ denotes the equivalent class unrder $\sim$. Also, $S \not\vdash 1+1+1=0$, so $[1+1+1] \neq [0]$.

So our structure does not satisfy $1+1=0 \vee 1+1+1 = 0$. Then we need to extend $S$ to maximal consistent.

$\bullet$ 4. If $S$ is 'fields with a sqaure root of $2$': field axioms + $(\exists x) (xx=1+1)$. Maybe no closed term $t$ has $[tt] = [1+1]$. So $s$ lacks 'witnesses'.\\
Solution: for each $(\exists x|p$ in $S$, add new constant $c$ to language, and add $p[c/x]$ to $S$. (e.g. $cc=1+1$).\\
Now no longer maximal consistent, so go back to step 3.\\
Problem: this might not terminate.
\begin{proof}
We have consistent $S$ in language $L_0 = L(\Omega,\Pi)$. Extend to maximal consistent $S_1$ (zorn), so for each sentence $p \in L$, we have $p \in S_1$, or $(\neg p) \in S_1$. Thus $S_1$ is complete (for every $p$, $S_1 \vdash p$ or $S_1 \vdash (\neg p)$). Add witnesses: for each $(\exists x) p$ in $S_1$, add new constant $c$ and axiom $p[c/x]$. We obtain $T_1$ in language $L_1 = L(\Omega \cup C_1,\Pi)$ that has \emph{witnesses} for $S_1$ (if $(\exists x) p \in S$, then some closed term $t$ has $p[t/x] \in T_1$). It's easy to check $T_1$ consistent. Now extend $T_1$ to maximal consistent $S_2$ (in $L$). Add witnesses, obtaining $T_2$ in language $L_2 =L(\Omega \cup C_1 \cup C_2, \Pi)$.\\
Continue inductively.\\
Put $\bar{S} = S_1 \cup S_2 \cup ...$. In language $\bar{L} = L(\Omega \cup C_1\cup C_2 \cup ...)$.\\
$\bullet$ $\bar{S}$ is consistent: If $\bar{S} \vdash \perp$, then some $S_n \vdash \perp$ (as proofs are finite), contradiction;\\
$\bullet$ $\bar{S}$ is complete: given sentence $p \in \bar{L}$, we have $p \in L_n$ for some $n$ (as $p$ mentions only finitely many constants), so $S_{n+1} \vdash p$ or $S_{n+1} \vdash (\neg p)$ (choice of $S_{n+1}$).\\
$\bullet$ $\bar{S}$ has witnesses (for itself): given $(\exists x)p \in \bar{S}$, we have $(\exists x)p \in S_n$ for some $n$. So $p[t/x] \in T_n$ for some closed term $t$ (choice of $T_n$), whence $p[t/x] \in \bar{S}$.
\end{proof}
\end{thm}

On set of closed terms of $\bar{L}$, define $s \sim t$ if $\bar{S} \vdash (s=t)$.

This is clearly an equivalent relationship. let $A$ be the set of equivalent clases. Make $A$ into an $\bar{L}$-structure by setting $f_A([t_1],...,[t_2]) = [ft_1...t_n]$ (each $f \in \bar{\Omega},\alpha(f) = n$, closed terms $t_1...t_n$), $\varphi_A = \{([t_1],...,[t_n]): \bar{S} \vdash \phi(t_1,...,t_n)\}$ (each $\phi \in \Pi$, $\alpha(\phi) = n$, closed terms $t_1...t_n$).

Claim: $\phi_A = 1 \iff \bar{S} \vdash p$ for each setnence $p \in \bar{L}$. (Then done: $A$ is a model of $\bar{S}$, so $A$ is a model of $S$.
\begin{proof}
An easy induction:\\
\emph{Atomic sentences}:\\
$\perp$: $\perp_A = 0$ and $\bar{S} \not\vdash \perp$.\\
$s=t$: 
\begin{equation*}
\begin{aligned}
\bar{S} \vdash (s=t) &\iff [s] = [t]\\
&\iff s_A = t_A\\
&\iff (s=t)_A = 1
\end{aligned}
\end{equation*}
$\phi(t_1...t_n)$: same.

\emph{Induction step}:\\
$p \implies q$: 
\begin{equation*}
\begin{aligned}
\bar{S} \vdash (p \implies q) &\iff \bar{S} \vdash (\neg p)\text{ or } \bar{S} \vdash q\\
&\iff p_A = 0\text{ or } q_A = 1 (induction)\\
&\iff (p \implies q)_A =1 
\end{aligned}
\end{equation*}
where the second step is because, say if the forward direction doesn't hold, then $\bar{S} \vdash p$, $\bar{S} \vdash (\neg q)$ (since $\bar{S}$ is complete), but then $\bar{S} \vdash \neg(p \implies q)$, contradiction).

$(\exists x) p$:
\begin{equation*}
\begin{aligned}
\bar{S} \vdash (\exists x) p &\iff \bar{S} \vdash p[t/x]\\
&\iff p[t/x]_A = 1\\
&\iff ((\exists x)p)_A = 1
\end{aligned}
\end{equation*}
for some closed term $t$. The last line is because $A$ is the set of equivalent classes of closed terms.
\end{proof}

By remark before theorem 3 we have

\begin{coro} (4,adequacy)\\
If $S \vDash p$, then $S \vdash o$.
\end{coro}

Hence:
\begin{thm} (5, G$\ddot{o}$del's completeness theorem for first-order logic)\\
Let $S$ be a set of sentences and $p$ a sentence (in language $L$). Then $S \vDash p \iff S \vdash p$.\\
The proof is just soundness + adequacy.
\end{thm}

Note:\\
$\bullet$ If $L$ is countable (i.e .$\Omega,\Pi$ countable), then we don't need Zorn's lemma;\\
$\bullet$ 'First-order' means variables range over elements of our structure (not, e.g., subsets).

\begin{thm} (6, compactness)\\
Let $S \subset L$ be a set of sentences. Then if every finite subset of $S$ has a model, then $S$ has a model.
\begin{proof}
This is trivial if we replace $\vDash$ with $\vdash$ (as proofs are finite).
\end{proof}
\end{thm}

Note: we have no decidability theorem -- how to check if $S \vDash t$?

Some consequences of completeness/compactness:\\
Can we axiomatise the class of finite groups? In other words, we want some sentences $S$ (in language of groups) s.t. a structure is a model for $S$ $\iff$ it is a finite group.

However, this is not possible.

\begin{coro} (7)\\
the class of finite groups cannot be axiomatised (in language of groups).
\begin{proof}
Suppose $S$ axiomatises finite groups. We add to $S$ the sentences:
\begin{equation*}
\begin{aligned}
(\exists x_1) (\exists x_2) (\neg (x_1 =x_2))\\
(\exists x_1) (\exists x_2) (\exists x_3) (\neg (x_1 = x_2) \wedge \neg(x_1 = x_3) \wedge \neg(x_2 = x_3))\\
...
\end{aligned}
\end{equation*}
which stands for $|G| \geq 2$, $|G| \geq 3$, etc.\\
Then ever finite subset has a model (e.g. $\Z_n$, $n$ large). However, the set itself has no model -- contradicting compactness.
\end{proof}
\end{coro}

Similarly,
\begin{coro} (7')\\
Let $S$ be a theory in a language $L$. Then if $S$ has arbitrarily large finite models, then it has an infinite model.
\begin{proof}
Add sentences as in corollary 7, and apply compactness theorem.
\end{proof}
\end{coro}

So we know \emph{finiteness is not a first-order property}.

\begin{coro} (8, upward L$\ddot{o}$wenheim-Skolem theorem)\\
If a theory $S$ has an infinite model, then it has an uncoutnable model.
\begin{proof}
Add uncoutnably many consttants $\{c_i:i \in I\}$ to the language, and add to $S$ the set of sentences $c_i \neq c_j$ (for each distinct $i,j \in I$). Then any finite subset has a model. So the whole set has a model by compactness.
\end{proof}
\end{coro}

Similarly, we could find a model into which $P(P(R))$ injects (choose $I = P(P(R))$). E.g., there exists an infinite field ($\Q$), so there exists field as big as $P(P(R))$.

\begin{coro} (9, downward L$\ddot{o}$wenheim-Skolem theorem):\\
Let $S$ be a theory in countable language $L$. If $S$ has a model, then it has a countable model.
\begin{proof}
The model constructed in theorem 3 is countable.
\end{proof}
\end{coro}

\subsection{Peano Arithmetic}
We try to make the usual axioms for $\N$ into a first-order theory.\\
$L : \Omega = \{0,s,+,\times\}$, $\Pi = \phi$, axioms:\\
1. $(\forall x) (\neg s(x) = 0)$;\\
2. $(\forall x)(\forall y) (s(x) = s(y) \implies x=y)$;\\
3. $(\forall y_1)...(\forall y_n) [(p[0/x] \cap (\forall x) (p \implies p[s(x)/x])) \implies (\forall x) p]$.\\
($y_i$ in 3 are parameters).\\
4. $(\forall x) (x+0=x)$;\\
5. $(\forall x)(\forall y) (x+s(y) = s(x+y))$;\\
6. $(\forall x) (x+0=0)$;\\
7. $(\forall x)(\forall y) (x\times(y) = (x+y)+x)$.

These axioms are called Peano Arithmetic or Formal Number Theory.

Note on axiom 3: first guess shold have been 
\begin{equation*}
\begin{aligned}
(p[0/x] \cap (\forall x|(p \implies p[s(x)/x])) \implies (\forall x) p
\end{aligned}
\end{equation*}
But then missing properties like $x \geq y$ ($y$ chosen earlier).

Then PA has an infinite model, so by upward L-S, PA has an uncountable model that is not isomorphic to $\N$ trivially. Doesn't this contradict the fact that the usual axioms characterise $\N$ uniquely?

Answer: axiom 3 is only 'first-order induction' -- even in $\N$ itself, it refers to only countably many subsets (as opposed to true induction).

A subset $S \subset \N$ is called \emph{definable} if there exists $p \in L$, free variable $x$, s.t. $\forall m \in \N$ we have: $m \in S \iff p[m/x]$ holds in $\N$ (where by $m$ we mean $1+1+...+1$ ($m$ times)).

e.g. set of squares: $p(x)$ is $(\exists y) (yy=x)$;\\
set of primes: $p(x)$ is: $\neg(x=0) \cap \neg(x=1) \neg (\forall y) (y |x) \implies ((y=1) \vee (y=x))$, where $y|x$ is a short hand for $(\exists z) (yz = x)$, and by $1$ we mean $s(0)$.\\
Powers of $2$: $p(x)$ is $(\forall y) ((y|x \wedge y\ prime) \implies (y=2))$.

Exercise: powers of 4; challenge: powers of 6.

Is PA complete? in other words, for each sentence $p$, PA $\vdash p$ or PA $\vdash \neg p$?

\begin{thm} (G$\ddot{o}$del's incompleteness theorem)\\
PA is not complete.\\
Take $p$ with PA $\not\vdash p$, $PA \not\vdash \neg p$. We have $p$ holding in $\N$ or $(\neg p)$ holding in $\N$. Conclution: $\exists$ sentence $p$ s.t. $p$ is true in $\N$, but $PA \not\vdash p$.
\end{thm}

This does not contradict completeness; it shows that if $p$ true in all models of PA, then PA $\vdash p$.

\newpage
\section{Set Theory}
Aim: what does 'the universe of sets' look like?

Key starting point: view set theory as 'just another finite-order theory'.

\subsection{Zermelo-Fraenkel set theory}
We have $L$: $\Omega = \phi$, $\Pi = \{\varepsilon\}$, $\alpha(\epsilon) = 2$.

We'll have the ZF axioms: 2 to get started, 4 to build things, and 3 you might not think of at first.

Then a 'universe of sets' will mean a model ($V,\epsilon$) of the ZF axioms.

1. \emph{Axiom of extension}:\\
If two sets have the same mebmers, then they are equal:\\
$(\forall x)(\forall y) ((\forall z) (z \in x \iff z \in y) \implies (x=y))$.

Note: converse is an instance of a logical axiom.

2. \emph{Axiom of separtion}:\\
We can form a subset of a set, or precisely, given set $x$ and property $p(z)$, we can form the set of all $z \in x$ such that $p(z)$ holds:\\
$(\forall t_1) ... (\forall t_n) (\forall x)(\exists y) (\forall z) (z \in y \iff (z \in x \wedge p))$\\
This is actually an axiom scheme: for each formula $p$ and free variables $t_i$.

Note: we do want parameters, e.g. to have $\{z \in x: t \in z\}$, $t$ chosen earlier.

3. \emph{Axiom of empty-set}:\\
There is a set with no members.\\
$(\exists x) (\forall y) (\neg y \in x)$.

We write $\phi$ for the unique (by extension axiom) such set $x$. This is just an abbreviation: so $p(\phi)$ means $(\exists x) ((\forall y) (\neg y \in x) \wedge p(x))$.

Similarly, write $\{z \in x: p (z)\}$ for the set guaranteed by separation.

4. \emph{Axiom of pair-set}:\\
We can form $\{x,y\}$.\\
$(\forall x)(\forall y)(\exists z) (\forall t) (t \in z \iff t = x \vee t = y)$.

We write $\{x,y\}$ for this set, and $\{x\}$ for $\{x,x\}$.\\
We can now define the 'ordered pair' $(x,y)$ to be $\{\{x\},\{x,y\}\}$.\\
It's easy to check that $(x,y) = (t,u) \implies x=t \wedge y=u$ (follows from axiom so far).\\
Say $x$ is an ordered pair if $(\exists y) (\exists z) (x=(y,z))$, and we say $f$ is a function to mean $(\forall x) (x \in f \implies x$ is an ordered pair) $\wedge (\forall x)(\forall y)(\forall z)((x,y) \in f \wedge (x,z) \in f \implies y=z)$.

Can now define the domain of a function as follows: write $x = Dom f$ if $(f$ is a function) $\wedge (\forall z)(z \in x \iff (\exists t)((z,t) \in f)))$.

And write $f:x \to y$ for $(f$ is a function) $\wedge (x=Dom f|\wedge (\forall z)((\exists t)((t,z) \in f) \implies z \in y))$.

5. \emph{Axiom of union}:\\
We can form unions.\\
$(\forall x)(\exists y)(\forall z)(z \in y \iff (\exists t) (z \in t \wedge t \in x))$.

6. \emph{Axiom of power-set}:\\
We can form power-sets.\\
$(\forall x)(\exists y) (\forall z) (z \in y \iff z \subset x)$.\\
Here by $z \subset x$ we mean $(\forall t)(t \in z \implies t \in x)$.

Notes:\\
1. write $\cup x$ and $\mathcal{P}(x)$ for these two sets. We can write $x \cup y$, etc.\\
2. No extra axiom needed for interseionts: we can form $\cap x$ ($x \neq \phi$) as a subset of $y$ any $y \in x$. So ok by separation.\\
3. We can now form $x \times y$ as a suitable subset of $\mathcal{P}\mathcal{P}(x\cup y)$ -- since if $t \in x,u \in y$, then $(t,u) = \{\{t\},\{t,u\}\}\in\mathcal{P}\mathcal{P}(x \cup y)$. And then we can form the set of all functions from $x$ to $y$, as a subset of $\mathcal{P}(x \times y)$.

The next three are more subtle:

7. \emph{Axiom of infinity}:\\
So far, $V$ (the branch symbol) must be inifinite. For example, write $x^+ = x \cup \{x\}$, then easy to check that $\phi,\phi^+,\phi^{++},...$ are all distinct. We often write $0$ for $\phi$, $1$ for $\phi^+$,$2$ for $\phi^{++}$, etc. So $1 = \{0\}, 2 = \{0,1\},3=\{0,1,2\}$,etc. But does the structure $(V,\epsilon)$ have an infinite set -- e.g. $x$ with $\phi \in x,\phi^+ \in x$, ...?

We say $x$ is a successor set if $(\phi \in x) \wedge (\forall y) (y \in x \implies y^+ \in x)$.

Now let's state the axiom:\\
There is an infintie set/there is a successor set.\\
$(\exists x) (x$ is a successor set$)$.

Note that any intersection of successor sets is a successor set, so there exists a least one, called $\omega$. This will be our version, in $V$, of the natural numbers.

Thus $(\forall x)(x \in \omega \iff (\forall y)(y$ a successor set $\implies x \in y))$.

Note that if $x \subset \omega$ is a successor set then $x=\omega$ by definition:\\
$(\forall x)(x \subset \omega \wedge \phi \in x \wedge (\forall y)(y \in x \implies y^+ \in x)) \implies x=\omega)$. This is induction: genuine induction, over all $x \subset \omega$ (as opposed to in PA).

Also, it's easy to check $(\forall x \in \omega) (\neg x^+ = \phi)$, and $(\forall x \in \omega)(\forall y \in \omega)(x^+ = y^+ \implies x=y)$.

Thus: $\omega$ satisfies (in $V$) all the usual axioms for the natural numbers.

Say $x$ is finite if $(\exists y) (y \in \omega \wedge x$ bijects with $y$).

And then $x$ is countable if $x$ is finite or $x$ bijects with $y$.

8. \emph{Axiom of Foundation}:\\
"Sets are build up from simpler sets". We want to disallow $x \in x$: note that $\{x\}$ has no $\varepsilon$-minimal member; and also disallow $x \in y \in x$: note $\{x,y\}$ has no $\varepsilon$-minimal element, etc. And we also want to disallow the infinite sequence $x_1 \in x_0$, $x_2 \in x_1$, $x_3 \in x_2$,..., in which case $\{x_0,x_1,...\}$ has no $\varepsilon$-minimal element.

The axiom: every (non-empty) set has an $\varepsilon$-minimal element.\\
$(\forall x) (x \neq \phi \implies (\exists y) (y \in x \wedge (\forall z) (z \in x \implies z \not\in y))$.

Bonus lecture on next Wednesday 1pm (proof of incompleteness theorem, consistency of ZF)

9. \emph{Axiom of Replacement}:\\
We often say "for each $i \in I$ have $A_i$ -- take $\{A_i: i \in I\}$. However, how do we know they form a set? Alternatively, how do we know that $i \to A_i$ is a function?\\
We want to say "the image of a set under something that looks like a function is a set".

A digression on classes:\\
Idea: $x \to \{x\}$ (for all $x$). This looks like a function, but it isn't: e.g. every function has a domain as functions are sets of ordered pairs, and the domain is just the left element of all those pairs. However, the 'domain' of $x \to \{x\}$ is not a set (the universal 'set').

For an $L$-structure $V$, a collection $C$ of elements of $V$ is called a \emph{class} if there is a formula $p$, free variables $x$ (and maybe more) s.t. $x \in C$ $\iff p(x)$ holds in $V$. E.g. $V$ is a class: take $p(x)$ to be $x=x$.

For any $t$, $\{x: t \in x\}$ is a class: take $p(x)$ to be $t \in x$.\\
Note that every set $y$ is a class: take $p(x)$ to be $x \in y$.

If $C$ is not a set (in $V$), i.e. $\not (\exists y)(\forall x)(x \in y \iff p(x))$, say $C$ is a proper class. E.g., $V$ is a proper class, as is $\{x:x$ infinite$\}$, where by infinite we mean not finite.

Similarly, a function-class is a collection $F$ of ordered pairs from $V$, s.t. for some formula $p$, free variables $x,y$ (and maybe more), have $(x,y) \in F \iff p(x,y)$, and if $(x,y) \in F, (x,z) \in F$, then $y=z$.\\
For example, $x \to \{X\}$ is a function class: take $p(x,y)$ to be $y=\{x\}$.

---End of digression---

Let's now state the axiom of replacement: "the image of a set under a function-class is a set.\\
$(\forall t_1) ... (\forall t_n) ([(\forall x)(\forall y)(\forall z) ((p \wedge p[z/y] ) \implies y=z)] \implies [(\forall x)(\exists y) (\forall z) (z \in y \iff (\exists t) (t \in x \wedge p[t/x,z/y])])$\\
For each formula $p$, free variables $x,y,t_1,...,t_n$, i.e., the image of $x$ under $p$ is a set.

Eg. for any set $x$, we can form $\{\{t\}:t \in x\}$ using function class $t \to \{t\}$.

This is a 'bad' example, as it didn't need replacement -- see later for 'good' examples.

Those are the ZF axioms.

Note:\\
1: Sometimes separation is called 'comprehension', and sometimes fundation is called 'regularity'.\\
2. ZF axioms do not include AC: ZF + AC is called ZFC, where axiom of choice is: "every family of (non-empty) sets has a choice function" -- $ (\forall f) (f$ is a function $\wedge (\forall x) (x \in Dom f \implies f(x) \neq \phi)) \implies (\exists y) (y$ is a function $ \wedge Dom y = Dom f \wedge (\forall x)(x \in Dom f \implies g(x) \in f(x))))$.

Goal: what does a model $(V,\epsilon)$ of ZF look like?

Remark: we haven't proved ZF consistent (i.e. $\exists$ model of ZF). Sadly, ZF $\not\vdash$ "ZF has a model", i.e. it cannot be proved in ordinary maths (ZF or ZFC).

\iffalse
\begin{equation*}
\begin{aligned}

\end{aligned}
\end{equation*}
\fi


\end{document}
