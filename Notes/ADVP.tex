\documentclass[a4paper]{article}

\setlength{\parindent}{0pt}
\setlength{\parskip}{1em}

\pagestyle{headings}

\usepackage{amssymb}
\usepackage{amsmath}
\usepackage{amsthm}
\usepackage{mathtools}
\usepackage{graphicx}
\usepackage{hyperref}
\usepackage{color}
\usepackage{microtype}
\usepackage{tikz}
\usepackage{pgfplots}
\usepackage{pgfplotstable}

\newcommand{\N}{\mathbb{N}}
\newcommand{\Q}{\mathbb{Q}}
\newcommand{\Z}{\mathbb{Z}}
\newcommand{\R}{\mathbb{R}}
\newcommand{\C}{\mathbb{C}}
\newcommand{\D}{\mathcal{D}}
\renewcommand{\S}{\mathcal{S}}
\renewcommand{\P}{\mathbb{P}}
\newcommand{\F}{\mathbb{F}}
\newcommand{\E}{\mathbb{E}}
\newcommand{\bra}{\langle}
\newcommand{\ket}{\rangle}


\graphicspath{{Image/}}

\hypersetup{
    colorlinks=true,
    linktoc=all,
    linkcolor=blue
}

\theoremstyle{definition}
\newtheorem*{axiom}{Axiom}
\newtheorem*{claim}{Claim}
\newtheorem*{conv}{Convention}
\newtheorem*{coro}{Corollary}
\newtheorem*{defi}{Definition}
\newtheorem*{eg}{Example}
\newtheorem*{lemma}{Lemma}
\newtheorem*{notation}{Notation}
\newtheorem*{prob}{Problem}
\newtheorem*{post}{Postulate}
\newtheorem*{prop}{Proposition}
\newtheorem*{rem}{Remark}
\newtheorem*{thm}{Theorem}

\DeclareMathOperator{\vdiv}{div}
\DeclareMathOperator{\grad}{grad}
\DeclareMathOperator{\curl}{curl}
\DeclareMathOperator{\Ann}{Ann}
\DeclareMathOperator{\Fit}{Fit}
\DeclareMathOperator{\Diag}{Diag}
\DeclareMathOperator{\tr}{tr}
\DeclareMathOperator{\im}{im}
\DeclareMathOperator{\Mat}{Mat}
\DeclareMathOperator{\Log}{Log}
\DeclareMathOperator{\Isom}{Isom}
\DeclareMathOperator{\Mesh}{Mesh}
\DeclareMathOperator{\Sym}{Sym}
\DeclareMathOperator{\Aut}{Aut}
\DeclareMathOperator{\cosech}{cosech}
\DeclareMathOperator{\Card}{Card}
\DeclareMathOperator{\Gal}{Gal}


\setcounter{section}{-1}

\begin{document}

\title{Advanced Probability}

\maketitle

\newpage

\tableofcontents

\newpage

\section{Reviews}

\subsection{Measure spaces}

Let $E$ be a set. Let $\mathcal{E}$ be a set of subsets of $E$. We say that $\mathcal{E}$ is a $\sigma$-algebra on $E$ if:\\
$\bullet$ $\phi \in \mathcal{E}$;\\
$\bullet$ $\mathcal{E}$ is closed under countable unions and complements.\\
In that case, $(E,\mathcal{E})$ is called a \emph{measurable space}.

We call the elements of $\mathcal{E}$ \emph{measurable sets}.

Let $\mu$ be a function $\mathcal{E} \to [0,\infty]$. We say $\mu$ is a measure if:\\
$\bullet$ $\mu(\phi) = 0$;
$\bullet$ $\mu$ is countably additive: for all sequences $(A_n)$ of disjoint elements of $\mathcal{E}$, then
\begin{equation*}
\begin{aligned}
\mu(\bigcup_n A_n) = \sum_n \mu(A_n)
\end{aligned}
\end{equation*}
In that case, the triple $(E,\mathcal{E},\mu)$ is called a \emph{measure space}.

Given a topological space $E$, there is a smallest $\sigma$-algebra containing all the open sets in $E$. This is the \emph{Borel $\sigma$-algebra of $\mathcal{E}$}, denoted $\mathcal{B}(E)$.

In particular, for the real line $\R$, we will just write $\mathcal{B} = \mathcal{B}(\R)$ for simplicity.

\subsection{Integration of measurable functions}

Let $(E,\mathcal{E})$ and $(E',\mathcal{E}')$ be measurable spaces. A function $f:E \to E'$ is \emph{measurable} if $f^{-1}(A) = \{x \in E: f(x) \in A\} \in \mathcal{E} \forall A \in \mathcal{E}'$.

If we refer to a measurable function $f$ without specifying range, the default is $(\R,\mathcal{B})$.

Similarly, if we refer to $f$ as a non-negative measurable function, then we mean $E'=[0,\infty]$, $\mathcal{E}' = \mathcal{B}([0,\infty])$.

It is worth notice that under this set of definitions, a non-negative measurable function might not be $\R$-measurable (since we allowed $\infty$).

We write $m\mathcal{E}^+$ for set of non-negative measurable functions.

\begin{thm}
Let $(E,\mathcal{E},\mu)$ be a measure space. There exists a unique map $\tilde{\mu}: m\mathcal{E}^+ \to [0,\infty]$ such that:\\
$\bullet$(a) $\tilde{\mu}(1_A) = \mu(A)$ for all $A \in \mathcal{E}$, where $1_A$ is the indicator function;\\
$\bullet$(b) $\tilde{\mu}(\alpha f + \beta g) = \alpha\tilde{\mu}(f) + \beta\tilde{\mu}(g)$ for all $\alpha,\beta \in [0,\infty)$, $f,g \in m\mathcal{E}^+$ (linearity);\\
$\bullet$(c) $\tilde{\mu}(f) = \lim_{n \to \infty} \tilde{\mu}(f_n)$ for any non-decreasing sequence $(f_n:n \in \N)$ in $m\mathcal{E}^+$ such that $f_n(x) \to f(x)$ for all $x \in E$ (monotone-convergence).

We'll only prove uniqueness. For existence, see II Probability and Measure notes.
\end{thm}

From now on, write $\mu$ for $\tilde{\mu}$.\\
We'll call $\mu(f)$ the \emph{integral} of $f$ w.r.t. $\mu$.\\
We also write $\int_E f d\mu = \int E f(x) \mu(dx)$.

A \emph{simple function} is a finite linear combination of indicator functions of measurable sets with positive coefficients, i.e. $f$ is simple if 
\begin{equation*}
\begin{aligned}
f =\sum_{k=1}^n \alpha_k 1_{A_k}
\end{aligned}
\end{equation*}
for some $n \geq 0$, $\alpha_k \in (0,\infty), A_k \in \mathcal{E} \forall k = 1,...,n$.

From (a) and (b), for $f$ simple,
\begin{equation*}
\begin{aligned}
\mu(f) =\ sum_{k=1}^n \alpha_k \mu(A_k)
\end{aligned}
\end{equation*}
Also, if $f,g \in m\mathcal{E}^+$ with $f \leq g$, then $f+h = g$ where $h = g - f \cdot 1_{f < \infty} \in m\mathcal{E}^+$. Then since $\mu(h) \geq 0$, (b) implies $\mu(f) \leq \mu(g)$.

Take $f \in m\mathcal{E}^+$. Define for $x \in E$, $n \in \N$,
\begin{equation*}
\begin{aligned}
f_n(x) = \left(2^{-n} \lfloor 2^n f(x)\rfloor \right) \wedge n
\end{aligned}
\end{equation*}
where $\wedge$ means taking the minimum. Note that $(f_n)$ is a non-decreasing sequence of simple functions that converges to $f$ pointwise everywhere on $E$. Then by (c),
\begin{equation*}
\begin{aligned}
\mu(f) = \lim_{n \to \infty} \mu(f_n)
\end{aligned}
\end{equation*}
So we have shown uniqueness: $\mu$ is uniquely determined by the measure (provided that it exists, which we're not going to show).

When is $\mu(f)$ zero (for $f \in m\mathcal{E}^+$)? For measurable functions $f,g$, we say $f=g$ \emph{almost everywhere} if 
\begin{equation*}
\begin{aligned}
\mu(\{x \in E: f(x) \neq g(x) \}) = 0
\end{aligned}
\end{equation*}
i.e. they only disagree on a measure-zero set.

We can show, for $f \in m\mathcal{E}^+$, that $\mu(f)=0$ if and only if $f=0$ almost everywhere.

Let $f$ be a measurable function. We say that $f$ is \emph{integrable} if $\mu(|f|) < \infty$.

Write $L^1 = L^1(E,\mathcal{E},\mu)$ for the set of all integrable functions. We extend the integral to $L^1$ by setting $\mu(f) = \mu(f^+) - \mu(f^-)$, where 
\begin{equation*}
\begin{aligned}
f^\pm (x) = 0 \vee (\pm f(x))
\end{aligned}
\end{equation*}
where $\vee$ means the maximum (so $f = f^+ - f^-$). Note that now $f^+,f^-$ are both non-negative, with disjoint support. Then we can show that $L^1$ is a vector space, and $\mu:L^1 \to \R$ is linear.

\begin{lemma} (Fatou's lemma)\\
Let $(f_n:n \in \N)$ be any sequence in $m\mathcal{E}^+$. Then 
\begin{equation*}
\begin{aligned}
\mu(\liminf_{n\to \infty} f_n) \leq \liminf_{n \to \infty} \mu(f_n)
\end{aligned}
\end{equation*}
The proof is a straight forward application of monotone convergence.\\
The only hard part is to remember which way the inequality is (consider a sliding block function to the right).
\end{lemma}

\begin{thm} (Dominated convergence)\\
Let $(f_n:n \in \N)$ be a sequence of measurable functions on $(E,\mathcal{E})$. Suppose $f_n(x)$ converges pointwise as $n \to \infty$, with limit $f(x)$ say. Suppose further that $|f_n| \leq g$ for all $n$, for some integrable function $g$. Then $f_n$ is integrable for all $n$, so is $f$, and $\mu(f_n) \to \mu(f)$ as $n \to \infty$.
\end{thm}

\begin{defi}
We call a measure space $(\Omega,\mathcal{F},\P)$ such that $\P(\Omega)=1$ a \emph{probability space}. In this setting, measurable functions correspond to random variables, measurable sets correspond to events, almost everywhere corresponds to almost surely, and the integral $\P(X)$ corresponds to the expectation $\E(X) = \int_\Omega X d\P$, sometimes written $\E_\P(X)$ if we need to specify the underlying measure.
\end{defi}

\newpage

\section{Conditional expectation}
Throughout this section we'll use the default probability space $(\Omega \mathcal{F},\P)$.
\subsection{The discrete case}

Suppose $(G_n:n \in \N)$ is a sequence of disjoint set in $\mathcal{F}$ such that $\cup_n G_n = \Omega$ (so a partition of the space $\Omega$). Let $X$ be an integrable random variable. Set $\mathcal{G} = \sigma(G_n:n \in \N)$, which in this case is $\{\cup_{n \in I} G_n:I \subseteq \N\}$, i.e. all countable unions of $G_n$. Define $Y = \sum_{n \in \N} \E(X|G_n) 1_{G_n}$, where $\E(X|G_n) = \E(X 1_{G_n}) / \P(G_n)$, except in the case where $\P(G_n)$ we define LHS to be 0 as well). Now note that $Y$ is $\mathcal{G}$-measurable, is integrable, and $\E(Y 1_A) = \E(X 1_A)$ for any $A \in \mathcal{G}$. We'll write $Y =\E(X|\mathcal{G})$ almost surely, and say $Y$ is \emph{a version of} conditional expectation of $X$ given $\mathcal{G}$.

\subsection{Gaussian case}
Let $(W,X)$ be a Gaussian (normal) random variable in $\R^2$. Take a coarser $\sigma$-algebra $\mathcal{G}$ generated by $W$, which is $\{\{W in B\}: B \in \mathcal{B}\}$. Consider for $a,b \in \R$, the random variable $Y = aW + b$. We can choose $a,b$ so that $\E(Y-X) = a \E(W)+b - \E(X) = 0$, and $cov(Y-X,W) = a\ var(W) - cov(X,W) =0$. Then $Y$ is $\mathcal{G}$-measurable, is integrable, and $\E(Y1_A) = \E(X1_A)$ for all $A \in \mathcal{G}$. To see this, note $Y-X$ and $W$ are independent (as their covariance is 0), and $A = \{W \in B\}$ for some $B \in \mathcal{B}$. So for $A \in \mathcal{G}$, $\E((Y-X)1_A) = \E(Y-X)\P(A) = 0$.

\subsection{Conditional density functions}
Let $(U,V)$ be a random variable in $\R^2$ with density functoin $f(u,v)$, i.e.
\begin{equation*}
\begin{aligned}
\P((U,V) \in A) = \int_A f(u,v) dudv
\end{aligned}
\end{equation*}

Take $\mathcal{G} = \sigma(U) = \{\{U \in B\}:B \in \mathcal{B}\}$. Take a Borel measurable function $h$ on $\R$ and set $X = h(V)$, assume $X \in L^1(\P)$. Note $U$ has density function 
\begin{equation*}
\begin{aligned}
f(u) = \int_\R f(u,v) dv
\end{aligned}
\end{equation*}

Define the conditional density function
\begin{equation*}
\begin{aligned}
f(v|u) = f(u,v) / f(u)
\end{aligned}
\end{equation*}
where we define $0/0 = 0$.

Now set $Y = g(U)$, where 
\begin{equation*}
\begin{aligned}
g(u) = \int_\R h(v) f(v|u) dv
\end{aligned}
\end{equation*}

Then $g$ is a Borel-measurable functoin on $\R$ (not obvious), so $Y$ is a $\mathcal{G}$-measurable random variable, and is integrable and for all $A = \{U \in B\} \in \mathcal{G}$, $\E(Y 1_A ) = \E(X 1_A)$. To see this, 
\begin{equation*}
\begin{aligned}
\E(Y1_A) &=\int_\R g(u) 1_B(u) f(u) du\\
&= \int_\R \int_\R h(v)f(v|u) dv 1_B(u) f(u) du\\
&=\E(X1_A)
\end{aligned}
\end{equation*}
where at the last step we use Fubini's theorem (introduced later) to swap integrals, and note that we can combine $\int f(v|u) f(u)$ to get $f(u,v)$.

\subsection{Product measure and Fubini's theorem}
Take finite (or countably infinite) measure spaces $(E_1,\mathcal{E}_1,\mu_1)$ and $(E_2,\mathcal{E}_2,\mu_2)$. Write $\mathcal{E}_1 \otimes \mathcal{E}_2$ for the $\sigma$-algebra on $E_1 \times E_2$ generated by sets of the form $A_1 \times A_2$ where $A_i \in \mathcal{i}$ for $i=1,2$. We call $\mathcal{E}_1 \otimes \mathcal{E}_2$ the \emph{product $\sigma$-algebra}.

\begin{thm}
There exists a unique measure $\mu=\mu_1 \otimes \mu_2$ on $(E_1 \times E_2, \mathcal{E}_1 \otimes \mathcal{E}_2)$ such that
\begin{equation*}
\begin{aligned}
\mu(A_1 \times A_2) = \mu_1(A_1) \mu_2(A_2)
\end{aligned}
\end{equation*}
for all $A_i \in \mathcal{E}_i$ for $i=1,2$.
\end{thm}

\begin{thm} (Fubini's theorem)\\
Let $f$ be a non-negative measurable function $(E_1 \times E_2, \mathcal{E}_1 \otimes \mathcal{E}_2)$. For $x_1 \in E_1$, define in the obvious way
\begin{equation*}
\begin{aligned}
f_{x_1}(x_2) = f(x_1,x_2)
\end{aligned}
\end{equation*}
Then $f_{x_1}$ is $\mathcal{E}_2$-measurable for all $x_1 \in E_1$. Now define $f_1(x_1) = \mu_2(f_{x_1})$. Then $f_1$ is $\mathcal{E}_1$ measurable and $\mu_1(f_1) = \mu(f)$ (see part II Prob and Measure notes for the integrable case). Define $\hat{f}$ on $E_2 \times E_1$ by 
\begin{equation*}
\begin{aligned}
\hat{f}(x_2,x_1) = f(x_1,x_2)
\end{aligned}
\end{equation*}
then we can show $\hat{f}$ is $\mathcal{E}_2 \otimes \mathcal{E}_1$-measurable, and 
\begin{equation*}
\begin{aligned}
(\mu_2 \otimes \mu_1) (\hat{f}) = (\mu_1 \otimes \mu_2) (f)
\end{aligned}
\end{equation*}
So by Fubini,
\begin{equation*}
\begin{aligned}
\mu_2(f_2) = \hat{f}(\hat{f}) = \mu(f) = \mu_1(f_1)
\end{aligned}
\end{equation*}
with obvious notations. This means
\begin{equation*}
\begin{aligned}
\int_{E_2}\left(\int_{E_1} f(x_1,x_2) \mu_1 (dx_1) \right) \mu_2 (dx_2) = \int_{E_1} \left(\int_{E_2} f(x_1,x_2) \mu_2(dx_2) \right) \mu_1(dx_1)
\end{aligned}
\end{equation*}
Note that this also holds for just $f$ integrable.
\end{thm}

\end{document}
