\documentclass[a4paper]{article}

\setlength{\parindent}{0pt}
\setlength{\parskip}{1em}

\pagestyle{headings}

\usepackage{amssymb}
\usepackage{amsmath}
\usepackage{amsthm}
\usepackage{mathtools}
\usepackage{graphicx}
\usepackage{hyperref}
\usepackage{color}
\usepackage{microtype}
\usepackage{tikz}
\usepackage{pgfplots}
\usepackage{pgfplotstable}

\newcommand{\N}{\mathbb{N}}
\newcommand{\Q}{\mathbb{Q}}
\newcommand{\Z}{\mathbb{Z}}
\newcommand{\R}{\mathbb{R}}
\newcommand{\C}{\mathbb{C}}
\newcommand{\D}{\mathcal{D}}
\renewcommand{\S}{\mathcal{S}}
\renewcommand{\P}{\mathbb{P}}
\newcommand{\F}{\mathbb{F}}
\newcommand{\E}{\mathbb{E}}

\graphicspath{{Image/}}

\hypersetup{
    colorlinks=true,
    linktoc=all,
    linkcolor=blue
}

\theoremstyle{definition}
\newtheorem*{axiom}{Axiom}
\newtheorem*{claim}{Claim}
\newtheorem*{conv}{Convention}
\newtheorem*{coro}{Corollary}
\newtheorem*{defi}{Definition}
\newtheorem*{eg}{Example}
\newtheorem*{lemma}{Lemma}
\newtheorem*{notation}{Notation}
\newtheorem*{prob}{Problem}
\newtheorem*{post}{Postulate}
\newtheorem*{prop}{Proposition}
\newtheorem*{rem}{Remark}
\newtheorem*{thm}{Theorem}

\DeclareMathOperator{\vdiv}{div}
\DeclareMathOperator{\grad}{grad}
\DeclareMathOperator{\curl}{curl}
\DeclareMathOperator{\Ann}{Ann}
\DeclareMathOperator{\Fit}{Fit}
\DeclareMathOperator{\Diag}{Diag}
\DeclareMathOperator{\tr}{tr}
\DeclareMathOperator{\im}{im}
\DeclareMathOperator{\Mat}{Mat}
\DeclareMathOperator{\Log}{Log}
\DeclareMathOperator{\Isom}{Isom}
\DeclareMathOperator{\Mesh}{Mesh}
\DeclareMathOperator{\Sym}{Sym}
\DeclareMathOperator{\Aut}{Aut}
\DeclareMathOperator{\cosech}{cosech}
\DeclareMathOperator{\Card}{Card}
\DeclareMathOperator{\Gal}{Gal}


\setcounter{section}{-1}

\begin{document}

\title{Mathematical Biology}

\maketitle

\newpage

\tableofcontents

\newpage

\section{Miscellaneous}

Course notes online: Julia Gog(www.damtp.cam.ac.uk/research/dd/teaeching, 2013-2017), Peter Haynes(www.damtp.cam.ac.uk/user/phh/mathbio.html)

Moodle page: Handwritten notes by lecture; Matlab/Python programming examples; solved exercises.

This course involves 3 models: Deterministic temporal models (11 lectures), Stochastic temporal models (5 lectures), Deterministic spatio-temporal models (8 lectures).

The focus of this course is biochemical reactions and population processes.

(some introductory speech)

\begin{eg} (1, Transient population)
If we use $n(t)$ to denote the size of a population, we may want to model $\frac{dn}{dt} = f(n)$ by an ODE, or maybe if we have several components $\mathbf{n}(t)$ then we may want to model $\frac{d\mathbf{n}}{dt} = \mathbf{f}(\mathbf{n})$ which is a system of ODEs.

Note that although $n$ should be an integer (discrete), when $n >> 1$ we may model it with continuous equations.
\end{eg}

\begin{eg} (2)
$n \to \partial_t P(n,t) = W \cdot P(n,t)$, Markov processes. Here $P(n,t)$ is a probablity(?), $n$ being a state, and $W$ being the transition matrix.
\end{eg}

\begin{eg} (3)\\
If we include spatial aspect, we may have $n(t)$ becoming $n(x,t)$. Now there might be 'diffusion': $\partial_t n(x,t) = f(n(x,t)) + D \nabla^2 (x,t)$ where $\nabla^2 = \frac{\partial^2}{\partial x^2}$; this is the reaction-diffusion equation.
\end{eg}

\newpage

\section{Birth-death models}
The general idea is that we have a population of size $n(t)$; per capita per unit time, we have births of rate $b$ and deaths of rate $d$. Then we can write $$n(t+\Delta t) = n(t) + bn\Delta t - dn \Delta t$$ So we have an ODE $$\frac{dn}{dt} = (b-d)n = rn$$ where $r = b-d$. This has an easy solution $n(t) = n_0 e^{rt}$, assuming $r$ is a constant. We see that if $r$ is positive then the population grows exponentially, and if $r$ is negative then the population decreases to 0 asymptotically.

Now probably $b$ and $d$ are related to $n$ by $b(n) = bn$ and $d(n) = dn^2$ due to competition. Then we have $$\frac{dn}{dt} = bn-dn^2$$ which we can definitely rewrite as $$\frac{dn}{dt}=\alpha n(1-n)$$ by some change of variable on $n$. Now

\begin{equation*}
\begin{aligned}
\frac{dn}{n(1-n)}&=\alpha dt\\
\implies \frac{dn}{n} + \frac{dn}{1-n} &= \alpha dt\\
\implies \ln n - \ln(1-n) &= \alpha t + c\\
\implies n &= \frac{n_0 e^{\alpha t}}{(1-n_0) + n_0 e^{\alpha t}}
\end{aligned}
\end{equation*}
where we are given that $t=0$, $n=n_0$. If $t \gg \frac{1}{\alpha}$, when $t \to \infty$ we have $n(t) \to 1$. Now we can investigate if the population size is stable, and if it has any fixed points.



\begin{equation*}
\begin{aligned}
1
\end{aligned}
\end{equation*}

\end{document}
