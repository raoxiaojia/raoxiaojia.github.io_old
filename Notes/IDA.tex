\documentclass[a4paper]{article}

\setlength{\parindent}{0pt}
\setlength{\parskip}{1em}

\pagestyle{headings}

\usepackage{amssymb}
\usepackage{amsmath}
\usepackage{amsthm}
\usepackage{mathtools}
\usepackage{graphicx}
\usepackage{hyperref}
\usepackage{color}
\usepackage{microtype}
\usepackage{tikz}
\usepackage{pgfplots}
\usepackage{pgfplotstable}

\newcommand{\N}{\mathbb{N}}
\newcommand{\Q}{\mathbb{Q}}
\newcommand{\Z}{\mathbb{Z}}
\newcommand{\R}{\mathbb{R}}
\newcommand{\C}{\mathbb{C}}
\newcommand{\D}{\mathcal{D}}
\renewcommand{\S}{\mathcal{S}}
\renewcommand{\P}{\mathbb{P}}
\newcommand{\F}{\mathbb{F}}
\newcommand{\E}{\mathbb{E}}
\newcommand{\bra}{\langle}
\newcommand{\ket}{\rangle}


\graphicspath{{Image/}}

\hypersetup{
    colorlinks=true,
    linktoc=all,
    linkcolor=blue
}

\theoremstyle{definition}
\newtheorem*{axiom}{Axiom}
\newtheorem*{claim}{Claim}
\newtheorem*{conv}{Convention}
\newtheorem*{coro}{Corollary}
\newtheorem*{defi}{Definition}
\newtheorem*{eg}{Example}
\newtheorem*{lemma}{Lemma}
\newtheorem*{notation}{Notation}
\newtheorem*{prob}{Problem}
\newtheorem*{post}{Postulate}
\newtheorem*{prop}{Proposition}
\newtheorem*{rem}{Remark}
\newtheorem*{thm}{Theorem}

\DeclareMathOperator{\vdiv}{div}
\DeclareMathOperator{\grad}{grad}
\DeclareMathOperator{\curl}{curl}
\DeclareMathOperator{\Ann}{Ann}
\DeclareMathOperator{\Fit}{Fit}
\DeclareMathOperator{\Diag}{Diag}
\DeclareMathOperator{\tr}{tr}
\DeclareMathOperator{\im}{im}
\DeclareMathOperator{\Mat}{Mat}
\DeclareMathOperator{\Log}{Log}
\DeclareMathOperator{\Isom}{Isom}
\DeclareMathOperator{\Mesh}{Mesh}
\DeclareMathOperator{\Sym}{Sym}
\DeclareMathOperator{\Aut}{Aut}
\DeclareMathOperator{\cosech}{cosech}
\DeclareMathOperator{\Card}{Card}
\DeclareMathOperator{\Gal}{Gal}


\setcounter{section}{-1}

\begin{document}

\title{Introduction to Discrete Analysis}

\maketitle

\newpage

\tableofcontents

\newpage

\section{Introduction}
asdasd

\newpage

\section{The discrete Fourier transform}

Let $N$ be a fixed positive integer. Write $\omega$ for $e^{2\pi i/N}$, and $\Z_N$ for $\Z/n\Z$. Let $f:\Z_N \to \C$. Given $r \in \Z_N$, define $\hat{f}(r)$ to be
\begin{equation*}
\begin{aligned}
\frac{1}{N} \sum_{x \in \Z_N} f(x) \omega^{-rx}
\end{aligned}
\end{equation*}
From now on we use the notation $\E_{x \in \Z_N}$ for $\frac{1}{N}\sum_{x \in \Z_N}$, so $\hat{f}(r) = \E_x f(x) e^{-\frac{2\pi irx}{N}}$.

If we write $\omega_r$ for the function $x \to \omega^{rx}$, and $\bra f,g\ket$ for $\E_x f(x) \overline{g(x)}$, then $\hat{f}(r) = \bra f,\omega_r \ket$. So the discrete fourier transforn is basically expanding the function $f$ in the set of orthonormal basis $\omega_r$.

Let us write $||f||_p$ for $\E_x |f(x)|^p)^{1/p}$ (the $L_p$-norm), and call the resulting space $L_p(\Z_n)$.

Important convention: we use \emph{averages} for the 'original functions' in 'physical spaces', and \emph{sums} for their Fourier transforms in 'frequency space' (referring to $\E$: $\bra,\ket$ is average in the original space but just $\sum$ in frequency space, i.e. for $\hat{f},\hat{g}$ etc.)

\begin{lemma} (1, Parseval's identity)\\
    If $f,g: \Z_n \to \C$, then $\bra \hat{f},\hat{g}\ket = \bra f,g\ket$.
    \begin{proof}
        \begin{equation*}
            \begin{aligned}
                \bra\hat{f},\hat{g}\ket &= \sum_r \hat{f}(r) \overline{\hat{g}(r)}\\
                &= \sum_r (\E_x f(x) \omega^{-rx}) (\overline{\E_y g(y) \omega^{-ry}})\\
                &= \E_x \E_y f(x) \overline{g(y)} \sum_r \omega^{-r(x-y)}\\
                &= \E_x \E_y f(x) \overline{g(y)} n\delta_{xy}\\
                &= \bra f,g\ket
            \end{aligned}
            \end{equation*}
    \end{proof}
\end{lemma}

\begin{lemma} (2, Convolution identity)\\
    \begin{equation*}
        \begin{aligned}
            \widehat{f*g}(r) = \hat{f}(r)\hat{g}(r)
        \end{aligned}
    \end{equation*}
    where
    \begin{equation*}
        \begin{aligned}
            (f*g)(x) = \E_{y+z=x}f(y)g(z) = \E_y f(y)g(x-y)
        \end{aligned}
    \end{equation*}
    \begin{proof}
        \begin{equation*}
            \begin{aligned}
                \widehat{f*g}(r) &= \E_x f*g(x)\omega^{-rx}\\
                &=\E_x \E_{y+z=x}f(y)g(z)\omega^{-rx}\\
                &=\E_x\E_{y+z=x} f(y)g(z) \omega^{-ry}\omega^{-rz}\\
                &=\E_y\E_z f(y) \omega^{-ry} g(z) \omega^{-rz}\\
                &=\hat{f}(r) \hat{g}(r)
            \end{aligned}
        \end{equation*}
    \end{proof}

\end{lemma}

\begin{lemma} (3, Inversion formula)\\
    \begin{equation*}
        \begin{aligned}
            f(x) = \sum_r \hat{f}(r) \omega^{rx}
        \end{aligned}
    \end{equation*}
    (note the sign of $\omega^{rx}$).
    \begin{proof}
        \begin{equation*}
            \begin{aligned}
                \sum_r \hat{f}(r) \omega^{rx} &= \sum_r \E_y f(y) \omega^{r(x-y)}\\
                &= \E_y f(y) \sum_r \omega^{r(x-y)}\\
                &= \E_y f(y) n\delta_{xy}\\
                &=f(x)
            \end{aligned}
        \end{equation*}
        This is really just the statement that we get the original vector back when we sum up its components.
    \end{proof}
\end{lemma}

Further observations: If $f$ is real-valued, then $\hat{f}(-r) = \E_x f(x) \omega^{rx} = \overline{\E_x f(x) \omega^{-rx}} = \overline{\hat{f}(r)}$.

If $A \subset \Z_n$, write $A$ (instead of $1_A,\chi_A$) for the characteristic function of $A$. Then $\hat{A}(0) = \E_x A(x) = \frac{|A|}{N}$, the \emph{density} of $A$.

Also, $||\hat{A}||^2_2 = \bra\hat{A},\hat{A}\ket = \bra A,A\ket = \E_x A(x)^2 = \E_x A(x) = \frac{|A|}{N}$, again the density.

Let $f:\Z_n \to \C$. Given $\mu \in \Z_n$, define $f_\mu(x)$ to be $f(\mu^{-1}x)$ (so we need $(\mu,N) = 1$). Then
\begin{equation*}
    \begin{aligned}
        \hat{f}_\mu(r) &= \E_x f_\mu (x) \omega^{-rx}\\
        &= \E_x f(x/\mu) \omega^{-rx}\\
        &=\E_x f(x) \omega^{-r\mu x}\\
        ^= \hat{f}(\mu r)
    \end{aligned}
\end{equation*}

\subsection{Roth's theorem}
\begin{thm} (4)
    For every $\delta > 0$, $\exists N$ s.t. if $A \subset \{1,...,N\}$ is a set of size at least $\delta N$, then $A$ must contain an arithmetic progression of length 3.\\
    This is also true for 4,5,..., but the proof is much harder -- Szemeredi's theorem.\\
    Basic strategy of proof: show that if $A$ has density $\delta$ and no AP of length 3 (3AP), then there's a long AP in $P \subset \{1,2,...,n\}$ s.t.
    \begin{equation*}
        \begin{aligned}
            |A \cap P| \geq (\delta+c(\delta)) |p|
        \end{aligned}
    \end{equation*}
    where $c(\delta)$ is some positive number. But then we can continue this argument to expand $A\cap P$ to infinity (note that $|A\cap P|$ is an integer, so each time increase by 1 at least).

    The best known relationship between $\delta$ and the $N$ required is around $\delta \sim \frac{c}{\log\log N}$ for some constant $c$.
\end{thm}

---Lecture 2---

\begin{lemma} (5)\\
    Let $N$ be odd, $A,B,C \subset \Z_N$ have densties $\alpha,\beta,\gamma$.\\
    If $\max_{r \neq 0} |\hat{A}(r)| \leq \frac{\alpha(\beta\gamma)^{1/2}}{2}$ and $\frac{\alpha\beta\gamma}{2} > \frac{1}{N}$, then there exists $x,d \in \Z_N$ with $d \neq 0$ s.t. $(x,x+d,x+2d) \in A \times B \times C$.
    \begin{proof}
        \begin{equation*}
            \begin{aligned}
                \E_{x,d} A(x) B(x+d) C(x+2d) &= \E_{x+z=2y} A(x)B(y)C(z)\\
                &=\E_u (\E_{x+z=u}A(x)C(z)) \E_{2y=u} B(y)\\
                &=\E_u A * C(u) B_2(u)\\
                &=\bra A*C,B_2\ket\\
                &=\bra\widehat{A*C},\hat{B}_2\ket\\
                &=\bra \hat{A}\hat{C},\hat{B}_2\ket\\
                &=\sum_r \hat{A}(r)\hat{C}(r)\hat{B}(-2r)\\
                &=\alpha\beta\gamma + \sum_{r \neq 0} \hat{A}(r)\hat{C}(r)\hat{B}(-2r)
            \end{aligned}
        \end{equation*}
        Recall here the notation is $B_2(u) = B(u/2)$. now
        \begin{equation*}
            \begin{aligned}
                |\sum_{r\neq 0} \hat{A}(r)\hat{B}(-2r)\hat{C}(r)| &\leq \frac{\alpha(\beta\gamma)^{1/2}}{2}\sum_{r \neq 0} |\hat{B}(-2r)||\hat{C}(r)|\\
                &\leq \frac{\alpha(\beta\gamma)^{1/2}}{2} \left(\sum_r |\hat{B}(-2r)^2\right)^{1/2}\left(\sum_r |\hat{C}(r)|^2\right)^{1/2} \ \text{By Cauchy-Schwarz}\\
                &=\frac{\alpha(\beta\gamma)^{1/2}}{2} ||\hat{B}||_2 ||\hat{C}||_2\\
                &= \frac{\alpha(\beta\gamma)^{1/2}}{2} ||B||_2||C||_2\\
                &= \frac{\alpha\beta\gamma}{2}
            \end{aligned}
        \end{equation*}
        The contribution to $\E_{x,d} A(x)B(x+d)C(x+2d)$ from $d=0$ is at most $\frac{1}{N}$, so if $\frac{\alpha\beta\gamma}{2} > \frac{1}{N}$, we are done.
    \end{proof}
\end{lemma}

Now let $A$ be a subset of $\{1,...,N\}$ with density $\geq \delta$ and let $B=C=A \cap [\frac{N}{3},\frac{2N}{3})$. If $B$ has density $<\frac{\delta}{5}$ (??), then either $A \cap [1,\frac{N}{3}]$ or $A \cap [\frac{2N}{3},N]$ has density at least $\frac{2\delta}{5}$. In that case we find an AP $P$ of length about $N/3$ such that $|A \cap P| / |P| \geq \frac{6\delta}{5}$.\\
Otherwise, we find that if $\max_{r \neq 0}|\hat{A}(r)| \leq \frac{\delta}{10}$ and $\frac{\delta^3}{50} > \frac{1}{N}$, then $A \times B \times C$ contains a 3AP, so $A$ contains a 3AP.\\
So if $A$ does not contain a 3AP, then either we find $P$ of length about $N/3$ with $|A\cap P| / |P| \geq \frac{6\delta}{5}$, or ther exists $r \neq 0$ s.t. $|\hat{A}(r)| \geq \frac{\delta}{10}$.

\begin{defi}
    If $X$ is a finite set and $f:X \to \C$, $Y \subset X$, write $osc(f|_Y)$ to mean $\max_{y_1,y_2 \in Y} |f(y_1)-f(y_2)|$ (I think \emph{amplitude} is a better word for this).
\end{defi}

\begin{lemma} (6)\\
    Let $r \in \Z_n$ and let $\varepsilon > 0$. Then there is a partition of $\{1,2,...,N\}$ into arithmetic progressions $P_i$ of length at least $c(\varepsilon)\sqrt{N}$ such that
    \begin{equation*}
        \begin{aligned}
            osc(\omega_r|_{P_i}) \leq \varepsilon
        \end{aligned}
    \end{equation*}
    for each $i$.
    \begin{proof}
        Let $t = \lfloor \sqrt{N}\rfloor$. Of the numbers $1,\omega^r,...,\omega^tr$, there must be two that differ by at most $\frac{2\pi}{t}$.\\
        If $|\omega^{ar}-\omega^{br}| \leq \frac{2\pi}{t}$ with $a < b$, then $|1-\omega^{dr}| \leq \frac{2\pi}{t}$ where $d=b-a$. Then $|\omega^{urd}-\omega^{vrd}| \leq |\omega^{urd}-\omega^{(u+1)rd}| + ... + |\omega^{(v-1)rd}-\omega^{vrd}| \leq \frac{2\pi}{t}(v-u)$.\\
        So if $P$ is a progression with common differnece $d$ and length $l$, then $osc(\omega_r|_P) \leq \frac{2\pi l}{t}$. So divide up $\{1,...,N\}$ into residue classes mod $d$, and partition each residue class into parts of length between $\frac{\varepsilon t}{4\pi}$ and $\frac{\varepsilon t}{2\pi}$ (possible, since $d \leq t \leq \sqrt{N}$).\\
        We are done, with $c(\varepsilon) = \frac{\varepsilon}{16}$ (a casual choice).
    \end{proof}
\end{lemma}
Now let us use the information that $r \neq 0$ and $|\hat{A}(r)| \geq \frac{\delta^2}{10}$.\\
Define the \emph{balanced function} $f$ of $A$ by $f(x) = A(x)=\frac{|A|}{N}$ for each $x$.\\
Note that $\hat{f}(0) = 0$ and $\hat{f}(r) = \hat{A}(r)$ for all $r \neq 0$.

Now let $P_1,...,P_m$ be given by Lemma 6 with $\varepsilon = \delta^2/20$. Then
\begin{equation*}
    \begin{aligned}
        \frac{\delta^2}{10} &\leq |\hat{f}(r)|\\
        &= \frac{1}{N} |\sum_x f(x) \omega^{-rx}|\\
        &\leq \frac{1}{N} \sum_{i=1}^m |\sum_{x \in P_i} f(x) \omega^{-rx}|\\
        &\leq \frac{1}{N} \sum_{i=1}^N \left[\left|\sum_{x \in P_i} f(x) \omega^{-rx_i}\right| + \left|\sum_{x \in P_i} f(x) (\omega^{-rx} - \omega^{-rx_i})\right|\right] \ x_i \in P_i \text{ arbitrary}\\
        &\leq \frac{1}{N} \sum_{i=1}^m |\sum_{x \in P_i} f(x)| + \frac{\delta^2}{20}
    \end{aligned}
\end{equation*}
Therefore $\sum_{i=1}^m \left|\sum_{x \in P_i} f(x)\right| \geq \frac{\delta^2 N}{20}$.

We also have $\sum_{i=1}^m \sum_{x\in P_i} f(x) = 0$, so
\begin{equation*}
    \begin{aligned}
        \sum_{i=1}^m \left(\left|\sum_{x \in P_i} f(x) \right| + \sum_{x \in P_i}f(x) \right) \geq \frac{\delta^2}{20} \sum_{i=1}^m |P_i|
    \end{aligned}
\end{equation*}
Therefore, 
\begin{equation*}
    \begin{aligned}
        &|\sum_{x \in P_i} f(x)| + \sum_{x \in P_i} f(x) \geq \frac{\delta^2}{20} |P_i|\\
        &\implies \sum_{x \in P_i} f(x) \geq \frac{\delta^2}{40}|P_i|\\
        &\implies |A \cap P_i| \geq \left(\delta+\frac{\delta^2}{40}\right)|P_i|
    \end{aligned}
\end{equation*}

---Lecture 3---

Now let $A \subset \Z_N$, $|A| \geq \delta N$. Then:\\
$\bullet$ either $A$ contains a 3AP,\\
$\bullet$ or $N$ is even,\\
$\bullet$ or $\exists P \subset \{1,...,N\}$, $|P| \geq N/3$ s.t. $|A \cap P| \geq \frac{6\delta}{5} |P|$,\\
$\bullet$ or $\exists P \subset \{1,...,N\}$, $|P|\geq \frac{\delta^2}{640}\sqrt{N}$ (casual) s.t. $|A\cap P| \geq (\delta+\frac{\delta^2}{40})|P|$.

Note that the third case is strictly worse than the fourth.\\
Well if the first is true then we're done. Suppose now the seconds hols. Write $N=N_1+N_2$ with $N_1,N_2$ odd, $N_1,N_2 \approx\frac{N}{2}$. Then $A$ has density at least $\delta$ in one of $\{1,...,N_1\}$ or $\{N_1+1,...,N_1+N_2\}$.\\
If (4) holds (note (3) $\implies$ (4)), then we pass to $P$ and start to again. After $\frac{40}{\delta}$ iterations, the density at least doubles. Therefore the toatl number of iterations we can have is at most $\frac{40}{\delta} + \frac{40}{2\delta}+... \leq \frac{80}{\delta}$.\\
If $\frac{\delta^2}{640} \sqrt{N} \geq N^{1/3}$ (to account for the above, and also for the possible use of (2)) at each iteration, and $\frac{\delta^3}{25} \geq N^{-1}$ (which follows from the first condition), then after $\frac{80}{\delta}$ iterations we have $N \geq N^{(1/3)^{80/\delta}}$, so the argument works provided
\begin{equation*}
    \begin{aligned}
        N^{(1/3)^{80\delta}} \geq \left(\frac{640}{\delta^2}\right)^6
    \end{aligned}
\end{equation*}
taking logs and simplify a bit, we need
\begin{equation*}
    \begin{aligned}
        &-\frac{80}{\delta} \log 3 + \log \log N \geq \log 6 + \log (\log 640 + 2 \log \frac{1}{\delta}))\\
        &\Leftarrow \log\log N \geq \frac{160}{\delta}\\
        &\Leftarrow \delta \geq \frac{160}{\log\log N}
    \end{aligned}
\end{equation*}

\subsection{Bogolyubov's method}
Let $K \subset \hat{\Z}_N$ and let $\delta > 0$. The \emph{Bohr set} $B(K,\delta)$ has two definitions (not exactly equivalent, but quite equivalent):\\
$\bullet$ (1) $B(K,\delta) = \{x \in \Z_N: rx \in [-\delta N,\delta N] \forall r \in K\}$ (arc-length definition);\\
$\bullet$ (2) $B(K,\delta) = \{x \in \Z_N: |1-\omega^{rx}| \leq \delta \forall r \in K\}$ (chord-length definition).

\begin{defi}
    Let $G$ be an abelian group and let $A,B$ be subsets of $G$. Then write $A + B = \{a+b: a \in A, b \in B\}$ and the obvious definition for $A-B$. We also write $rA = \{a_1+...+a_r:a_1,...,a_r\in A\}$ (note this might be different than what you think this notation should mean).
\end{defi}

\begin{lemma} (7)\\
    Let $A \subset \Z_N$ be a set of density $\alpha$. Then $2A-2A$ contains a Bohr set $B(K,1/4)$ (arc) with $|K| \leq \alpha^{-2}$.
    \begin{proof}
        Observe that $x \in 2A-2A$ iff $A*A*(-A)*(-A)(x) \neq 0$ (this makes more sense if we write it as $\E_{a+b-c-d=x} A(a)A(b)A(c)A(d) \neq 0$, i.e. we are basically just counting the number of ways $x$ can be written as $a+b-c-d$ where $a,b,c,d \in A$.)\\
        But
        \begin{equation*}
            \begin{aligned}
                A*A*(-A)*(-A)(x) &= \sum_r \widehat{A*A*(-A)*(-A)}(r) \omega^{rx} \text{ inversion formula}\\
                &= \sum |\hat{A}(r)|^4 \omega^{rx}
            \end{aligned}
        \end{equation*}
        Let $K = \{r: |\hat{A}(r)| \geq \alpha^{3/2}\}$. Then $\alpha = ||\hat{A}||_2^2 = \sum_r |\hat{A}(r)|^2 \geq \alpha^3 |K|$.\\
        So $|K| \leq \alpha^{-2}$.\\
        Now suppose that $x \in B(K,1/4)$. Then
        \begin{equation*}
            \begin{aligned}
                \sum_r |\hat{A}(r)|^4 \omega^{rx} &= \alpha^4 + \sum_{r \in K, r \neq 0} |\hat{A}(r)|^4 \omega^{rx} + \sum_{r \not\in K} |\hat{A}(r)|^4 \omega^{rx}
            \end{aligned}
        \end{equation*}
        The real part of the second term is non-negative since $rx \in [-N/4,N/4]$ when $r \in K$.\\
        Also 
        \begin{equation*}
            \begin{aligned}
                \left|\sum_{r \not\in K} |\hat{A}(r)|^4 \omega^{rx}\right| &\leq \sum_{r \not\in K} |\hat{A}(r)|^4\\
                &< \alpha^3 \sum_{r \not\in K} |\hat{A}(r)|^2\\
                &\leq \alpha^4
            \end{aligned}
        \end{equation*}
        So it follows that the real part of $\sum_r |\hat{A}(r)|^4 \omega^{rx} > 0$, i.e. it is non-zero. So $x \in 2A - 2A$.
    \end{proof}
\end{lemma}

\begin{lemma} (8)\\
    Let $K \subset \Z_N$ and let $\delta > 0$. Then:\\
    (i) $B(K,\delta)$ (arc) has density at least $\delta^{|K|}$;\\
    (ii) $B(K,\delta)$ contains a mod-$N$ artihmetic progression of length at least $\delta N^{1/|K|}$.
    \begin{proof}
        (i) Let $K = \{r_1,...,r_k\}$. Consider the $N$ $k$-tuples $(r_1x,...,r_kx) \in \Z_N^k$. If we intersect this set of $k$-tuples with a random 'box' $[t_1,t_1+\delta N] \times ... \times [t_k,t_k+\delta N]$ (here we are thinking $t_i$ as real numbers), then the expected number of the $k$-tuples in the box is $\delta^k N$ (since each one has a probability $\delta^k$).\\
        But if $(r_1x,...,r_kx)$ and $(r_1y,...,r_ky)$ belong to this box, then $x-y \in B(K,\delta)$.\\
        (ii) If we take $\eta > N^{-1/k}$, then by (i) we get that $|B(K,\eta)| > 1$, therefore at least 2. So $\exists x \in B(K,\eta)$ s.t. $x \neq 0$. But then $dx\in B(K,d\eta)$ for every $d$.\\
        So if $d\eta \leq \delta$ then $dx \in B(K,\delta)$. That gives us an AP of length at least $\frac{\delta}{\eta}$. So we get one of length at least $\delta N^{1/k}$.
    \end{proof}
\end{lemma}

\begin{defi}
    Let $A,b$ be subsets of Abelian groups and let $\phi:A \to B$. Then $\phi$ is a \emph{Freiman homomorphism of order $k$} if 
    \begin{equation*}
        \begin{aligned}
            a_1+...+a_k=a_{k+1}+...+a_{2k} \implies \phi(a_1)+...+\phi(a_k) = \phi(a_{k+1})+...+\phi(a_{2k})
        \end{aligned}
    \end{equation*}
    If $k=2$, we call this just a \emph{Freiman homomorphism}. In that case, the condition is equivalent to $a-b=c-d \implies \phi(a)-\phi(b)=\phi(c)-\phi(d)$.\\
    If $\phi$ has an inverse which is also a F-homomorphism of order $k$, then $\phi$ is a F-\emph{isomorphism} of order $k$.
\end{defi}

\begin{lemma} (9)\\
    Assume $0 \not \in K$, and $N$ is prime. If $\delta < 1/4$, then $B(K,\delta)$ (arc) is Freiman isomorphic to the intersection in $\R^{|K|}$ of $[-\delta N, \delta N]^{|K|}$ with some lattice $\Lambda$.
    \begin{proof}
        Let $K = \{r_1,...,r_k\}$, and let $\Lambda = N\Z^k + \{(r_1 x,...,r_kx):x \in \Z\}$. Write $\mathbf{r}$ for $(r_1,...,r_k)$. Claim that $B(K,\delta) \cong \Lambda \cap [-\delta N, \delta N]^k$.\\
        Define a map $\phi:B(K,\delta) \to \Lambda \cap [-\delta N, \delta N]^k$ by sending $x$ to $(\bra r_1 x\ket, ...,\bra r_k x\ket)$ where $\bra u\ket$ means the least-modulus residue $u$ mod $N$.\\
        If $x+y = z+w$, then $\mathbf{r} x +\mathbf{r} y = \mathbf{r}z+\mathbf{r}w$ in $\Z_N^k$. But for each $i$, $\bra r_ix\ket+\bra r_iy\ket-\bra r_iz\ket-\bra r_iw\ket \in [-4\delta N, 4\delta N]$. Since $\delta< 1/4$, that implies that $\bra r_ix\ket +\bra r_iy\ket-\bra r_iz\ket-\bra r_iw\ket =0$. So $\bra\mathbf{r}x\ket+\bra\mathbf{r}y\ket=\bra\mathbf{r}z\ket+\bra\mathbf{r}w\ket$.\\
        That already implies that $\phi$ is an injection.\\
        If $\mathbf{r}x + \mathbf{a} N \in [-\delta N,\delta N]^k$, then $r_i x \in [-\delta N,\delta N]$ mod $N$ for each $i$. So $x \in B(K,\delta)$ and $\phi(x) = \mathbf{r}x + \mathbf{a} N$. So $\phi$ is a surjection as well.\\
        If $\mathbf{r}x+\mathbf{a}N + \mathbf{r}y+\mathbf{b}N = \mathbf{r}z+\mathbf{c}N+\mathbf{r}w+\mathbf{d}N$, then $r_1(x+y) = r_1(z_w)$ mod $N$, so $x+y=z+w$ mod $N$. So the inverse of $\phi$ is also a Freiman homomorphism.
    \end{proof}
\end{lemma}

\begin{lemma} (10)\\
    Let $\Lambda$ be a lattice and $C$ be a symmetric convex body, both in $\R^k$. Then $|\Lambda \cap C| \leq 5^k |\Lambda \cap \frac{C}{2}$|.
    \begin{proof}
        let $x_1,...,x_m$ be a maximal subset of $\Lambda \cap C$ such that for all $i \neq j$, $x_j \not\in x_i + \frac{C}{2}$. Then by maximality, the sets $x_i+\frac{C}{2}$ over(are?) all of $\Lambda \cap C$. Also, the sets $x_i+\frac{C}{4}$ are disjoint subsets of $\R^k$, and they are all contained in $C+\frac{C}{4} = \frac{5}{4} C$. So $m \leq \frac{vol(\frac{5}{4}C)}{vol(\frac{1}{4}C)} = 5^k$.
    \end{proof}
\end{lemma}

\begin{coro} (11)\\
    If $N$ is prime, $0 \not \in K$, $|K| = k$, $\delta < 1/4$, then $|B(K,\delta)| \leq 5^k |B(K,\frac{\delta}{2})|$.
\end{coro}

\newpage

\section{Sumsets and their structure}
It's to be shown that $|A+A| \leq K|A|$ $\implies$ $|rA-sA| \leq K^{r+s}|A|$ (Ruzsa).

\begin{lemma} (1, Petridis)\\
    Let $A_0,B$ be finite subsets of an Abelian group such that $|A_0+B| \leq K_0|A_0|$. Then there exist a non-empty subset $A \subset A_0$ and $K \leq K_0$ s.t. $|A+B+C| \leq K|A+C|$ for every finite subset $C$ of the group.
    \begin{proof}
        Let $A$ minimize the ratio $\frac{|A+B|}{|A|}$, and let the minimal ratio be $K$.\\
        Claim: this works. We prove this by induction on $C$.\\
        If $C =\phi$, then the result holds. Now assume it for $C$ and let $x \not\in C$. Then $A+(C \cup \{x\}) = (A+C) \cup [(A+x) \setminus (A'+x)]$ where $A' = \{a \in A:a+x \in A+C\}$.\\
        This is a disjoint union, so $|A+(C\cup \{x\})| = |A+C|+|A|-|A'|$, $A+B+(C\cup\{x\}) = (A+B+C) \cup ((A+B+x)\setminus (A'+B+x))$, since if $a+x \in A+C$ then $a+B+x \subset A+B+C$. So
        \begin{equation*}
            \begin{aligned}
                |A+B+(C\cup\{x\})| &\leq |A+B+C|+|A+B|-|A'+B|\\
                &\leq K|A+C|+K|A|-K|A'|
            \end{aligned}
        \end{equation*}
        by induction and minimality property of $A$.
    \end{proof}
\end{lemma}

---Lecture 5---

\begin{coro} (2)\\
    If $A,B$ are finite subsets of an abelian group, and $|A+B| \leq K^r|A|$, then there exists $A'\subset A$, $A' \neq \phi$ such that $|A'+rB| \leq K|A'|$ for every positive integer $r$.
    \begin{proof}
        Choose $A'$ as we choose $A$ in lemma 1. Then $|A'+rB| = |A'+B+(r-1)B| \leq K|A'+(r-1)B|$.\\
        And $|A'+B| \leq K|A'|$, so we are done by induction.
    \end{proof}
\end{coro}

\begin{coro} (3)\\
    If $|A+A| \leq K|A|$ or $|A-A| \leq K|A|$, then $|rA| \leq K^r|A|$.
    \begin{proof}
        Set $B=A$ or $-A$ in corollary 2. (think about this)
    \end{proof}
\end{coro}

\begin{lemma} (4, Ruzsa triangle inequality)\\
    Let $A,B,C$ be finite subsets of an abliean group. Then $|A||B-C| \leq |A-B||A-C|$.
    \begin{proof}
        Define a map $\phi: A\times(B-C) \to (A-B) \times (A-C)$. Given $(a,x)$ with $a\in A, x \in B-C$, choose, \emph{somehow}, $b(x) \in B$ and $c(x) \in C$ s.t. $b(x) - c(x) = x$, and set $\phi(a,x) = (a-b(x),a-c(x))$. Note that $(a-c(x)) - (a-b(x)) = b(x)-c(x) = x$ (!). And then, having worked out $x$, we know $b(x)$, and $a=a-b(x)+b(x)$, so $a$ is determined too. So $\phi$ is an injection.
    \end{proof}
\end{lemma}

The proof was easy, but why is this called the triangle inequality? We can rewrite it as 
\begin{equation*}
    \begin{aligned}
        \frac{|B-C|}{|B|^{1/2}|C|^{1/2}} \leq \frac{|A-B|}{|A|^{1/2}|B|^{1/2}} \cdot \frac{|A-C|}{|A|^{1/2}|C|^{1/2}}
    \end{aligned}
\end{equation*}
So if we define the \emph{Ruzsa distance} $d(A,B) = \frac{|A-B|}{|A|^{1/2}|B|^{1/2}}$, then the inequality says $d(B,C) \leq d(A,B) d(A,C)$.

\begin{coro} (5)\\
    If $|A+B| \leq K|A|$, then $|rB-sB| \leq K^{r+s}|A|$ for all $r,s$.
    \begin{proof}
        Pick $a'$ as before. Then by corollary 2 with $B$ replaced by $-B$, $|A'-rB| \leq K^r|A'|$ and $|A'-sB| \leq K^s|A'|$.\\
        Therefore, by Rusza triangle inequality (lemma 4), $|A'||rB-sB| \leq K^{r+s}|A'|^2$, so $|rB-sB| \leq K^{r+s} |A|$.
    \end{proof}
\end{coro}

One finally corollary:
\begin{coro} (6, Plunnecke's theorem)\\
    If $|A+A| \leq K|A|$ or $|A-A| \leq K|A$, then $|rA-sA| \leq K^{r+s} |A|$.
    \begin{proof}
        Just apply corollary 5 with $B=-A$ or $B=A$.
    \end{proof}
\end{coro}

\begin{lemma} (7, Ruzsa's embedding lemma)\\
    Let $A \subset \Z$ be finite and suppose that $|kA-kA| \leq C|A|$. Then there exists a prime $p \leq 4C|A|$ and a subset $A'\subset A$ of size at least $|A|/k$ such that $A'$ is Freiman isomorphic of order $k$ to a subset of $\Z_p$.
    \begin{proof}
        Consider the following composition of maps $\Z \xrightarrow{\text{ reduce mod }q}\Z_q \xrightarrow{\times \text{ by random non-zero } r} \Z_q \xrightarrow{\text{least non-negative residue}} \Z \xrightarrow{\text{reduce mod }p} \Z_p$, where $q$ is a prime bigger than $diam A$ and $p$ is a prime $\in (2C|A|,4C|A|]$ by Bertrand's postulate.\\
        Let $|phi$ be the composition. THe first, second and fourth parts are group homomorphisms, and thus Freiman homomorphisms of all order. Also, the third map is a Freiman homomorphism of order $k$ if you restrict to a subinterval of $[0,q-1]$ of lemgth $\leq q/k$. To see this, write $\bra u \ket$ for least non-negative residue. Then if $I$ has length $\leq q/k$ (and therefore $<q/k$) and $u_{1},...,u_{2k} \in I$, then if $i_1+...+u_k-u_{k+1}-...-k_{2k} = 0$, then $\bra u_1 \ket + ... + \bra u_k \ket - \bra u_{k+1} \ket - ... - \bra u_{2k} \ket \equiv 0 \pmod q$, and also has modulus less than $q$. So it is zero.\\
        By the pigeonhole principle, for any $r$ we can find $I$ of length $\leq q/k$ such that $A' = \{a \in A: ra \in I\}$ has size at least $|A| / k$.\\
        Then $\phi|_{A'}$ is a Freiman homomorphism of order $k$. It's now remain to prove that $\phi$ is an isomorphism to its image, i.e. we must show that if $a_1+...+a_k-a_{k+1}-...-a_{2k} \neq 0 (a_i \in A)$, then
        $$ \bra ra_1\ket + ... + \bra ra_k\ket - \bra ra_{k+1} \ket - ... - \bra ra_{2k} \ket \not\equiv 0 \pmod p$$
        But if the $a_i$ are chosen sucth that the $ra_i$ all belong to the same interval of length $\leq q/k$, then 
        $$|\bra ra_1\ket + ... + \bra ra_k\ket - \bra ra_{k+1} \ket - ... - \bra ra_k\ket| < q$$
        and is congruent to $r(a_1+...+a_k-a_{k+1}-...-a_{2k}) \pmod q$.\\
        So all that can go wrong is if $r(a_1+...+a_k-a_{k+1} - ... - a_{2k}$ is $xp$ for some $x \neq 0$, with $|x| < q/p$. The number of values to avoid is at most $2q/p$, so for each $a_1+...+a_k-a_{k+1} - ... - a_{2k}$, the probability of going wrong if $r$ is chosen randomly is at most $2/p$. So since $|kA-kA| \leq C|A|$, the probability of going wrong is at most $\frac{2}{p} C|A|$. Since $p > 2C|A|$, there exists $r$ s.t. we get an Freiman isomorphism of order $k$.
    \end{proof}
\end{lemma}

\end{document}
