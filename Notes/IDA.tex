\documentclass[a4paper]{article}

\setlength{\parindent}{0pt}
\setlength{\parskip}{1em}

\pagestyle{headings}

\usepackage{amssymb}
\usepackage{amsmath}
\usepackage{amsthm}
\usepackage{mathtools}
\usepackage{graphicx}
\usepackage{hyperref}
\usepackage{color}
\usepackage{microtype}
\usepackage{tikz}
\usepackage{pgfplots}
\usepackage{pgfplotstable}

\newcommand{\N}{\mathbb{N}}
\newcommand{\Q}{\mathbb{Q}}
\newcommand{\Z}{\mathbb{Z}}
\newcommand{\R}{\mathbb{R}}
\newcommand{\C}{\mathbb{C}}
\newcommand{\D}{\mathcal{D}}
\renewcommand{\S}{\mathcal{S}}
\renewcommand{\P}{\mathbb{P}}
\newcommand{\F}{\mathbb{F}}
\newcommand{\E}{\mathbb{E}}

\graphicspath{{Image/}}

\hypersetup{
    colorlinks=true,
    linktoc=all,
    linkcolor=blue
}

\theoremstyle{definition}
\newtheorem*{axiom}{Axiom}
\newtheorem*{claim}{Claim}
\newtheorem*{conv}{Convention}
\newtheorem*{coro}{Corollary}
\newtheorem*{defi}{Definition}
\newtheorem*{eg}{Example}
\newtheorem*{lemma}{Lemma}
\newtheorem*{notation}{Notation}
\newtheorem*{prob}{Problem}
\newtheorem*{post}{Postulate}
\newtheorem*{prop}{Proposition}
\newtheorem*{rem}{Remark}
\newtheorem*{thm}{Theorem}

\DeclareMathOperator{\vdiv}{div}
\DeclareMathOperator{\grad}{grad}
\DeclareMathOperator{\curl}{curl}
\DeclareMathOperator{\Ann}{Ann}
\DeclareMathOperator{\Fit}{Fit}
\DeclareMathOperator{\Diag}{Diag}
\DeclareMathOperator{\tr}{tr}
\DeclareMathOperator{\im}{im}
\DeclareMathOperator{\Mat}{Mat}
\DeclareMathOperator{\Log}{Log}
\DeclareMathOperator{\Isom}{Isom}
\DeclareMathOperator{\Mesh}{Mesh}
\DeclareMathOperator{\Sym}{Sym}
\DeclareMathOperator{\Aut}{Aut}
\DeclareMathOperator{\cosech}{cosech}
\DeclareMathOperator{\Card}{Card}
\DeclareMathOperator{\Gal}{Gal}


\setcounter{section}{-1}

\begin{document}

\title{Introduction to Discrete Analysis}

\maketitle

\newpage

\tableofcontents

\newpage

\section{Introduction}
asdasd

\newpage

\section{The discrete Fourier transform}

Let $N$ be a fixed positive integer. Write $\omega$ for $e^{2\pi i/N}$, and $\Z_N$ for $\Z/n\Z$. Let $f:\Z_N \to \C$. Given $r \in \Z_N$, define $\hat{f}(r)$ to be
\begin{equation*}
\begin{aligned}
\frac{1}{N} \sum_{x \in \Z_N} f(x) \omega^{-rx}
\end{aligned}
\end{equation*}
From now on we use the notation $\E_{x \in \Z_N}$ for $\frac{1}{N}\sum_{x \in \Z_N}$, so $\hat{f}(r) = \E_x f(x) e^{-\frac{2\pi irx}{N}}$.

If we write $\omega_r$ for the function $x \to \omega^{rx}$, and $\bra f,g\ket$ for $\E_x f(x) \overline{g(x)}$, then $\hat{f}(r) = \bra f,\omega_r \ket$. So the discrete fourier transforn is basically expanding the function $f$ in the set of orthonormal basis $\omega_r$.

Let us write $||f||_p$ for $\E_x |f(x)|^p)^{1/p}$ (the $L_p$-norm), and call the resulting space $L_p(\Z_n)$.

Important convention: we use \emph{averages} for the 'original functions' in 'physical spaces', and \emph{sums} for their Fourier transforms in 'frequency space' (referring to $\E$: $\bra,\ket$ is average in the original space but just $\sum$ in frequency space, i.e. for $\hat{f},\hat{g}$ etc.)

\begin{lemma} (1, Parseval's identity)\\
    If $f,g: \Z_n \to \C$, then $\bra \hat{f},\hat{g}\ket = \bra f,g\ket$.
    \begin{proof}
        \begin{equation*}
            \begin{aligned}
                \bra\hat{f},\hat{g}\ket &= \sum_r \hat{f}(r) \overline{\hat{g}(r)}\\
                &= \sum_r (\E_x f(x) \omega^{-rx}) (\overline{\E_y g(y) \omega^{-ry}})\\
                &= \E_x \E_y f(x) \overline{g(y)} \sum_r \omega^{-r(x-y)}\\
                &= \E_x \E_y f(x) \overline{g(y)} n\delta_{xy}\\
                &= \bra f,g\ket
            \end{aligned}
            \end{equation*}
    \end{proof}
\end{lemma}

\begin{lemma} (2, Convolution identity)\\
    \begin{equation*}
        \begin{aligned}
            \widehat{f*g}(r) = \hat{f}(r)\hat{g}(r)
        \end{aligned}
    \end{equation*}
    where
    \begin{equation*}
        \begin{aligned}
            (f*g)(x) = \E_{y+z=x}f(y)g(z) = \E_y f(y)g(x-y)
        \end{aligned}
    \end{equation*}
    \begin{proof}
        \begin{equation*}
            \begin{aligned}
                \widehat{f*g}(r) &= \E_x f*g(x)\omega^{-rx}\\
                &=\E_x \E_{y+z=x}f(y)g(z)\omega^{-rx}\\
                &=\E_x\E_{y+z=x} f(y)g(z) \omega^{-ry}\omega^{-rz}\\
                &=\E_y\E_z f(y) \omega^{-ry} g(z) \omega^{-rz}\\
                &=\hat{f}(r) \hat{g}(r)
            \end{aligned}
        \end{equation*}
    \end{proof}

\end{lemma}

\begin{lemma} (3, Inversion formula)\\
    \begin{equation*}
        \begin{aligned}
            f(x) = \sum_r \hat{f}(r) \omega^{rx}
        \end{aligned}
    \end{equation*}
    (note the sign of $\omega^{rx}$).
    \begin{proof}
        \begin{equation*}
            \begin{aligned}
                \sum_r \hat{f}(r) \omega^{rx} &= \sum_r \E_y f(y) \omega^{r(x-y)}\\
                &= \E_y f(y) \sum_r \omega^{r(x-y)}\\
                &= \E_y f(y) n\delta_{xy}\\
                &=f(x)
            \end{aligned}
        \end{equation*}
        This is really just the statement that we get the original vector back when we sum up its components.
    \end{proof}
\end{lemma}

Further observations: If $f$ is real-valued, then $\hat{f}(-r) = \E_x f(x) \omega^{rx} = \overline{\E_x f(x) \omega^{-rx}} = \overline{\hat{f}(r)}$.

If $A \subset \Z_n$, write $A$ (instead of $1_A,\chi_A$) for the characteristic function of $A$. Then $\hat{A}(0) = \E_x A(x) = \frac{|A|}{N}$, the \emph{density} of $A$.

Also, $||\hat{A}||^2_2 = \bra\hat{A},\hat{A}\ket = \bra A,A\ket = \E_x A(x)^2 = \E_x A(x) = \frac{|A|}{N}$, again the density.

Let $f:\Z_n \to \C$. Given $\mu \in \Z_n$, define $f_\mu(x)$ to be $f(\mu^{-1}x)$ (so we need $(\mu,N) = 1$). Then
\begin{equation*}
    \begin{aligned}
        \hat{f}_\mu(r) &= \E_x f_\mu (x) \omega^{-rx}\\
        &= \E_x f(x/\mu) \omega^{-rx}\\
        &=\E_x f(x) \omega^{-r\mu x}\\
        ^= \hat{f}(\mu r)
    \end{aligned}
\end{equation*}

\subsection{4, Roth's theorem}
\begin{thm}
    For every $\delta > 0$, $\exists N$ s.t. if $A \subset \{1,...,N\}$ is a set of size at least $\delta N$, then $A$ must contain an arithmetic progression of length 3.\\
    This is also true for 4,5,..., but the proof is much harder -- Szemeredi's theorem.\\
    Basic strategy of proof: show that if $A$ has density $\delta$ and no AP of length 3 (3AP), then there's a long AP in $P \subset \{1,2,...,n\}$ s.t.
    \begin{equation*}
        \begin{aligned}
            |A \cap P| \geq (\delta+c(\delta)) |p|
        \end{aligned}
    \end{equation*}
    where $c(\delta)$ is some positive number. But then we can continue this argument to expand $A\cap P$ to infinity (note that $|A\cap P|$ is an integer, so each time increase by 1 at least).

    The best known relationship between $\delta$ and the $N$ required is around $\delta \sim \frac{c}{\log\log N}$ for some constant $c$.
\end{thm}

---Lecture 2---

\begin{lemma} (5)\\
    Let $N$ be odd, $A,B,C \subset \Z_N$ have densties $\alpha,\beta,\gamma$.\\
    If $\max_{r \neq 0} |\hat{A}(r)| \leq \frac{\alpha(\beta\gamma)^{1/2}}{2}$ and $\frac{\alpha\beta\gamma}{2} > \frac{1}{N}$, then there exists $x,d \in \Z_N$ with $d \neq 0$ s.t. $(x,x+d,x+2d) \in A \times B \times C$.
    \begin{proof}
        \begin{equation*}
            \begin{aligned}
                \E_{x,d} A(x) B(x+d) C(x+2d) &= \E_{x+z=2y} A(x)B(y)C(z)\\
                &=\E_u (\E_{x+z=u}A(x)C(z)) \E_{2y=u} B(y)\\
                &=\E_u A * C(u) B_2(u)\\
                &=\bra A*C,B_2\ket\\
                &=\bra\widehat{A*C},\hat{B}_2\ket\\
                &=\bra \hat{A}\hat{C},\hat{B}_2\ket\\
                &=\sum_r \hat{A}(r)\hat{C}(r)\hat{B}(-2r)\\
                &=\alpha\beta\gamma + \sum_{r \neq 0} \hat{A}(r)\hat{C}(r)\hat{B}(-2r)
            \end{aligned}
        \end{equation*}
        Recall here the notation is $B_2(u) = B(u/2)$. now
        \begin{equation*}
            \begin{aligned}
                |\sum_{r\neq 0} \hat{A}(r)\hat{B}(-2r)\hat{C}(r)| &\leq \frac{\alpha(\beta\gamma)^{1/2}}{2}\sum_{r \neq 0} |\hat{B}(-2r)||\hat{C}(r)|\\
                &\leq \frac{\alpha(\beta\gamma)^{1/2}}{2} \left(\sum_r |\hat{B}(-2r)^2\right)^{1/2}\left(\sum_r |\hat{C}(r)|^2\right)^{1/2} \ \text{By Cauchy-Schwarz}\\
                &=\frac{\alpha(\beta\gamma)^{1/2}}{2} ||\hat{B}||_2 ||\hat{C}||_2\\
                &= \frac{\alpha(\beta\gamma)^{1/2}}{2} ||B||_2||C||_2\\
                &= \frac{\alpha\beta\gamma}{2}
            \end{aligned}
        \end{equation*}
        The contribution to $\E_{x,d} A(x)B(x+d)C(x+2d)$ from $d=0$ is at most $\frac{1}{N}$, so if $\frac{\alpha\beta\gamma}{2} > \frac{1}{N}$, we are done.
    \end{proof}
\end{lemma}

Now let $A$ be a subset of $\{1,...,N\}$ with density $\geq \delta$ and let $B=C=A \cap [\frac{N}{3},\frac{2N}{3})$. If $B$ has density $<\frac{\delta}{5}$ (??), then either $A \cap [1,\frac{N}{3}]$ or $A \cap [\frac{2N}{3},N]$ has density at least $\frac{2\delta}{5}$. In that case we find an AP $P$ of length about $N/3$ such that $|A \cap P| / |P| \geq \frac{6\delta}{5}$.\\
Otherwise, ew find that if $\max_{r \neq 0}|\hat{A}(r)| \leq \frac{\delta}{10}$ and $\frac{\delta^3}{50} > \frac{1}{N}$, then $A \times B \times C$ contains a 3AP, so $A$ contains a 3AP.\\
So if $A$ does not contain a 3AP, then either we find $P$ of length about $N/3$ with $|A\cap P| / |P| \geq \frac{6\delta}{5}$, or ther exists $r \neq 0$ s.t. $|\hat{A}(r)| \geq \frac{\delta}{10}$.

\begin{defi}
    If $X$ is a finite set and $f:X \to \C$, $Y \subset X$, write $osc(f|_Y)$ to mean $\max_{y_1,y_2 \in Y} |f(y_1)-f(y_2)|$.
\end{defi}

\begin{lemma} (6)\\
    Let $r \in \Z_n$ and let $\varepsilon > 0$. Then there is a partition of $\{1,2,...,N\}$ into arithmetic progressions $P_i$ of length at least $c(\varepsilon)\sqrt{N}$ such that
    \begin{equation*}
        \begin{aligned}
            osc(\omega_r|_{P_i}) \leq \varepsilon
        \end{aligned}
    \end{equation*}
    for each $i$.
    \begin{proof}
        Let $t = \lfloor \sqrt{N}\rfloor$. Of the numbers $1,\omega^r,...,\omega^tr$, there must be two that differ by at most $\frac{2\pi}{t}$.\\
        If $|\omega^{ar}-\omega^{br}| \leq \frac{2\pi}{t}$ with $a < b$, then $|1-\omega^{dr}| \leq \frac{2\pi}{t}$ where $d=b-a$. Then $|\omega^{urd}-\omega^{vrd}| \leq |\omega^{urd}-\omega^{(u+1)rd}| + ... + |\omega^{(v-1)rd}-\omega^{vrd}| \leq \frac{2\pi}{t}(v-u)$.\\
        So if $P$ is a progression with common differnece $d$ and length $l$, then $osc(\omega_r|_P) \leq \frac{2\pi l}{t}$. So divide up $\{1,...,N\}$ into residue classes mod $d$, and partition each residue class into parts of length between $\frac{\varepsilon t}{4\pi}$ and $\frac{\varepsilon t}{2\pi}$ (possible, since $d \leq t \leq \sqrt{N}$).\\
        We are done, with $c(\varepsilon) = \frac{\varepsilon}{16}$ (a casual choice).
    \end{proof}
\end{lemma}
Now let us use the information that $r \neq 0$ and $|\hat{A}(r)| \geq \frac{\delta^2}{10}$.\\
Define the \emph{balanced function} $f$ of $A$ by $f(x) = A(x)=\frac{|A|}{N}$ for each $x$.\\
Note that $\hat{f}(0) = 0$ and $\hat{f}(r) = \hat{A}(r)$ for all $r \neq 0$.

Now let $P_1,...,P_m$ be given by Lemma 6 with $\varepsilon = \delta^2/20$. Then
\begin{equation*}
    \begin{aligned}
        \frac{\delta^2}{10} &\leq |\hat{f}(r)|\\
        &= \frac{1}{N} |\sum_x f(x) \omega^{-rx}|\\
        &\leq \frac{1}{N} \sum_{i=1}^m |\sum_{x \in P_i} f(x) \omega^{-rx}|\\
        &\leq \frac{1}{N} \sum_{i=1}^N \left[\left|\sum_{x \in P_i} f(x) \omega^{-rx_i}\right| + \left|\sum_{x in P_i} f(x) (\omega^{-rx} - \omega^{-rx_i})\right|\right] \ x_i \in P_i \text{ arbitrary}\\
        &\leq \frac{1}{N} \sum_{i=1}^m |\sum_{x \in P_i} f(x)| + \frac{\delta^2}{20}
    \end{aligned}
\end{equation*}
Therefore $\sum_{i=1}^m |\sum_{x \in P_i} f(x)| \geq \frac{\delta^2 N}{20}$.

We also have $\sum_{i=1}^m \sum_{x\in P_i} f(x) = 0$, so
\begin{equation*}
    \begin{aligned}
        \sum_{i=1}^m \left(\left|\sum_{x \in P_i} f(x) \right| + \sum_{x \in P_i}f(x) \right) \geq \frac{\delta^2}{20} \sum_{i=1}^m |P_i|
    \end{aligned}
\end{equation*}
Therefore, 
\begin{equation*}
    \begin{aligned}
        &|\sum_{x \in P_i} f(x)| + \sum_{x \in P_i} f(x) \geq \frac{\delta^2}{20} |P_i|\\
        &\implies \sum_{x \in P_i} f(x) \geq \frac{\delta^2}{40}|P_i|\\
        &\implies |A \cap P_i| \geq (\delta+\frac{\delta^2}{40})|P_i|
    \end{aligned}
\end{equation*}

\end{document}
