\documentclass[a4paper]{article}

\setlength{\parindent}{0pt}
\setlength{\parskip}{1em}

\pagestyle{headings}

\usepackage{amssymb}
\usepackage{amsmath}
\usepackage{amsthm}
\usepackage{mathtools}
\usepackage{graphicx}
\usepackage{hyperref}
\usepackage{color}
\usepackage{microtype}
\usepackage{tikz}
\usepackage{pgfplots}
\usepackage{pgfplotstable}

\newcommand{\N}{\mathbb{N}}
\newcommand{\Q}{\mathbb{Q}}
\newcommand{\Z}{\mathbb{Z}}
\newcommand{\R}{\mathbb{R}}
\newcommand{\C}{\mathbb{C}}
\newcommand{\D}{\mathcal{D}}
\renewcommand{\S}{\mathcal{S}}
\renewcommand{\P}{\mathbb{P}}
\newcommand{\F}{\mathbb{F}}
\newcommand{\E}{\mathbb{E}}
\newcommand{\bra}{\langle}
\newcommand{\ket}{\rangle}


\graphicspath{{Image/}}

\hypersetup{
    colorlinks=true,
    linktoc=all,
    linkcolor=blue
}

\theoremstyle{definition}
\newtheorem*{axiom}{Axiom}
\newtheorem*{claim}{Claim}
\newtheorem*{conv}{Convention}
\newtheorem*{coro}{Corollary}
\newtheorem*{defi}{Definition}
\newtheorem*{eg}{Example}
\newtheorem*{lemma}{Lemma}
\newtheorem*{notation}{Notation}
\newtheorem*{prob}{Problem}
\newtheorem*{post}{Postulate}
\newtheorem*{prop}{Proposition}
\newtheorem*{rem}{Remark}
\newtheorem*{thm}{Theorem}

\DeclareMathOperator{\vdiv}{div}
\DeclareMathOperator{\grad}{grad}
\DeclareMathOperator{\curl}{curl}
\DeclareMathOperator{\Ann}{Ann}
\DeclareMathOperator{\Fit}{Fit}
\DeclareMathOperator{\Diag}{Diag}
\DeclareMathOperator{\tr}{tr}
\DeclareMathOperator{\im}{im}
\DeclareMathOperator{\Mat}{Mat}
\DeclareMathOperator{\Log}{Log}
\DeclareMathOperator{\Isom}{Isom}
\DeclareMathOperator{\Mesh}{Mesh}
\DeclareMathOperator{\Sym}{Sym}
\DeclareMathOperator{\Aut}{Aut}
\DeclareMathOperator{\cosech}{cosech}
\DeclareMathOperator{\Card}{Card}
\DeclareMathOperator{\Gal}{Gal}


\setcounter{section}{-1}

\begin{document}

\title{Introduction to Discrete Analysis}

\maketitle

\newpage

\tableofcontents

\newpage

\section{Introduction}
asdasd

\newpage

\section{The discrete Fourier transform}

Let $N$ be a fixed positive integer. Write $\omega$ for $e^{2\pi i/N}$, and $\Z_N$ for $\Z/n\Z$. Let $f:\Z_N \to \C$. Given $r \in \Z_N$, define $\hat{f}(r)$ to be
\begin{equation*}
\begin{aligned}
\frac{1}{N} \sum_{x \in \Z_N} f(x) \omega^{-rx}
\end{aligned}
\end{equation*}
From now on we use the notation $\E_{x \in \Z_N}$ for $\frac{1}{N}\sum_{x \in \Z_N}$, so $\hat{f}(r) = \E_x f(x) e^{-\frac{2\pi irx}{N}}$.

If we write $\omega_r$ for the function $x \to \omega^{rx}$, and $\bra f,g\ket$ for $\E_x f(x) \overline{g(x)}$, then $\hat{f}(r) = \bra f,\omega_r \ket$. So the discrete fourier transforn is basically expanding the function $f$ in the set of orthonormal basis $\omega_r$.

Let us write $||f||_p$ for $\E_x |f(x)|^p)^{1/p}$ (the $L_p$-norm), and call the resulting space $L_p(\Z_n)$.

Important convention: we use \emph{averages} for the 'original functions' in 'physical spaces', and \emph{sums} for their Fourier transforms in 'frequency space' (referring to $\E$: $\bra,\ket$ is average in the original space but just $\sum$ in frequency space, i.e. for $\hat{f},\hat{g}$ etc.)

\begin{lemma} (Parseval's identity)\\
    If $f,g: \Z_n \to \C$, then $\bra \hat{f},\hat{g}\ket = \bra f,g\ket$.
    \begin{proof}
        \begin{equation*}
            \begin{aligned}
                \bra\hat{f},\hat{g}\ket &= \sum_r \hat{f}(r) \overline{\hat{g}(r)}\\
                &= \sum_r (\E_x f(x) \omega^{-rx}) (\overline{\E_y g(y) \omega^{-ry}})\\
                &= \E_x \E_y f(x) \overline{g(y)} \sum_r \omega^{-r(x-y)}\\
                &= \E_x \E_y f(x) \overline{g(y)} n\delta_{xy}\\
                &=\bra f,g\ket
            \end{aligned}
            \end{equation*}
    \end{proof}
\end{lemma}
\end{document}
