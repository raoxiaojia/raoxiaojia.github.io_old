\documentclass[a4paper]{article}

\setlength{\parindent}{0pt}
\setlength{\parskip}{1em}

\pagestyle{headings}

\usepackage{amssymb}
\usepackage{amsmath}
\usepackage{amsthm}
\usepackage{mathtools}
\usepackage{graphicx}
\usepackage{hyperref}
\usepackage{color}
\usepackage{microtype}
\usepackage{tikz}
\usepackage{pgfplots}
\usepackage{pgfplotstable}

\newcommand{\N}{\mathbb{N}}
\newcommand{\Q}{\mathbb{Q}}
\newcommand{\Z}{\mathbb{Z}}
\newcommand{\R}{\mathbb{R}}
\newcommand{\C}{\mathbb{C}}
\newcommand{\D}{\mathcal{D}}
\renewcommand{\S}{\mathcal{S}}
\renewcommand{\P}{\mathbb{P}}
\newcommand{\F}{\mathbb{F}}
\newcommand{\E}{\mathbb{E}}

\graphicspath{{Image/}}

\hypersetup{
    colorlinks=true,
    linktoc=all,
    linkcolor=blue
}

\theoremstyle{definition}
\newtheorem*{axiom}{Axiom}
\newtheorem*{claim}{Claim}
\newtheorem*{conv}{Convention}
\newtheorem*{coro}{Corollary}
\newtheorem*{defi}{Definition}
\newtheorem*{eg}{Example}
\newtheorem*{lemma}{Lemma}
\newtheorem*{notation}{Notation}
\newtheorem*{prob}{Problem}
\newtheorem*{post}{Postulate}
\newtheorem*{prop}{Proposition}
\newtheorem*{rem}{Remark}
\newtheorem*{thm}{Theorem}

\DeclareMathOperator{\vdiv}{div}
\DeclareMathOperator{\grad}{grad}
\DeclareMathOperator{\curl}{curl}
\DeclareMathOperator{\Ann}{Ann}
\DeclareMathOperator{\Fit}{Fit}
\DeclareMathOperator{\Diag}{Diag}
\DeclareMathOperator{\tr}{tr}
\DeclareMathOperator{\im}{im}
\DeclareMathOperator{\Mat}{Mat}
\DeclareMathOperator{\Log}{Log}
\DeclareMathOperator{\Isom}{Isom}
\DeclareMathOperator{\Mesh}{Mesh}
\DeclareMathOperator{\Sym}{Sym}
\DeclareMathOperator{\Aut}{Aut}
\DeclareMathOperator{\cosech}{cosech}
\DeclareMathOperator{\Card}{Card}
\DeclareMathOperator{\Gal}{Gal}


\setcounter{section}{-1}

\begin{document}

\title{Applied Probability}

\maketitle

\newpage

\tableofcontents

\newpage

\section{Miscellaneous}

Some speech

Google lecture's name to find his homepage and example sheets or probably some notice of a change of room

\newpage

\section{Poisson process}

Suppose we have a Geiger counter. We model the "click process" as a family $\{N(t) : t \geq 0\}$, where $N(t)$ denotes the total number of ticks up to time $t$. Now note that $N(t) \in \{0,1,...\}$, $N(s) \leq N(t)$ if $s \leq t$, $N$ increases by unit jumps, and $N(0) = 0$. We also assert that $N$ is right-continuous, i.e. $\lim_{x \to t^+} N(x) = N(t)$.

\begin{defi} (infinitesimal definition)\\
A \emph{Poisson process} with intensity $\lambda$ is a process $N=(N(t):t \geq 0)$ which takes values in $S = \{0,1,2,...\}$, s.t.:\\
(a) $N(0) = 0$, $N(s) \leq N(t)$ if $s \leq t$;\\
(b) 
\begin{equation*}
\begin{aligned}
\P(N(t+h)=n+m | N(t) = n) = \left\{\begin{array}{ll}
\lambda h + o(h) & m=1\\
o(h) & m>1\\
1-\lambda h & m=0
\end{array}
\right.
\end{aligned}
\end{equation*}
Recall that $g(h) = o(h)$ means that $\frac{g(h)}{h} \to 0$ as $h \to 0$;\\
(c) if $s<t$, then $N(t)-N(s)$ is independent of all arrivals prior to $s$.
\end{defi}

\begin{thm}
$N(t)$ has the Poisson distribution with parameter $\lambda t$.
\begin{proof}
Study $N(t+h)$ given $N(t)$. We have 
\begin{equation*}
\begin{aligned}
\P(N(t+h) =j) &= \sum_{i\leq j} \P(N(t+h) \\
&= j|N(t) = i) \P(N(t) = i) \\
&= (1-\lambda h) \P(N(t) = j) + \lambda h \P(N(t) = h-1) + o(h)
\end{aligned}
\end{equation*}
So
\begin{equation*}
\begin{aligned}
\frac{\P(N(t+h)=j) - \P(N(t) = j)}{h} = -\lambda \P(N(t) = j) + \lambda \P (N(t) = j-1) + \frac{o(h)}{h}
\end{aligned}
\end{equation*}
write $p_n(t) = \P(N(t) = n)$, then let $h \to 0^+$ we get
\begin{equation*}
\begin{aligned}
p'_j(t) &= -\lambda p_j(t) + \lambda p_{j-1}(t) & j \geq 1\\
p'_0(t) &= -\lambda p_0(t) &
\end{aligned}
\end{equation*}
with boundary condition $p_0(0) = 1$.\\
We solve $p_0$ to get $p_0(t) = e^{-\lambda(t)}$. Then we can use this to inductively solve $p_1,p_2,...$ to get the desired result.
\end{proof}
\end{thm}

An alternative derivation from the differential equations:\\
Let $G(s,t) = \sum_j s^j p_j(t)$. Now we take the set of differential equation, multiplying each one by $s^j$, then we get
\begin{equation*}
\begin{aligned}
\frac{\partial G}{\partial t} = \lambda (s-1) G
\end{aligned}
\end{equation*}
Then we have $$G(s,t) = A(s) e^{\lambda (s-1) t}$$ We also have $G(s,0)=1$ so we should be able to plug in a suitable value of $s$ to get the desired result (I probably missed that).

\begin{defi}(Holding/interarrival times)
In a poisson process (pp) with parameter $\lambda$, let $N(t)$ denote the total number of "clicks". Define the arrival times $T_0 = 0$, $T_n = \inf \{t \geq 0: N(t) = n\}$, i.e. the first time $t$ that $N$ reaches $n$ (note right continuity of $N$). We also define the interarrival times $X_n = T_n - T_{n-1}$.
\end{defi}

\includegraphics[scale=0.5]{image/AP_01.png}

\begin{thm}
Suppose $X_1,X_2,...$ are known. Let $T_n = \sum_1^n x_i$, note $N(t) = \max\{n:T_n \leq t\}$. Then the random variables $X_1,X_2,...$ are independent and they have the exponential distribution with parameter $\lambda$ ($Exp(\lambda)$).
\begin{proof}
\begin{equation*}
\begin{aligned}
\P(X_1 > t) = \P(N(t) = 0) = e^{-\lambda t}
\end{aligned}
\end{equation*}
So $X_1$ has $Exp(\lambda)$ distribution. Now consider $\P(X_2>t|X_1=t_1)$. This doesm't look to make much sense as $X_1$ has a continuous distribution so $\P(X_1=t_1) = 0$; however we could could consider the conditional densiy as $f_{X|Y} (x|y) =\frac{f_{X,Y}(x,y)}{f_Y(y)}$. Then $\P(X_2>t|X_1=t_1) = \P($ no arrivals in $(t_1,t_1+t)|X_1=t_1) = \P($ no arrivals in $(t_1,t_1+t)$ by independence. This is then equal to $\P($ no arrivals in $(0,t)) = \P(N(t)=0) = e^{-\lambda t}$. Then continue by induction.
\end{proof}
\end{thm}

\begin{prop} (properties of a poisson process $N$)\\
(a) $N$ has stationary independent increments, i.e.:\\
(i) If $0<t_1<...<t_n$, then $N(t_1),N(t_2)-N(t_1),...,N(t_n)-N(t_{n-1})$ are independent;\\
(ii) $N(s+t)-N(s) \xrightarrow{d} N(t)-N(0)$.\\
Amongst processes which are right continuous, non-decreasing, has only jump discontinuities of size 1, (i) and (ii) are characteristics of the Poisson process, meaning that Poisson process is the only process that has those two properties.\\
(b) Thinning:\\
Suppose insects arrive as a poisson process with parameter $\lambda$. Each insect is a mosquito with probability $\alpha$, or a skeet with probability $1-\alpha$, and the occurences of the two insects are independent. Then\\
(i) the mosquito-arrival process $F$ is a $PP(\alpha\lambda)$, 
(ii) the skeet-arrival process is $S$ a $PP((1-\alpha)\lambda)$, and 
(iii) these processes are independent.
\begin{proof}
(i) and (ii) are immediate by infinitestimal definition of a poisson process. For (iii), by independence we mean that $\P(F(t_1)=f_1,S(t_1)=s_1,...,F(t_n)=f_n,S(t_n)=s_n) = \P(F(t_1)=f_1,...,F(t_n)=f_n) \P(S(t_1)=s_1,...,S(t_n)=s_n)$ $\forall t_1,...,t_n,f_1,...,f_n,s_1,...,s_n)$.

The simple case is 
\begin{equation*}
\begin{aligned}
\P(F(t) = f, S(t) = s) &= \frac{(\lambda t)^{f+s} e^{-\lambda t}}{(f+s)!}{{f+s} \choose f} \alpha^f (1-\alpha)^s\\
&= \frac{(\alpha\lambda t)^f}{f!} e^{-\alpha\lambda t} \frac{((1-\alpha) \lambda t)^s}{s!} e^{-(1-\alpha)\lambda t}\\
&= \P(F(t) = f) \P(S(t) = s)
\end{aligned}
\end{equation*}
\end{proof}
(c) Superposition:\\
$F$: Flies arrive as $PP(\lambda_1)$;\\
$S$: Skeets arrive as $PP(\lambda_2)$, and these processes are independent. Then $N=F+S$ is a $PP(\lambda_1+\lambda_2$. This follows by infinitesimal construction of $PP$.\\
(d) Given $N(t) = n$, write $\mathbf{T} = (T_1,...,T_n)$, $\mathbf{t} = (t_1,...,t_n)$, we have $f_{\mathbf{T}}(\mathbf{t} | N(t) = n) =\left(\frac{1}{t}\right)^n n! L(\mathbf{t})$, where $L(\mathbf{t})=1$ iff $t_1<t_2<...<t_n$.
\begin{proof}
Next time.
\end{proof}
\end{prop}


\iffalse
\begin{equation*}
\begin{aligned}

\end{aligned}
\end{equation*}
\fi
\end{document}
