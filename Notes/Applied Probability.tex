\documentclass[a4paper]{article}

\setlength{\parindent}{0pt}
\setlength{\parskip}{1em}

\pagestyle{headings}

\usepackage{amssymb}
\usepackage{amsmath}
\usepackage{amsthm}
\usepackage{mathtools}
\usepackage{graphicx}
\usepackage{hyperref}
\usepackage{color}
\usepackage{microtype}
\usepackage{tikz}
\usepackage{pgfplots}
\usepackage{pgfplotstable}

\newcommand{\N}{\mathbb{N}}
\newcommand{\Q}{\mathbb{Q}}
\newcommand{\Z}{\mathbb{Z}}
\newcommand{\R}{\mathbb{R}}
\newcommand{\C}{\mathbb{C}}
\newcommand{\D}{\mathcal{D}}
\renewcommand{\S}{\mathcal{S}}
\renewcommand{\P}{\mathbb{P}}
\newcommand{\F}{\mathbb{F}}
\newcommand{\E}{\mathbb{E}}
\newcommand{\bra}{\langle}
\newcommand{\ket}{\rangle}


\graphicspath{{Image/}}

\hypersetup{
    colorlinks=true,
    linktoc=all,
    linkcolor=blue
}

\theoremstyle{definition}
\newtheorem*{axiom}{Axiom}
\newtheorem*{claim}{Claim}
\newtheorem*{conv}{Convention}
\newtheorem*{coro}{Corollary}
\newtheorem*{defi}{Definition}
\newtheorem*{eg}{Example}
\newtheorem*{lemma}{Lemma}
\newtheorem*{notation}{Notation}
\newtheorem*{prob}{Problem}
\newtheorem*{post}{Postulate}
\newtheorem*{prop}{Proposition}
\newtheorem*{rem}{Remark}
\newtheorem*{thm}{Theorem}

\DeclareMathOperator{\vdiv}{div}
\DeclareMathOperator{\grad}{grad}
\DeclareMathOperator{\curl}{curl}
\DeclareMathOperator{\Ann}{Ann}
\DeclareMathOperator{\Fit}{Fit}
\DeclareMathOperator{\Diag}{Diag}
\DeclareMathOperator{\tr}{tr}
\DeclareMathOperator{\im}{im}
\DeclareMathOperator{\Mat}{Mat}
\DeclareMathOperator{\Log}{Log}
\DeclareMathOperator{\Isom}{Isom}
\DeclareMathOperator{\Mesh}{Mesh}
\DeclareMathOperator{\Sym}{Sym}
\DeclareMathOperator{\Aut}{Aut}
\DeclareMathOperator{\cosech}{cosech}
\DeclareMathOperator{\Card}{Card}
\DeclareMathOperator{\Gal}{Gal}


\setcounter{section}{-1}

\begin{document}

\title{Applied Probability}

\maketitle

\newpage

\tableofcontents

\newpage

\section{Miscellaneous}

Some speech

Google lecture's name to find his homepage and example sheets or probably some notice of a change of room

\newpage

\section{Poisson process}

Suppose we have a Geiger counter. We model the "click process" as a family $\{N(t) : t \geq 0\}$, where $N(t)$ denotes the total number of ticks up to time $t$. Now note that $N(t) \in \{0,1,...\}$, $N(s) \leq N(t)$ if $s \leq t$, $N$ increases by unit jumps, and $N(0) = 0$. We also assert that $N$ is right-continuous, i.e. $\lim_{x \to t^+} N(x) = N(t)$.

\begin{defi} (infinitesimal definition)\\
A \emph{Poisson process} with intensity $\lambda$ is a process $N=(N(t):t \geq 0)$ which takes values in $S = \{0,1,2,...\}$, s.t.:\\
(a) $N(0) = 0$, $N(s) \leq N(t)$ if $s \leq t$;\\
(b) 
\begin{equation*}
\begin{aligned}
\P(N(t+h)=n+m | N(t) = n) = \left\{\begin{array}{ll}
\lambda h + o(h) & m=1\\
o(h) & m>1\\
1-\lambda h & m=0
\end{array}
\right.
\end{aligned}
\end{equation*}
Recall that $g(h) = o(h)$ means that $\frac{g(h)}{h} \to 0$ as $h \to 0$;\\
(c) if $s<t$, then $N(t)-N(s)$ is independent of all arrivals prior to $s$.
\end{defi}

\begin{thm}
$N(t)$ has the Poisson distribution with parameter $\lambda t$.
\begin{proof}
Study $N(t+h)$ given $N(t)$. We have 
\begin{equation*}
\begin{aligned}
\P(N(t+h) =j) &= \sum_{i\leq j} \P(N(t+h) \\
&= j|N(t) = i) \P(N(t) = i) \\
&= (1-\lambda h) \P(N(t) = j) + \lambda h \P(N(t) = h-1) + o(h)
\end{aligned}
\end{equation*}
So
\begin{equation*}
\begin{aligned}
\frac{\P(N(t+h)=j) - \P(N(t) = j)}{h} = -\lambda \P(N(t) = j) + \lambda \P (N(t) = j-1) + \frac{o(h)}{h}
\end{aligned}
\end{equation*}
write $p_n(t) = \P(N(t) = n)$, then let $h \to 0^+$ we get
\begin{equation*}
\begin{aligned}
p'j(t) = -\lambda p_j(t) + \lambda p_{j-1}(t) & j \geq 1\\
p'0_(t) = -\lambda p_0(t) &
\end{aligned}
\end{equation*}
with boundary condition $p_0(0) = 1$.\\
We solve $p_0$ to get $p_0(t) = e^{-\lambda(t)}$. Then we can use this to inductively solve $p_1,p_2,...$ to get the desired result.
\end{proof}
\end{thm}

An alternative derivation from the differential equations:\\
Let $G(s,t) = \sum_j s^j p_j(t)$. Now we take the set of differential equation, multiplying each one by $s^j$, then we get
\begin{equation*}
\begin{aligned}
\frac{\partial G}{\partial t} = \lambda (s-1) G
\end{aligned}
\end{equation*}
Then we have $$G(s,t) = A(s) e^{\lambda (s-1) t}$$ We also have $G(s,0)=1$ so we should be able to plug in a suitable value of $s$ to get the desired result (I probably missed that).





\iffalse
\begin{equation*}
\begin{aligned}

\end{aligned}
\end{equation*}
\fi
\end{document}
